\documentclass[border=3pt,tikz]{standalone}
% -- Retrieving git hash
\usepackage{xstring}
\usepackage{catchfile}
\CatchFileDef{\HEAD}{../.git/refs/heads/main}{}
\newcommand{\gitrevision}{%
\StrLeft{\HEAD}{7}%
}


\usepackage{dsfont}  % Provides \mathds for better fonts than amsfonts


\mdtheorem[style=defistyle]{defi}{Definition}[chapter]%[section]
\mdtheorem[style=satzstyle]{satz}[defi]{Satz}
\mdtheorem[style=satzstyle]{lemma}[defi]{Lemma}
\mdtheorem[style=satzstyle]{cor}[defi]{Korollar}
\mdtheorem[style=bspstyle]{bsp}[defi]{Beispiel}



\usepackage{tikz}
\usepackage{ifthen}
\usepackage{xcolor}
\usepackage{scrlayer-scrpage}



% \documentclass[../relativity_main.tex]{subfiles}



\pgfkeys{
	/addlightsignal/.is family, /addlightsignal,
	default/.style = {v = 0, color = red},
	v/.estore in = \lightsignalVelocity,
	vtwo/.estore in = \lightsignalVelocityTwo,
	color/.estore in = \lightsignalColor
}

% Mandatory argument: x position of first event, y position of first event, x position of second event, y position of second event
% Optional arguments: relative velocity (important: if negative, must be given as (-1) * v where v is the absolute value, otherwise error), color
\newcommand{\addlightsignal}[5][]{%
	\pgfkeys{/addlightsignal, default, #1}

    % Evaluate the Lorentz transformation
    %\FPeval{\calcgamma}{1/((1-(#3)^2)^.5)}
    \FPeval{\calcgamma}{1/((1-((\lightsignalVelocity)*(\lightsignalVelocity)))^.5)} % More robust, allows negative v
    \FPeval{\calcbetagamma}{\calcgamma*\lightsignalVelocity}

	% Draw event
	% \draw[thick, \lightsignalColor, cm={\calcgamma,\calcbetagamma,\calcbetagamma,\calcgamma,(0,0)}] (#2, #3) -- (#4, #5);
    
    \FPeval{\calcgammaone}{1/((1-((\lightsignalVelocity)*(\lightsignalVelocity)))^.5)}
    \FPeval{\calcbetagammaone}{\calcgammaone*\lightsignalVelocity}
    \FPeval{\calcgammatwo}{1/((1-((\lightsignalVelocityTwo)*(\lightsignalVelocityTwo)))^.5)}
    \FPeval{\calcbetagammatwo}{\calcgammatwo*\lightsignalVelocityTwo}

    \draw ({\calcgammaone*#2 + \calcbetagammaone*#3}, {\calcgammaone*#3 + \calcbetagammaone*#2}) -- ({\calcgammatwo*#4 + \calcbetagammatwo*#5}, {\calcgammatwo*#5 + \calcbetagammatwo*#4});
}




\tikzset{
  vlorentz/.style={
    % /pgf/gamma/.expanded=\FPeval{#1},
    % /pgf/betagamma/.expanded=\FPeval{#2},
    % Apply the calculated shifts
    % cm={(\pgfkeysvalueof{/pgf/number1} cm,\pgfkeysvalueof{/pgf/number2} cm)}
    % 
    % /pgf/gamma/.expanded=\FPeval{\calcgamma}{1/((1-((#1)*(#1)))^.5)}, % More robust, allows negative v
    % /pgf/betagamma/.expanded=\FPeval{\calcbetagamma}{\calcgamma*#1},
    % cm={\calcgamma,\calcbetagamma,\calcbetagamma,\calcgamma,(0,0)}
    % cm={\pgfkeysvalueof{/pgf/gamma},\pgfkeysvalueof{/pgf/betagamma},\pgfkeysvalueof{/pgf/betagamma},\pgfkeysvalueof{/pgf/gamma},(0,0)}
    % cm={0.5,0.2,0.2,0.5,(0,0)}
    % cm={#1,0.0,#1,0.0,(0,0)}
    % cm={1/((1-((#1)*(#1)))^.5,1/((1-((#1)*(#1)))^.5*#1,1/((1-((#1)*(#1)))^.5*#1,1/((1-((#1)*(#1)))^.5,(0,0)}
    cm={1/((1-((#1)*(#1)))^.5,#1*1/((1-((#1)*(#1)))^.5,#1*1/((1-((#1)*(#1)))^.5,1/((1-((#1)*(#1)))^.5,(0,0)}
  }
}


\begin{document}


\begin{tikzpicture}[thick]
    \spacetimediagram[lightcone=false]{5}

    % \def\v{0.0}
    % \def\v{0.25}
    \def\v{0.5}

    % \addworldline[v=\v]{0}{0}{0}{6}  % Observer
    % \addworldline[v=\v]{2}{0}{2}{6}  % Observer

    % \addworldline[v=\v, color=lightyellow]{0}{0}{2}{2}
    % \addworldline[v=\v, color=lightyellow]{2}{2}{0}{4}
    
    % \addworldline[v=\v, color=lightyellow]{2}{0}{0}{2}
    % \addworldline[v=\v, color=lightyellow]{0}{2}{2}{4}



    % \def\v{0.2}
    \def\v{0.0}
    % \def\vtwo{0.4}
    \def\vtwo{0.5}

    \addevent[v=\vtwo]{2}{2}  % Test
    
    
    % \FPeval{\calcgammaone}{1/((1-((\v)*(\v)))^.5)}
    % \FPeval{\calcbetagammaone}{\calcgammaone*\v}
    % \FPeval{\calcgammatwo}{1/((1-((\vtwo)*(\vtwo)))^.5)}
    % \FPeval{\calcbetagammatwo}{\calcgammatwo*\vtwo}

    % \def\xone{0}
    % \def\tone{0}
    % \def\xtwo{2}
    % \def\ttwo{2}

    % \draw ({\calcgammaone*(\xone + \calcbetagammaone*\tone)}, {\calcgammaone*(\tone + \calcbetagammaone*\xone)}) -- ({\calcgammatwo*(\xtwo + \calcbetagammatwo*\ttwo)}, {\calcgammatwo*(\ttwo + \calcbetagammatwo*\xtwo)});
    

    % \def\initsep{1}
    % % Ah shit, that means clocks are already desynchronized in t=0

    % \addworldline[v=\v]{0}{0}{0}{6}  % Observer
    % \addworldline[v=\vtwo]{\initsep}{0}{\initsep}{6}  % Observer

    % \addlightsignal[v=\v, vtwo=\vtwo]{0}{0}{\initsep}{\initsep}
    
    % % \addlightsignal[v=\v, vtwo=\vtwo]{\initsep}{\initsep}{0}{2*\initsep}
    % % \addlightsignal[v=\vtwo, vtwo=\v]{\initsep}{\initsep}{0}{2*\initsep}
    
    % % \addlightsignal[v=\v, vtwo=\vtwo]{0}{2}{\initsep}{\initsep}
    % % \addlightsignal[v=\vtwo, vtwo=\v]{\initsep}{\initsep}{0}{2}

    % \addlightsignal[v=\vtwo, vtwo=\v]{\initsep}{\initsep}{0}{2*\initsep*\calcgammaone*\calcgammaone*\calcgammaone}  % Pure luck that it looks good with v=0.2, vtw=0.4

    % % \FPeval{\calcdopplerone}{((1+\v)(1-\v))^.5}
    % % \FPeval{\calcdopplertwo}{((1+\vtwo)(1-\vtwo))^.5}
    % \FPeval{\calcdopplerone}{((1-\v)(1+\v))^.5}
    % \FPeval{\calcdopplertwo}{((1-\vtwo)(1+\vtwo))^.5}
    % \FPeval{\calcdoppleronetwo}{\calcdopplerone*\calcdopplertwo}
    
    % % \addevent{0}{\calcdopplerone*\initsep}
    % % \addevent[v=\v]{0}{\calcdopplerone*\initsep}
    % % \addevent[v=\vtwo]{0}{\calcdopplerone*\initsep}
    
    % % \addevent[v=\vtwo]{\initsep}{\initsep}
    % % \addevent[v=\vtwo]{\initsep}{\calcdopplerone*\initsep}
    % % \addevent[v=\vtwo]{\initsep}{\calcdopplertwo*\initsep}
    % % \addevent[v=\vtwo]{\initsep}{\calcdoppleronetwo*\initsep}

    % \addevent[]{0}{\calcdoppleronetwo*\initsep}
    % \addevent[]{0}{\calcdoppleronetwo*\calcdoppleronetwo*2*\initsep}
    
    % \addevent[v=\vtwo]{0}{\calcdoppleronetwo*\initsep}




    \def\v{0.0}
    % \def\v{0.2}
    \def\vtwo{0.5}
    \def\initsep{0}

    \def\tsignal{1}

    \addworldline[v=\v]{0}{0}{0}{6}  % Observer
    \addworldline[v=\vtwo]{0}{0}{0}{6}  % Observer

    \addlightsignal[v=\v, vtwo=\vtwo]{0}{\tsignal}{\tsignal}{2*\tsignal}
    
    \addlightsignal[v=\vtwo, vtwo=\v]{\tsignal}{2*\tsignal}{0}{3*\tsignal}

    \FPeval{\calcdopplerone}{((1+\v)/(1-\v))^.5}
    \FPeval{\calcdopplertwo}{((1+\vtwo)/(1-\vtwo))^.5}
    % \FPeval{\calcdopplerone}{((1-\v)/(1+\v))^.5}
    % \FPeval{\calcdopplertwo}{((1-\vtwo)/(1+\vtwo))^.5}
    \FPeval{\calcdoppleronetwo}{\calcdopplerone*\calcdopplertwo}
    
    % \addevent{0}{\calcdopplerone*\initsep}
    % \addevent[v=\v]{0}{\calcdopplerone*\initsep}
    % \addevent[v=\vtwo]{0}{\calcdopplerone*\initsep}
    
    % \addevent[v=\vtwo]{\initsep}{\initsep}
    % \addevent[v=\vtwo]{\initsep}{\calcdopplerone*\initsep}
    % \addevent[v=\vtwo]{\initsep}{\calcdopplertwo*\initsep}
    % \addevent[v=\vtwo]{\initsep}{\calcdoppleronetwo*\initsep}


    \addevent[v=\v]{0}{\tsignal}
    
    % \addevent[v=\vtwo]{0}{\calcdoppleronetwo*\tsignal}
    \FPeval{\vonetwo}{(\v+\vtwo)/(1+\v*\vtwo)}
    \addevent[v=\vonetwo]{0}{\calcdoppleronetwo*\tsignal}

    % \addevent[v=\v]{0}{\calcdoppleronetwo*\tsignal}
    \addevent[v=\v]{0}{\calcdoppleronetwo*\calcdoppleronetwo*\tsignal}


    % Does not work yet for both moving. Fixed below
\end{tikzpicture}



\begin{tikzpicture}
    \spacetimediagram[lightcone=false]{5}
    
	% \draw[] (0,1) node[circle, fill, red] {};
	% \draw[vlorentz=0.5, ultra thick] (0,1) node[circle, fill, red] {};

    % \begin{scope}[vlorentz=0.2]
	%     \draw[vlorentz=0.5, ultra thick] (0,1) node[circle, fill, red] {};
    % \end{scope}

    % Works!!!

    

    % \def\v{0.2}
    % \def\v{-0.1}
    \def\v{0.0}
    \def\vtwo{0.5}
    \def\initsep{0}
    
    \def\tsignal{1}

    \FPeval{\vonetwo}{(\v+\vtwo)/(1+\v*\vtwo)}
    \FPeval{\vonetworel}{(\v-\vtwo)/(1-\v*\vtwo)}

    \begin{scope}[vlorentz=0.2]  % YESSS, WORKS

    \addworldline[v=\v]{0}{0}{0}{6}  % Observer
    \addworldline[v=\vonetwo]{0}{0}{0}{6}  % Observer

    
    \FPeval{\calcdopplerone}{((1+\v)/(1-\v))^.5}
    \FPeval{\calcdopplertwo}{((1+\vtwo)/(1-\vtwo))^.5}
    \FPeval{\calcdoppleronetwo}{\calcdopplerone*\calcdopplertwo}
    

    \addevent[v=\v]{0}{\tsignal}
    % \addevent[v=\vtwo]{0}{\calcdoppleronetwo*\tsignal}

    % \addevent[v=\vonetwo]{0}{\calcdoppleronetwo*\tsignal}  % Must account for time dilation here, right?
    \addevent[v=\vonetwo]{0}{\calcdoppleronetwo*\tsignal}

    % \addevent[v=\v]{0}{\calcdoppleronetwo*\tsignal}
    \addevent[v=\v]{0}{\calcdoppleronetwo*\calcdoppleronetwo*\tsignal}


    \addevent[v=\vonetwo, color=blue]{0}{\tsignal}
    \addevent[v=\v, color=blue]{0}{\calcdoppleronetwo*\tsignal}
    \addevent[v=\vonetwo, color=blue]{0}{\calcdoppleronetwo*\calcdoppleronetwo*\tsignal}
    \end{scope}
\end{tikzpicture}



% \makeatother
\makeatletter

\pgfkeys{
	/synchrobs/.is family, /synchrobs,
	default/.style = {vone=0.0, vtwo=0.5, tsignal=1, tmin=0.0, tmax=6.0, initsep=0.0},
	vone/.estore in = \synchrobs@vone,
	vtwo/.estore in = \synchrobs@vtwo,
	tsignal/.estore in = \synchrobs@tsignal,
	tmin/.estore in = \synchrobs@tmin,
	tmax/.estore in = \synchrobs@tmax,
	initsep/.estore in = \synchrobs@initsep,
}


\usetikzlibrary{decorations.markings,decorations.pathmorphing}

\tikzset{
    light/.style={
        lightyellow,
        line width=0.8,
        decorate,
        decoration={
            snake,
            amplitude=0.67,
            segment length=4.2,
            post length=4.2,
        }
    }
}

\newcommand{\EinsteinSynchronizedObservers}[1][]{%
	\pgfkeys{/synchrobs, default, #1}

    \def\vone{\synchrobs@vone}
    \def\vtwo{\synchrobs@vtwo}
    \def\tsignal{\synchrobs@tsignal}
    
    \FPeval{\vtotal}{(\vone+\vtwo)/(1+\vone*\vtwo)}
    \FPeval{\vrelative}{(\vtwo-\vone)/(1-\vone*\vtwo)}
    % \FPeval{\vrelative}{(\vone-\vtwo)/(1-\vone*\vtwo)}
    % \FPeval{\vrelative}{(\vtwo-\vone)/(1+\vone*\vtwo)}
    % \FPeval{\vrelative}{(\vtwo-\vone)/(1+\vone*\vtwo)}
    
    \FPeval{\dopplerone}{((1+\vone)/(1-\vone))^.5}
    \FPeval{\dopplertwo}{((1+\vtwo)/(1-\vtwo))^.5}
    % \FPeval{\doppleronetwo}{\dopplertwo}
    \FPeval{\doppleronetwo}{\dopplerone*\dopplertwo}
    
    % \FPeval{\dopplertwo}{((1+\vrelative)/(1-\vrelative))^.5}

    \FPeval{\doppleronetwo}{\dopplertwo/\dopplerone}
    % \FPeval{\doppleronetwo}{\dopplerone/\doppletwo}


    \begin{scope}[vlorentz=\vone]
        \coordinate (A) at (0, \tsignal);
        \coordinate (E) at (0, \doppleronetwo*\tsignal);
        \coordinate (B) at (0, \doppleronetwo*\doppleronetwo*\tsignal);
    \end{scope}

    % \begin{scope}[vlorentz=\vtotal]
    \begin{scope}[vlorentz=\vtwo]
        \coordinate (C) at (\initsep, \tsignal);
        \coordinate (F) at (\initsep, \doppleronetwo*\tsignal);
        \coordinate (D) at (\initsep, \doppleronetwo*\doppleronetwo*\tsignal);
    \end{scope}


    % \begin{scope}[vlorentz=\vone]
    %     \coordinate (A) at (0, \tsignal);
    %     \coordinate (E) at (0, \doppleronetwo*\tsignal);
    %     \coordinate (B) at (0, \doppleronetwo*\doppleronetwo*\tsignal);
        
    %     \begin{scope}[vlorentz=\vtwo]
    %         \coordinate (C) at (0, \tsignal);
    %         \coordinate (F) at (0, \doppleronetwo*\tsignal);
    %         \coordinate (D) at (0, \doppleronetwo*\doppleronetwo*\tsignal);
    %     \end{scope}
    % \end{scope}


    % -- Light from 1 to 2
    % \begin{scope}[vlorentz=\vtotal]
    %     \draw[lightyellow] (0, \tsignal) -- (0, \doppleronetwo*\tsignal);
    %     \draw[lightyellow] (0, \doppleronetwo*\tsignal) -- (0, \doppleronetwo*\doppleronetwo*\tsignal);
    % \end{scope}
    
    % \draw[light] (A) -- (F) -- (B);  % Light from 1 to 2 -> not so nice with decorate
    \draw[light] (A) -- (F);
    \draw[light] (F) -- (B);
    % \draw[light] (C) -- (E) -- (D);  % Light from 2 to 1
    \draw[light] (C) -- (E);
    \draw[light] (E) -- (D);

    
    % -- Observer worldlines
	\draw[thick, ->, vlorentz=\vone] (0, \synchrobs@tmin) -- (0, \synchrobs@tmax);
	\draw[thick, ->, vlorentz=\vtwo] (\synchrobs@initsep, \synchrobs@tmin) -- (\synchrobs@initsep, \synchrobs@tmax);

    % TODO: clip here?

    % -- Draw point halfway between emission and reception
	\draw[vlorentz=\vone] (0, {(1+\doppleronetwo*\doppleronetwo)/2*\tsignal}) node[circle, fill, red] {};
	\draw[vlorentz=\vtwo] (0, {(1+\doppleronetwo*\doppleronetwo)/2*\tsignal}) node[circle, fill, red] {};
}

% \makeatletter
\makeatother


\begin{tikzpicture}
    % \spacetimediagram[]{6}

    % \EinsteinSynchronizedObservers[vone=0.0,tsignal=1,tmax=4.2]

    % \EinsteinSynchronizedObservers[vone=0.2,vtwo=0.7,tsignal=1,tmax=4.2]

    \EinsteinSynchronizedObservers[vone=-0.1,vtwo=0.5,tsignal=1,tmax=4.2]
\end{tikzpicture}



\end{document}