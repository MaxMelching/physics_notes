\documentclass[border=3pt,tikz]{standalone}
% -- Add bib file for relativity
\addbibresource{biblio.bib}

\definecolor{caltechorange}{RGB}{239, 119, 35}
\definecolor{headercolor}{named}{caltechorange}

\definecolor{amethyst}{rgb}{0.6, 0.4, 0.8}
\definecolor{persianblue}{rgb}{0.11, 0.22, 0.73}


% -- Customized headline style
\IfFileExists{gwheadline.sty}{
	% -- Available at https://github.com/MaxMelching/gwheadline
	\usepackage[
		% templatefile="eccentric_template.txt",
	]{gwheadline}

	\KOMAoptions{headsepline=false}
	
	\clearmainofpairofpagestyles  % Reset everything
	% \clearpairofpagestyles  % Also reset custom styles -> that's not what it does...

	
	\tikzset{
		gwline@linestyle/.style={
			smooth,
			very thick,
			headercolor,
		}
	}
	
	\ifthenelse{\boolean{@twoside}}{
	  \lehead{{%
		\gwheadline[templatefile="None"]{}%
		\hypersetup{hidelinks}\thepage/\pageref{LastPage}}%
	  }
	  \rehead{\leftmark}
	
	  \lohead{%
		\gwheadline{}%
		\leftmark%
	  }
	  \rohead{{\hypersetup{hidelinks}\thepage/\pageref{LastPage}}}
	}{
	  \ihead{\headmark}
	  %\ohead[]{Seite~\thepage}  % Like it better in footer
	  \cfoot{{\hypersetup{hidelinks}Page~\thepage~of~\pageref{LastPage}}}
	  % -- NOTE: if lastpage is removed as package, use \pagemark instead of \thepage
	}
	
	
	% -- Some custom pagestyles to handle things like big floats. Use them via
	% -- \thisfloatpagestyle{onlyheader}. Non-trivial task because floats may be
	% -- inserted at different page compared to text surrounding them (package fings
	% -- page they appear on and sets pagestyle there).
	\usepackage{floatpag}  % Provides \floatpagestyle
	
	\ifthenelse{\boolean{@twoside}}{
	  \newpairofpagestyles{onlyheader}{%
		\lehead{{%
		  \gwheadline[templatefile="None"]{}%
		  \hypersetup{hidelinks}\thepage/\pageref{LastPage}}%
		}%
		\rehead{\leftmark}%
		\lohead{%
		  \gwheadline{}%
		  \leftmark%
		}%
		\rohead{{\hypersetup{hidelinks}\thepage/\pageref{LastPage}}}
	  }
	

	  \newpairofpagestyles{onlyfooter}{%
		% \lehead{}%
		% \rehead{}%
		% \lohead{}%
		% \rohead{}%
		% %
		% \cfoot{}%
	  }
	}{
		\newpairofpagestyles{onlyheader}{%
		\lehead{{%
		\gwheadline[templatefile="None"]{}%
		\hypersetup{hidelinks}\thepage/\pageref{LastPage}}%
		}%
		\rehead{\leftmark}%
		\lohead{%
		\gwheadline{}%
		\leftmark%
		}%
		\rohead{{\hypersetup{hidelinks}\thepage/\pageref{LastPage}}}
		}

		\newpairofpagestyles{onlyfooter}{%
			% \lehead{}%
			% \rehead{}%
			% \lohead{}%
			% \rohead{}%
			% %
			% \cfoot{}%
		}
	}
}{
  \ifthenelse{\boolean{@twoside}}{
	\newpairofpagestyles{onlyheader}{%
	  \lehead{{%
		\hypersetup{hidelinks}\thepage/\pageref{LastPage}}%
	  }%
	  \rehead{\leftmark}%
	  \lohead{%
		\leftmark%
	  }%
	  \rohead{{\hypersetup{hidelinks}\thepage/\pageref{LastPage}}}
	}
  
	\newpairofpagestyles{onlyfooter}{%
	  % \lehead{}%
	  % \rehead{}%
	  % \lohead{}%
	  % \rohead{}%
	  % %
	  % \cfoot{}%
	}
  }{
	\newpairofpagestyles{onlyheader}{%
	  \automark[section]{section}%
	  %\automark*[section]{subsection}%
	  \ihead[]{\headmark}%
	  %
	  % -- Only change to previous settings is here
	  \cfoot{}%
	}
  
	\newpairofpagestyles{onlyfooter}{%
	  \automark[section]{section}%
	  %\automark*[section]{subsection}%
	  % -- Only change to previous settings is here
	  \ihead[]{}%
	  %
	  \cfoot{{\hypersetup{linkcolor=black}Page~\thepage~of~\pageref{LastPage}}}%
	}
  }
}


% -- Retrieving git hash
\usepackage{xstring}
\usepackage{catchfile}
\CatchFileDef{\HEAD}{../.git/refs/heads/main}{}
\newcommand{\gitrevision}{%
	\StrLeft{\HEAD}{7}%
}


\usepackage{clock}  % For nice clock diagrams


% Idea for simpler syntax: renew \boxed command from amsmath; seems to work like fbox, so maybe background color can be changed there

\usepackage[most]{tcolorbox}
%\colorlet{eqcolor}{}
\definecolor{atomictangerine}{rgb}{1.0, 0.6, 0.4}
\definecolor{bittersweet}{rgb}{1.0, 0.44, 0.37}
\definecolor{burntsienna}{rgb}{0.91, 0.45, 0.32}
\definecolor{coral}{rgb}{1.0, 0.5, 0.31}
\definecolor{outrageousorange}{rgb}{1.0, 0.43, 0.29}

\tcbset{
	on line, 
	boxsep=4pt, left=0pt,right=0pt,top=0pt,bottom=0pt,
%	colframe=cyan,colback=cyan!42,
	% Testing with new colors
%	colframe=coral,colback=coral!42, % Very nice, decent
%	colframe=burntsienna,colback=burntsienna!42, % A bit darker than coral, but still Very nice
%	colframe=bittersweet,colback=bittersweet!42, % More red than coral, thus a bit more punch, but nice still
	colframe=atomictangerine,colback=atomictangerine!42, % More of a classical orange, very light and thus decent -> might be favorite right now
%	colframe=outrageousorange,colback=outrageousorange!42, % A bit stronger color, but nice
	highlight math style={enhanced}
}

\renewcommand{\eqbox}[1]{\tcbhighmath{#1}}

\newcommand{\manyqquad}{\qquad \qquad \qquad \qquad}  % Four seems to be sweet spot



\mdtheorem[style=defistyle]{defi}{Definition}[chapter]%[section]
\mdtheorem[style=satzstyle]{thm}[defi]{Theorem}
\mdtheorem[style=satzstyle]{prop}[defi]{Property}
\mdtheorem[style=satzstyle]{post}[defi]{Postulate}
\mdtheorem[style=satzstyle]{lemma}[defi]{Lemma}
\mdtheorem[style=satzstyle]{cor}[defi]{Corollary}
\mdtheorem[style=bspstyle]{ex}[defi]{Example}



% -- Spacetime diagrams -------------------------------------------------------
\usetikzlibrary{arrows.meta}
%\tikzset{>={Latex[scale=1.2]}}
\tikzset{>={Stealth[inset=0,angle'=27]}}

%\usepackage{tsemlines}  % To draw Dragon stuff; Bard says this works with emline, not pstricks
%\def\emline#1#2#3#4#5#6{%
%       \put(#1,#2){\special{em:moveto}}%
%       \put(#4,#5){\special{em:lineto}}}


% Inspiration: https://de.overleaf.com/latex/templates/minkowski-spacetime-diagram-generator/kqskfzgkjrvq, https://www.overleaf.com/latex/examples/spacetime-diagrams-for-uniformly-accelerating-observers/kmdvfrhhntzw

% Ah, nice for accelerated observers: https://de.wikipedia.org/wiki/Zeitdilatation#Bewegung_mit_konstanter_Beschleunigung

\usepackage{fp}
\usetikzlibrary{math}  % Allows declaration of variables, very convenient
\usepackage{pgfkeys}

\usepackage{tikz}
\usetikzlibrary{calc,intersections,through,backgrounds} % https://tex.stackexchange.com/questions/31398/tikz-intersection-of-two-lines


% Define fixed color for light rays
\definecolor{lightyellow}{rgb}{0,0,0}  % Start with black
\colorlet{lightyellow}{black!10!yellow}  % Mix with 10% of yellow


\pgfkeys{
	/spacetimediagram/.is family, /spacetimediagram,
	default/.style = {stepsize = 1, xlabel = $x$, ylabel = $ct$, color = black, onlypositive = false, lightcone = true, grid = true},
	stepsize/.estore in = \diagramStepsize,
	xlabel/.estore in = \diagramxlabel,
	ylabel/.estore in = \diagramylabel,
	color/.estore in = \diagramColor,
	onlypositive/.estore in = \diagramPositiveVals,
	lightcone/.estore in = \diagramLightCone,
	grid/.estore in = \diagramGrid
}
	%lightcone/.estore in = \diagramlightcone  % Maybe also make optional?
	% Maybe add argument if grid is drawn or markers along axis? -> nope, they are really important

% Mandatory argument: grid size
% Optional arguments: stepsize (sets grid scale), xlabel, ylabel, color
\newcommand{\spacetimediagram}[2][]{%
	\pgfkeys{/spacetimediagram, default, #1}

	\ifthenelse{\equal{\diagramPositiveVals}{false}}{
		\tikzmath{\helpval = 1;}
	}{
		\tikzmath{\helpval = 0;}
	}

    % Draw the x ct grid
	\ifthenelse{\equal{\diagramGrid}{true}}{
	    \draw[\diagramColor, step=\diagramStepsize, gray!30, very thin] (-#2 * \diagramStepsize * \helpval, -#2 * \diagramStepsize * \helpval) grid (#2 * \diagramStepsize, #2 * \diagramStepsize);
	}{}

    % Draw the x and ct axes
    \draw[->, thick, \diagramColor] (-#2 * \diagramStepsize * \helpval - \diagramStepsize * \helpval, 0) -- (#2 * \diagramStepsize + \diagramStepsize, 0);
    \draw[->, thick, \diagramColor] (0, -#2 * \diagramStepsize * \helpval - \diagramStepsize * \helpval) -- (0, #2 * \diagramStepsize + \diagramStepsize);

	% Draw the x and ct axes labels
    \draw (#2 * \diagramStepsize + \diagramStepsize + 0.2, 0) node[\diagramColor] {\diagramxlabel};
    \draw (0, #2 * \diagramStepsize + \diagramStepsize + 0.2) node[\diagramColor] {\diagramylabel};

	% Draw light cone
	\ifthenelse{\equal{\diagramLightCone}{true}}{
		\draw[lightyellow, thick] (-#2 * \diagramStepsize * \helpval, -#2 * \diagramStepsize * \helpval) -- (#2 * \diagramStepsize, #2 * \diagramStepsize);
		\draw[lightyellow, thick] (-#2 * \diagramStepsize * \helpval, #2 * \diagramStepsize * \helpval) -- (#2 * \diagramStepsize * \helpval, -#2 * \diagramStepsize * \helpval);
	}{}
}
% Hmmm, replace spacetimediagram code by just calling \addobserver with velocity of zero? Then light cone stuff would have to be added to observer, but this is doable (could add it with default value false, but then enable it for spacetime diagram)



\pgfkeys{
	/addobserver/.is family, /addobserver,
	default/.style = {grid = true, stepsize = 1, xpos = 0, ypos = 0, xlabel = $x'$, ylabel = $c t'$, color = blue},
	grid/.estore in = \observerGrid,
	stepsize/.estore in = \observerStepsize,
	xpos/.estore in = \observerxpos,
	ypos/.estore in = \observerypos,
	xlabel/.estore in = \observerxlabel,
	ylabel/.estore in = \observerylabel,
	color/.estore in = \observerColor
}

% Mandatory argument: grid size, relative velocity (important: if negative, must be given as (-1) * v where v is the absolute value, otherwise error)
% Optional arguments: stepsize (sets grid scale), xlabel, ylabel
\newcommand{\addobserver}[3][]{%
	\pgfkeys{/addobserver, default, #1}

    % Evaluate the Lorentz transformation -> we rotate axes matrix ((gamma, beta*gamma), (beta*gamma, gamma)) and no minus in front of beta because what we want is that event with coordinates (x, ct) has coordinate x'=gamma(x - beta t); since we do not rotate the event, axes have to be rotated in "opposite" direction; then reading of coordinates from there works in the desired way
	% Equivalent way of seeing it: we give coordinates in (ct', x') frame and then have to get the corresponding ones in (ct, x) frame, this transformation has a +
    %\FPeval{\calcgamma}{1/((1-(#3)^2)^.5)}
    \FPeval{\calcgamma}{1/((1-((#3)*(#3)))^.5)} % More robust, allows negative v
    \FPeval{\calcbetagamma}{\calcgamma*#3}

	% Draw the x' and ct' axes
	\draw[->, thick, cm={\calcgamma,\calcbetagamma,\calcbetagamma,\calcgamma,(\observerxpos,\observerypos)}, \observerColor] (-#2 * \observerStepsize - \observerStepsize, 0) -- (#2 * \observerStepsize + \observerStepsize, 0);
    \draw[->, thick, cm={\calcgamma,\calcbetagamma,\calcbetagamma,\calcgamma,(\observerxpos,\observerypos)}, \observerColor] (0, -#2 * \observerStepsize - \observerStepsize) -- (0, #2 * \observerStepsize + \observerStepsize);

	% Check if grid shall be drawn
	\ifthenelse{\equal{\observerGrid}{true}}{%#
		% Draw transformed grid
		\draw[step=\observerStepsize, \observerColor, very thin, cm={\calcgamma,\calcbetagamma,\calcbetagamma,\calcgamma,(\observerxpos,\observerypos)}] (-#2 * \observerStepsize, -#2 * \observerStepsize) grid (#2 * \observerStepsize, #2 * \observerStepsize);
	}{} % Do nothing in else case

	% Draw the x' and ct' axes labels
    \draw[cm={\calcgamma,\calcbetagamma,\calcbetagamma,\calcgamma,(\observerxpos,\observerypos)}, \observerColor] (#2 * \observerStepsize + \observerStepsize + 0.2, 0) node {\observerxlabel};
    \draw[cm={\calcgamma,\calcbetagamma,\calcbetagamma,\calcgamma,(\observerxpos,\observerypos)}, \observerColor] (0, #2 * \observerStepsize + \observerStepsize + 0.2) node {\observerylabel};
}



\pgfkeys{
	/addevent/.is family, /addevent,
	default/.style = {v = 0, label =, color = red, label placement = below, radius = 5},
	v/.estore in = \eventVelocity,
	label/.estore in = \eventLabel,
	color/.estore in = \eventColor,
	label placement/.estore in = \eventLabelPlacement,
	radius/.estore in = \circleEventRadius
}

% Mandatory argument: x position, y position
% Optional arguments: relative velocity (important: if negative, must be given as (-1) * v where v is the absolute value, otherwise error), label, color, label placement
\newcommand{\addevent}[3][]{%
	\pgfkeys{/addevent, default, #1}

    % Evaluate the Lorentz transformation
    %\FPeval{\calcgamma}{1/((1-(#3)^2)^.5)}
    \FPeval{\calcgamma}{1/((1-((\eventVelocity)*(\eventVelocity)))^.5)} % More robust, allows negative v
    \FPeval{\calcbetagamma}{\calcgamma*\eventVelocity}

	% Draw event
	\draw[cm={\calcgamma,\calcbetagamma,\calcbetagamma,\calcgamma,(0,0)}] (#2,#3) node[circle, fill, \eventColor, minimum size=\circleEventRadius, label=\eventLabelPlacement:\eventLabel] {};
}



\pgfkeys{
	/addworldline/.is family, /addworldline,
	default/.style = {v = 0, color = red},
	v/.estore in = \worldlineVelocity,
	color/.estore in = \worldlineColor
}

% Mandatory argument: x position of first event, y position of first event, x position of second event, y position of second event
% Optional arguments: relative velocity (important: if negative, must be given as (-1) * v where v is the absolute value, otherwise error), color
\newcommand{\addworldline}[5][]{%
	\pgfkeys{/addworldline, default, #1}

    % Evaluate the Lorentz transformation
    %\FPeval{\calcgamma}{1/((1-(#3)^2)^.5)}
    \FPeval{\calcgamma}{1/((1-((\worldlineVelocity)*(\worldlineVelocity)))^.5)} % More robust, allows negative v
    \FPeval{\calcbetagamma}{\calcgamma*\worldlineVelocity}

	% Draw event
	\draw[thick, \worldlineColor, cm={\calcgamma,\calcbetagamma,\calcbetagamma,\calcgamma,(0,0)}] (#2, #3) -- (#4, #5);
}



\pgfkeys{
	/lightcone/.is family, /lightcone,
	default/.style = {stepsize = 1, xpos = 0, ypos = 0, color = yellow, fill opacity = 0.42},
	stepsize/.estore in = \lightconeStepsize,
	xpos/.estore in = \lightconexpos,
	ypos/.estore in = \lightconeypos,
	color/.estore in = \lightconeColor,
	fill opacity/.estore in = \lightconeFillOpacity
}

% Mandatory arguments: cone size
% Optional arguments: stepsize (scale of grid), xpos, ypos, color, fill opacity
\newcommand{\lightcone}[2][]{
	\pgfkeys{/lightcone, default, #1}
	% Draw light cone -> idea: go from event location into the directions (1, 1), (-1, 1) for upper part of cone and then in directions (-1, -1), (1, -1) for lower part of cone
	\draw[\lightconeColor, fill, fill opacity=\lightconeFillOpacity] (\lightconexpos * \lightconeStepsize - #2 * \lightconeStepsize, \lightconeypos * \lightconeStepsize + #2 * \lightconeStepsize) -- (\lightconexpos, \lightconeypos) -- (\lightconexpos * \lightconeStepsize + #2 * \lightconeStepsize, \lightconeypos * \lightconeStepsize + #2 * \lightconeStepsize);
	\draw[\lightconeColor, fill, fill opacity=\lightconeFillOpacity] (\lightconexpos * \lightconeStepsize - #2 * \lightconeStepsize, \lightconeypos * \lightconeStepsize - #2 * \lightconeStepsize) -- (\lightconexpos, \lightconeypos) -- (\lightconexpos * \lightconeStepsize + #2 * \lightconeStepsize, \lightconeypos * \lightconeStepsize - #2 * \lightconeStepsize);
}



\pgfkeys{
	/addacceleratedworldline/.is family, /addacceleratedworldline,
	default/.style = {tstart = 0, tend = 1, v = 0, color = red},
	tstart/.estore in = \acceleratedworldlineTstart,
	tend/.estore in = \acceleratedworldlineTend,
	v/.estore in = \acceleratedworldlineVelocity,
	color/.estore in = \acceleratedworldlineColor
}

% Mandatory argument: x position of first event, y position of first event, x position of second event, y position of second event
% Optional arguments: relative velocity (important: if negative, must be given as (-1) * v where v is the absolute value, otherwise error), color
% starting points in #2, #3 and then starting velocity in #4, acceleration in #5; optional v is for Lorentz transform of world line
\newcommand{\addacceleratedworldline}[5][]{%
	\pgfkeys{/addacceleratedworldline, default, #1}

    % Evaluate the Lorentz transformation
    \FPeval{\calcgamma}{1/((1-((\acceleratedworldlineVelocity)*(\acceleratedworldlineVelocity)))^.5)} % More robust, allows negative v
    \FPeval{\calcbetagamma}{\calcgamma*\acceleratedworldlineVelocity}

	\FPeval{\calcgammanod}{1/((1-((#4)*(#4)))^.5)}

	% Draw event
	\draw[variable=\t, domain=\acceleratedworldlineTstart:\acceleratedworldlineTend, thick, smooth, \acceleratedworldlineColor, cm={\calcgamma,\calcbetagamma,\calcbetagamma,\calcgamma,(0,0)}] plot ({#2 + 1 * (#5)^(-1) * (sqrt(1 + (#5 * \t + #4 * \calcgammanod)^2) - \calcgammanod)}, {#3 + \t}); % Dividing by a (#5) yields error for negative a, this version not

	%\draw[domain=-4:4, very thick, smooth, variable=\t, color=purple] plot ({sqrt(1 + \t * \t)}, {\t});
}


\DeclareMathOperator{\hodge}{{\star}}
\newcommand{\fvec}[1]{\underline{#1}}
\newcommand{\tensor}[1]{\mathbf{#1}}
\newcommand{\unitvec}[1]{\vec{e}_{#1}}  % V1
% \newcommand{\unitvec}[1]{\hat{#1}}  % V2
% \renewcommand{\grad}{\vec{\nabla}}  % If arrowdel option of physics must be disabled


\usepackage{tikz}
\usepackage{ifthen}
\usepackage{xcolor}
\usepackage{scrlayer-scrpage}



% \documentclass[../relativity_main.tex]{subfiles}



\pgfkeys{
	/addlightsignal/.is family, /addlightsignal,
	default/.style = {v = 0, color = red},
	v/.estore in = \lightsignalVelocity,
	vtwo/.estore in = \lightsignalVelocityTwo,
	color/.estore in = \lightsignalColor
}

% Mandatory argument: x position of first event, y position of first event, x position of second event, y position of second event
% Optional arguments: relative velocity (important: if negative, must be given as (-1) * v where v is the absolute value, otherwise error), color
\newcommand{\addlightsignal}[5][]{%
	\pgfkeys{/addlightsignal, default, #1}

    % Evaluate the Lorentz transformation
    %\FPeval{\calcgamma}{1/((1-(#3)^2)^.5)}
    \FPeval{\calcgamma}{1/((1-((\lightsignalVelocity)*(\lightsignalVelocity)))^.5)} % More robust, allows negative v
    \FPeval{\calcbetagamma}{\calcgamma*\lightsignalVelocity}

	% Draw event
	% \draw[thick, \lightsignalColor, cm={\calcgamma,\calcbetagamma,\calcbetagamma,\calcgamma,(0,0)}] (#2, #3) -- (#4, #5);
    
    \FPeval{\calcgammaone}{1/((1-((\lightsignalVelocity)*(\lightsignalVelocity)))^.5)}
    \FPeval{\calcbetagammaone}{\calcgammaone*\lightsignalVelocity}
    \FPeval{\calcgammatwo}{1/((1-((\lightsignalVelocityTwo)*(\lightsignalVelocityTwo)))^.5)}
    \FPeval{\calcbetagammatwo}{\calcgammatwo*\lightsignalVelocityTwo}

    \draw ({\calcgammaone*#2 + \calcbetagammaone*#3}, {\calcgammaone*#3 + \calcbetagammaone*#2}) -- ({\calcgammatwo*#4 + \calcbetagammatwo*#5}, {\calcgammatwo*#5 + \calcbetagammatwo*#4});
}




\tikzset{
  vlorentz/.style={
    % /pgf/gamma/.expanded=\FPeval{#1},
    % /pgf/betagamma/.expanded=\FPeval{#2},
    % Apply the calculated shifts
    % cm={(\pgfkeysvalueof{/pgf/number1} cm,\pgfkeysvalueof{/pgf/number2} cm)}
    % 
    % /pgf/gamma/.expanded=\FPeval{\calcgamma}{1/((1-((#1)*(#1)))^.5)}, % More robust, allows negative v
    % /pgf/betagamma/.expanded=\FPeval{\calcbetagamma}{\calcgamma*#1},
    % cm={\calcgamma,\calcbetagamma,\calcbetagamma,\calcgamma,(0,0)}
    % cm={\pgfkeysvalueof{/pgf/gamma},\pgfkeysvalueof{/pgf/betagamma},\pgfkeysvalueof{/pgf/betagamma},\pgfkeysvalueof{/pgf/gamma},(0,0)}
    % cm={0.5,0.2,0.2,0.5,(0,0)}
    % cm={#1,0.0,#1,0.0,(0,0)}
    % cm={1/((1-((#1)*(#1)))^.5,1/((1-((#1)*(#1)))^.5*#1,1/((1-((#1)*(#1)))^.5*#1,1/((1-((#1)*(#1)))^.5,(0,0)}
    cm={1/((1-((#1)*(#1)))^.5,#1*1/((1-((#1)*(#1)))^.5,#1*1/((1-((#1)*(#1)))^.5,1/((1-((#1)*(#1)))^.5,(0,0)}
  }
}


\begin{document}


\begin{tikzpicture}[thick]
    \spacetimediagram[lightcone=false]{5}

    % \def\v{0.0}
    % \def\v{0.25}
    \def\v{0.5}

    % \addworldline[v=\v]{0}{0}{0}{6}  % Observer
    % \addworldline[v=\v]{2}{0}{2}{6}  % Observer

    % \addworldline[v=\v, color=lightyellow]{0}{0}{2}{2}
    % \addworldline[v=\v, color=lightyellow]{2}{2}{0}{4}
    
    % \addworldline[v=\v, color=lightyellow]{2}{0}{0}{2}
    % \addworldline[v=\v, color=lightyellow]{0}{2}{2}{4}



    % \def\v{0.2}
    \def\v{0.0}
    % \def\vtwo{0.4}
    \def\vtwo{0.5}

    \addevent[v=\vtwo]{2}{2}  % Test
    
    
    % \FPeval{\calcgammaone}{1/((1-((\v)*(\v)))^.5)}
    % \FPeval{\calcbetagammaone}{\calcgammaone*\v}
    % \FPeval{\calcgammatwo}{1/((1-((\vtwo)*(\vtwo)))^.5)}
    % \FPeval{\calcbetagammatwo}{\calcgammatwo*\vtwo}

    % \def\xone{0}
    % \def\tone{0}
    % \def\xtwo{2}
    % \def\ttwo{2}

    % \draw ({\calcgammaone*(\xone + \calcbetagammaone*\tone)}, {\calcgammaone*(\tone + \calcbetagammaone*\xone)}) -- ({\calcgammatwo*(\xtwo + \calcbetagammatwo*\ttwo)}, {\calcgammatwo*(\ttwo + \calcbetagammatwo*\xtwo)});
    

    % \def\initsep{1}
    % % Ah shit, that means clocks are already desynchronized in t=0

    % \addworldline[v=\v]{0}{0}{0}{6}  % Observer
    % \addworldline[v=\vtwo]{\initsep}{0}{\initsep}{6}  % Observer

    % \addlightsignal[v=\v, vtwo=\vtwo]{0}{0}{\initsep}{\initsep}
    
    % % \addlightsignal[v=\v, vtwo=\vtwo]{\initsep}{\initsep}{0}{2*\initsep}
    % % \addlightsignal[v=\vtwo, vtwo=\v]{\initsep}{\initsep}{0}{2*\initsep}
    
    % % \addlightsignal[v=\v, vtwo=\vtwo]{0}{2}{\initsep}{\initsep}
    % % \addlightsignal[v=\vtwo, vtwo=\v]{\initsep}{\initsep}{0}{2}

    % \addlightsignal[v=\vtwo, vtwo=\v]{\initsep}{\initsep}{0}{2*\initsep*\calcgammaone*\calcgammaone*\calcgammaone}  % Pure luck that it looks good with v=0.2, vtw=0.4

    % % \FPeval{\calcdopplerone}{((1+\v)(1-\v))^.5}
    % % \FPeval{\calcdopplertwo}{((1+\vtwo)(1-\vtwo))^.5}
    % \FPeval{\calcdopplerone}{((1-\v)(1+\v))^.5}
    % \FPeval{\calcdopplertwo}{((1-\vtwo)(1+\vtwo))^.5}
    % \FPeval{\calcdoppleronetwo}{\calcdopplerone*\calcdopplertwo}
    
    % % \addevent{0}{\calcdopplerone*\initsep}
    % % \addevent[v=\v]{0}{\calcdopplerone*\initsep}
    % % \addevent[v=\vtwo]{0}{\calcdopplerone*\initsep}
    
    % % \addevent[v=\vtwo]{\initsep}{\initsep}
    % % \addevent[v=\vtwo]{\initsep}{\calcdopplerone*\initsep}
    % % \addevent[v=\vtwo]{\initsep}{\calcdopplertwo*\initsep}
    % % \addevent[v=\vtwo]{\initsep}{\calcdoppleronetwo*\initsep}

    % \addevent[]{0}{\calcdoppleronetwo*\initsep}
    % \addevent[]{0}{\calcdoppleronetwo*\calcdoppleronetwo*2*\initsep}
    
    % \addevent[v=\vtwo]{0}{\calcdoppleronetwo*\initsep}




    \def\v{0.0}
    % \def\v{0.2}
    \def\vtwo{0.5}
    \def\initsep{0}

    \def\tsignal{1}

    \addworldline[v=\v]{0}{0}{0}{6}  % Observer
    \addworldline[v=\vtwo]{0}{0}{0}{6}  % Observer

    \addlightsignal[v=\v, vtwo=\vtwo]{0}{\tsignal}{\tsignal}{2*\tsignal}
    
    \addlightsignal[v=\vtwo, vtwo=\v]{\tsignal}{2*\tsignal}{0}{3*\tsignal}

    \FPeval{\calcdopplerone}{((1+\v)/(1-\v))^.5}
    \FPeval{\calcdopplertwo}{((1+\vtwo)/(1-\vtwo))^.5}
    % \FPeval{\calcdopplerone}{((1-\v)/(1+\v))^.5}
    % \FPeval{\calcdopplertwo}{((1-\vtwo)/(1+\vtwo))^.5}
    \FPeval{\calcdoppleronetwo}{\calcdopplerone*\calcdopplertwo}
    
    % \addevent{0}{\calcdopplerone*\initsep}
    % \addevent[v=\v]{0}{\calcdopplerone*\initsep}
    % \addevent[v=\vtwo]{0}{\calcdopplerone*\initsep}
    
    % \addevent[v=\vtwo]{\initsep}{\initsep}
    % \addevent[v=\vtwo]{\initsep}{\calcdopplerone*\initsep}
    % \addevent[v=\vtwo]{\initsep}{\calcdopplertwo*\initsep}
    % \addevent[v=\vtwo]{\initsep}{\calcdoppleronetwo*\initsep}


    \addevent[v=\v]{0}{\tsignal}
    
    % \addevent[v=\vtwo]{0}{\calcdoppleronetwo*\tsignal}
    \FPeval{\vonetwo}{(\v+\vtwo)/(1+\v*\vtwo)}
    \addevent[v=\vonetwo]{0}{\calcdoppleronetwo*\tsignal}

    % \addevent[v=\v]{0}{\calcdoppleronetwo*\tsignal}
    \addevent[v=\v]{0}{\calcdoppleronetwo*\calcdoppleronetwo*\tsignal}


    % Does not work yet for both moving. Fixed below
\end{tikzpicture}



\begin{tikzpicture}
    \spacetimediagram[lightcone=false]{5}
    
	% \draw[] (0,1) node[circle, fill, red] {};
	% \draw[vlorentz=0.5, ultra thick] (0,1) node[circle, fill, red] {};

    % \begin{scope}[vlorentz=0.2]
	%     \draw[vlorentz=0.5, ultra thick] (0,1) node[circle, fill, red] {};
    % \end{scope}

    % Works!!!

    

    % \def\v{0.2}
    % \def\v{-0.1}
    \def\v{0.0}
    \def\vtwo{0.5}
    \def\initsep{0}
    
    \def\tsignal{1}

    \FPeval{\vonetwo}{(\v+\vtwo)/(1+\v*\vtwo)}
    \FPeval{\vonetworel}{(\v-\vtwo)/(1-\v*\vtwo)}

    \begin{scope}[vlorentz=0.2]  % YESSS, WORKS

    \addworldline[v=\v]{0}{0}{0}{6}  % Observer
    \addworldline[v=\vonetwo]{0}{0}{0}{6}  % Observer

    
    \FPeval{\calcdopplerone}{((1+\v)/(1-\v))^.5}
    \FPeval{\calcdopplertwo}{((1+\vtwo)/(1-\vtwo))^.5}
    \FPeval{\calcdoppleronetwo}{\calcdopplerone*\calcdopplertwo}
    

    \addevent[v=\v]{0}{\tsignal}
    % \addevent[v=\vtwo]{0}{\calcdoppleronetwo*\tsignal}

    % \addevent[v=\vonetwo]{0}{\calcdoppleronetwo*\tsignal}  % Must account for time dilation here, right?
    \addevent[v=\vonetwo]{0}{\calcdoppleronetwo*\tsignal}

    % \addevent[v=\v]{0}{\calcdoppleronetwo*\tsignal}
    \addevent[v=\v]{0}{\calcdoppleronetwo*\calcdoppleronetwo*\tsignal}


    \addevent[v=\vonetwo, color=blue]{0}{\tsignal}
    \addevent[v=\v, color=blue]{0}{\calcdoppleronetwo*\tsignal}
    \addevent[v=\vonetwo, color=blue]{0}{\calcdoppleronetwo*\calcdoppleronetwo*\tsignal}
    \end{scope}
\end{tikzpicture}



% \makeatother
\makeatletter

\pgfkeys{
	/synchrobs/.is family, /synchrobs,
	default/.style = {vone=0.0, vtwo=0.5, tsignal=1, tmin=0.0, tmax=6.0, initsep=0.0},
	vone/.estore in = \synchrobs@vone,
	vtwo/.estore in = \synchrobs@vtwo,
	tsignal/.estore in = \synchrobs@tsignal,
	tmin/.estore in = \synchrobs@tmin,
	tmax/.estore in = \synchrobs@tmax,
	initsep/.estore in = \synchrobs@initsep,
}


\usetikzlibrary{decorations.markings,decorations.pathmorphing}

\tikzset{
    light/.style={
        lightyellow,
        line width=0.8,
        decorate,
        decoration={
            snake,
            amplitude=0.67,
            segment length=4.2,
            post length=4.2,
        }
    }
}

\newcommand{\EinsteinSynchronizedObservers}[1][]{%
	\pgfkeys{/synchrobs, default, #1}

    \def\vone{\synchrobs@vone}
    \def\vtwo{\synchrobs@vtwo}
    \def\tsignal{\synchrobs@tsignal}
    
    \FPeval{\vtotal}{(\vone+\vtwo)/(1+\vone*\vtwo)}
    \FPeval{\vrelative}{(\vtwo-\vone)/(1-\vone*\vtwo)}
    % \FPeval{\vrelative}{(\vone-\vtwo)/(1-\vone*\vtwo)}
    % \FPeval{\vrelative}{(\vtwo-\vone)/(1+\vone*\vtwo)}
    % \FPeval{\vrelative}{(\vtwo-\vone)/(1+\vone*\vtwo)}
    
    \FPeval{\dopplerone}{((1+\vone)/(1-\vone))^.5}
    \FPeval{\dopplertwo}{((1+\vtwo)/(1-\vtwo))^.5}
    % \FPeval{\doppleronetwo}{\dopplertwo}
    \FPeval{\doppleronetwo}{\dopplerone*\dopplertwo}
    
    % \FPeval{\dopplertwo}{((1+\vrelative)/(1-\vrelative))^.5}

    \FPeval{\doppleronetwo}{\dopplertwo/\dopplerone}
    % \FPeval{\doppleronetwo}{\dopplerone/\doppletwo}


    \begin{scope}[vlorentz=\vone]
        \coordinate (A) at (0, \tsignal);
        \coordinate (E) at (0, \doppleronetwo*\tsignal);
        \coordinate (B) at (0, \doppleronetwo*\doppleronetwo*\tsignal);
    \end{scope}

    % \begin{scope}[vlorentz=\vtotal]
    \begin{scope}[vlorentz=\vtwo]
        \coordinate (C) at (\initsep, \tsignal);
        \coordinate (F) at (\initsep, \doppleronetwo*\tsignal);
        \coordinate (D) at (\initsep, \doppleronetwo*\doppleronetwo*\tsignal);
    \end{scope}


    % \begin{scope}[vlorentz=\vone]
    %     \coordinate (A) at (0, \tsignal);
    %     \coordinate (E) at (0, \doppleronetwo*\tsignal);
    %     \coordinate (B) at (0, \doppleronetwo*\doppleronetwo*\tsignal);
        
    %     \begin{scope}[vlorentz=\vtwo]
    %         \coordinate (C) at (0, \tsignal);
    %         \coordinate (F) at (0, \doppleronetwo*\tsignal);
    %         \coordinate (D) at (0, \doppleronetwo*\doppleronetwo*\tsignal);
    %     \end{scope}
    % \end{scope}


    % -- Light from 1 to 2
    % \begin{scope}[vlorentz=\vtotal]
    %     \draw[lightyellow] (0, \tsignal) -- (0, \doppleronetwo*\tsignal);
    %     \draw[lightyellow] (0, \doppleronetwo*\tsignal) -- (0, \doppleronetwo*\doppleronetwo*\tsignal);
    % \end{scope}
    
    % \draw[light] (A) -- (F) -- (B);  % Light from 1 to 2 -> not so nice with decorate
    \draw[light] (A) -- (F);
    \draw[light] (F) -- (B);
    % \draw[light] (C) -- (E) -- (D);  % Light from 2 to 1
    \draw[light] (C) -- (E);
    \draw[light] (E) -- (D);

    
    % -- Observer worldlines
	\draw[thick, ->, vlorentz=\vone] (0, \synchrobs@tmin) -- (0, \synchrobs@tmax);
	\draw[thick, ->, vlorentz=\vtwo] (\synchrobs@initsep, \synchrobs@tmin) -- (\synchrobs@initsep, \synchrobs@tmax);

    % TODO: clip here?

    % -- Draw point halfway between emission and reception
	\draw[vlorentz=\vone] (0, {(1+\doppleronetwo*\doppleronetwo)/2*\tsignal}) node[circle, fill, red] {};
	\draw[vlorentz=\vtwo] (0, {(1+\doppleronetwo*\doppleronetwo)/2*\tsignal}) node[circle, fill, red] {};
}

% \makeatletter
\makeatother


\begin{tikzpicture}
    % \spacetimediagram[]{6}

    % \EinsteinSynchronizedObservers[vone=0.0,tsignal=1,tmax=4.2]

    % \EinsteinSynchronizedObservers[vone=0.2,vtwo=0.7,tsignal=1,tmax=4.2]

    \EinsteinSynchronizedObservers[vone=-0.1,vtwo=0.5,tsignal=1,tmax=4.2]
\end{tikzpicture}



\end{document}