%\documentclass[ART_main.tex]{subfiles}
\documentclass[ngerman, DIV=11, BCOR=0mm, paper=a4, fontsize=11pt, parskip=half, twoside=false, titlepage=true]{scrreprt}
%\graphicspath{ {Bilder/} {../Bilder/} }


\usepackage[singlespacing]{setspace}
\usepackage{lastpage}
\usepackage[automark, headsepline]{scrlayer-scrpage}
\clearscrheadings
\setlength{\headheight}{\baselineskip}
%\automark[part]{section}
\automark[chapter]{chapter}
\automark*[chapter]{section} %mithilfe des * wird nur ergänzt; bei vorhandener section soll also das in der Kopfzeile stehen
\automark*[chapter]{subsection}
\ihead[]{\headmark}
%\ohead[]{Seite~\thepage}
\cfoot{\hypersetup{linkcolor=black}Seite~\thepage~von~\pageref{LastPage}}

\usepackage[utf8]{inputenc}
\usepackage[ngerman, english]{babel}
\usepackage[expansion=true, protrusion=true]{microtype}
\usepackage{amsmath}
\usepackage{amsfonts}
\usepackage{amsthm}
\usepackage{amssymb}
\usepackage{mathtools}
\usepackage{mathdots}
\usepackage{aligned-overset} % otherwise, overset/underset shift alignment
\usepackage{upgreek}
\usepackage[free-standing-units]{siunitx}
\usepackage{esvect}
\usepackage{graphicx}
\usepackage{epstopdf}
\usepackage[hypcap]{caption}
\usepackage{booktabs}
\usepackage{flafter}
\usepackage[section]{placeins}
\usepackage{pdfpages}
\usepackage{textcomp}
\usepackage{subfig}
\usepackage[italicdiff]{physics}
\usepackage{xparse}
\usepackage{wrapfig}
\usepackage{color}
\usepackage{multirow}
\usepackage{dsfont}
\numberwithin{equation}{chapter}%{section}
\numberwithin{figure}{chapter}%{section}
\numberwithin{table}{chapter}%{section}
\usepackage{empheq}
\usepackage{tikz-cd}%für Kommutationsdiagramme
\usepackage{tikz}
\usepackage{pgfplots}
\usepackage{mdframed}
\usepackage{floatpag} % to have clear pages where figures are
%\usepackage{sidecap} % for caption on side -> not needed in the end
\usepackage{subfiles} % To put chapters into main file

\usepackage{hyperref}
\hypersetup{colorlinks=true, breaklinks=true, citecolor=linkblue, linkcolor=linkblue, menucolor=linkblue, urlcolor=linkblue} %sonst z.B. orange bei linkcolor

\usepackage{imakeidx}%für Erstellen des Index
\usepackage{xifthen}%damit bei \Def{} das Index-Arugment optional gemacht werden kann

\usepackage[printonlyused]{acronym}%withpage -> seems useless here

\usepackage{enumerate} % for custom enumerators

\usepackage{listings} % to input code

\usepackage{csquotes} % to change quotation marks all at once


%\usepackage{tgtermes} % nimmt sogar etwas weniger Platz ein als default font, aber wenn dann nur auf Text anwenden oder?
\usepackage{tgpagella} % traue mich noch nicht ^^ Bzw macht ganze Formatierung kaputt und so sehen Definitionen nicer aus
%\usepackage{euler}%sieht nichtmal soo gut aus und macht Fehler
%\usepackage{mathpazo}%macht iwie überall pagella an...
\usepackage{newtxmath}%etwas zu dick halt im Vergleich dann; wenn dann mit pagella oder überall Times gut

\setkomafont{chapter}{\fontfamily{qpl}\selectfont\Huge}%{\rmfamily\Huge\bfseries}
\setkomafont{chapterentry}{\fontfamily{qpl}\selectfont\large\bfseries}%{\rmfamily\large\bfseries}
\setkomafont{section}{\fontfamily{qpl}\selectfont\Large}%{\rmfamily\Large\bfseries}
%\setkomafont{sectionentry}{\rmfamily\large\bfseries} % man kann anscheinend nur das oberste Element aus toc setzen, hier also chapter
\setkomafont{subsection}{\fontfamily{qpl}\selectfont\large}%{\rmfamily\large}
\setkomafont{paragraph}{\rmfamily}%\bfseries\itshape}%\underline
\setkomafont{title}{\fontfamily{qpl}\selectfont\Huge\bfseries}%{\Huge\bfseries}
\setkomafont{subtitle}{\fontfamily{qpl}\selectfont\LARGE\scshape}%{\LARGE\scshape}
\setkomafont{author}{\Large\slshape}
\setkomafont{date}{\large\slshape}
\setkomafont{pagehead}{\scshape}%\slshape
\setkomafont{pagefoot}{\slshape}
\setkomafont{captionlabel}{\bfseries}



\definecolor{mygreen}{rgb}{0.8,1.00,0.8}
\definecolor{mycyan}{rgb}{0.76,1.00,1.00}
\definecolor{myyellow}{rgb}{1.00,1.00,0.76}
\definecolor{defcolor}{rgb}{0.10,0.00,0.60} %{1.00,0.49,0.00}%orange %{0.10,0.00,0.60}%aquamarin %{0.16,0.00,0.50}%persian indigo %{0.33,0.20,1.00}%midnight blue
\definecolor{linkblue}{rgb}{0.00,0.00,1.00}%{0.10,0.00,0.60}


% auch gut: green!42, cyan!42, yellow!24


\setlength{\fboxrule}{0.76pt}
\setlength{\fboxsep}{1.76pt}

%Syntax Farbboxen: in normalem Text \colorbox{mygreen}{Text} oder bei Anmerkungen in Boxen \fcolorbox{black}{myyellow}{Rest der Box}, in Mathe-Umgebung für farbige Box \begin{empheq}[box = \colorbox{mycyan}]{align}\label{eq:} Formel \end{empheq} oder farbigen Rand \begin{empheq}[box = \fcolorbox{mycyan}{white}]{align}\label{eq:} Formel \end{empheq}

% Idea for simpler syntax: renew \boxed command from amsmath; seems to work like fbox, so maybe background color can be changed there

\usepackage[most]{tcolorbox}
%\colorlet{eqcolor}{}
\tcbset{on line, 
        boxsep=4pt, left=0pt,right=0pt,top=0pt,bottom=0pt,
        colframe=cyan,colback=cyan!42,
        highlight math style={enhanced}
        }

\newcommand{\eqbox}[1]{\tcbhighmath{#1}}


\newcommand{\manyqquad}{\qquad \qquad \qquad \qquad}  % Four seems to be sweet spot



\newcommand{\rem}[1]{\fcolorbox{yellow!24}{yellow!24}{\parbox[c]{0.985\textwidth}{\textbf{Remark}: #1}}}%vorher: black als erste Farbe, das macht Rahmen schwarz%vorher: black als erste Farbe, das macht Rahmen schwarz

%\newcommand{\anm}[1]{\footnote{#1}}

\newcommand{\anmind}[1]{\fcolorbox{yellow!24}{yellow!24}{\parbox[c]{0.92 \textwidth}{\textbf{Anmerkung}: #1}}}
% wegen Einrückung in itemize-Umgebungen o.Ä.

\newcommand{\eqboxold}[1]{\fcolorbox{white}{cyan!24}{#1}}

\newcommand{\textbox}[1]{\fcolorbox{white}{cyan!24}{#1}}


\newcommand{\Def}[2][]{\textcolor{defcolor}{\fontfamily{qpl}\selectfont \textit{#2}}\ifthenelse{\isempty{#1}}{\index{#2}}{\index{#1}}}%{\colorbox{green!0}{\textit{#1}}}
% zwischendurch Test mit \textbf{#1} noch (wurde aber viel größer)

% habe jetzt Schrift/ font pagella reingehauen (mit qpl), ist mega; wobei Times auch toll (ptm statt qpl)

% wenn Farbe doch doof, einfach beide auf white :D




\mdfdefinestyle{defistyle}{topline=false, rightline=false, linewidth=1pt, frametitlebackgroundcolor=gray!12}

\mdfdefinestyle{satzstyle}{topline=true, rightline=true, leftline=true, bottomline=true, linewidth=1pt}

\mdfdefinestyle{bspstyle}{%
rightline=false,leftline=false,topline=false,%bottomline=false,%
backgroundcolor=gray!8}


\mdtheorem[style=defistyle]{defi}{Definition}[chapter]%[section]
\mdtheorem[style=satzstyle]{thm}[defi]{Theorem}
\mdtheorem[style=satzstyle]{prop}[defi]{Property}
\mdtheorem[style=satzstyle]{post}[defi]{Postulate}
\mdtheorem[style=satzstyle]{lemma}[defi]{Lemma}
\mdtheorem[style=satzstyle]{cor}[defi]{Corollary}
\mdtheorem[style=bspstyle]{ex}[defi]{Example}




% if float is too long use \thisfloatpagestyle{onlyheader}
\newpairofpagestyles{onlyheader}{%
\setlength{\headheight}{\baselineskip}
\automark[section]{section}
%\automark*[section]{subsection}
\ihead[]{\headmark}
%
% only change to previous settings is here
\cfoot{}
}




% Spacetime diagrams
%\usepackage{tikz}
%\usetikzlibrary{arrows.meta}
% -> setting styles sufficient
%\tikzset{>={Latex[scale=1.2]}}
\tikzset{>={Stealth[inset=0,angle'=27]}}

%\usepackage{tsemlines}  % To draw Dragon stuff; Bard says this works with emline, not pstricks
%\def\emline#1#2#3#4#5#6{%
%       \put(#1,#2){\special{em:moveto}}%
%       \put(#4,#5){\special{em:lineto}}}


% Inspiration: https://de.overleaf.com/latex/templates/minkowski-spacetime-diagram-generator/kqskfzgkjrvq, https://www.overleaf.com/latex/examples/spacetime-diagrams-for-uniformly-accelerating-observers/kmdvfrhhntzw

\usepackage{fp}
\usepackage{pgfkeys}


\pgfkeys{
	/spacetimediagram/.is family, /spacetimediagram,
	default/.style = {stepsize = 1, xlabel = $x$, ylabel = $c t$},
	stepsize/.estore in = \diagramStepsize,
	xlabel/.estore in = \diagramxlabel,
	ylabel/.estore in = \diagramylabel
}
	%lightcone/.estore in = \diagramlightcone  % Maybe also make optional?
	% Maybe add argument if grid is drawn or markers along axis? -> nope, they are really important

% Mandatory argument: grid size
% Optional arguments: stepsize (sets grid scale), xlabel, ylabel
\newcommand{\spacetimediagram}[2][]{%
	\pgfkeys{/spacetimediagram, default, #1}

    % Draw the x ct grid
    \draw[step=\diagramStepsize, gray!30, very thin] (-#2 * \diagramStepsize, -#2 * \diagramStepsize) grid (#2 * \diagramStepsize, #2 * \diagramStepsize);

    % Draw the x and ct axes
    \draw[->, thick] (-#2 * \diagramStepsize - \diagramStepsize, 0) -- (#2 * \diagramStepsize + \diagramStepsize, 0);
    \draw[->, thick] (0, -#2 * \diagramStepsize - \diagramStepsize) -- (0, #2 * \diagramStepsize + \diagramStepsize);

	% Draw the x and ct axes labels
    \draw (#2 * \diagramStepsize + \diagramStepsize + 0.2, 0) node {\diagramxlabel};
    \draw (0, #2 * \diagramStepsize + \diagramStepsize + 0.2) node {\diagramylabel};

	% Draw light cone
	\draw[black!10!yellow, thick] (-#2 * \diagramStepsize, -#2 * \diagramStepsize) -- (#2 * \diagramStepsize, #2 * \diagramStepsize);
	\draw[black!10!yellow, thick] (-#2 * \diagramStepsize, #2 * \diagramStepsize) -- (#2 * \diagramStepsize, -#2 * \diagramStepsize);
}



\pgfkeys{
	/addobserver/.is family, /addobserver,
	default/.style = {grid = true, stepsize = 1, xpos = 0, ypos = 0, xlabel = $x'$, ylabel = $c t'$},
	grid/.estore in = \observerGrid,
	stepsize/.estore in = \observerStepsize,
	xpos/.estore in = \observerxpos,
	ypos/.estore in = \observerypos,
	xlabel/.estore in = \observerxlabel,
	ylabel/.estore in = \observerylabel
}

% Mandatory argument: grid size, relative velocity (important: if negative, must be given as (-1) * v where v is the absolute value, otherwise error)
% Optional arguments: stepsize (sets grid scale), xlabel, ylabel
\newcommand{\addobserver}[3][]{%
	\pgfkeys{/addobserver, default, #1}

    % Evaluate the Lorentz transformation
    %\FPeval{\calcgamma}{1/((1-(#3)^2)^.5)}
    \FPeval{\calcgamma}{1/((1-((#3)*(#3)))^.5)} % More robust, allows negative v
    \FPeval{\calcbetagamma}{\calcgamma*#3}

	% Draw the x' and ct' axes
	\draw[->, thick, cm={\calcgamma,\calcbetagamma,\calcbetagamma,\calcgamma,(\observerxpos,\observerypos)}, blue] (-#2 * \observerStepsize - \observerStepsize, 0) -- (#2 * \observerStepsize + \observerStepsize, 0);
    \draw[->, thick, cm={\calcgamma,\calcbetagamma,\calcbetagamma,\calcgamma,(\observerxpos,\observerypos)}, blue] (0, -#2 * \observerStepsize - \observerStepsize) -- (0, #2 * \observerStepsize + \observerStepsize);

	% Check if grid shall be drawn
	\ifthenelse{\equal{\observerGrid}{true}}{%#
		% Draw transformed grid
		\draw[step=\diagramStepsize, blue, very thin, cm={\calcgamma,\calcbetagamma,\calcbetagamma,\calcgamma,(\observerxpos,\observerypos)}] (-#2 * \diagramStepsize, -#2 * \diagramStepsize) grid (#2 * \diagramStepsize, #2 * \diagramStepsize);
	}{} % Do nothing in else case

	% Draw the x' and ct' axes labels
    \draw[cm={\calcgamma,\calcbetagamma,\calcbetagamma,\calcgamma,(\observerxpos,\observerypos)}, blue] (#2 * \observerStepsize + \observerStepsize + 0.2, 0) node {\observerxlabel};
    \draw[cm={\calcgamma,\calcbetagamma,\calcbetagamma,\calcgamma,(\observerxpos,\observerypos)}, blue] (0, #2 * \observerStepsize + \observerStepsize + 0.2) node {\observerylabel};
}



\pgfkeys{
	/addevent/.is family, /addevent,
	default/.style = {v = 0, label =, color = red, label placement = below, radius = 5pt},
	v/.estore in = \eventVelocity,
	label/.estore in = \eventLabel,
	color/.estore in = \eventColor,
	label placement/.estore in = \eventLabelPlacement,
	radius/.estore in = \circleEventRadius
}

% Mandatory argument: x position, y position
% Optional arguments: relative velocity (important: if negative, must be given as (-1) * v where v is the absolute value, otherwise error), label, color, label placement
\newcommand{\addevent}[3][]{%
	\pgfkeys{/addevent, default, #1}

    % Evaluate the Lorentz transformation
    %\FPeval{\calcgamma}{1/((1-(#3)^2)^.5)}
    \FPeval{\calcgamma}{1/((1-((\eventVelocity)*(\eventVelocity)))^.5)} % More robust, allows negative v
    \FPeval{\calcbetagamma}{\calcgamma*\eventVelocity}

	% Draw event
	\draw[cm={\calcgamma,\calcbetagamma,\calcbetagamma,\calcgamma,(0,0)}, red] (#2,#3) node[circle, fill, \eventColor, minimum size=\circleEventRadius, label=\eventLabelPlacement:\eventLabel] {};
}



\pgfkeys{
	/lightcone/.is family, /lightcone,
	default/.style = {stepsize = 1, xpos = 0, ypos = 0, color = yellow, fill opacity = 0.42},
	stepsize/.estore in = \lightconeStepsize,
	xpos/.estore in = \lightconexpos,
	ypos/.estore in = \lightconeypos,
	color/.estore in = \lightconeColor,
	fill opacity/.estore in = \lightconeFillOpacity
}

% Mandatory arguments: cone size
% Optional arguments: stepsize (scale of grid), xpos, ypos, color, fill opacity
\newcommand{\lightcone}[2][]{
	\pgfkeys{/lightcone, default, #1}
	% Draw light cone -> idea: go from event location into the directions (1, 1), (-1, 1) for upper part of cone and then in directions (-1, -1), (1, -1) for lower part of cone
	\draw[\lightconeColor, fill, fill opacity=\lightconeFillOpacity] (\lightconexpos * \lightconeStepsize - #2 * \lightconeStepsize, \lightconeypos * \lightconeStepsize + #2 * \lightconeStepsize) -- (\lightconexpos, \lightconeypos) -- (\lightconexpos * \lightconeStepsize + #2 * \lightconeStepsize, \lightconeypos * \lightconeStepsize + #2 * \lightconeStepsize);
	\draw[\lightconeColor, fill, fill opacity=\lightconeFillOpacity] (\lightconexpos * \lightconeStepsize - #2 * \lightconeStepsize, \lightconeypos * \lightconeStepsize - #2 * \lightconeStepsize) -- (\lightconexpos, \lightconeypos) -- (\lightconexpos * \lightconeStepsize + #2 * \lightconeStepsize, \lightconeypos * \lightconeStepsize - #2 * \lightconeStepsize);
}





\begin{document}

\chapter{General Relativity}

SR dealt with uniformly moving frames, now we want to use the insights gained there to generalize discussions to accelerated frames -- this is what general relativity does (as it turns out, acceleration is very closely related to gravity, so GR is a theory of gravity as well)

-> wrong, SR can handle acceleration (contrary to popular belief I feel)! GR is really about incorporating gravity




	\section{Generalizing Relativity}
		\subsection{What is wrong with Newton (and SR)?}
gravitational redshift and instantaneous effect of gravity



		\subsection{Einstein Postulates}
do postulates by Einstein again as start, but now the ones for GR; weak equivalence principle + Einstein equivalence principle

-> what about Mach principle?


		\subsection{Notes}
Penrose has incredibly well written section 17.9 on intuition about metric and light cone structure in GR



\newpage



	\section{Giulini Lectures}
from 10 on (until 16) he deals with GWs, noice



\newpage



	\section{Gravitational Physics Summary -- Physics Part}

\rem{we use units of $c = G = 1$ (\Def{geometric units}).}

\rem{we adopt the Einstein summation convention where repeated combinations of upper and lower indices are summed over, that is we abbreviate $\sum_\mu x^\mu y_\mu = x^\mu y_\mu$.}

		\subsection{Newtonian Gravity}
Newtonian gravity can be captured by his famous formula
\begin{equation}
\eqbox{
F_g = - \frac{m_1 m_2}{r^2}
}
\end{equation}
which describes the gravitational force that an object with mass $m_1$ exerts onto another object with mass $m_2$. From Newton's second law, we know that the same force is exerted from the second object onto the first.


This force can also be brought into the form
\begin{equation}\label{eq:newton_potential}
F_g = m_2 \dv{r}\qty(\frac{m_1}{r}) = - m_2 \dv{\Phi_g}{r}
\end{equation}
which tells us that gravitation is a conservative force with potential 
\begin{equation}
\eqbox{
\Phi_g = - \frac{m_1}{r}
}
\end{equation}
produced by some object with mass $m_1$. We get the conservative property from equation \eqref{eq:newton_potential} alone because gravitational force only has a radial and no angular component (thinking in polar/spherical coordinates) such that any derivative with respect to angular coordinates %(like $\pdv{\Phi_g}{\theta}$)
 vanishes. Thus, \eqref{eq:newton_potential} is equivalent to the more general condition for conservative forces,
\begin{equation}\label{eq:cons_force}
\eqbox{
\vec{F} = - \vec{\nabla} \Phi
}
\manyqquad
F^k = - \delta^{k l} \pdv{\Phi}{x^l} \, .
\end{equation}
As a consequence, knowing the potential is sufficient to know how gravity acts. Thus, we are interested in how to determine $\Phi$ and this can be done using the Poisson equation. For a point particle, it takes the form
\begin{equation}
\eqbox{
\Delta \Phi = \nabla^2 \Phi = 0
}
\end{equation}
and for a continuous mass distribution $\rho\qty(\vec{x})$
\begin{equation}\label{eq:field_eq_newton}
\eqbox{
\Delta \Phi\qty(\vec{x}) = 4 \pi \rho\qty(\vec{x})
}
\manyqquad
\delta^{ij} \pdv[2]{\Phi(\vec{x})}{x^i}{x^j} = 4 \phi \rho\qty(\vec{x}) \, .
\end{equation}

Another perspective is not to look at forces $\vec{F}$, but at associated accelerations which comes from Newton's second law
\begin{equation}
\eqbox{
\vec{F} = m \vec{a} = m \dv[2]{\vec{r}}{t}
}
\manyqquad
F^k = m a^k = m \ddot{r}^k
\end{equation}
or at momenta $\vec{p}$ which are defined by
\begin{equation}
\eqbox{
\vec{F} = \dv{\vec{p}}{t} \quad \Leftrightarrow \quad \vec{p} = m \vec{v}
}
\manyqquad
p^k = m v^k \, .
\end{equation}



\begin{ex}[Gravity on Earth]
The gravity exerted by Earth on objects with mass $m$ (assuming they stand on Earth's surface for now) is
\begin{equation}
F_g = - m \frac{m_e}{r_e^2} = - m g
\end{equation}
Comparing that with Newton's second formula, $F = m a$, we see that such an object experiences an acceleration
\begin{equation}
a = - g = - 9.81 \frac{\metre}{\second^2} = - 1.1 \cdot 10^{-16} \frac{1}{\metre} \, .
\end{equation}
	\rem{note that we implicitly assume that gravitational mass and inertial mass are equal here. This is a non-trivial statement, which has been experimentally verified with high accuracy.}

To see how much potential energy is needed to lift objects of mass $m$ to a height $h \ll r_e$ above Earth's surface, we can do a Taylor expansion around $h = 0$:
\begin{align*}
\Phi_g = - \frac{m_e}{r_e + h} &\simeq - \eval{\frac{m_e}{r_e + h}}_{h = 0} + h \eval{\dv{h}\qty(- \frac{m_e}{r_e + h})}_{h = 0} + \mathcal{O}(h^2)
\\
&= - \frac{m_e}{r_e} + h \eval{\frac{m_e}{(r_e + h)^2}}_{h = 0} + \mathcal{O}(h^2)
\\
&= - \frac{m_e}{r_e} + h \frac{m_e}{r_e^2} + \mathcal{O}(h^2) = - \frac{m_e}{r_e} + h g + \mathcal{O}(h^2)
\end{align*}
%From that, we obtain the gravitational potential energy at $h$ to first order as the difference
However, the first contribution is nothing but the energy at Earth's surface. The potential energy that at $h$ and thus the energy which is needed to lift an object of mass $m$ to this height $h$ (which is what one is interested in most of the time) is given to first order by the difference
\begin{equation}
\Phi_g =  - \frac{m_e}{r_e} + g h - - \frac{m_e}{r_e} = g h \, .
\end{equation}
This corresponds to gauging our measurements such that Earth's surface is the value with zero potential energy.
\end{ex}


We see that gravity is related to a potential and thus to potential energy. Hence, we expect an objects energy to change if it moves in a gravitational field (in radial direction). This has interesting consequences, for example because light will also be affected by this.

\begin{ex}[Gravitational Redshift]
somehow we could build perpetual motion machine is the argument, don't get it

I rather think about it like this (should be equivalent): SR tells us that photons have a certain mass $m = \frac{E}{c^2}$; therefore, it is also affected by a gravitational potential and to move against gravity, some of its energy has to be converted; that corresponds to a change in frequency, (since $f = \frac{E}{h}$):
\begin{equation}
\frac{f_\text{top}}{f_\text{bottom}} = \frac{E_\text{top}}{E_\text{bottom}} = \frac{m - m g h}{m} = 1 - g h
\end{equation}
\rem{in script, this is only true to first order, so derivation might be wrong... Result there reads $\frac{1}{m + m g h}$. Ahhh, because there things are defined differently: photon starts from top, thus it has more energy at ground}
\end{ex}

A natural consequence because time is inversely proportional to frequency (time differences are) is that clocks tick faster at higher altitude, i.e.~for a stronger gravitational potential. It is also possible to derive it in reverse order, that is by showing that clocks tick slower in a stronger gravitational field. This causes a change in frequency and thus also a redshift.



%reason for puzzling and seemingly inconsistent results: a frame where gravity acts is \emph{not} inertial (because we have external force in gravity, right?)!



	\subsection{Special Relativity}
The theory of relativity is about how physical laws depend on the observer. We will begin with the theory of \Def{special relativity} (SR), which generalizes Newtonian dynamics.

\begin{defi}[Inertial Frame]
An \Def{intertial frame of reference} is a coordinate frame where $\vec{F} = m \vec{a}$ holds. In particular, that means objects move with constant speed when no force is acting on them.
\end{defi}
From the definition we can immediately see that there is no unique inertial frame because we can always get other inertial frames from existing ones by looking at frames which move with constant speed with respect to them.

We also see that all laws of physics hold equally in all inertial frames because
\begin{equation}
\vec{F} = m \vec{a} = m \dv{\vec{v}(t)}{t} = m \dv{\vec{v}'(t)}{t}
\end{equation}
as long as $\vec{v} - \vec{v}' = \text{const}$. That also means only laws and quantities which are invariant under transformations between inertial frames have physical meaning. In particular, that means coordinates of events have no physical meanings. As an alternative, we can look at distances between events which turn out to be invariant because they are related to a metric:
\begin{align*}
\qty(\Delta s)^2 &= - \qty(\Delta t)^2 + \qty(\Delta x)^2 + \qty(\Delta y)^2 + \qty(\Delta z)^2
\\
&= \underline{\Delta x} \cdot \eta \cdot \underline{\Delta x} = \mqty(\Delta t & \Delta x & \Delta y & \Delta z) \cdot \mqty(-1 & 0 & 0 & 0 \\ 0 & 1 & 0 & 0 \\ 0 & 0 & 1 & 0 \\ 0 & 0 & 0 & 1) \cdot \mqty(\Delta t \\ \Delta x \\ \Delta y \\ \Delta z)
\\
&= \eta_{\mu \nu} \qty(\Delta x)^\mu \qty(\Delta x)^\nu = \qty(\Delta x)^\mu \qty(\Delta x)_\mu \, .
\end{align*}
Making these differences $\Delta$ infinitesimally small gives the line element of the metric $\eta$
\begin{equation}
ds^2 := \qty(ds)^2 = - c^2 dt^2 + dx^2 + dy^2 + dz^2 \, .
\end{equation}\\

Let us now think about gravity in special relativity. In principle, the Newtonian description is kept, but some effects can be examined in different manner now, e.g.~due to the new notion/tool of different inertial frames. However, a frame where gravity acts is \emph{not} inertial (because we have external force in gravity, right?)! Thus, to do physics on Earth, we have to find a reference frame in which the effect of gravity is cancelled out. Obviously, earths surface is not sufficient and neither is a uniformly moving one. In free fall, however, we experience no gravity, that is a freely falling frame cancels out the effect of gravity. This can be stated more formally:
\begin{prop}[Weak Equivalence Principle]
The effects of a gravitational field are indistinguishable from an accelerated frame of reference.
\end{prop}
Basically, that means only a freely falling frame can serve as an inertial frame on Earth. That raises the question what happens to the laws of physics in such a freely falling frame.

\begin{prop}[Einstein Equivalence Principle]
The laws of physics in a freely falling frame are locally described by SR without gravity. For this reason, such a frame is also called \Def[local inertial frame]{local inertial frame (LIF)}.
\end{prop}
Speaking strictly mathematically, \enquote{locally} means in an infinitesimally small neighbourhood of points. The degree to which this can be extended in practice depends on the physical effects of interest.

Since gravity acts radially, its direction changes on different places around Earth. That implies there is no uniform direction of acceleration, so there can be no global freely falling frame/LIF. Other properties of gravity which are known from experience are the following:
\begin{itemize}
\item[(a)] All bodies which start with the same initial velocity move through a gravitational field along the same curve

\item[(b)] Bodies which move initially parallel to each other in a freely falling frame do not necessarily move parallel at all times if an external gravitational field is present (this effect is due to \Def{tidal forces} acting on them)
\end{itemize}

Property (b) can be further examined and quantified. To do that, we note that for a particle with world line $x^k(\tau)$ we have
\begin{equation*}
\dv[2]{x^k}{\tau} = - \delta^{k l} \pdv{\Phi}{x^l} = - \delta^{k l} \eval{\pdv{\Phi}{x^l}}_{\vec{x}}
\end{equation*}
because of \eqref{eq:cons_force} and Newton's second law $F^k = m \dv[2]{x^k}{\tau}$. Similarly, for another particle starting close to the first one (i.e.~with world line $x^k + \xi^k$, where $\abs{\xi^k \xi_k} \ll 1$)
\begin{align*}
\dv[2]{\qty(x^k + \xi^k)}{\tau} &= - \delta^{k l} \eval{\pdv{\Phi}{x^l}}_{\vec{x} + \vec{\xi}}
\\
&\simeq - \delta^{k l} \eval{\pdv{\Phi}{x^l}}_{\vec{x}} - \delta^{k l} \xi^m \pdv{x^m} \eval{\pdv{\Phi}{x^l}}_{\vec{x}}
\end{align*}
where we used a Taylor approximation to first order. The linearity of derivatives yields
\begin{equation}\label{eq:tidal_newton}
\eqbox{
\dv[2]{\xi}{\tau} = \dv[2]{\qty(x^k + \xi^k)}{\tau} - \dv[2]{x^k}{\tau} = - \delta^{k l} \pdv[2]{\Phi}{x^m}{x^l} \, \xi^m
}
\end{equation}
	\rem{note that the evaluation is still at the point $\vec{x}$, not at something related to $\xi^k$!}

This is the \Def{Newtonian deviation equation}. We see that tidal forces are governed by the tidal acceleration tensor $\pdv[2]{\Phi}{x^m}{x^l}$. Tidal forces are a way to detect gravity as opposed to constant acceleration (which would affect the world lines $x^k$ and $x^k + \xi^k$ equally)



	\subsection{Curved Spacetime}
One problem in SR is that the Newtonian description of gravity is still taken to be valid. That, however, is a problem because there are many inconsistencies between them, for example the instantaneous effect of gravity (gravitational redshift is also puzzling). However, we can come up with generalized description: for anybody familiar with differential/Riemannian geometry, the effects (a), (b) of gravity stated above sound very much like the ones associated with a curved space. This motivates the (mathematical) description of gravity as a geometrical effect in Minkowskian spacetime (which will become a curved space in this process). Many relations known from SR will remain, but with different quantities and most prominently, a different metric other than $\eta$. The basic goal of \Def{general relativity} (GR) will be to find ways to derive the metric which contains information about spacetime curvature and thus gravity.\\


The approach in this subsection will always be to look how generalizations can be made using the metric and other tools of geometry, while recovering SR in a LIF. Such a check, however, has not been made for the metric itself yet! Thus, we will now look at how the mathematical term \enquote{locally} is to be thought of. Taking an arbitrary metric with components $g_{\mu \nu}$ in some basis, we can always transform to other coordinates using the tensor transformation law. This

say something about how LIF can be characterized using metric; is important we always want to recover results in there, for example statements on timelike etc. (around 2.36)\\


The most basic thing we need to know is how test particles move in curved spaces. We will start with the case where no force is present, i.e.~free movement. The Newtonian theory/SR gives us $\ddot{\vec{a}} = 0$ and thus a movement on straight lines. This has to be reproduced locally (that is in a LIF), but originating from a more general concept. Finding this generalization is based on the observation that tangent vectors remain constant along straight lines. That leads to the \Def{geodesic equation}
\begin{equation}
\eqbox{
\nabla_{\underline{t}} t^\beta = t^\alpha \nabla_\alpha t^\beta = 0
}\, ,
\end{equation}
which just expresses that $t^\beta$ remains constant as long as we take the derivative along the curve that it is tangent to (which explains the $t^\alpha \nabla_\alpha$ part). Consequently, the world lines test particles are \Def{geodesics}.

One can obtain the same statement from a completely different approach: by demanding that world lines are the curves in Minkowski space which extremize the proper time/distance
\begin{equation}
\tau_{AB} = \int_A^B d\tau = \int_A^B \sqrt{- g_{\alpha \beta} dx^\alpha dx^\beta} = \int_0^1 \sqrt{- g_{\alpha \beta} \dv{x^\alpha}{\sigma} \dv{x^\beta}{\sigma}} \, d\sigma
\end{equation}
between two events $A, B$.

In both cases, we obtain the following coordinate version of the geodesic equation:
\begin{equation}
\eqbox{
\dv[2]{x^\beta}{\sigma} + \Gamma_{\alpha \delta}^\beta \dv{x^\alpha}{\sigma} \dv{x^\delta}{\sigma}
}
\end{equation}

\begin{proof}
In the first approach, we use
\begin{align*}
t^\alpha \nabla_\alpha t^\beta &= \dv{x^\alpha}{\sigma} \nabla_\alpha \dv{x^\beta}{\sigma}
\\
&= \dv{x^\alpha}{\sigma} \qty(\pdv{x^\alpha} \dv{x^\beta}{\sigma} + \Gamma_{\alpha \delta}^\beta \dv{x^\delta}{\sigma})
\\
&= \dv{x^\alpha}{\sigma} \pdv{x^\alpha} \dv{x^\beta}{\sigma} + \Gamma_{\alpha \delta}^\beta \dv{x^\alpha}{\sigma} \dv{x^\delta}{\sigma}
\\
\underset{\text{chain rule}}&{=} \dv{\sigma} \dv{x^\beta}{\sigma} + \Gamma_{\alpha \delta}^\beta \dv{x^\alpha}{\sigma} \dv{x^\delta}{\sigma}
\\
&= \dv[2]{x^\beta}{\sigma} + \Gamma_{\alpha \delta}^\beta \dv{x^\alpha}{\sigma} \dv{x^\delta}{\sigma} \, .
\end{align*}

In the second approach, we derive the Euler-Lagrange equations by varying the proper time integral:
\begin{equation}
\eqbox{
\pdv{L}{x^\alpha} = \dv{\sigma} \pdv{L}{dx^\alpha / d\sigma}
}
\end{equation}
where we introduced the Lagrangian
\begin{equation}
\eqbox{
L = L\qty(x^\alpha, \dv{x^\alpha}{\sigma}) = \sqrt{- g_{\alpha \beta} \dv{x^\alpha}{\sigma} \dv{x^\beta}{\sigma}}
} \, .
\end{equation}
Notable and useful properties in this context are that $L$ is constant along the geodesic because
\begin{align*}
\dv{\sigma} \qty(g_{\alpha \beta} \dv{x^\alpha}{\sigma} \dv{x^\beta}{\sigma}) &= t^\gamma \partial_\gamma \qty(g_{\alpha \beta} t^\alpha t^\beta) = t^\gamma \nabla_\gamma \qty(g_{\alpha \beta} t^\alpha t^\beta)
\\
&= t^\alpha t^\beta \nabla_\gamma g_{\alpha \beta} + g_{\alpha \beta} t^\beta t^\gamma \nabla_\gamma t^\alpha + g_{\alpha \beta} t^\alpha t^\gamma \nabla_\gamma t^\beta = 0 \, .
\end{align*}
Note that this does \emph{not} imply $\partial_\alpha L = 0$ (so the Euler-Lagrange equations still make sense). It does, however, mean that we are free to change the parametrization from $\sigma$ to any \Def{affine parameter} $\sigma' = a \sigma + b, \; a, b \in \mathbb{R}$ while only picking up a factor $\frac{1}{a}$. The proper time $\tau$ is defined as the parameter $\sigma'$ with
\begin{equation}
L = 1 \qquad \Leftrightarrow \qquad g_{\alpha \beta} \dv{x^\alpha}{\tau} \dv{x^\beta}{\tau} = -1
\end{equation}
which comes from the known normalization of the four-velocity $U^\alpha = \dv{x^\alpha}{\tau}$. Therefore, we can always replace $d\sigma \rightarrow d\tau = L d\sigma$ or vice versa.
\end{proof}
The advantage of having two approaches is that, in some cases, calculating the Christoffel symbols might be easier or might have already been done, while in others the Euler-Lagrange equations reveal very useful properties of the problem/system (for example in case one coordinate does not appear explicitly in $L$; then, we have immediately found a quantity which is conserved by the system, i.e.~it does not change as time evolves/as we vary $\sigma$ when going along the geodesic, this quantity being $\pdv{L}{dx^\alpha / d\sigma}$).

A physical consequence from $L = \text{const}$ is that geodesics/world lines which are timelike/null/spacelike somewhere have this property everywhere! Test particles (with mass $m > 0$) always move along timelike geodesics while massless particles like photons move along null geodesics (spacelike ones violate causality).



		\subsection{Weakly Curved Spacetimes}
basic idea here: re-derive first subsection from GR (confirm that it reproduces Newtonian results), that is using the spacetime metric
\begin{equation}
ds^2 = - \qty(1 + 2 \Phi) dt^2 + \qty(1 - 2\Phi) \qty(dx^2 + dy^2 + dz^2)
\end{equation}
which is the first order approximation in case the Newtonian potential $\Phi = - \frac{m}{r}$ fulfils $\abs{\Phi} \ll 1$; we will see where it comes from later on, will be used as motivation for now\\


we can formulate the geodesic equation for test particles in many ways, for example
\begin{equation}
U^\alpha \nabla_\alpha U^\beta = 0 \manyqquad p^\alpha \nabla_\alpha p^\beta
\end{equation}
where $U^\alpha = \dv{x^\alpha}{\tau}$ is the four-velocity of a particle with world line $x^\alpha$ and $p^\alpha = m U^\alpha$ is the four-momentum (sometimes preferred quantity because it is well-defined also for photons)

for the first/time component $p^0 = E \approx m$ ($c^2 = 1$, this is an approximation to lowest order) we obtain
\begin{equation}
0 = m \dv{p^0}{\tau} + \Gamma^0_{\alpha \beta} p^\alpha p^\beta \approx m \dv{p^0}{\tau} + \Gamma^0_{0 0} \qty(p^0)^2 \qquad \Leftrightarrow \qquad \dv{p^0}{\tau} \approx - m \pdv{\Phi}{t}
\end{equation}
which matches the Newtonian result that if the gravitational field does not change over time, then the energy $p^0$ will be conserved over time

similarly, the equations for the spatial components
\begin{equation}
0 = m \dv{p^k}{\tau} + \Gamma^k_{\alpha \beta} p^\alpha p^\beta \approx m \dv{p^k}{\tau} + \Gamma^k_{0 0} m^2 \qquad \Leftrightarrow \qquad \dv{p^k}{\tau} \approx - m \delta^{k l} \pdv{\Phi}{x^l}
\end{equation}
match the Newtonian result \eqref{eq:cons_force} that gravity acts as a conservative force\\


In the Newtonian case, tidal forces \eqref{eq:tidal_newton} could be used to detect gravity as opposed to constant acceleration; something similar should be possible for curvature (which is how we described gravity now) and indeed, we obtain the \Def{geodesic deviation equation}
\begin{equation}\label{eq:geod_dev_eq}
\eqbox{
\nabla_{\underline{U}} \nabla_{\underline{U}} \xi^\beta = U^\sigma \nabla_\sigma U^\alpha \nabla_\alpha \xi^\beta = - R^\beta{}_{\gamma \delta \epsilon} U^\gamma \xi^\delta U^\epsilon
} \, .
\end{equation}
Therefore, gravity/curvature is determined and measured by the \Def{Riemann curvature tensor}
\begin{equation}
\eqbox{
R^\beta{}_{\gamma \delta \epsilon} = \pdv{\Gamma^\beta{}_{\gamma \epsilon}}{x^\delta} - \pdv{\Gamma^\beta{}_{\gamma \delta}}{x^\epsilon} + \Gamma^\beta{}_{\delta \mu} \Gamma^\mu{}_{\gamma \epsilon} - \Gamma^\beta{}_{\epsilon \mu} \Gamma^\mu{}_{\gamma \delta}
} \, .
\end{equation}
A helpful way to think about the Riemann tensor is as the \enquote{difference}/commutator of subsequent derivatives, which is motivated by the following equation:
\begin{equation}
\eqbox{
\qty[\nabla_\alpha, \nabla_\beta] V^\mu = R^\mu{}_{\gamma \alpha \beta} V^\gamma
} \, .
\end{equation}


We can simplify this equation by choosing a LIF with $\underline{e'_0} = \underline{e_\tau}$ (one basis vector along proper time). Since this coordinate system is parallel transported along the geodesic,
\begin{equation*}
U^\gamma \nabla_\gamma \dv{x'^\beta}{x^\alpha} = U^\gamma \nabla_\gamma \qty(\underline{e'^\beta})_\alpha = 0
\end{equation*}
and we obtain
\begin{equation}
\dv[2]{\xi^\beta}{\tau} = \pdv{x'^\beta}{x^\alpha} U^\sigma \nabla_\sigma U^\delta \nabla_\delta \xi^\alpha = - \pdv{x'^\beta}{x^\alpha} R^\alpha{}_{\gamma \delta \epsilon} \pdv{x^\gamma}{\tau} \xi^\delta \pdv{x^\epsilon}{\tau} = - R'^\beta{}_{\tau \delta \tau} \xi'^\delta
\end{equation}
	\rem{note that the evaluation is still at the point $\vec{x}$, not at something related to $\xi^k$!}
due to the transformation law for vectors and tensors (?). This equation looks very much like the Newtonian expression \eqref{eq:tidal_newton}. Moreover, we see that the Riemann tensor might play a role which is similar to the tidal acceleration tensor $\pdv[2]{\Phi}{x^m}{x^l}$. Hence, it might also play a similar role in causing gravity...



	\subsection{Einstein Equation}
We have seen how Newtonian theory can be generalized, for example by going from a potential $\Phi$ for the gravity field to the metric $g_{\mu \nu}$. Many effects could be derived from that, e.g.~equations of motion and the effect of gravity in tidal forces -- but we have no way of determining the metric yet! Thus, we will now look for a field equation like \eqref{eq:field_eq_newton}. As it turns out, there is no way of truly deriving the result. Instead, we can only motivate it sufficiently.\\


First of all, we need an analogue to the mass distribution $\rho\qty(\vec{x})$ to describe the effect of matter. This should be a tensorial quantity in accordance with previous generalizations.


This definition also allows to summarize several conservation formulas into one, coordinate-independent equation:
\begin{equation}
\nabla_\alpha T^{\alpha \beta} = 0 \, .
\end{equation}
This gives us
\begin{equation}
\partial_t \rho + \partial_i \pi^i = 0 \manyqquad \partial_t \pi^i + \partial_j T^{i j} = 0
\end{equation}
which corresponds to energy and momentum conservation (which is the continuum version of $F^k = m a^k$ since spatial components of the stress-energy tensor describe forces).\\


Now that we have something which causes gravity, we need to equate it with something that describes gravity -- a natural quantity would be the Riemann tensor $R^\alpha{}_{\beta \gamma \delta}$. However, the stress-energy tensor only has two indices, while the Riemann tensor has four. Thus, we have to get rid of two indices, which can be done by contraction. Since $T = T^\alpha{}_\alpha \neq 0$, we are only interested a $2$-tensor trace component, which turns out to be the \Def{Ricci tensor} 
\begin{equation}
\eqbox{
R_{\alpha \beta} = R^\mu{}_{\alpha \mu \beta}
} = R_{\beta \alpha} \, .
\end{equation}

However, the Ricci tensor is not divergent free, so a field equation of the kind $R^{\alpha \beta} = b T^{\alpha \beta}$ cannot hold. Luckily, we can construct divergent-free tensor from it by rearranging
\begin{equation*}
\nabla_\alpha R^{\alpha \beta} = \frac{1}{2} g^{\alpha \beta} \nabla_\alpha R
\end{equation*}
which can be obtained using the Bianchi identity. Since $\nabla_\alpha g^{\alpha \beta} = 0$,
\begin{equation}
\nabla_\alpha \qty(R^{\alpha \beta} - \frac{1}{2} g^{\alpha \beta} R) = 0
\end{equation}
and we see that the \Def{Einstein tensor}
\begin{equation}
\eqbox{
G^{\alpha \beta} = R^{\alpha \beta} - \frac{1}{2} g^{\alpha \beta} R = G^{\beta \alpha}
}
\end{equation}
fulfils the required conservation law. Here, $R$ is the \Def{Ricci scalar} (or \Def{scalar curvature})
\begin{equation}
\eqbox{
R = R^\alpha_\alpha
} \, ,
\end{equation}
part of another trace component of the Riemann tensor. Thus, the field equation of interest (also called \Def{Einstein equation}) is
\begin{equation}\label{eq:einstein1}
\eqbox{
G^{\alpha \beta} = 8 \pi T^{\alpha \beta}
}
\end{equation}
where the constant of proportionality $8 \pi$ is determined by requiring that they reduce to the Newtonian weak-field limit. Now, remembering that $\nabla_\gamma g^{\alpha \beta} = 0$ tells us we are free to add a term of the kind $a g^{\alpha \beta}$ to the field equation. Therefore,
\begin{equation}\label{eq:einstein2}
\eqbox{
G^{\alpha \beta} + \Lambda g^{\alpha \beta} = 8 \pi T^{\alpha \beta}
}
\end{equation}
is an equally valid version of the Einstein equation. $\Lambda$ is the (in)famous \Def{cosmological constant} and its contribution can be thought of as \enquote{stress-energy of empty space}. However, this can be absorbed in $T^{\alpha \beta}$, so we will mostly stick to using \eqref{eq:einstein1}. The Einstein equation can be interpreted following this famous quote by John Archibald Wheeler:
\begin{center}
\enquote{\textit{Spacetime tells matter how to move; matter tells spacetime how to curve.}}
\end{center}

%-> we need something with two indices now... To see how that might go, we can decompose Riemann tensor into part with non-vanishing and vanishing trace; since $T = T^\alpha{}_\alpha \neq 0$, we are only interested a $2$-tensor trace component, which turns out to be the \Def{Ricci tensor} $R_{\alpha \beta} = R^\mu{}_{\alpha \mu \beta}$ (is symmetric); the other trace component is (proportional to) the \Def{Ricci scalar} $R = R^\alpha{}_\alpha$, but this is not a tensor, so we will omit it for now

%-> problem: Ricci tensor is not divergent free, but has to be fulfilled if equal to $T^{\alpha \beta}$! However, we can construct divergent-free tensor from it by rearranging the result $\nabla_\alpha R^{\alpha \beta} = \frac{1}{2} g^{\alpha \beta} \nabla_\alpha R$ which can be obtained using the Bianchi identity; since $\nabla_\alpha g^{\alpha \beta} = 0$, $\nabla_\alpha \qty(R^{\alpha \beta} - \frac{1}{2} g^{\alpha \beta} R) =: \nabla_\alpha G^{\alpha \beta} = 0$ and we see that the \Def{Einstein tensor} $G^{\alpha \beta}$ fulfils the required conservation law.


To simplify things one can look empty space first, where $T^{\alpha \beta} = T = 0$. For the Einstein equation, that means
\begin{equation*}
G^{\alpha \beta} = 0 \, .
\end{equation*}
But since trace of the Einstein equation yields $R = - 8 \pi T$ ($= 0$ in vacuum), we obtain the following \Def{vacuum Einstein equation}:
\begin{equation}
\eqbox{
R^{\alpha \beta} = 0
} \, .
\end{equation}


Now we will come to few examples of solutions of the Einstein equation.% (spoiler: there are not too many).
\begin{ex}[Solutions of the Einstein equation]
\begin{enumerate}[(a)]
\item Minkowski space with $g_{\alpha \beta} = \eta_{\alpha \beta}$ and
\begin{equation}
ds^2 = - dt^2 + dx^2 + dy^2 + dz^2
\end{equation}


\item Schwarzschild black hole: unique (non-trivial) solution for a static, spherically symmetric spacetime containing a mass $M$ (a \Def{black hole})
\begin{equation}
ds^2 = - \qty(1 - \frac{2 M}{r}) dt^2 + \frac{1}{1 - \frac{2 M}{r}} dr^2 + r^2 \qty(d\theta^2 + \sin(\theta)^2 d\phi^2)
\end{equation}


\item weak (Newtonian) gravitational potential: Schwarzschild solution far away from the source (such that it can be approximated as point mass)
\begin{align}
ds^2 &= - \qty(1 + 2 \Phi) dt^2 + \qty(1 - 2 \Phi) \qty(dR^2 + R^2 d\theta^2 + R^2 \sin(\theta)^2 d\phi^2)
\\
&= - \qty(1 + 2 \Phi) dt^2 + \qty(1 - 2 \Phi) \qty(dx^2 + dy^2 + dz^2)
\end{align}

Is obtained from defining the potential $\Phi = - \frac{r}{r - M}$, Taylor-expanding to first order in $\Phi$ and then making the gauge transformation $r \rightarrow R = r - M$.


\item Kerr solution of a rotating black hole
\begin{equation}
\begin{split}
ds^2 &= - \frac{\Delta - a^2 \sin(\theta)^2}{\rho^2} dt^2 - 2 a \frac{2 M r \sin(\theta)^2}{\rho^2} dt d\phi + \frac{\rho^2}{\Delta} dr^2
\\
&\quad + \frac{\qty(r^2 + a^2)^2 - a^2 \Delta \sin(\theta)^2}{\rho^2} \sin(\theta)^2 d\phi^2 + \rho^2 d\theta^2
\end{split}
\end{equation}
where $\Delta = r^2 - 2 M r + a^2, \rho = r^2 + a^2 \cos(\theta)^2$ and $a$ parametrizes the rotation speed


\item Friedmann-Lemaitre-Robertson-Walker (FLWR) metric of an isotropic, homogenous, expanding universe
\begin{equation}
ds^2 = - dt^2 - a(t)^2 \qty[\frac{1}{1 - k r^2} dr^2 + r^2 \qty(d\theta^2 + \sin(\theta)^2 d\phi^2)]
\end{equation}
\end{enumerate}
\end{ex}
As we can see, there are not too many analytical solutions. This is because the Einstein equation forms a coupled system of $10$ non-linear, second-order partial differential equations, which makes them very hard to solve in general (even numerically).



		\subsection{Existence of Gravitational Waves}

gravitational wave = GW



		\subsection{Effect of Gravitational Waves}
Gravitational waves are small perturbations of spacetime, so they should have a measurable effect. To find this effect, we will now look at a particles in the TT gauge. We assume it to be at some position $x^\alpha$ and at rest at $t = 0$, that is
\begin{equation*}
\eval{U^\alpha}_{t = 0} = \eval{\dv{x^\alpha}{\tau}}_{t = 0} = \qty(1, 0, 0, 0) \, .
\end{equation*}
Evaluating the geodesic equation at $t = 0$ yields
\begin{equation}
\begin{split}
\dv{U^\alpha}{\tau} &= \dv[2]{x^\alpha}{\tau} = - \Gamma^\alpha{}_{\mu \nu} \dv{x^\mu}{\tau} \dv{x^\nu}{\tau} = - \Gamma^\alpha{}_{tt}
\\
&= - \frac{1}{2} g^{\alpha \mu} \qty(2 \partial_t g_{t \mu} - \partial_\mu g_{tt}) = 0 \, .
\end{split}
\end{equation}
The four-velocity remains constant, so the coordinates of the particle do not change! That, however, does not mean GWs have no effect -- after all, coordinates have no invariant meaning anyway. Instead, we have to look at physically meaningful quantities like the proper distance between particles which is related to the metric and thus may be affected by a GW.

To simplify calculations, we will now assume a GW propagating in $z$-direction, with linear polarization and $h_+ = h_+(t - z), h_{\cross} = 0$. The proper distance between two particles which are initially separated by $L$ in $x$-direction then becomes
\begin{equation}
L_x(t) = \int_0^L \sqrt{g_{xx} dx dx} = \int_0^L \sqrt{1 + h_+} \, dx \approx \int_0^L \qty(1 + \frac{h_+}{2}) \, dx = L \qty(1 + \frac{h_+(t)}{2})
\end{equation}
assuming that the wavelength is much longer than $L$ (whence the amplitude $h_+$ does not change much during the propagation from $x = 0$ to $x = L$). Therefore, a GW causes a (time-dependent) relative length change
\begin{equation}
\frac{\Delta L_x}{L} = \frac{L_x - L}{L} \approx \frac{h_+(t)}{2}
\end{equation}
which is often called \Def{strain}.


Another way to quantify the effect of a GW is to look at geodesic deviation. In a LIF, we recall that the corresponding equation \eqref{eq:geod_dev_eq} reads
\begin{equation*}
\dv[2]{\xi^\alpha}{\tau} = - R^\alpha{}_{\tau \mu \tau} \xi^\mu \, .
\end{equation*}
It is possible to calculate the relevant components of the Riemann tensor in TT gauge:
\begin{equation}
R^x{}_{\tau x \tau} = - \frac{1}{2} \pdv[2]{h_+}{\tau} \qquad \qquad R^y{}_{\tau x \tau} = - \frac{1}{2} \pdv[2]{h_{\cross}}{\tau} \qquad \qquad R^y{}_{\tau y \tau} = \frac{1}{2} \pdv[2]{h_+}{\tau} \, .
\end{equation}
Assuming an initial separation in $x$-direction again, i.e.~$\underline{\xi} = \qty(0, \xi, 0, 0)$, we obtain
\begin{equation}
\dv[2]{\xi^x}{\tau} = \frac{\xi}{2} \pdv[2]{h_+}{\tau} \manyqquad \dv[2]{\xi^y}{\tau} = \frac{\xi}{2} \pdv[2]{h_{\cross}}{\tau} \, .
\end{equation}
For an initial separation in $y$-direction, $\underline{\xi} = \qty(0, 0, \xi, 0)$, the roles are reversed:
\begin{equation}
\dv[2]{\xi^x}{\tau} = \frac{\xi}{2} \pdv[2]{h_{\cross}}{\tau} \manyqquad \dv[2]{\xi^y}{\tau} = - \frac{\xi}{2} \pdv[2]{h_+}{\tau} \, .
\end{equation}
Consequently, GWs exert a force which changes the proper distance between particles, i.e.~they stretch and squeeze spacetime between particles (not only in a LIF). Moreover, we see that the two polarizations have an equivalent effect, they are just rotated against each other.







\end{document}