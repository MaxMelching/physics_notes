\documentclass[relativity_main.tex]{subfiles}
%\documentclass[ngerman, DIV=11, BCOR=0mm, paper=a4, fontsize=11pt, parskip=half, twoside=false, titlepage=true]{scrreprt}
%\graphicspath{ {Bilder/} {../Bilder/} }


\usepackage[singlespacing]{setspace}
\usepackage{lastpage}
\usepackage[automark, headsepline]{scrlayer-scrpage}
\clearscrheadings
\setlength{\headheight}{\baselineskip}
%\automark[part]{section}
\automark[chapter]{chapter}
\automark*[chapter]{section} %mithilfe des * wird nur ergänzt; bei vorhandener section soll also das in der Kopfzeile stehen
\automark*[chapter]{subsection}
\ihead[]{\headmark}
%\ohead[]{Seite~\thepage}
\cfoot{\hypersetup{linkcolor=black}Seite~\thepage~von~\pageref{LastPage}}

\usepackage[utf8]{inputenc}
\usepackage[ngerman, english]{babel}
\usepackage[expansion=true, protrusion=true]{microtype}
\usepackage{amsmath}
\usepackage{amsfonts}
\usepackage{amsthm}
\usepackage{amssymb}
\usepackage{mathtools}
\usepackage{mathdots}
\usepackage{aligned-overset} % otherwise, overset/underset shift alignment
\usepackage{upgreek}
\usepackage[free-standing-units]{siunitx}
\usepackage{esvect}
\usepackage{graphicx}
\usepackage{epstopdf}
\usepackage[hypcap]{caption}
\usepackage{booktabs}
\usepackage{flafter}
\usepackage[section]{placeins}
\usepackage{pdfpages}
\usepackage{textcomp}
\usepackage{subfig}
\usepackage[italicdiff]{physics}
\usepackage{xparse}
\usepackage{wrapfig}
\usepackage{color}
\usepackage{multirow}
\usepackage{dsfont}
\numberwithin{equation}{chapter}%{section}
\numberwithin{figure}{chapter}%{section}
\numberwithin{table}{chapter}%{section}
\usepackage{empheq}
\usepackage{tikz-cd}%für Kommutationsdiagramme
\usepackage{tikz}
\usepackage{pgfplots}
\usepackage{mdframed}
\usepackage{floatpag} % to have clear pages where figures are
%\usepackage{sidecap} % for caption on side -> not needed in the end
\usepackage{subfiles} % To put chapters into main file

\usepackage{hyperref}
\hypersetup{colorlinks=true, breaklinks=true, citecolor=linkblue, linkcolor=linkblue, menucolor=linkblue, urlcolor=linkblue} %sonst z.B. orange bei linkcolor

\usepackage{imakeidx}%für Erstellen des Index
\usepackage{xifthen}%damit bei \Def{} das Index-Arugment optional gemacht werden kann

\usepackage[printonlyused]{acronym}%withpage -> seems useless here

\usepackage{enumerate} % for custom enumerators

\usepackage{listings} % to input code

\usepackage{csquotes} % to change quotation marks all at once


%\usepackage{tgtermes} % nimmt sogar etwas weniger Platz ein als default font, aber wenn dann nur auf Text anwenden oder?
\usepackage{tgpagella} % traue mich noch nicht ^^ Bzw macht ganze Formatierung kaputt und so sehen Definitionen nicer aus
%\usepackage{euler}%sieht nichtmal soo gut aus und macht Fehler
%\usepackage{mathpazo}%macht iwie überall pagella an...
\usepackage{newtxmath}%etwas zu dick halt im Vergleich dann; wenn dann mit pagella oder überall Times gut

\setkomafont{chapter}{\fontfamily{qpl}\selectfont\Huge}%{\rmfamily\Huge\bfseries}
\setkomafont{chapterentry}{\fontfamily{qpl}\selectfont\large\bfseries}%{\rmfamily\large\bfseries}
\setkomafont{section}{\fontfamily{qpl}\selectfont\Large}%{\rmfamily\Large\bfseries}
%\setkomafont{sectionentry}{\rmfamily\large\bfseries} % man kann anscheinend nur das oberste Element aus toc setzen, hier also chapter
\setkomafont{subsection}{\fontfamily{qpl}\selectfont\large}%{\rmfamily\large}
\setkomafont{paragraph}{\rmfamily}%\bfseries\itshape}%\underline
\setkomafont{title}{\fontfamily{qpl}\selectfont\Huge\bfseries}%{\Huge\bfseries}
\setkomafont{subtitle}{\fontfamily{qpl}\selectfont\LARGE\scshape}%{\LARGE\scshape}
\setkomafont{author}{\Large\slshape}
\setkomafont{date}{\large\slshape}
\setkomafont{pagehead}{\scshape}%\slshape
\setkomafont{pagefoot}{\slshape}
\setkomafont{captionlabel}{\bfseries}



\definecolor{mygreen}{rgb}{0.8,1.00,0.8}
\definecolor{mycyan}{rgb}{0.76,1.00,1.00}
\definecolor{myyellow}{rgb}{1.00,1.00,0.76}
\definecolor{defcolor}{rgb}{0.10,0.00,0.60} %{1.00,0.49,0.00}%orange %{0.10,0.00,0.60}%aquamarin %{0.16,0.00,0.50}%persian indigo %{0.33,0.20,1.00}%midnight blue
\definecolor{linkblue}{rgb}{0.00,0.00,1.00}%{0.10,0.00,0.60}


% auch gut: green!42, cyan!42, yellow!24


\setlength{\fboxrule}{0.76pt}
\setlength{\fboxsep}{1.76pt}

%Syntax Farbboxen: in normalem Text \colorbox{mygreen}{Text} oder bei Anmerkungen in Boxen \fcolorbox{black}{myyellow}{Rest der Box}, in Mathe-Umgebung für farbige Box \begin{empheq}[box = \colorbox{mycyan}]{align}\label{eq:} Formel \end{empheq} oder farbigen Rand \begin{empheq}[box = \fcolorbox{mycyan}{white}]{align}\label{eq:} Formel \end{empheq}

% Idea for simpler syntax: renew \boxed command from amsmath; seems to work like fbox, so maybe background color can be changed there

\usepackage[most]{tcolorbox}
%\colorlet{eqcolor}{}
\tcbset{on line, 
        boxsep=4pt, left=0pt,right=0pt,top=0pt,bottom=0pt,
        colframe=cyan,colback=cyan!42,
        highlight math style={enhanced}
        }

\newcommand{\eqbox}[1]{\tcbhighmath{#1}}


\newcommand{\manyqquad}{\qquad \qquad \qquad \qquad}  % Four seems to be sweet spot



\newcommand{\rem}[1]{\fcolorbox{yellow!24}{yellow!24}{\parbox[c]{0.985\textwidth}{\textbf{Remark}: #1}}}%vorher: black als erste Farbe, das macht Rahmen schwarz%vorher: black als erste Farbe, das macht Rahmen schwarz

%\newcommand{\anm}[1]{\footnote{#1}}

\newcommand{\anmind}[1]{\fcolorbox{yellow!24}{yellow!24}{\parbox[c]{0.92 \textwidth}{\textbf{Anmerkung}: #1}}}
% wegen Einrückung in itemize-Umgebungen o.Ä.

\newcommand{\eqboxold}[1]{\fcolorbox{white}{cyan!24}{#1}}

\newcommand{\textbox}[1]{\fcolorbox{white}{cyan!24}{#1}}


\newcommand{\Def}[2][]{\textcolor{defcolor}{\fontfamily{qpl}\selectfont \textit{#2}}\ifthenelse{\isempty{#1}}{\index{#2}}{\index{#1}}}%{\colorbox{green!0}{\textit{#1}}}
% zwischendurch Test mit \textbf{#1} noch (wurde aber viel größer)

% habe jetzt Schrift/ font pagella reingehauen (mit qpl), ist mega; wobei Times auch toll (ptm statt qpl)

% wenn Farbe doch doof, einfach beide auf white :D




\mdfdefinestyle{defistyle}{topline=false, rightline=false, linewidth=1pt, frametitlebackgroundcolor=gray!12}

\mdfdefinestyle{satzstyle}{topline=true, rightline=true, leftline=true, bottomline=true, linewidth=1pt}

\mdfdefinestyle{bspstyle}{%
rightline=false,leftline=false,topline=false,%bottomline=false,%
backgroundcolor=gray!8}


\mdtheorem[style=defistyle]{defi}{Definition}[chapter]%[section]
\mdtheorem[style=satzstyle]{thm}[defi]{Theorem}
\mdtheorem[style=satzstyle]{prop}[defi]{Property}
\mdtheorem[style=satzstyle]{post}[defi]{Postulate}
\mdtheorem[style=satzstyle]{lemma}[defi]{Lemma}
\mdtheorem[style=satzstyle]{cor}[defi]{Corollary}
\mdtheorem[style=bspstyle]{ex}[defi]{Example}




% if float is too long use \thisfloatpagestyle{onlyheader}
\newpairofpagestyles{onlyheader}{%
\setlength{\headheight}{\baselineskip}
\automark[section]{section}
%\automark*[section]{subsection}
\ihead[]{\headmark}
%
% only change to previous settings is here
\cfoot{}
}




% Spacetime diagrams
%\usepackage{tikz}
%\usetikzlibrary{arrows.meta}
% -> setting styles sufficient
%\tikzset{>={Latex[scale=1.2]}}
\tikzset{>={Stealth[inset=0,angle'=27]}}

%\usepackage{tsemlines}  % To draw Dragon stuff; Bard says this works with emline, not pstricks
%\def\emline#1#2#3#4#5#6{%
%       \put(#1,#2){\special{em:moveto}}%
%       \put(#4,#5){\special{em:lineto}}}


% Inspiration: https://de.overleaf.com/latex/templates/minkowski-spacetime-diagram-generator/kqskfzgkjrvq, https://www.overleaf.com/latex/examples/spacetime-diagrams-for-uniformly-accelerating-observers/kmdvfrhhntzw

\usepackage{fp}
\usepackage{pgfkeys}


\pgfkeys{
	/spacetimediagram/.is family, /spacetimediagram,
	default/.style = {stepsize = 1, xlabel = $x$, ylabel = $c t$},
	stepsize/.estore in = \diagramStepsize,
	xlabel/.estore in = \diagramxlabel,
	ylabel/.estore in = \diagramylabel
}
	%lightcone/.estore in = \diagramlightcone  % Maybe also make optional?
	% Maybe add argument if grid is drawn or markers along axis? -> nope, they are really important

% Mandatory argument: grid size
% Optional arguments: stepsize (sets grid scale), xlabel, ylabel
\newcommand{\spacetimediagram}[2][]{%
	\pgfkeys{/spacetimediagram, default, #1}

    % Draw the x ct grid
    \draw[step=\diagramStepsize, gray!30, very thin] (-#2 * \diagramStepsize, -#2 * \diagramStepsize) grid (#2 * \diagramStepsize, #2 * \diagramStepsize);

    % Draw the x and ct axes
    \draw[->, thick] (-#2 * \diagramStepsize - \diagramStepsize, 0) -- (#2 * \diagramStepsize + \diagramStepsize, 0);
    \draw[->, thick] (0, -#2 * \diagramStepsize - \diagramStepsize) -- (0, #2 * \diagramStepsize + \diagramStepsize);

	% Draw the x and ct axes labels
    \draw (#2 * \diagramStepsize + \diagramStepsize + 0.2, 0) node {\diagramxlabel};
    \draw (0, #2 * \diagramStepsize + \diagramStepsize + 0.2) node {\diagramylabel};

	% Draw light cone
	\draw[black!10!yellow, thick] (-#2 * \diagramStepsize, -#2 * \diagramStepsize) -- (#2 * \diagramStepsize, #2 * \diagramStepsize);
	\draw[black!10!yellow, thick] (-#2 * \diagramStepsize, #2 * \diagramStepsize) -- (#2 * \diagramStepsize, -#2 * \diagramStepsize);
}



\pgfkeys{
	/addobserver/.is family, /addobserver,
	default/.style = {grid = true, stepsize = 1, xpos = 0, ypos = 0, xlabel = $x'$, ylabel = $c t'$},
	grid/.estore in = \observerGrid,
	stepsize/.estore in = \observerStepsize,
	xpos/.estore in = \observerxpos,
	ypos/.estore in = \observerypos,
	xlabel/.estore in = \observerxlabel,
	ylabel/.estore in = \observerylabel
}

% Mandatory argument: grid size, relative velocity (important: if negative, must be given as (-1) * v where v is the absolute value, otherwise error)
% Optional arguments: stepsize (sets grid scale), xlabel, ylabel
\newcommand{\addobserver}[3][]{%
	\pgfkeys{/addobserver, default, #1}

    % Evaluate the Lorentz transformation
    %\FPeval{\calcgamma}{1/((1-(#3)^2)^.5)}
    \FPeval{\calcgamma}{1/((1-((#3)*(#3)))^.5)} % More robust, allows negative v
    \FPeval{\calcbetagamma}{\calcgamma*#3}

	% Draw the x' and ct' axes
	\draw[->, thick, cm={\calcgamma,\calcbetagamma,\calcbetagamma,\calcgamma,(\observerxpos,\observerypos)}, blue] (-#2 * \observerStepsize - \observerStepsize, 0) -- (#2 * \observerStepsize + \observerStepsize, 0);
    \draw[->, thick, cm={\calcgamma,\calcbetagamma,\calcbetagamma,\calcgamma,(\observerxpos,\observerypos)}, blue] (0, -#2 * \observerStepsize - \observerStepsize) -- (0, #2 * \observerStepsize + \observerStepsize);

	% Check if grid shall be drawn
	\ifthenelse{\equal{\observerGrid}{true}}{%#
		% Draw transformed grid
		\draw[step=\diagramStepsize, blue, very thin, cm={\calcgamma,\calcbetagamma,\calcbetagamma,\calcgamma,(\observerxpos,\observerypos)}] (-#2 * \diagramStepsize, -#2 * \diagramStepsize) grid (#2 * \diagramStepsize, #2 * \diagramStepsize);
	}{} % Do nothing in else case

	% Draw the x' and ct' axes labels
    \draw[cm={\calcgamma,\calcbetagamma,\calcbetagamma,\calcgamma,(\observerxpos,\observerypos)}, blue] (#2 * \observerStepsize + \observerStepsize + 0.2, 0) node {\observerxlabel};
    \draw[cm={\calcgamma,\calcbetagamma,\calcbetagamma,\calcgamma,(\observerxpos,\observerypos)}, blue] (0, #2 * \observerStepsize + \observerStepsize + 0.2) node {\observerylabel};
}



\pgfkeys{
	/addevent/.is family, /addevent,
	default/.style = {v = 0, label =, color = red, label placement = below, radius = 5pt},
	v/.estore in = \eventVelocity,
	label/.estore in = \eventLabel,
	color/.estore in = \eventColor,
	label placement/.estore in = \eventLabelPlacement,
	radius/.estore in = \circleEventRadius
}

% Mandatory argument: x position, y position
% Optional arguments: relative velocity (important: if negative, must be given as (-1) * v where v is the absolute value, otherwise error), label, color, label placement
\newcommand{\addevent}[3][]{%
	\pgfkeys{/addevent, default, #1}

    % Evaluate the Lorentz transformation
    %\FPeval{\calcgamma}{1/((1-(#3)^2)^.5)}
    \FPeval{\calcgamma}{1/((1-((\eventVelocity)*(\eventVelocity)))^.5)} % More robust, allows negative v
    \FPeval{\calcbetagamma}{\calcgamma*\eventVelocity}

	% Draw event
	\draw[cm={\calcgamma,\calcbetagamma,\calcbetagamma,\calcgamma,(0,0)}, red] (#2,#3) node[circle, fill, \eventColor, minimum size=\circleEventRadius, label=\eventLabelPlacement:\eventLabel] {};
}



\pgfkeys{
	/lightcone/.is family, /lightcone,
	default/.style = {stepsize = 1, xpos = 0, ypos = 0, color = yellow, fill opacity = 0.42},
	stepsize/.estore in = \lightconeStepsize,
	xpos/.estore in = \lightconexpos,
	ypos/.estore in = \lightconeypos,
	color/.estore in = \lightconeColor,
	fill opacity/.estore in = \lightconeFillOpacity
}

% Mandatory arguments: cone size
% Optional arguments: stepsize (scale of grid), xpos, ypos, color, fill opacity
\newcommand{\lightcone}[2][]{
	\pgfkeys{/lightcone, default, #1}
	% Draw light cone -> idea: go from event location into the directions (1, 1), (-1, 1) for upper part of cone and then in directions (-1, -1), (1, -1) for lower part of cone
	\draw[\lightconeColor, fill, fill opacity=\lightconeFillOpacity] (\lightconexpos * \lightconeStepsize - #2 * \lightconeStepsize, \lightconeypos * \lightconeStepsize + #2 * \lightconeStepsize) -- (\lightconexpos, \lightconeypos) -- (\lightconexpos * \lightconeStepsize + #2 * \lightconeStepsize, \lightconeypos * \lightconeStepsize + #2 * \lightconeStepsize);
	\draw[\lightconeColor, fill, fill opacity=\lightconeFillOpacity] (\lightconexpos * \lightconeStepsize - #2 * \lightconeStepsize, \lightconeypos * \lightconeStepsize - #2 * \lightconeStepsize) -- (\lightconexpos, \lightconeypos) -- (\lightconexpos * \lightconeStepsize + #2 * \lightconeStepsize, \lightconeypos * \lightconeStepsize - #2 * \lightconeStepsize);
}






\begin{document}

\chapter{Bridging The Gap Between Special And General Relativity}

\begin{center}
The last chapter was dedicated to learning the effects of special relativity in an intuitive-based manner. It has already been teased that a mathematical discussion of the same subject can be equally as fruitful, but we have not really started it yet. Not only that, the formalization of relativistic physics leads naturally to the theory of general relativity, where such a description is required.

Therefore, this chapter in some ways builds a bridge between special and general relativity because we start to treat everything on the same footing.
\end{center}


% Sources: Dragon, Giulini, Einstein paper and book, Penrose



\newpage



	\section{Minkowski Space}
Throughout the last sections, we discovered more and more how space and time work in relativity and how they are related. Important contributions to that picture were made by the insights of Einstein regarding synchronization of clocks as well as Lorentz and Poincaré, who developed a corrected version of the Galilei transform.

It might be clear to the reader that this implies space and time are not independent anymore, but instead have to be treated on the same footing. Historically speaking, however, this final step was not made until Minkowski proposed his viewpoint that physics should take place in a four-dimensional \Def{spacetime}. This unification is an essential part of how the theory of relativity is described in modern literature, in particular because it allows a description in terms of a well-developed mathematical theory -- the theory of manifolds. We will also adopt the usage from now on.


% over the course of the last sections, we saw more and more how space time work in relativity and how they are related; Einstein's insights and Lorentz transform show how clocks work differently than their Newtonian pendants once relativity principle is incorporated, they are not independent anymore; but it is only with the Minkowskian viewpoint that this connection is fully unveiled/completed by uniting them into one spacetime (which goes along with unified mathematical formulation on sound/fond footing)


% ok, this might be nice: in spacetime diagrams, we see the idea how space and time (which are, as we know now, no separate notions) can be combined into one entity -- describe them as coordinates in one space; this space is called Minkowski space and spacetime diagrams can be seen as a visualization of it; SR is basically which geometry Minkowski space possesses and mathematically, we can describe that in metric; this metric we get from invariant notion we have already seen




%		\subsection{Basic Definitions}
First, it is necessary to state what spacetime is in a formal, mathematical way.


\begin{defi}[Minkowski Space]
	In special relativity, \Def{Spacetime} is described as a $4$-manifold $M$ with one time and three spatial coordinates. Another common name for $M$ is \Def{Minkowski space} with corresponding symbol $\mathbb{M}$ (i.e.~we use $M = \mathbb{M}$).

	\Def{Events} $E$ can be identified with points in $M$ and thus specified using \Def{inertial Cartesian coordinates} (ICCs)
	\begin{equation}
		\eqbox{
		x^\mu = (ct, x, y, z) = (ct, \vec{x})
		}
	\end{equation}
	where $(x, y, z)$ are the Cartesian coordinates of Euclidean space and $t$ the coordinate time measured by (synchronized) clocks in this space.
	
	
	\Def{World lines} are curves $\Gamma = \Gamma(\sigma): I = [a, b] \rightarrow M$.
\end{defi}
We adopt the convention to rescale the time-component to have the same units as the spatial components. To avoid these conversion factors, we could choose units where $c = 1$, as it is common practice when dealing with relativity. However, we have elected not to do this here. Another note on the coordinates is that choosing inertial Cartesian coordinates is by no means necessary\footnote{In fact, coordinates themselves have \emph{no} physical meaning.}, just like spherical coordinates are equally suited to describe physics in Euclidean space as Cartesian ones. It is simply more convenient for now and we maintain generality of our discussion because the theory of manifolds provides us with natural ways of changing coordinates. This mathematical way of demanding coordinate-independent statements and using invariant notions like tensors is completely equivalent to what relativity demanded: physics has to be invariant of the observer, so only special quantities like proper times have an invariant and thus physical meaning (coordinate times or positions do not). This correspondence of underlying ideas and concepts is the reason why describing relativity in the language of manifolds is a very natural idea.


Having defined spacetime, we can see why space-time diagrams have been called spacetime diagrams: they visualize the new-defined entity $M$, just like coordinate planes with $x$-, $y$-axis visualize Euclidean space. This also explains why they are often called \Def{Minkowski diagrams}.\\


While points and vectors do not necessarily have a direct identification with each other, elements of the vector space $\mathbb{M}$ are also points in $M$ because both are elements of $\mathbb{R}^4$ (which further motivates $M = \mathbb{R}^4$).\footnote{To be more precise, $\mathbb{M}$ is the affine space of $\mathbb{R}^4$, i.e.~we keep the space itself, but not the attached notions of distances and angles. In particular, there is no origin (in accordance with relativity principle).} For this reason, points $x^\mu = (ct, x, y, z)$ in coordinates are often also interpreted as (the components of) a four-vector $\underline{x}$ (so that $\mathbb{M}$ really forms a vector space, justifying the name Minkowski \emph{space}). Over the course of the next sections, we will develop the mathematical tools to describe spacetime. As we will learn throughout this section, it has more structure than what is natively given by \enquote{pure} manifolds, namely a metric. Therefore, $\mathbb{M}$ will turn out to be a \Def{pseudo-Riemannian manifold} or \Def{Lorentz-manifold}, $\mathbb{M} = (\mathbb{R}^4, \eta)$. At this point, the departure from Euclidean space with metric $g$ and line element $ds^2 = dt^2 + dx^2 + dy^2 + dz^2$ begins.


%introduce four-velocity as tangent vector; then also state that as of now, we are not able to take norm of it -> motivates metric section




		\subsection{Metric \& Inner Product}
We have now seen how physics can be conveniently described using a 4D manifold, which we called spacetime. Points on this manifold are events and we can change coordinates or inertial frames using Lorentz-transformations. Moreover, there are several quantities that can be defined naturally on manifolds, for example curves, vectors, and covectors (maps that take vectors as input). While manifolds do have an additional natural structure, this is given by topology. In physics, however, we are also interested in statements concerned with distances between events and to measure them we need additional structure. More specifically, we have to specify a metric that will allow measuring distances, as well as norms of vectors via the induced inner product.


Mathematically, metrics are objects called tensors and they have the convenient property that they are invariant under coordinate changes. Therefore, distances are physically meaningful statements because they do not depend on the inertial frame we compute them on. In the tradition of invariant quantities that have been encountered so far, we may guess that the metric will be related to light in some way. From the universality of the speed if light $c$, distances $s$ are equivalent to times $t$ for light, $s = c t$. Because of that, a natural measure for distances is the time elapsed on a clock, i.e.~the geometric structure of Minkowski space is determined by Minkowski's theorem \ref{thm:minkowski_moving_clocks}. Instead of denoting time with the usual variable $t$, we will now switch to the \Def{proper time} $\tau$ since the time elapsed a clock between events $(0, 0, 0, 0)$ and $(ct, x, y, z)$ is
\begin{equation}\label{eq:proper_time_uniform}
	\eqbox{
	\tau = \sqrt{1 - \frac{v^2}{c^2}} \, t = \sqrt{1 - \qty(\frac{x}{ct})^2 - \qty(\frac{y}{ct})^2 - \qty(\frac{z}{ct})^2} \, t = \sqrt{t^2 - (x^2 + y^2 + z^2) / c^2}
	}
\end{equation}
This distance notion depends on the trajectory taken by the clock/corresponding observer (more specifically, on the uniform velocity $v$), but will in general not be equal to $t$, which is the time measured simultaneously by a clock resting in the corresponding frame (but we can still compute $\tau$ from this $t$ because it directly accounts for time dilation). This does \emph{not} mean $\tau$ depends on the coordinates we use to compute it, i.e.~the specific inertial frame chosen, the striking factor is the path taken by the clock we measure time for.



\begin{ex}[Proper Time vs.~Coordinate Time]
	We will use an example to elaborate a bit more on the meaning of all the symbols in \eqref{eq:proper_time_uniform}. Say we are in an inertial frame with coordinates $(ct, x, y, z)$. 
	
	The time elapsed between two events 
	A clock resting in this frame will measure the proper time between events $E_1 = (ct_1, 0, 0, 0), E_2 = (ct_2, 0, 0, 0)$, i.e.~it will show
	\begin{equation*}
		\tau_{E_2, E_1} = t_2 - t_1
	\end{equation*}
	to be elapsed between them. If we look at events $E_3 = (ct_1, x_1, y_1, z_1), E_4 = (ct_2, x_2, y_2, z_2)$, however, it will still measure $t_2 - t_1$. This is not equal to the proper time
	\begin{equation*}
		\tau_{E_3, E_4} = \sqrt{(t_2 - t_1)^2 - ((x_2 - x_1)^2 + (y_2 - y_1)^2 + (z_2 - z_1)^2) / c}
	\end{equation*}
	between these events, which would be measured by a clock moving on the straight world line that connects them.
	
	
	
	Going to a frame with coordinates $(ct', x', y', z')$, which moves uniformly between $E_3$ and $E_4$, the situation is different. A clock resting in this frame moves on a trajectory between $E_3$ and $E_4$, which means the time $t'$ measured by it now coincides with the proper time between these two events,
	\begin{equation*}
		\tau_{E_3, E_4} = t'_2 - t'_1 \, .
	\end{equation*}
	This is because the spatial coordinates of $E_3$ and $E_4$ are equal in the primed frame, the Lorentz transformation automatically incorporates all spatial movement happening in unprimed coordinates into $t'$. However,
	\begin{equation*}
		t'_2 - t'_1 = \tau_{E_1, E_2} = \sqrt{(t'_2 - t'_1)^2 - ((x'_2 - x'_1)^2 + (y'_2 - y'_1)^2 + (z'_2 - z'_1)^2) / c}
	\end{equation*}
	because the trajectory connecting $E_1$ and $E_2$ is not parallel to $ct'$. This is the mutuality of time dilation. Nonetheless, it shows a general way how different observers can agree on times: by using the proper time between events. Not only do the numbers agree in this case, conceptually it also makes a lot of sense to look at times which are measured by clocks which actually \enquote{see} both events, i.e.~move on a world line connecting them, instead of using clocks far away from the event.\\
	
	
	This whole behaviour might seem familiar, we have already encountered it in example \ref{ex:twin_paradox_2}, where the twin paradox has been discussed in detail. Here it already showed how different observers do not necessarily agree on times $t$ their own clocks measure between two events, but they do agree on times they infer to be measured by a clock moving on a trajectory connecting these two events -- the proper time. Moreover, since all observers agree on the proper time, one can immediately infer effects like time dilation, which becomes as easy as
	\begin{equation*}
		t'_2 - t'_1 = \tau'_{E_3, E_4} = \tau_{E_3, E_4} < t_2 - t_1 \, ,
	\end{equation*}
	the primed, moving observers measures smaller times than the unprimed, resting one.
\end{ex}

As we have seen in this example, in some coordinates it is very easy to compute proper times because the object or particle we look at is at rest in this frame, which implies that the coordinate time $t$ is already the proper time, $\tau = t$. These coordinates are often given a special name.
\begin{defi}[Instantaneous Rest Frame]
	A frame $(t, x, y, z)$ where
	\begin{equation}
		\eqbox{
		%dx = dy = dz = 0
		%\qquad \Rightarrow \qquad d\tau = dt
		x = y = z = 0
		\qquad \Rightarrow \qquad \tau = t
		}
	\end{equation}
	along the world line of a particle is called \Def{instantaneous rest frame} or \Def{comoving frame}.
\end{defi}
This definition can also be extended to non-uniform velocities $v = v(t)$ because instantaneous rest frames taken at different times are related by Lorentz transforms. We will now see how the general definition of the proper time may be extended to this case.



			\paragraph{Generalized Proper Time}
What was denoted with $\tau$ in equation \eqref{eq:proper_time_uniform}, in reality is a difference $\Delta \tau$ of proper times (just measured with respect to $\tau = 0$), and the same is true for the coordinates $(ct, x, y, z)$. Making these differences infinitesimally small, i.e.~$\Delta \rightarrow d$, we obtain the infinitesimal distance or \Def{proper time element}\footnote{We adopt the common notation $dx^2 := (dx)^2 := dx \otimes dx$ ($\otimes$ is the tensor product, usually omitted here).}
\begin{equation}\label{eq:proper_time_element}
	\eqbox{
	d\tau^2 = dt^2 - \frac{dx^2 + dy^2 + dz^2}{c^2} = (1 - \frac{v^2}{c^2}) \, dt^2 = dt^2 / \gamma^2
	} \, .
\end{equation}
From the general form $ds^2 = \eta_{\mu \nu} dx^\mu dx^\nu$ of line elements, one can immediately read off the components of the \Def{Minkowski metric} $\eta$, which can be conveniently arranged in a matrix
\begin{equation}\label{eq:minkowski_metric}
	\eqbox{
	\eta = \mqty(1 & 0 & 0 & 0 \\ 0 & -1 & 0 & 0 \\ 0 & 0 & -1 & 0 \\ 0 & 0 & 0 & -1)
	} \, .
\end{equation}


Potentially contrary to popular belief, the scope special relativity is not restricted to observers with uniform velocities. Using some of the quantities and tools we have just derived, it is possible to extend the description of certain dynamics to observers moving with non-uniform velocities, i.e.~accelerating ones. For distances, this is possible using integration on manifolds, where the length of a curve $\Gamma(\sigma)$ defined on an interval $I \subset \mathbb{R}$ is
\begin{equation}
	L(\Gamma) = \int_\Gamma ds = \int_\Gamma \sqrt{g_{\mu \nu} dx^\mu dx^\nu} = \int_I d\Gamma := \int_I \sqrt{g_{\mu \nu} t^\mu t^\nu} \, d\sigma = \int_I \sqrt{g(\underline{t}, \underline{t})} \, d\sigma \, .
\end{equation}
$\underline{t} = \dv{\Gamma(\sigma)}{\sigma}$ is the tangent vector (field) along $\Gamma$. Because we are equipped with a metric $g_{\mu \nu} = \eta_{\mu \nu}$ and corresponding line element $d\tau$, we can compute proper times for arbitrary kinds of movements.
\begin{post}[Clock Postulate]\label{post:clock_postulate}
	Given a world line $\Gamma$ parametrized by $\sigma \in I = [a, b]$, i.e.
	\begin{equation*}
		\Gamma(\sigma): \sigma \mapsto (t, x, y, z) = (t(\sigma), x(\sigma), y(\sigma), z(\sigma)) \, ,
	\end{equation*}
	the proper time elapsed along $\Gamma$ is
	\begin{equation}\label{eq:proper_time_general}
		\eqbox{
		\begin{split}
		\tau & = \int_\Gamma d\tau = \int_\Gamma dt / \gamma(t) = \int_\Gamma \sqrt{1 - \frac{v(t)^2}{c^2}} \, dt
		\\
		%&= \int_a^b \dv{\tau}{\sigma} \, d\sigma = \int_a^b \sqrt{\qty(\dv{t}{\sigma})^2 - \frac{1}{c^2} \qty(\dv{\vec{x}}{\sigma})^2} \, d\sigma
		&= \int_a^b \dv{\tau}{\sigma} \, d\sigma = \int_a^b \sqrt{\qty(\dv{t}{\sigma})^2 - \frac{1}{c^2} \dv{x^\alpha}{\sigma} \dv{x^\alpha}{\sigma}} \, d\sigma
		\end{split}
		} \, .
	\end{equation}
\end{post}
Note that while this \enquote{derivation} we provided here does make a lot sense, there is no guarantee for it to be correct -- accelerating clocks could be vastly different from uniformly moving and resting ones. For this reason, we call \eqref{eq:proper_time_general} a postulate rather than a property. Once again, experiments have tested this postulate to very high accelerations of $\approx 10^{16} g$ ($g = 9.81 \frac{\metre}{\second^2}$), which showed no dependence on it (only on the speed $v$).

Intuitively, we can see that \eqref{eq:proper_time_general} works because it utilizes infinitesimal steps $d\tau$ where $v(t)$ does not change, so we can apply what we know at this point and integrate up the results from all points (idea is similar to the rectification of curves). It includes the case of uniform movement $v = \text{const.}$, whence the integrand $\gamma$ is constant and evaluation of the integral simply yields $\tau = \gamma (t_b - t_a)$. In case $v = v(t)$, the most convenient formula to use from \eqref{eq:proper_time_general} really depends on what is given -- it may be the parametrization that is explicitly known or the velocity as a function of time.

Instead of $\sigma$, one could also choose arbitrary linear combinations $\sigma' = e \sigma + f$ ($e, f \in \mathbb{R}$) to parametrize $\Gamma$ and then use $d\sigma' = e \, d\sigma$, which means that the integral in \eqref{eq:proper_time_general} is invariant under changes of this \Def{affine parameter}. A very common choice is $\sigma = \tau$ and we will later see how this specific parametrization can be characterized.\\


Another remark on this definition is related to the metric. In coordinates $x'^\mu$ which are not inertial Cartesian ones (nothing prevents us from using spherical coordinates for the spatial part, as an example), the metric is very likely to have different components $\eta'_{\mu \nu} \neq \eta_{\mu \nu}$. In this case, the last equality would not hold true anymore, although analogous formulas can be obtained from (again, assuming a world line $\Gamma$ with tangent vector $\underline{t}$)
\begin{equation}\label{eq:proper_time_non_cartesian}
	\eqbox{
	%\tau = \int_\Gamma d\tau = \frac{1}{c} \int_I \sqrt{\eta'(\underline{t}, \underline{t})} \, d\sigma = \frac{1}{c} \int_I \sqrt{\eta'_{\mu \nu} dx^\mu(\underline{t}) dx^\nu(\underline{t})} \, d\sigma = \frac{1}{c} \int_I \sqrt{\eta'_{\mu \nu} t^\mu t^\nu} \, d\sigma
	\tau = \int_\Gamma d\tau = \frac{1}{c} \int_\Gamma \sqrt{\eta_{\mu \nu} dx^\mu dx^\nu} = \frac{1}{c} \int_I \sqrt{\eta_{\mu \nu} t^\mu t^\nu} \, d\sigma = \frac{1}{c} \int_I \sqrt{\eta'_{\mu \nu} t'^\mu t'^\nu} \, d\sigma
	} \, ,
\end{equation}
which holds due to the transformation rule for integrals (note that the vector components have to be transformed as well, $t^\mu \rightarrow t'^\mu$). This rule is also what implies the invariance of $\tau$ under the coordinates/frame it is computed in, which is a mathematical way of stating that all observers agree on the proper time.

	\rem{the factor $1 / c$ appears due to the coordinates being $(c t, x, y, z)$, which implies $\eta_{\mu \nu} dx^\mu dx^\nu = d(c t)^2 - dx^2 - dy^2 - dz^2 = \frac{1}{c^2} d\tau^2$.}



			\paragraph{Proper Distance}
While it is an important task to measure times, an invariant notion of distances is also important. Distances are commonly associated with metrics and line elements as well, for example
\begin{equation*}
	dx^2 + dy^2 + dz^2
\end{equation*}
that measures lengths in the three-dimensional Euclidean space we live in. But what is an analogous quantity for four-dimensional Minkowski space?\\


We have already derived a line element suitable for relativistic analyses of spacetime, $d\tau$ as given by \eqref{eq:proper_time_element}, and using the work put in there to define distances is a straightforward idea. Let us start by looking at the conceptually simplest case of measuring distance between two events $E_1$ at $(t_1, x_1, y_1, z_1)$ and $E_2$ at $(t_2, x_2, y_2, z_2)$, where the rod placed between $(x_1, y_1, z_1)$ and $(x_2, y_2, z_2)$ is at rest. In this case, the distance measured by this rod is
\begin{equation}\label{eq:proper_length}
	s_{E_1, E_2} = \sqrt{(x_2 - x_1)^2 + (y_2 - y_1)^2 + (z_2 - z_1)^2)}
\end{equation}
independently of the times $t_1, t_2$. This quantity is called the \Def{proper length} of the measuring rod. However, for a general and thus observer-independent quantity, we have to remember that space and time are not separated anymore. This is because for an observer moving relative to the rod, the time at which he takes the measurements have a clear impact on the length that he measures the rod to have (since the endpoints of the rod are moving in this inertial frame). For invariant statements, we will now define $s$ in terms of the proper time $\tau$. Clearly, if we choose $t_1 = t_2$ in the rest frame of the rod, its proper length is
\begin{equation*}
	s_{E_1, E_2} = \sqrt{(x_2 - x_1)^2 + (y_2 - y_1)^2 + (z_2 - z_1)^2} = c \sqrt{- \tau_{E_1, E_2}^2} \, ,
\end{equation*}
confer \eqref{eq:proper_time_uniform}. Different frames will not measure $E_2$ and $E_1$ to be simultaneous, which means the general def Further imposing the condition of simultaneity, the same distances is measured from a different reference frame, the \enquote{generalized proper length} is
\begin{equation}\label{eq:proper_distance_uniform}
	\eqbox{
	s_{E_1, E_2} = c \sqrt{- \tau_{E_1, E_2}^2} = \sqrt{(x_2 - x_1)^2 + (y_2 - y_1)^2 + (z_2 - z_1)^2 - c^2 (t_2 - t_1)^2}
	}\, .
\end{equation}
This quantity is called \Def{proper distance} and allows for an invariant notion of spatial distance between events.\footnote{Of course, one can compute it for events that are not simultaneous in the current frame as well. The result will be the proper length in some inertial frame, namely the one where the events are simultaneous.} Just like the proper time automatically incorporated a correction for time dilation so that all inertial frames get the same results, the proper distance corrects for length contraction occurring if the length of a rod is measured by an observer moving relative to it. This is a non-trivial statement, so it is worthwhile to verify it explicitly.

\begin{proof}
	We have discussed this scenario for one spatial dimension in subsection \ref{subsec:time_dilation_lorentz}. One of the results was that we have to adjust the transformed time $t'_2$ in order to achieve $t'_2 = t'_1$ (cf.~\eqref{eq:length_contraction_correct}). As \eqref{eq:not_simultaneous} shows, the adjustment was $\Delta t' = v / c^2 (x'_2 - x'_1)$. For $s$, only the transformed and \emph{not} the corrected coordinates are used. This, however, turns out to be the crucial part in getting an invariant result, i.e.~why no length contraction occurs in $s$. We can verify this by inserting all the quantities we have mentioned into it:
	% Previous version, false I think
	%One of the results was that the distance measured for a rod with length $x_2 - x_1$ from coordinates moving relative to the rod is $x'_2 - x'_1 = (x_2 - x_1) / \gamma$ if we adjust the transformed time $t'_2$ slightly to achieve $t'_2 = t'_1$ (cf.~\eqref{eq:length_contraction_correct}). As \eqref{eq:not_simultaneous} shows, the adjustment was $\Delta t' = v / c^2 (x'_2 - x'_1)$ and in some sense, this correction has to be un-corrected in order to get the same results as in unprimed coordinates. We can see that this is actually done by $s$ if we insert all the quantities we have mentioned into it:
	\begin{align*}
		(x_2 - x_1)^2 \overset{!}&{=} (x'_2 - x'_1)^2 - c^2 (t'_2 - t'_1)^2 = (x'_2 - x'_1)^2 - c^2 (\Delta t)^2
		\\
		&= (x'_2 - x'_1)^2 - c^2 \frac{v^2}{c^4} (x'_2 - x'_1)^2 = (x'_2 - x'_1)^2 \qty(1 - v^2 / c^2) = (x_2 - x_1)^2
	\end{align*}
	where \eqref{eq:length_contraction_not_correct} was used, which shows that $(x'_1 - x'_2) = \gamma (x_2 - x_1)$. This calculation proofs that observers from all inertial frames agree on proper distances.
\end{proof}


Just like the proper time, proper distances between events are also path-dependent. The proper time element \eqref{eq:proper_time_element} can be used to define the analogous \Def{(proper) line element}
\begin{equation}\label{eq:proper_line_element}
	\eqbox{
	ds^2 = - c^2 d\tau^2 = dx^2 + dy^2 + dz^2 - c^2 dt^2
	} 
\end{equation}
which is suitable to define the proper distance covered when following an arbitrary, not necessarily straight world line $\Gamma$ (with parameter $\sigma \in I \subset \mathbb{R}$ and tangent vector $\underline{t}$),
\begin{equation}\label{eq:proper_distance_general}
	\eqbox{
	s = \int_\Gamma ds = \int_\Gamma \sqrt{- \eta_{\mu \nu} dx^\mu dx^\nu} = \int_I \sqrt{- \eta_{\mu \nu} t^\mu t^\nu} \, d\sigma
	} \, .
\end{equation}


At this point, we shall make a remark regarding the signs: in principle, one can choose $\eta$ to have only one minus (in the time component) without changing physics, if signs in equations where the metric occurs are changed accordingly. For example, the proper time and line elements would read
\begin{equation}
	d\tau^2 = - \frac{1}{c^2} \eta_{\mu \nu} dx^\mu dx^\nu
	\manyqquad
	ds^2 = \eta_{\mu \nu} dx^\mu dx^\nu
\end{equation}
instead. Good arguments exist for both conventions, but in the way metric and line element have been introduced here (via the proper time), it was more natural to adopt the convention with signature $(+, -, -, -)$.

The fact that this overall sign does not matter is supported by the following alternative derivation of the metric. It is also based $c$ being constant, but instead of constructing clocks etc.~explicitly, it uses that light propagates as a spherical wave with velocity $c$. Writing this equation in multiple inertial frames yields
\begin{equation}
	c^2 t^2 = x^2 + y^2 + z^2 \; , \; c^2 t'^2 = x'^2 + y'^2 + z'^2 \quad \Leftrightarrow \quad c^2 t^2 - x^2 - y^2 - z^2 \overset{!}{=} c^2 t'^2 - x'^2 - y'^2 - z'^2 \, .
\end{equation}
by the relativity principle. This points to the invariant proper distance we have called $s$ or equivalently (upon multiplication of both sides with $1 / c^2$) the proper time $\tau$.



			\paragraph{Proper Time \& Distance in Spacetime Diagrams}
It is the invariance of proper time, distance that allows us to infer statements about the geometry of Minkowski space from them (where we now assume $v = \text{const.}$ again and thus use \eqref{eq:proper_time_uniform}, \eqref{eq:proper_distance_uniform}). In Euclidean space, points of constant distance $s$ lie on a circle around the origin, which is determined by the equation $s^2 = x^2 + y^2 + z^2$.

In Minkowski space, events of equal (temporal) distance lie on a hyperboloid of constant proper times $\tau$ and are determined by
\begin{equation*}
	c^2 \tau^2 = c^2 t^2 - x^2 - y^2 - z^2 \, .
\end{equation*}
Setting $c \tau = 1$ to find a unit time length and neglecting two spatial dimensions yields
\begin{equation}
	\eqbox{
	c^2 t^2 = 1 + x^2
	} \, .
\end{equation}
This equation defines a hyperbola, whose intersection with time axes in spacetime diagrams determines the \enquote{length of one time unit}, a very convenient way to see how moving clocks are perceived to be slower from a resting observers point of view (fig.~\ref{fig:minkowski_with_eichhyperbel}).\\


In the same manner, one can visualize events of equal spatial distance by looking at
\begin{equation*}
%	x'^2 = \gamma^2 (x - vt)^2
	s^2 = x^2 + y^2 + z^2 = c^2 t^2 - c^2 \tau^2 = x'^2 + y'^2 + z'^2 + c^2 t^2 - c^2 t'^2
	\, .
\end{equation*}
In spacetime diagrams, two of the three spatial dimensions are neglected and the unit length as a distance from the origin can be found by setting $t' = 0, x' = 1$. Therefore, we obtain
\begin{equation}
	\eqbox{
	x^2 = 1 + c^2 t^2
	} \, ,
\end{equation}
another hyperbolic equation. The corresponding curve is visualized in figure \ref{fig:minkowski_with_eichhyperbel}. Similarly to what was said about time before, this curve determines the \enquote{length of one spatial unit} in spacetime diagrams by its intersection with the $x$-axis of any observer that we choose to put into the diagram.



\begin{figure}
	\centering

	\begin{tikzpicture}[scale=1.2]
		% Add basis for diagram
		\spacetimediagram{4}
	
		% Add other observers
		\addobserver{2}{0.5}
		\addobserver[color=black!30!green, xlabel=$x''$, ylabel=$ct''$]{2}{(-1)*0.75}

		% Add eichhyperbel for time. From $1 = c^2 t^2 - x^2$ we get parametrization $(x, ct) = (x, \sqrt{1 + x^2})$
		\draw[domain=-4:4, very thick, smooth, variable=\x, color=orange] plot ({\x}, {sqrt(1 + \x * \x)});
		% Add eichhyperbel for space. From $1 = x^2 - c^2 t^2$ we get parametrization $(x, ct) = (\sqrt{1 + c^2 t^2}, ct)
		\draw[domain=-4:4, very thick, smooth, variable=\t, color=purple] plot ({sqrt(1 + \t * \t)}, {\t});
	\end{tikzpicture}

	\caption[Plot of hyperbolas $\pm 1 = c^2 t^2 - x^2$]{Plot of the curves fulfilling $\pm 1 = c^2 t^2 - x^2$ (orange for $+$, purple for $-$). Only the positive solutions are shown to make diagram less confusing. In addition, three three inertial frames are drawn: one rest frame in black and frames moving with $v = 0.5 c, -0.75$ relative to resting one in blue, green, respectively.\\
	As we can see, this yields same time steps as Lorentz transform, the orange hyperbola intersects the time-axes of all observers perfectly at one time unit on their respective axis (as it should). Similarly, the brown hyperbola intersects the space-axes of all observers at one length unit on their respective axis.}
	\label{fig:minkowski_with_eichhyperbel}
\end{figure}



			\paragraph{Inner Product}
The notion of a metric allows for the construction of a rich theory. We have seen how it can be used to define distances and now we will deal with another important structure on manifolds, which has already been used in \eqref{eq:proper_time_non_cartesian}.

\begin{defi}[Minkowski Inner Product]
The \Def{Minkowski inner product} of two vectors $\underline{v}, \underline{w}$ is
\begin{equation}\label{eq:inner_prod}
	\eqbox{
	\underline{v} \cdot \underline{w} := \eta(\underline{v}, \underline{w}) = \eta_{\mu \nu} dx^\mu(\underline{v}) dx^\nu(\underline{w}) = \eta_{\mu \nu} v^\mu w^\nu
	} \, .
\end{equation}

The induced norm of a vector is
\begin{equation}\label{eq:vector_norm}
	\eqbox{
	\norm{\underline{v}}^2 = \eta(\underline{v}, \underline{v}) = \eta_{\mu \nu} v^\mu v^\nu
	} \, .
\end{equation}
This is also the line element of the curve that $\underline{t}$ is tangent to.
\iffalse
% This is not needed, right?
Based on that, a general way to define the distance between events $E_1$ at $\underline{x_1} = (ct_1, x_1, y_1, z_1)$ and $E_2$ at $\underline{x_2} = (ct_2, x_2, y_2, z_2)$ in an inertial Cartesian frame is
\begin{equation}
	\eqbox{
	\begin{split}
	d(E_1, E_2) &= \sqrt{(x_2 - x_1)^2 + (y_2 - y_1)^2 + (z_2 - z_1)^2 - c^2 (t_2 - t_1)^2}
	\\
	&= c \, \sqrt{- \tau_{E_1, E_2}^2} = \eta_{\mu \nu} (x_1^\mu - x_2^\mu) (x_1^\nu - x_2^\nu)
	\\
	&= \min_{\Gamma: \; \Gamma(a) = E_1, \Gamma(b) = E_2} L(\Gamma)
	= \min_{\Gamma: \; \Gamma(a) = E_1, \Gamma(b) = E_2} \int_\Gamma ds
	% = \min_{\Gamma: \; \Gamma(a) = E_1, \Gamma(b) = E_2} c \int_a^b \norm{\underline{v}} \, d\sigma
	\end{split}
	}
\end{equation}
%where we have chosen $\Gamma(\sigma)$ to map from $I = [a, b]$.% and $\underline{v} = \dv{\Gamma(\sigma)}{\sigma}$.
i.e.~as the (proper) length of the straight world line connecting them.
\fi
\end{defi}
Strictly mathematically speaking, this does not define an inner product because it is not positive-definite, \enquote{only} non-degenerate. For this reason, $\eta$ is also called \Def{pseudo-Riemannian metric}. For simplicity (in typical physicist-manner) it is commonly called Minkowski inner product despite that. Due to the non-degeneracy of $\eta$, vectors can have negative or even vanishing norm. Other consequences include unit vectors now being defined by $\norm{\underline{t}}^2 = \pm 1$ instead of just $+ 1$ in case of Euclidean geometry.


Based on \eqref{eq:inner_prod}, we naturally get an equivalent, more formal way of quantifying causality.
\begin{defi}[Timelike, Lightlike, Spacelike]\label{defi:causality_v2}
	A world line $\Gamma$ with tangent vector $\underline{t}$ is called
	\begin{itemize}
		\item \Def{timelike}, if $\abs{v} < c \; \Leftrightarrow \; c^2 d\tau^2 = - ds^2 = \eta(\underline{t}, \underline{t}) > 0$ along $\Gamma$
		
		
		\item \Def{null/lightlike}, if $\abs{v} = c \; \Leftrightarrow \; c^2 d\tau^2 = - ds^2 = \eta(\underline{t}, \underline{t}) = 0$ along $\Gamma$
		
		
		\item \Def{spacelike}, if $\abs{v} > c \; \Leftrightarrow \; c^2 d\tau^2 = - ds^2 = \eta(\underline{t}, \underline{t}) < 0$ along $\Gamma$
	\end{itemize}
\end{defi}
This very compact characterization translates to events connected by world lines and adds a mathematical counterpart to the intuitive definition \ref{defi:causality_v1} from before. In particular, they show that due to them being related to the metric $\eta$, the causal relation of two events does not depend on the coordinates we use to denote/visualize them.

Using this definition, one can also interpret causality in terms of proper times and distances along the world line that connect them -- and vice versa. First of all, while one can compute these quantities for arbitrary events, we immediately see that proper times make \enquote{more sense} for timelike- or null-separated events than they do for spacelike-separated ones, while proper distances make \enquote{more sense} for spacelike- or null-separated events than they do for timelike-separated events. \enquote{Making sense} here refers to the value of $d\tau$ and $ds$, respectively, which can be real or, in the cases that make less sense, be complex. Complex quantities, however, are unphysical, at least in the scope of relativity. This is also related to the interpretation of each quantity: proper time can be thought of as the time measured by a clock which moves between the events and this clock can only move with $\abs{v} \leq c$ (i.e.~not on spacelike world lines); distances, on the other hand, often make most sense when measured simultaneously (i.e.~as a proper length in a certain frame), whence the corresponding events are naturally spacelike-separated (due to their spatial distance being greater than $0$, which is their temporal separation). Consequently, proper distances are not to be thought of as the distance covered by a clock moving along the world line (in fact, this interpretation is more appropriate for the proper time).

	\rem{this is related to the fact that for events separated by a timelike interval, there exists an inertial frame where they happen at the same place, while for spacelike separation, there exists one where they happen at the same time.}



-> Penrose continues to make interesting point on that on page 407: light cones more fundamental than metric



		\subsection{Lorentz Transformations 2}\label{subsec:lorentz_transformation_2}
Lorentz transformations have been introduced in section \ref{sec:lorentz_transformation_1}, from which we know how they look like, that they correctly reproduce effects of relativity and how they can be visualized in spacetime diagrams. Nonetheless, it is worthwhile to revisit them now because of the knowledge we have gained since then, in particular regarding the mathematical interpretation of spacetime as a 4-manifold. We have already seen how this formulation comes with a natural structure, like vectors. A crucial part of the theory of manifolds is yet to be discussed. After all, one of the key motivations to use manifolds was that while they are described in terms of coordinates, but their properties exist independently of them, i.e.~they do not depend on the specific choice of coordinates. Consequently, changing coordinates is an important part of the whole theory and the coordinate transformations of spacetime are those between different inertial frames -- Lorentz transformations. This interpretation is the first time we encounter the more general role they play since there is a rich mathematical structure related to them.



-> for more on group-theoretic nature (and mathematical treatment of SR in general), I can recommend \cite{giulini_srt_indepth}



-> after metric is nice, then we can present mathematical viewpoint; do not change norm of four-vector (this is well-known property of rotations, but there is also a second class of Lorentz transformations, which are called boosts; they represent what we have derived in previous section, change to other inertial frame that moves uniformly with respect to first one); also note that they admit group structure, i.e.~note properties here; maybe even introduce rapidity and note connection of Lie group and Lie algebra? But not elaborate on this

maybe just note that Lorentz transformation = change of coordinates/charts (which is corresponding term in language of manifolds); one condition to obtain them is by demanding norm of four-vector does not change (ahh, might be confusing because I think Nolting refers to this kind of in Euclidean space; what he calls norm is really proper time passing on clock which moves from origin to point, isn't it? Would make sense to demand this better stay the same after transformation after we have put so much effort into invariant definition) -> I like this, but does not fit before introduction of norm; so maybe make subsection on transformations after metric?



Lorentz-scalar or Lorentz-invariant is scalar quantity, which does not change under Lorentz-transformation; example is proper time, but also mass etc.~are of this nature





when interpreting Minkowski space as a manifold and working with coordinates/charts $\xi$, we know the results should be independent of $\xi$; in particular, that means they hold in other charts as well and changing coordinates is an important part; the basis changes even have a distinct name, \Def{Lorentz transformation}; this is basically group theory due to the symmetries that Minkowski space possesses (known from logic and experiments) ?right?



hmmm, parallel transformations form a group (as Einstein states); commutativity is trivial, but associativity not:
\begin{align}
	(v_1 +_R v_2) +_R v_3 &= \frac{v_1 + v_2}{1 + v_1 v_2 / c^2} +_R v_3 = \frac{\frac{v_1 + v_2}{1 + v_1 v_2 / c^2} + v_3}{1 + \frac{v_1 + v_2}{1 + v_1 v_2 / c^2} v_3 / c^2}
	\notag\\
	&= \frac{\frac{v_1 + v_2 + v_3 + v_1 v_2 v_3 / c^2}{1 + v_1 v_2 / c^2}}{1 + \frac{v_1 v_3 + v_2 v_3}{1 + v_1 v_2 / c^2} / c^2}
	= \frac{v_1 + v_2 + v_3 + v_1 v_2 v_3 / c^2}{1 + v_1 v_2 / c^2 + (v_1 v_3 + v_2 v_3) / c^2}
	\notag\\
	&= \frac{v_1 + v_2 + v_3 + v_1 v_2 v_3 / c^2}{1 + (v_1 v_2 + v_1 v_3 + v_2 v_3) / c^2}
\end{align}
this last expression is very symmetric, so we might suspect already that it is associative; essentially by repeating these calculations, we obtain that $v_1 +_R (v_2 +_R v_3)$ evaluates to the same expression, so parallel transformations (boosts) admit group structure with group operation being $+_R$




\newpage







	\section{Relativistic Kinematics}
Up until this point, we have mostly dealt with uniform motion. To be able to make realistic physics, however, this is not sufficient because rarely is a motion ever fully uniform. Thus, in order to deal with relativistic dynamics we have to find ways of treating non-uniform, accelerated motion as well.


It seems to be a common misconception that such topics cannot be treated using the tools of special relativity.\footnote{Many authors claim that many people claim this. I have no idea how frequently this actually happens, but it one reason is probably that during the beginnings of relativity, the line between special and general relativity was not so clear (and the nomenclature was not, i.e.~what was meant by the \enquote{special} and \enquote{general} parts). Nowadays, though, the consensus among scientists is that the line is gravity -- acceleration falls into the scope of special relativity.} It should be stressed that, no matter what may be claimed otherwise, special relativity \emph{can} certainly handle acceleration. Without any problems, the tools developed over the course of the last sections is able to visualize world lines of accelerating observers and calculate proper times/lengths for them using \eqref{eq:proper_time_general}. However, all of these calculations are still made in inertial frames (which plays an important role in why the equations are valid in the first place), what about accelerating reference frames? As it turns out, this is a topic that cannot be generalized so easily from the theory dealing with uniform velocities -- but nonetheless, it is possible and we will talk about it briefly.

%However, all of these calculations are still made in inertial frames (which plays an important role in why the equations are valid in the first place), what about accelerating reference frames? As it turns out, this is a topic that cannot be generalized so easily from the theory dealing with uniform velocities. This is because the Lorentz transformation used to convert inertial frames into each other was based on the constancy of $c$ -- but looking back at postulate \ref{post:c_constant}, this constancy is only postulated for \emph{inertial}, i.e.~non-accelerating, frames.


Acceleration is not the only topic that has to be developed in a relativistic manner, though. Many other notions will have to be looked at and reformulated using four-vectors in Minkowski space, for example momentum, energy, force. To start this development, we have to think about velocities at first. Excellent sources for this section are \cite{Fleury_2019, Faraoni_2013}.

	\rem{We elect to formulate everything as general as possible, which means we go from using just ICCs (which are nothing but coordinates for inertial frames) to arbitrary frames of reference. This will require e.g.~using covariant derivates $\nabla$ because the Christoffel symbols do not generally vanish, unlike in inertial frames.}

%Before we can deal with accelerations in more mathematical fashion, however, we have to think about velocities first. Excellent sources for this section are \cite{Fleury_2019, Faraoni_2013}.



		\subsection{Four-Velocity}
Trajectories or world lines are nothing but curves on manifolds, as we have already seen. A natural question, however, was left unanswered until now: what is the velocity of such a trajectory? Since its role is to quantify how fast and in which direction the trajectory is changing, velocity now becomes a tangent vector. In analogy to the previous Newtonian description, we may try
\begin{equation}
	\underline{u} = \dv{t}
	\qquad \Leftrightarrow \qquad
	u^\alpha = \dv{x^\alpha}{t} = (c, \vec{v}) \, .
\end{equation}
Thinking back to the previous sections on clocks and proper time, though, we can immediately see how this definition is flawed: coordinate times $t$ are not invariant, i.e.~the velocity of a world line would change depending on the observer. At the same time, we can immediately come up with a solution: replacing $t$ with the proper time $\tau$ a clock would measure along the world line. This leads to the \Def{four-velocity}
\begin{equation}\label{eq:four_velocity}
	\eqbox{
	\underline{u} = \dv{\tau} = u^\alpha \dv{x^\alpha}
	}
\end{equation}
with components (in arbitrary coordinates $x^\alpha$)
\begin{equation}\label{eq:four_velocity_components}
	\eqbox{
	u^\alpha = \dv{x^\alpha}{\tau} = (c \dv{t}{\tau}, \dv{\vec{x}}{\tau}) = \dv{t}{\tau}~ (c, \dv{\vec{x}}{t}) = \gamma \qty(c, \vec{v})
	} \, ,
\end{equation}
$\vec{v}$ is the \enquote{regular} three-velocity, which finds its way into the equation by a simple application of the chain rule. Clearly, although $u^\alpha$ is a four-vector, it has only three independent components because the first one is constant. An interesting corollary from this constancy is that the temporal component $u^0$ never vanishes, so time never stops (even if there is no spatial motion, $v = 0\; \Leftrightarrow \; u^1 = u^2 = u^3 = 0$).

An important thing to note is that $\underline{u}$ is only defined for timelike world lines because $d\tau = 0$ for lightlike ones (dividing by zero is not defined), while $d\tau$ is complex for spacelike ones (regardless of whether one can make sense of this derivative mathematically, complex velocities do not make sense physically).



Information we get from $\vec{v}$ is the direction of an object (through the unit vector $\vec{v} / \norm{v}$) and how fast (through $v = \norm{v}$). As a tangent vector to certain points $x^\alpha(\tau)$ on the corresponding world line $\Gamma$, $\underline{u}$ naturally contains information about the direction, so what about $\norm{\underline{u}}$?
\begin{equation}
	\eqbox{
	\norm{\underline{u}}^2 = \eta_{\mu \nu} u^\mu u^\nu = \eta_{\mu \nu} \dv{x^\mu}{\tau} \dv{x^\nu}{\tau} = \frac{\eta_{\mu \nu} dx^\mu dx^\nu}{d\tau^2} = c^2 \frac{d\tau^2}{d\tau^2} = c^2
	} \, .
\end{equation}
The four-velocity possesses a \enquote{built-in} normalization, it has a constant magnitude, the speed of light -- no matter at which actual speed $v$ the particle moves (and also, independent of the time $\tau$ we evaluate it at, i.e.~independent of the point in manifold)! Not only does that imply all observers trivially agree on its value, it also points to the special role of $\tau$ as affine parameter $\sigma$ of the world line. Note that the calculation here only works because the Minkowski metric has constant components. Equivalently, we could calculate
\begin{equation*}
	\norm{\underline{u}}^2 = \eta_{\mu \nu} u^\mu u^\nu = \gamma^2 (c^2 - v^2) = c^2 \frac{1 - v^2 / c^2}{1 - v^2 / c^2} = c^2 \, .
\end{equation*}



		\subsection{Four-Acceleration}
Now we can come to acceleration. In principle, treating it does not seem to hard, it should just be a matter of taking one more derivative of $\underline{x}$. Indeed, one can treat acceleration this way -- but we have to realize that we still work in inertial, i.e.~non-accelerated, frames when doing that (which is completely valid, just important to note). To start this subsection, we will also go this route of taking a derivative of $\underline{u}$. For tangent vectors, however, this is not as straightforward as taking derivative of $x^\alpha$, instead one has to use a covariant derivative. If we to that, things work out beautifully and we obtain
\begin{equation}\label{eq:four_acceleration}
	\eqbox{
	\underline{a} = \nabla_{\underline{u}} \underline{u}
	}
	\manyqquad
	\eqbox{
	a^\alpha = (\nabla_{\underline{u}} \underline{u})^\alpha = \dv{u^\alpha}{\tau} + \Gamma^\alpha_{\beta \delta} u^\beta u^\delta
	} \, .
\end{equation}

In inertial frames (i.e.~in ICCs), $\nabla_{\underline{u}} = \dv{\tau}$ and we obtain more explicit versions:
\begin{equation}\label{key}
	\eqbox{
	\underline{a} = \dv{\underline{u}}{\tau} = a^\alpha \dv{x^\alpha}
	}
	%\manyqquad
	\qquad \qquad
	\eqbox{
	a^\alpha = \dv[2]{x^\alpha}{\tau} = \dv{u^\alpha}{\tau} = \gamma \dv{u^\alpha}{t} = \gamma (c \dv{\gamma}{t}, \dv{\gamma}{t} \vec{v} + \gamma \vec{a})
	}
\end{equation}
where $\vec{a} = \dv{v}{t}$ is the familiar three-acceleration (which is sometimes also called \Def{proper acceleration}). One can even further massage this expression:
\begin{align}
	\eqbox{a^\alpha} &= \gamma (c \dv{\gamma}{t}, \dv{\gamma}{t} \vec{v} + \gamma \vec{a}) = \gamma \dv{\gamma}{t} \, (c , \dv{\gamma}{t} \vec{v}) + \gamma^2 \qty(0, \vec{a})
	\notag\\
	&= \gamma \dv{t} \qty(\frac{1}{\sqrt{1 - \vec{v} \cdot \vec{v} / c^2}}) \qty(c , \dv{\gamma}{t} \vec{v}) +  \gamma^2 \qty(0, \vec{a})
	\notag\\
	&= \gamma \dv{t} \qty(1 - \vec{v} \cdot \vec{v} / c^2) \frac{-1}{2} \frac{1}{(1 - \vec{v} \cdot \vec{v} / c^2)^{3 / 2}} \qty(c , \dv{\gamma}{t} \vec{v}) +  \gamma^2 \qty(0, \vec{a})
	\notag\\
	&\eqbox{
	= \frac{\gamma^4}{c^2} \, \vec{v} \cdot \vec{a} \, \qty(c, \vec{v}) + \gamma^2 \qty(0, \vec{a})
	}
\end{align}
Note that $\vec{v} \cdot \vec{a}$ is a scalar product of three-vectors, which means it refers to the one from 3D Euclidean space here, $x^k y_k$.

Essentially, the four-acceleration consists of two terms, the first one being proportional to the four-velocity and the second one being the four-acceleration measured in the rest frame of of the object whose world line we are looking at (whence $\dv{\gamma}{t} = 0$, constant velocity). The second term could have been expected, it is just the regular three-acceleration multiplied with $\qty(dt / d\tau)^2$, which appears due to the use of $\tau$ instead of $t$ (and becomes $1$ in the rest frame of the world line). The first term, however, is new and interesting: it depends on the relative orientation of $\vec{v}, \vec{a}$; it scales with $\gamma^4 = (\gamma^2)^2$ and thus becomes dominant for high velocities (while both are on same scale for small $v$); lastly, it influences the time component and it determines how much the three-velocity $\vec{v}$ adds to the \enquote{pure} three-acceleration $\vec{a}$.\\
%Since $\gamma = \gamma(v)$ does not depend explicitly on $a$ too, this is exactly in line with what the clock postulate \eqref{eq:proper_time_general} told us: times measured by accelerating clocks do not depend explicitly on $a$; that is not to say acceleration has no influence because it certainly does, but only in changing the velocity $v = v(t)$ (that the proper time depends on through $\gamma$).\\
%	\rem{is that true here? Because as we can see clearly from $u^0 = \gamma c$, the change of time really only depends on $v$... And also, these formulas above \emph{do} depend on $a$, clearly. Yes, only related to $v$, but still. So maybe statement is that clock paradox holds at all times and only this shows that is depends on just $v$}




The norm of $\underline{u}$ was a very interesting result, how about the one of $\underline{a}$? For simplicity, we will compute it in ICCs (which is fine because norm is invariant scalar):
\begin{align}\label{eq:norm_four_acc}
	%\norm{\underline{a}} &= a^\mu a_\mu = \gamma^2 \qty(c^2 \qty(\dv{\gamma}{t})^2 - \qty(\dv{\gamma}{t} \vec{v} + \gamma \vec{a})^2)
	%\notag\\
	%&= \gamma^2 \qty(c^2 \qty(\dv{\gamma}{t})^2 - \qty(\dv{\gamma}{t})^2 v^2 - 2 \dv{\gamma}{t} \gamma \vec{v} \cdot \vec{a} - \gamma^2 a^2)
	%\notag
	% V2, using other definition
	\eqbox{\norm{\underline{a}}^2} &= a^\mu a_\mu = \frac{\gamma^8}{c^4} \qty(\vec{v} \cdot \vec{a})^2 c^2 - \qty(\frac{\gamma^4}{c^2} \qty(\vec{v} \cdot \vec{a}) \vec{v} + \gamma^2 \vec{a})^2
	\notag\\
	&= \frac{\gamma^8}{c^4} \qty(\vec{v} \cdot \vec{a})^2 c^2 - \frac{\gamma^8}{c^4} \qty(\vec{v} \cdot \vec{a})^2 v^2 - 2 \frac{\gamma^6}{c^2} \qty(\vec{v} \cdot \vec{a})^2 - \gamma^4 a^2
	\notag\\
	&= \frac{\gamma^8}{c^4} \qty(\vec{v} \cdot \vec{a})^2 \qty(c^2 - v^2) - 2 \frac{\gamma^6}{c^2} \qty(\vec{v} \cdot \vec{a})^2 - \gamma^4 a^2
	\notag\\
	&= \frac{\gamma^8}{c^4} \qty(\vec{v} \cdot \vec{a})^2 \frac{c^2}{\gamma^2} - 2 \frac{\gamma^6}{c^2} \qty(\vec{v} \cdot \vec{a})^2 - \gamma^4 a^2
	\notag\\
	&\eqbox{= - \frac{\gamma^6}{c^2} \qty(\vec{v} \cdot \vec{a})^2 - \gamma^4 a^2} %= \gamma^4 \qty(1 + \gamma^2 \frac{v^2}{c^2}) a^2
%	\notag\\
%	&= \gamma^4 \qty(\frac{1 - v^2 / c^2}{1 - v^2 / c^2} + \frac{v^2 / c^2}{1 - v^2 / c^2}) a^2 \eqbox{= \gamma^6 a^2}
\end{align}
%I think it is not correct to write out $\qty(\vec{v} \cdot \vec{a})^2$ in this way; more thoughts, maybe we can: $\vec{v} \cdot \vec{a} = \vec{v} \cdot \dv{\vec{v}}{t} = \frac{1}{2} \dv{t} (\vec{v} \cdot \vec{v}) = \frac{1}{2} \dv{v^2}{t} = a v$, right (but maybe $\dv{v}{t} \neq a$)?

Two notable special cases exist:
\begin{align}
	\vec{v} &\parallel \vec{a} \qquad \Rightarrow \qquad \vec{v} \cdot \vec{a} = v a \qquad \Rightarrow \qquad \eqbox{\norm{a}^2 = \gamma^6 a^2}
	\label{eq:norm_four_acc_parallel}\\
	\vec{v} &\perp \vec{a} \qquad \Rightarrow \qquad \vec{v} \cdot \vec{a} = 0 \qquad \Rightarrow \qquad \eqbox{\norm{a}^2 = \gamma^4 a^2}
	\label{eq:norm_four_acc_perpendicular}
\end{align}
This amounts to the fact that accelerations of world lines measured in different coordinates are not necessarily equal (just like four-velocities can differ between inertial frames).\\


Another notable property of $\underline{a}$ is its orientation relative to $\underline{u}$ as measured by the Minkowski inner product $\underline{a} \cdot \underline{u} = a^\mu u_\mu$:
\begin{equation}\label{eq:vel_acc_orthogonal}
	\eqbox{
	0 = \dv{c^2}{\tau} = \dv{\tau} u^\mu u_\mu = 2 a^\mu u_\mu
	} \, ,
\end{equation}
the four-acceleration is orthogonal to the four-velocity (note: this does not mean $\vec{a}$ is orthogonal to $\vec{v}$, orthogonality is meant strictly in a Minkowskian way).



\begin{ex}[Uniformly Accelerated World Lines]
	maybe treat it here? Can be done analytically
	
	\url{https://de.wikipedia.org/wiki/Zeitdilatation#Bewegung_mit_konstanter_Beschleunigung}

	see fig.~\ref{fig:accelerated_observers}
\end{ex}



\begin{figure}
	\centering
	
	\begin{tikzpicture}[scale=1.2]
		\spacetimediagram{4}

		\addacceleratedworldline[tstart=-4, tend=4]{0}{0}{0}{0.5}
		\addacceleratedworldline[tstart=-4, tend=4]{0}{0}{0}{-0.5}
		\addacceleratedworldline[tstart=-4, tend=4, color=blue]{0}{0}{0}{0.9}
		\addacceleratedworldline[tstart=-4, tend=4, color=green]{0}{0}{-0.5}{-0.5}

		%\addacceleratedworldline[tstart=-4, tend=4, v=0.5]{0}{0}{0}{1} % Same as blue line where v0=0.5, is this correct? So I guess not using optinoal v argument for now is good idea
	\end{tikzpicture}
	
	\caption[Uniformly accelerated world lines]{Uniformly accelerated world lines.\\
	The $(v_0, a_0)$-values are $(0, \pm 0.5 c / \second)$ in red, $(0, 0.9 c / \second)$ in blue and $(- 0.5 c, -0.5 c / \second)$, choosing one time unit in the diagram to be a second. We can clearly see the hyperbolic shapes of all world lines, which are characteristic for uniform acceleration.}
	\label{fig:accelerated_observers}
\end{figure}



			\paragraph{Accelerated Frames}
It has been mentioned a number of times until now that accelerated frames of reference are harder to deal with than inertial ones. Why is that? To see it, let us think back to how the Lorentz transformation has been derived. One cornerstone was the constancy of $c$ in vacuum, which had strong implications on how inertial frames are to be converted into each other. Looking back at postulate \ref{post:c_constant}, this constancy is only postulated for inertial, i.e.~non-accelerating, frames. Therefore, the Lorentz transformation cannot be used anymore and deriving general transformations is much harder because we have lost this property of $c$ being constant that was so helpful when deriving relativistic quantities. Instead, several alternatives exist.


In case of uniform acceleration, one can utilize \Def{Rindler coordinates} as the coordinates of the corresponding accelerated frame. We will focus on a different idea, though. The line of thought leading to this idea is as follows (where we assume to look at a world line in some inertial, but otherwise arbitrary, frame of reference): at each point $p$ on an accelerated world line $\Gamma$, the object moves with a certain velocity (which does not change in $p$, of course, because it is a point). Therefore, by going into the inertial frame that moves with this velocity relative to the initial one, we find the rest frame of the object at this specific point $p$, the so-called \Def{momentarily comoving inertial frame} (MCIF).\footnote{Other names are I have seen which should refer to the same thing are instantaneously comoving frame or momentarily comoving reference frame (MCRF).} The same procedure can be applied to find the rest frame at every point on the $\Gamma$, which means we can make sense of the notion of a rest frame along $\Gamma$. Moreover, the MCIFs at different points are just inertial frames with different relative velocities, which means we can transform between them using Lorentz transformations. Note the resemblance of this idea to what the clock postulate \ref{post:clock_postulate} stated for the computation of proper times: we replace $v \rightarrow v(t)$, so that what we have learned about uniform motion can be applied. This, however, only works for infinitesimal intervals, which means the replacement only works in a specific point. For the general result, some more work is needed, namely transforming between the points (which is now possible using Lorentz transformations).\footnote{Two very good visualizations of this are provided here: \url{https://en.wikipedia.org/wiki/Spacetime_diagram\#Accelerating_observers}.}


In practice, this whole procedure is still complicated, but at least the idea should now be clear. Furthermore, this procedure being complicated does not completely prevent us from studying accelerated world lines, though. After all, one can still examine their behaviour from inertial frames. Only for some phenomena, it is more convenient to actually analyze the situation from within an accelerated frame.\\


-> on transforming into MCIF: \url{https://math.ucr.edu/home/baez/physics/Relativity/SR/clock.html}


A final note concerns acceleration in such a MCIF. While one works in rest frames here, this rest is only momentarily. Hence, there is still an infinitesimal change in velocity (affecting the neighbouring points), i.e.~acceleration. More precisely,
\begin{equation}\label{eq:four_acceleration_mcif}
	\eqbox{
	a^\alpha = (0, \vec{a})
	} \qquad \Leftrightarrow
	\eqbox{
	\norm{\underline{a}}^2 = \vec{a} \cdot \vec{a} = a^2
	} \, .
\end{equation}
The four-acceleration has only spatial parts here and consequently, its magnitude is equal to the magnitude of the three-acceleration. However, this norm being constant does not mean the components $a^\alpha$ are equal in different MCIFs. Moreover, even if this was the case, coordinate axes of MCIFs at different points will, in general, point into different directions (as seen from another, fixed inertial frame), so constant spatial components in MCIFs would not be equivalent to the three-acceleration pointing into the same direction at all time (as seen from the fixed frame again). Yet another addition to this is that constancy of $\norm{\underline{a}}$ does not necessarily imply uniform acceleration because in a general, perhaps even inertial, frame the magnitude of each component $a^\alpha$ need not be constant (for example, $\underline{a}$ having only spatial components in MCIFs tells us nothing about components in other inertial frames).

-> do we even have constancy of $a$ at different points? I mean at a fixed point it has to be constant in all frames, clearly, but at different points? For non-uniform acceleration, this would not be true, would it?



			\paragraph{Accelerated Clocks}
Having gained more knowledge as to how acceleration works in relativity, it is appropriate to provide some thoughts on the clock postulate \ref{post:clock_postulate} now.\footnote{Many of these thoughts are inspired by articles in the special relativity section of \cite{baez_physics_page}.}


The main statement/takeaway was that even for non-uniform motion $v = v(t)$, the infinitesimal proper time interval still has the form
\begin{equation*}
	d\tau = \sqrt{1 - \frac{v^2}{c^2}} \, dt
\end{equation*}
and involves no additional terms proportional to acceleration (which also implies the speed of light is $c$, at least locally, i.e.~in the corresponding MCIFs). That is not to say acceleration has no effect on a clock's rate at all -- the corresponding changing speeds do change $d\tau$, but it is really only through $v$ in $\gamma(v)$.\footnote{This is despite \eqref{eq:norm_four_acc} having a non-zero time component, which means that the rate at which the clock rate itself changes is not constant. But still, the clock postulate (which is the only relevant statement in this regard) tells us that this does still not lead to an additional contribution from $a$ in $d\tau$.} To incorporate that, integrals over world lines have been introduced, which essentially sum up all the infinitesimal time intervals $d\tau$. An equivalent interpretation using accelerated frames is that one sums up the times going by in each MCIF along the world line, where the speed is momentarily constant such that the formula for uniform motion can be used -- the clock rate on accelerating world line is at all points equal to rate of clock in the corresponding MCIF at each point. Since these frames have to be changed in every point, these times are also the infinitesimal ones $d\tau$ and summing them all up results in the same formula.


The clock postulate applies to more than just the rate of clocks. Proper distances can be measured using integrals over the infinitesimal results obtained for uniform motion as well (which should be clear since $ds$ can be expressed using $d\tau$) and as it turns out, many other quantities involving the Lorentz-factor $\gamma$ can be generalized to $v = v(t)$ by simply using this non-uniform velocity in $\gamma$, for example $\underline{u}$.


However, it must always be remembered that the clock postulate cannot be proven rigorously (and its statement is by no means obvious, everything could very well depend change once $a \neq 0$). One can only \emph{argue} that it makes sense the way it is (like we just did) and there is also experimental evidence supporting it.



%-> note: acceleration itself does not enter when we compute proper distances and times, although they are what causes world lines which are not straight (and thus arbitrary ones; hesitant to say curved world lines because could be confused with world lines in curved space); that is not to say acceleration does not affect them because it certainly does, namely through the corresponding change in velocity (but there is no separate effect that only an acceleration shows)



\begin{ex}[Rotating Disk]
	cool: \url{https://en.wikipedia.org/wiki/Proper_time#Example_2:_The_rotating_disk}

	here, $\vec{v} \perp \vec{a}$ and therefore, no induced effect of acceleration at all (since $v = \text{const.}$)
\end{ex}



\begin{ex}[Twin Paradox 3]
	acceleration \emph{does} effect the rate of clocks!!! The claim we made during twin paradox was just that acceleration is not the reason for different ageing, uniform motion is sufficient for that; but rapid braking in rocket for example will definitely affect time on clock in this rocket -- and analyzing situation from within the rocket is really hard (and thus statements like it runs slower or faster; is much easier to analyze this from inertial frames)

	straightforward (not necessarily easy) way is find parametrization of world line, then take derivative with respect to coordinate time $t$, then do integral (potentially numerically)
\end{ex}



		\subsection{Four-Momentum \& Energy}
Newtonian dynamics in its general form was formulated in terms of momentum, the product of mass and velocity. Continuing the analogous formulation, the \Def{four-momentum} is
\begin{equation}\label{eq:four_momentum}
	\eqbox{
	\underline{p} = m \underline{u}
	}
	\manyqquad
	\eqbox{
	p^\alpha = m u^\alpha = \gamma (m c, m \vec{v}) = (\gamma m c, \vec{p}_r)
	} \, .
\end{equation}
% note that this is tangent vector with four independent component (mentioned on Wikipedia, due to multiplication with Lorentz scalar)
The spatial components can be understood as a \Def{relativistic momentum}
\begin{equation}\label{eq:rel_momentum}
	\eqbox{
	\vec{p}_r = \gamma \vec{p} = \gamma m \vec{v} = \frac{m \vec{v}}{\sqrt{1 - v^2 / c^2}}
	}
\end{equation}
and are clearly just a generalization of the \enquote{usual} Newtonian momentum $\vec{p}$, picking up a factor of $\gamma$ in comparison to the Newtonian definition. As \eqref{eq:taylor_expansion_gamma} tells us, this factor is $\approx 1$ for $v \ll c$, so we retrieve $\vec{p}_r = \vec{p} = m \vec{v}$ in this case.

%-> hmmm, but isn't mass the same for everyone? Because $p^\mu p_\mu = m c^2$ and this is invariant...


Conversely, the time component is harder to interpret. Since whatever it represents is very likely to have a Newtonian counterpart, let us look at $p^0$ in case $v \ll c$. Continuing \eqref{eq:taylor_expansion_gamma} shows
\begin{align}\label{eq:taylor_expansion_gamma_2}
	\eqbox{\frac{1}{\sqrt{1 - \frac{v^2}{c^2}}}} &\simeq \eval{\frac{1}{\sqrt{1 - \frac{v^2}{c^2}}}}_{v = 0} + \eval{\frac{\frac{1}{2} \cdot \frac{-2 v}{c}}{(1 - \frac{v^2}{c^2})^{3/2}}}_{v / c = 0} \frac{v}{c} + \frac{1}{2} \eval{\dv{v / c} \qty(\frac{\frac{1}{2} \cdot \frac{-2 v}{c}}{(1 - \frac{v^2}{c^2})^{3/2}})}_{v / c = 0} \frac{v^2}{c^2} + \dots
	\notag\\
	&= \eval{\frac{1}{\sqrt{1 - \frac{v^2}{c^2}}}}_{v = 0} + \eval{\frac{\frac{1}{2} \cdot \frac{-2 v}{c}}{(1 - \frac{v^2}{c^2})^{3/2}}}_{v = 0} \frac{v}{c} + \frac{1}{2} \eval{\qty(\frac{\qty(\frac{-1}{2} \cdot \frac{-2 v}{c})^2}{(1 - \frac{v^2}{c^2})^{5/2}} + \frac{\frac{-1}{2} \cdot (-2)}{(1 - \frac{v^2}{c^2})^{3/2}})}_{v / c = 0} \frac{v^2}{c^2} + \dots
	\notag\\
	&\eqbox{= 1 + \frac{v^2}{2 c^2} + \mathcal{O}\qty(\frac{v^3}{c^3})} \, .
\end{align}
and thus
\begin{equation*}
	c p^0 = \gamma m c^2 \approx m c^2 + \frac{m v^2}{2} \, .
\end{equation*}
The first term is a constant with dimension of an energy that we do not know yet, but the second term looks familiar: it is kinetic energy. For this reason, we identify
\begin{equation}\label{eq:rel_energy}
	\eqbox{
	p^0 = \frac{E}{c}
	}
	\manyqquad
	\eqbox{
	E = \gamma m c^2 = \frac{m c^2}{\sqrt{1 - v^2 / c^2}}
	} \, ,
\end{equation}
$E$ being the \Def{relativistic energy} of a free particle. Apparently, the kinetic energy (which is the energy of a free particle) has more contributions than just the familiar one $\sim v^2$. Since the terms of higher order in $v$ only become relevant for velocities comparable to $c$, however, it is no surprise that they are discovered by special relativity and not known from Newtonian mechanics. A rather interesting consequence arises for the other velocity limit $v = 0$ whence
\begin{equation}\label{eq:e=mc^2}
	\eqbox{
	E = m c^2
	} \, .
\end{equation}
Here we have the most famous formula of physics (remember I mentioned second most before; we have now finally completed quest for most famous one). Just like many insights from relativity, this formula has profound implications: it tells us that mass and energy are equivalent (strictly speaking, we can see that from \eqref{eq:rel_energy} already, but there we might still suspect it comes from kinetic part). Even particles at rest still have energy, namely an amount proportional to what is often called \Def{rest mass} $m$ in this context.\footnote{Because sometimes, a \Def{relativistic mass} $m(v) = \gamma m = \frac{m}{\sqrt{1 - v^2 / c^2}}$ is defined. We will not use this definition.} This \Def{rest energy} scales with $c^2$ and therefore, even very small masses contain huge amounts of energy (which shows e.g.~in the fact that we gain energy from nuclei nowadays). This motivates defining the \Def{relativistic kinetic energy} as
\begin{equation}\label{eq:rel_kin_energy}
	\eqbox{
	T = E - m c^2 = (\gamma - 1) \, m c^2
	} \, .
\end{equation}


Equivalently, since
\begin{equation}\label{eq:four_momentum_norm}
	\eqbox{
	m^2 c^2 = m^2 u^\mu u_\mu = p^\mu p_\mu = (p^0)^2 - \vec{p}^2 = E^2 / c^2 - \vec{p}_r^2
	}
\end{equation}
we can also write
\begin{equation}\label{eq:rel_energy_2}
	\eqbox{
	E = \sqrt{(m c^2)^2 + c^2 \vec{p}_r^2}
	} \, .
\end{equation}



One major advantage of using momenta as tangent vectors instead of velocities is that $p^\alpha$ can be defined for timelike \emph{and} null vectors. A priori,
\begin{equation*}
	p^\alpha = m u^\alpha = m \dv{x^\alpha}{d\tau} = \frac{0}{0}
\end{equation*}
since $d\tau = 0$ for photons and they have no mass, i.e.~$m = 0$. Apparently, we gain no insight from this definition. However, we do know that photons have energy
\begin{equation}
	\eqbox{
	E = h f
	} \, ,
\end{equation}
so
\begin{equation*}
	0 = p^\mu p_\mu = \frac{h^2 f^2}{c^2} - p_r^2 \, .
\end{equation*}
Therefore, we can associate the spatial components of photon's momentum with a three-vector of magnitude
\begin{equation}\label{eq:photon_momentum}
	\eqbox{
	p_r = \hbar k = \frac{h}{\lambda} = \frac{h f}{c} = \frac{E}{c}
	}
\end{equation}
pointing into the direction of some unit vector $\vec{e}_k$, which is the direction that the photon propagates into. This vector $\vec{k} = k \vec{e}_k$ is called \Def{wave vector}.



%redshift leads to higher energy, but frequency also changes such that all in all speed of light remains constant (might be wrong this way, but heard something like that)



		\subsection{Four-Force \& Equation of Motion}
Our goal in this section will be to find a way to treat dynamics in a (special-)relativistic way, which is equivalent to finding the equation of motion. As it turns out, there is not much work to do. Despite some aspects of Newton's theory being limited in their scope, it is not entirely wrong and nothing we have found out until now speaks against continuing to use Newton's second law (in an appropriate four-vector version).

One difference in notation will be the use of coordinate-independent language, i.e.~covariant derivative over \enquote{regular}, just like it has been used to define the four-acceleration, but the interpretation and idea remains the same. Equating this derivative of $\underline{P}$ with the \Def{four-force} $\underline{F}$ yields the following equation of motion:
\begin{equation}\label{eq:eq_of_motion}
	\eqbox{
	\underline{F} = \nabla_{\dv{\tau}} \underline{p} = \nabla_{\underline{u}} \underline{p} = \frac{1}{m} \nabla_{\underline{p}} \underline{p}
	}
	\manyqquad
	\eqbox{
	F^\alpha = (\nabla_{\underline{u}} \underline{p})^\alpha = \dv{p^\alpha}{\tau} + \Gamma^\alpha_{\beta \delta} u^\beta p^\delta
	}
\end{equation}

Using the product rule for $\nabla$, we can write that out a little further to obtain
\begin{equation}\label{eq:eq_of_motion_2}
	\eqbox{
	\underline{F}% = (\nabla_{\underline{u}} m) \underline{u} + m \nabla_{\underline{u}} \, \underline{u} 
	= \dv{m}{\tau} \underline{u} + m \nabla_{\underline{u}} \underline{u}
	}
	\manyqquad
	\eqbox{
	F^\alpha = \dv{m}{\tau} u^\alpha + m \dv{u^\alpha}{\tau} + \Gamma^\alpha_{\beta \delta} u^\beta p^\delta
	}
\end{equation}
where $\nabla_{\underline{u}} m = \dv{m}{\tau}$ since the mass is a scalar function. Therefore, we can write the components of $\underline{F}$ as (dropping the assumption $m = \text{const.}$)
\begin{equation}\label{eq:four_force}
	\eqbox{
	\underline{F} = (\frac{1}{c} \dv{E}{\tau}, \dv{\vec{p}_r}{\tau}) = \gamma (\frac{1}{c} \dv{E}{t}, \dv{\gamma}{t} \vec{p} + \gamma \vec{F}) = (F^0, \gamma \vec{F}_r)
	}
\end{equation}
with
\begin{equation}\label{eq:rel_four_force}
	\eqbox{
	\vec{F}_r = \dv{\vec{p}_r}{t}
	}
	\manyqquad
	\eqbox{
	\vec{F} = \dv{\vec{p}}{t}
	}
\end{equation}
being the relativistic and non-relativistic three-forces, respectively and $F^0$ being the a \enquote{force in time} or, perhaps more illustrative, a change of energy $P^0$ over time, i.e.~the \Def{power} of the force. This power is not independent of the other, spatial force components as a quick calculation in shows (assuming ICCs for simplicity):
\begin{align}
	F^\mu u_\mu &= \dv{m}{\tau} u^\mu u_\mu + m \, a^\mu u_\mu \underset{\eqref{eq:vel_acc_orthogonal}}{=} \dv{m}{\tau} c^2 = \gamma c^2 \dv{m}{t}
	\notag\\
	%&= F^0 u_0 - \dv{\gamma}{t} \vec{p} \cdot \gamma \vec{v} - \gamma \vec{F} \cdot \gamma \vec{v}
	%notag\\
	%&= \gamma^2 \dv{E}{t} - \dv{\gamma}{t} m v^2 - \gamma \qty(\dv{m}{t} \vec{p} \cdot \vec{v} + m \vec{a} \cdot \vec{v})
	&= F^0 u_0 - \gamma \dv{\vec{p}_r}{\tau} = \gamma^2 \dv{E}{t} - \gamma^2 \dv{\vec{p}_r}{t}
	\notag\\
	\Leftrightarrow \qquad &
	\eqbox{
	F^0 = \frac{\gamma}{c} \dv{E}{t} = \gamma c \dv{m}{t} + \frac{\gamma}{c} \dv{\vec{p}_r}{t} \cdot \vec{v}
	%\underset{m = \text{const.}}{=} \gamma \dv{(\gamma \vec{p})}{t} \cdot \vec{v} % Rather put to where m = const is treated
	}
\end{align}
Therefore, any change in energy corresponds either to a change in mass or to the (Euclidean) scalar product between what is often interpreted as a relativistic three-force in $\dv{(\gamma \vec{p})}{t}$ and the velocity $\vec{v}$, i.e.~work done on the system.\\


For constant mass $m$, \eqref{eq:eq_of_motion_2} simplifies to
\begin{equation}\label{eq:eq_of_motion_constant_mass}
	\eqbox{
	\underline{F} = m \nabla_{\underline{u}} \underline{u} = m \underline{a}
	}
	\manyqquad
	\eqbox{
	F^\alpha = m \dv{u^\alpha}{\tau} + \Gamma^\alpha_{\beta \delta} u^\beta p^\delta = m \, a^\alpha
	} \, ,
\end{equation}
which clearly resembles Newton's second law \eqref{eq:newton_2}. In fact, it reduces to this exact equation if we assume ICCs, where $\nabla_{\dv{\tau}} = \dv{\tau}$ and thus
\begin{equation*}
	\eqbox{
	\underline{F} = m \dv{\underline{u}}{\tau}
	}
	\manyqquad
	\eqbox{
	F^\alpha = m \dv{u^\alpha}{\tau}
	} \, .
\end{equation*}
Split up into components, that means
\begin{equation}\label{eq:four_force_iccs_m_const}
	\eqbox{
	\frac{1}{c} \dv{E}{t} = \dv{\vec{p}_r}{t} \cdot \vec{v}
	}
	\manyqquad
	\eqbox{
	\vec{F} = m \, \vec{a}
	} \, .
\end{equation}
Both of these equations are known from Newtonian mechanics as well.\\



In the force-free case, i.e.~for a free particle, \eqref{eq:eq_of_motion} becomes
\begin{equation}\label{eq:geodesic_eq}
	\eqbox{
	\nabla_{\underline{p}} \underline{p} = 0
	}
	\manyqquad
	\eqbox{
	u^\alpha \nabla_\alpha p^\beta = u^\alpha (\nabla_\alpha \underline{p})^\beta = 0
	} \, .
\end{equation}
This is the \Def{geodesic equation}. It expresses that $U^\beta$ remains constant as long as we take the derivative along the curve that it is tangent to (which explains the $u^\alpha \nabla_\alpha$ part). Consequently, the world lines of freely moving test particles are \Def{geodesics}.


For a free particle with constant mass $m$, we can further simplify
\begin{equation}\label{eq:geodesic_eq_simplified}
	\eqbox{
	\nabla_{\underline{u}} \underline{u} = 0
	}
	\manyqquad
	\eqbox{
	u^\alpha \nabla_\alpha u^\beta = u^\alpha (\nabla_\alpha \underline{u})^\beta = 0
	} \, .
\end{equation}


Coordinate versions of equations \eqref{eq:eq_of_motion}, \eqref{eq:eq_of_motion_constant_mass} are
\begin{equation}\label{eq:geodesic_eq_coordinates}
	\eqbox{
	F^\alpha = \dv{m}{\tau} \dv{x^\alpha}{\tau} + m \dv[2]{x^\alpha}{\tau} + m \, \Gamma_{\beta \delta}^\alpha \dv{x^\beta}{\tau} \dv{x^\delta}{\tau}
	}
	%\manyqquad
	\qquad \qquad
	\eqbox{
	F^\alpha = \dv[2]{x^\alpha}{\tau} + \Gamma_{\beta \delta}^\alpha \dv{x^\beta}{\tau} \dv{x^\delta}{\tau}
	}
\end{equation}
from which the one for the geodesic equation follow by setting $F^\alpha = 0$.

\begin{proof}
We can get to \eqref{eq:geodesic_eq_coordinates} via a straightforward calculation to express $\nabla_{\underline{u}} \underline{u}$ in coordinates. In fact, one can prove this for an arbitrary tangent vector $\underline{t}$ along $\Gamma$, which is parametrized by an arbitrary parameter $\sigma$ and not necessarily the proper time $\tau$.
\begin{align*}
	t^\alpha \nabla_\alpha t^\beta &= \dv{x^\alpha}{\sigma} \nabla_\alpha \dv{x^\beta}{\sigma}
	\\
	&= \dv{x^\alpha}{\sigma} \qty(\pdv{x^\alpha} \dv{x^\beta}{\sigma} + \Gamma_{\alpha \delta}^\beta \dv{x^\delta}{\sigma})
	\\
	&= \dv{x^\alpha}{\sigma} \pdv{x^\alpha} \dv{x^\beta}{\sigma} + \Gamma_{\alpha \delta}^\beta \dv{x^\alpha}{\sigma} \dv{x^\delta}{\sigma}
	\\
	\underset{\text{chain rule}}&{=} \dv{\sigma} \dv{x^\beta}{\sigma} + \Gamma_{\alpha \delta}^\beta \dv{x^\alpha}{\sigma} \dv{x^\delta}{\sigma}
	\\
	&= \dv[2]{x^\beta}{\sigma} + \Gamma_{\alpha \delta}^\beta \dv{x^\alpha}{\sigma} \dv{x^\delta}{\sigma} \qedhere
\end{align*}
\end{proof}



Faraoni has interesting stuff regarding momentum conservation, relates $F^\alpha u_\alpha$ to $\dv{m}{\tau}$


interesting take, \enquote{forces are not convenient}: \url{https://phys.libretexts.org/Bookshelves/University_Physics/Book%3A_Mechanics_and_Relativity_(Idema)/15%3A_Relativistic_Forces_and_Waves/15.01%3A_The_Force_Four-Vector}; is this because of way it transforms?

maybe something similar discussed here: \url{https://de.wikipedia.org/wiki/Beschleunigung_(spezielle_Relativit%C3%A4tstheorie)#Beschleunigung_und_Kraft}



\begin{ex}[Rotating Frame 2]
	Christoffel symbols are zero in ICCs (more generally: inertial frames -> right?), but not necessarily in other coordinates; in non-inertial frames, they lead to extra terms, the fictitious forces; for example, Coriolis and centripetal come out of there very naturally
\end{ex}



		\subsection{Lagrangian Approach}
review Lagrangian approach here

say that geodesic equation (which is equation of motion for free particle) can be derived by extremizing proper time elapsed along world line as well (minimizing or maximizing depends on sign of metric we choose, right? Ah no, should always be minimizing proper time, but how this translates into requirement for metric is different)


we have dependence on world line of our distance measure; this is nothing unusual in mathematical theory of metric spaces, but it raises an important physical question: what is the preferred trajectory of particles, i.e.~what is the time that usually elapses for them? Turns out that it is extremal proper time (minimal for us, depends on sign convention of metric, right?), which yields straight lines


relativistic Lagrangian is
\begin{equation}
	\eqbox{
	L = - (m c^2 + U) \sqrt{1 - \frac{v^2}{c^2}}
	}
\end{equation}
with potential energy $U$; we get that from (see Fleury 3.66) action, which is naturally defined as
\begin{equation}
	\eqbox{
	S = \int L \, d\tau = - \int m c^2 + U \, d\tau = - \int (m c^2 + U) \sqrt{1 - \frac{v^2}{c^2}} \, dt
	}
\end{equation}



\newpage



	\section{Relativistic Electrodynamics}
now we come to the key reason why relativity as formulated by Einstein exists in the first place, electrodynamics



\end{document}