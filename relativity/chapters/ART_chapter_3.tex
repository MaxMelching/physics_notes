%\documentclass[ART_main.tex]{subfiles}
\documentclass[DIV=11, BCOR=0mm, paper=a4, fontsize=11pt, parskip=half, twoside=false, titlepage=true]{scrartcl}

\usepackage{subfiles}


\usepackage[singlespacing]{setspace} 
\usepackage{lastpage}
\usepackage[automark, headsepline]{scrlayer-scrpage}
\clearscrheadings
\setlength{\headheight}{\baselineskip}
\automark{section} % mit [] wird Argument in [] für links, {} rechts genommen
\automark*{subsection} % write section in footline instead of chapter (if there is one)
%\automark*{subsection}
\ihead{\headmark}
%\ohead[]{Seite~\thepage}
\cfoot{{\hypersetup{linkcolor=black}Page~\thepage~of~\pageref{LastPage}}}

\usepackage[utf8]{inputenc}
\usepackage[ngerman, english]{babel}
\usepackage[expansion=true, protrusion=true]{microtype}
\usepackage{amsmath}
\usepackage{amsfonts}
\usepackage{amsthm}
\usepackage{amssymb}
\usepackage{mathtools}
\usepackage{mathdots}
\usepackage{upgreek}
\usepackage[free-standing-units]{siunitx}
\usepackage{esvect}
\usepackage{graphicx}
\usepackage{epstopdf}
\usepackage[hypcap]{caption}
\usepackage{booktabs}
\usepackage{flafter}
\usepackage[section]{placeins}
\usepackage{pdfpages}
\usepackage{textcomp}
\usepackage{subfig}
\usepackage{floatpag} % to have clear pages where figures are
\usepackage[italicdiff]{physics}
\usepackage{xparse}
\usepackage{wrapfig}
\usepackage{color}
\usepackage{xcolor}
\usepackage{colortbl}
\usepackage{multirow}
\usepackage{array} % needed to define fancy table cells
\usepackage{diagbox} % needed for double colored table cells
\usepackage{dsfont}
\numberwithin{equation}{section}
\numberwithin{figure}{section}
\numberwithin{table}{section}
\usepackage{empheq}
\usepackage{tikz}
\usepackage{tikz-cd}%für Kommutationsdiagramme
\usepackage{forest}%Baumdiagramme
\usepackage{mdframed}

\usepackage{hyperref}
\hypersetup{colorlinks=true, breaklinks=true, citecolor=linkblue, linkcolor=linkblue, menucolor=linkblue, urlcolor=linkblue} %sonst z.B. orange bei linkcolor

\usepackage{imakeidx}%für Erstellen des Index
\usepackage{xifthen}%damit bei \Def{} das Index-Arugment optional gemacht werden kann

\usepackage[printonlyused]{acronym}%withpage -> seems useless here

\usepackage{enumerate} % for custom enumerators

\usepackage{listings} % to input code

\usepackage{csquotes} % to change quotation marks all at once

%\usepackage[nottoc, notlot, notlof, chapter]{tocbibind} %macht automatisch ins TOC, auch index und andere Sachen; so ungenummert, es geht aber auch mit Option numbib -> nicht nötig jetzt

%\usepackage[maxcitenames=3, backend=biber]{biblatex}%vlt hätte maxnames=2 gepasst


%man muss wohl Pakete mit Matheschrift zuerst laden
%\usepackage{mathpazo}%hä lol, das stellt überall pagella ein, erlaubt aber noch Modifikation?! Besser als pagella einzeln laden sogar -> ah, man kann aber z.B. noch Times auch einstellen hinterher; sieht jetzt aber nicht unbedingt überragend aus, Times da mein Favorit
%\usepackage{euler} %macht Fehler und sieht nichtmal so nice aus

%Versuch nur in Mathe Modus anzumachen, geht wohl in pdflatex nicht
%\usepackage{xfrac,unicode-math}
%\defaultfontfeatures{Scale=MatchLowercase}
%\setmathfont{TeX Gyre Termes Math}{version=termes}
%\setmathfont{TeX Gyre Pagella Math}{version=pagella}

% Versuch zwei -> nope, man braucht wohl XeLatex
%\usepackage{fontenc,xunicode}
%\setmathrm{Optima}

% Version 3
\usepackage{newtxmath} %geil, macht Times an in Mathe (ist stark, wenn auch zu dick bei Nutzen von Standard Computer Modern); muss auf jeden Fall rein bei Schrift Times, sonst sieht das im Vergleich viel zu dünn aus (auch bei pagella eigentlich)
%newtxtext funktioniert nicht, aber dafür ist ja auch tgtermes da

%\usepackage{tgtermes}
%\usepackage{cmbright}%ihhhhhhhh
\usepackage{tgpagella}
\setkomafont{section}{\rmfamily\Large\bfseries}
\setkomafont{sectionentry}{\large\bfseries}
\setkomafont{subsection}{\rmfamily\large\scshape}%textsc%\textsl auch not bad
\setkomafont{title}{\bfseries}%von pagella ein
\setkomafont{subtitle}{\Large\scshape}
\setkomafont{author}{\Large\slshape}
%\setkomafont{date}{\Large\slshape}
\setkomafont{pagehead}{\scshape}
\setkomafont{pagefoot}{\slshape}
\setkomafont{captionlabel}{\bfseries}
%\mathversion{qpl}



\definecolor{mygreen}{rgb}{0.8,1.00,0.8}
\definecolor{mycyan}{rgb}{0.76,1.00,1.00}
\definecolor{myyellow}{rgb}{1.00,1.00,0.76}
\definecolor{defcolor}{rgb}{0.10,0.00,0.60} %{1.00,0.49,0.00}%orange %{0.10,0.00,0.60}%aquamarin %{0.16,0.00,0.50}%persian indigo %{0.33,0.20,1.00}%midnight blue
\definecolor{linkblue}{rgb}{0.00,0.00,1.00}%{0.10,0.00,0.60}


% auch gut: green!42, cyan!42, yellow!24

%Syntax Farbboxen: in normalem Text \colorbox{mygreen}{Text} oder bei Anmerkungen in Boxen \fcolorbox{black}{myyellow}{Rest der Box}, in Mathe-Umgebung für farbige Box \begin{empheq}[box = \colorbox{mycyan}]{align}\label{eq:} Formel \end{empheq} oder farbigen Rand \begin{empheq}[box = \fcolorbox{mycyan}{white}]{align}\label{eq:} Formel \end{empheq}

\setlength{\fboxrule}{0.76pt}
\setlength{\fboxsep}{1.76pt}

\newcommand{\anm}[1]{\fcolorbox{black}{yellow!24}{\parbox[c]{0.985\textwidth}{\textbf{Anmerkung}: #1}}}

%\newcommand{\anm}[1]{\footnote{#1}}

\newcommand{\anmind}[1]{\fcolorbox{black}{yellow!24}{\parbox[c]{0.92 \textwidth}{\textbf{Anmerkung}: #1}}}
% wegen Einrückung in itemize-Umgebungen o.Ä.

\newcommand{\eqbox}{\fcolorbox{white}{cyan!24}}

\newcommand{\textbox}[1]{\fcolorbox{white}{cyan!24}{#1}}


\newcommand{\Def}[2][]{\textcolor{defcolor}{\fontfamily{ptm}\selectfont \textit{#2}}\ifthenelse{\isempty{#1}}{\index{#2}}{\index{#1}}}%{\colorbox{green!0}{\textit{#1}}}
% zwischendurch Test mit \textbf{#1} noch (wurde aber viel größer)

% habe jetzt Schrift (font) pagella reingehauen, ist mega

% wenn Farbe doch doof, einfach beide auf white :D




\mdfdefinestyle{defistyle}{topline=false, rightline=false, linewidth=1pt, frametitlebackgroundcolor=gray!12}

\mdfdefinestyle{satzstyle}{topline=true, rightline=true, leftline=true, bottomline=true, linewidth=1pt}

\mdfdefinestyle{bspstyle}{%
rightline=false,leftline=false,topline=false,%bottomline=false,%
backgroundcolor=gray!8}% tried imitation of spruce from beamer with black!20!white


\mdtheorem[style=defistyle]{defi}{Definition}[section]
\mdtheorem[style=satzstyle]{thm}[defi]{Theorem}
\mdtheorem[style=satzstyle]{lem}[defi]{Lemma}
\mdtheorem[style=satzstyle]{cor}[defi]{Corollary}
\mdtheorem[style=satzstyle]{prop}[defi]{Property}
\mdtheorem[style=bspstyle]{ex}[defi]{Example}
% just have one, Property, instead of Theorem, Lemma, Corollary?


\newtheoremstyle{rem}
  {\topsep}{\topsep}
  {}{}%{\centering}{0.1\textwidth}
  {\bfseries}{\textbf{remark}:}
  { }{}
\theoremstyle{rem}
% might be unnecessary now

\mdfdefinestyle{remstyle}{%
rightline=false,leftline=false,topline=false,bottomline=false,%
backgroundcolor=myyellow,innerleftmargin=.4\baselineskip,innerrightmargin=.4\baselineskip,leftmargin=-.4\baselineskip,rightmargin=-.4\baselineskip}%setting the indentations is important because otherwise, everything will be indented (.4\baselineskip is default and looks natural, so this is chosen; the effects of margin and innermargin have to be balanced)
%,frametitle={\textbf{remark}: }}%frametitle also makes linebreak

\newmdenv[style=remstyle]{remark}%{remark}
%\newmdtheoremenv[style=remstyle]{rem}{remark}
%\mdtheorem[style=remstyle]{rem}{remark:}%allows use of \begin{rem*} for no numbering

%\newcommand{remark}[1]{\begin{rem*}: #1\end{rem*}}
%use of begin, end is not allowed before \begin{document}


%Lösung (also Umgehen von Verbot \begin{} in Präambel) kommt von: https://www.mrunix.de/forums/showthread.php?59532-begin-und-end-in-newcommand
\def\brem#1\erem{\begin{remark}#1\end{remark}}
\newcommand{\rem}[1]{\brem \textbf{remark:} #1\erem}
% finally, now \rem{} is a shortcut for \begin{remark} etc.

% new line not always wanted for remarks, thus change to this here
\usepackage{soul}
\sethlcolor{myyellow}
\newcommand{\question}[1]{\hl{#1}}


% Anpassung von itemize-Symbolen
\renewcommand{\labelitemi}{$\blacktriangleright$}%{$\vartriangleright$}
\renewcommand{\labelitemii}{\textbf{--}} % is also default there
\renewcommand{\labelitemiii}{$\bullet$}


% Shortcuts -> falls man Abkürzung mal ändern will, muss man dann nicht den ganzen Text durchgehen
\usepackage{xspace} %weil man sonst \gw{} callen muss, damit Leerzeichen danach erkannt werden.
\newcommand{\gw}{{\hypersetup{linkcolor=black}\ac{gw}}\xspace}
\newcommand{\gws}{{\hypersetup{linkcolor=black}\acp{gw}}\xspace}

\newcommand{\mi}{{\hypersetup{linkcolor=black}\ac{mi}}\xspace}

\newcommand{\art}{{\hypersetup{linkcolor=black}\ac{art}}\xspace}

% wenn was nicht klappt, dann \gw{} callen
% mit diesem Ding leider kene Nutzung in Überschriften möglich

%\newcommand{\Var}{{\fontfamily{ptm}\selectfont\text{var}}}
%\newcommand{\Cov}{{\fontfamily{ptm}\selectfont\text{cov}}}
%\newcommand{\Corr}{{\fontfamily{ptm}\selectfont\text{corr}}}

% this is better, auto-select fonts etc
\DeclareMathOperator{\Var}{var}
\DeclareMathOperator{\Cov}{cov}
\DeclareMathOperator{\Corr}{corr}


%\renewcommand{\bibname}{References}
\addto\captionsenglish{\renewcommand{\bibname}{References}}



% if float is too long use \thisfloatpagestyle{onlyheader}
\newpairofpagestyles{onlyheader}{%
\setlength{\headheight}{\baselineskip}
\automark[section]{section}
%\automark*[section]{subsection}
\ihead[]{\headmark}
%
% only change to previous settings is here
\cfoot{}
}


\newpairofpagestyles{onlyfooter}{%
\setlength{\headheight}{\baselineskip}
\automark[section]{section}
%\automark*[section]{subsection}
\ihead[]{}
%
% only change to previous settings is here
\cfoot{{\hypersetup{linkcolor=black}Page~\thepage~of~\pageref{LastPage}}}
}



% for dartboard (from https://de.overleaf.com/latex/templates/dartboard/bhpfmdvjsjmk)
\tikzstyle{wired}=[draw=gray!30, line width=0.15mm]
\tikzstyle{number}=[anchor=center, color=white]
%%%<
\usepackage{verbatim}
%%%>
\begin{comment}
:Title: Dartboard
:Tags: Foreach; Node positioning
:Author: Roberto Bonvallet
:Slug: dartboard
\end{comment}

% Sectors are numbered 0-19 counterclockwise from the top.

% \strip{color}{sector}{outer_radius}{inner_radius}
\newcommand{\strip}[4]{
    \filldraw[#1, wired]
      ({18 *  #2}      :                   #3) arc
      ({18 *  #2}      : {18 * (#2 + 1)} : #3) --
      ({18 * (#2 + 1)} :                   #4) arc
      ({18 * (#2 + 1)} : {18 *  #2}      : #4) -- cycle;
}

% \sector{color}{sector}{radius}
\newcommand{\sector}[3]{
    \filldraw[#1, wired]
      (0, 0) --
      ({18 * #2} :                   #3) arc
      ({18 * #2} : {18 * (#2 + 1)} : #3) -- cycle;
}


\begin{document}

\chapter{General Relativity}

	\section{Gravitational Physics Summary -- Physics Part}

\rem{we use units of $c = G = 1$ (\Def{geometric units})}

\rem{we adopt the Einstein summation convention where repeated indices are summed over, that is abbreviating $\sum_\mu x^\mu y_\mu = x^\mu y_\mu$}

		\subsection{Newtonian Gravity}
Newtonian gravity can be captured by his famous formula
\begin{equation}
\eqbox{
F_g = - \frac{m_1 m_2}{r^2}
}
\end{equation}
which describes the gravitational force that an object with mass $m_1$ exerts onto another object with mass $m_2$ (from Newton's second law, we know that the same force is exerted from the second object onto the first)


this can also be brought into the form
\begin{equation}\label{eq:newton_potential}
F_g = m_2 \dv{r}\qty(\frac{m_1}{r}) = - m_2 \dv{\Phi_g}{r}
\end{equation}
which tells us that gravitation is a conservative force with potential $\Phi_g = - \frac{m_1}{r}$ (produced by some object with mass $m_1$). This is because it only has a radial and no angular component (thinking in polar/spherical coordinates) such that any derivative with respect to angular coordinates such as $\pdv{\Phi_g}{\theta}$ vanishes. Thus, \eqref{eq:newton_potential} is equivalent to the more general condition for conservative forces,
\begin{equation}\label{eq:cons_force}
\eqbox{
\vec{F} = - \vec{\nabla} \Phi
}
\manyqquad
F^k = - \delta^{k l} \pdv{\Phi}{x^l} \, .
\end{equation}
as a consequence, knowing the potential is sufficient to know how gravity acts; for a point particle, we can determine it from $\Delta \Phi = \nabla^2 \Phi = 0$ and for a continuous mass distribution $\rho\qty(\vec{x})$ from
\begin{equation}
\eqbox{
\Delta \Phi\qty(\vec{x}) = 4 \pi \rho\qty(\vec{x})
}
\manyqquad
\delta^{ij} \pdv[2]{\Phi(\vec{x})}{x^i}{x^j} = 4 \phi \rho\qty(\vec{x})
\end{equation}

on this note, it is important to mention that instead forces $\vec{F}$ are also equivalent to accelerations $\vec{a}$ (due to Newton's second law $\vec{F} = m \vec{a}$) and momenta $\vec{p}$ (due to $\vec{F} = \dv{\vec{p}}{\tau}$)


\begin{ex}[Gravity on Earth]
The gravity exerted by Earth on objects with mass $m$ (assuming they stand on Earth's surface for now) is
\begin{equation}
F_g = - m \frac{m_e}{r_e^2} = - m g
\end{equation}
Comparing that with Newton's second formula, $F = m a$, we see that such an object experiences an acceleration
\begin{equation}
a = - g = - 9.81 \frac{\metre}{\second^2} = - 1.1 \cdot 10^{-16} \frac{1}{\metre} \, .
\end{equation}

To see how much potential energy is needed to lift objects of mass $m$ to a height $h \ll r_e$ above Earth's surface, we can do a Taylor expansion around $h = 0$:
\begin{align*}
\Phi_g = - \frac{m_e}{r_e + h} &\simeq - \eval{\frac{m_e}{r_e + h}}_{h = 0} + h \eval{\dv{h}\qty(- \frac{m_e}{r_e + h})}_{h = 0} + \mathcal{O}(h^2)
\\
&= - \frac{m_e}{r_e} + h \eval{\frac{m_e}{(r_e + h)^2}}_{h = 0} + \mathcal{O}(h^2)
\\
&= - \frac{m_e}{r_e} + h \frac{m_e}{r_e^2} + \mathcal{O}(h^2) = - \frac{m_e}{r_e} + h g + \mathcal{O}(h^2)
\end{align*}
%From that, we obtain the gravitational potential energy at $h$ to first order as the difference
However, the first contribution is nothing but the energy at Earth's surface. The potential energy that at $h$ and thus the energy which is needed to lift an object of mass $m$ to this height $h$ (which is what one is interested in most of the time) is given to first order by the difference
\begin{equation}
\Phi_g =  - \frac{m_e}{r_e} + g h - - \frac{m_e}{r_e} = g h \, .
\end{equation}
This corresponds to gauging our measurements such that Earth's surface is the value with zero potential energy.
\end{ex}


problem: the Newtonian description of gravity as a force is not consistent with relativity because it acts instantaneously



turns out that gravitational mass and inertial mass are equal


\begin{ex}[Gravitational Redshift]
somehow we could build perpetual motion machine is the argument, don't get it

I rather think about it like this (should be equivalent): SR tells us that photons have a certain mass $m = \frac{E}{c^2}$; therefore, it is also affected by a gravitational potential and to move against gravity, some of its energy has to be converted; that corresponds to a change in frequency, (since $f = \frac{E}{h}$):
\begin{equation}
\frac{f_\text{top}}{f_\text{bottom}} = \frac{E_\text{top}}{E_\text{bottom}} = \frac{m - m g h}{m} = 1 - g h
\end{equation}
\rem{in script, this is only true to first order, so derivation might be wrong... Result there reads $\frac{1}{m + m g h}$. Ahhh, because there things are defined differently: photon starts from top, thus it has more energy at ground}
\end{ex}

natural consequence because time is inversely proportional to frequency (time differences are): clocks tick faster at higher altitude, i.e.~for a stronger gravitational potential



%reason for puzzling and seemingly inconsistent results: a frame where gravity acts is \emph{not} inertial (because we have external force in gravity, right?)!



	\subsection{Special Relativity}
\begin{defi}[Inertial Frame]
An \Def{intertial frame of reference} is a coordinate frame where $\vec{F} = m \vec{a}$ holds. In particular, that means objects move with constant speed when no force is acting on them.
\end{defi}

there is no unique inertial frame; all laws of physics have to hold equally in all inertial frames; therefore, only quantities which are invariant under transformations between inertial frames have physical meaning

we can always get other inertial frames from existing ones by looking at some which move with constant speed with respect to them; as a consequence, coordinates of events have no physical meanings; however, distances between events turn out to have an invariant (and thus physical) meaning:
\begin{align*}
\qty(\Delta s)^2 &= - \qty(\Delta t)^2 + \qty(\Delta x)^2 + \qty(\Delta y)^2 + \qty(\Delta z)^2
\\
&= \underline{\Delta x} \cdot \eta \cdot \underline{\Delta x} = \mqty(\Delta t & \Delta x & \Delta y & \Delta z) \cdot \mqty(-1 & 0 & 0 & 0 \\ 0 & 1 & 0 & 0 \\ 0 & 0 & 1 & 0 \\ 0 & 0 & 0 & 1) \cdot \mqty(\Delta t \\ \Delta x \\ \Delta y \\ \Delta z)
\\
&= \eta_{\mu \nu} \qty(\Delta x)^\mu \qty(\Delta x)^\nu = \qty(\Delta x)^\mu \qty(\Delta x)_\mu
\end{align*}
making these differences $\Delta$ infinitesimally small, i.e.~by going to $ds^2 := \qty(ds)^2$, we obtain the line element of the metric $\eta$



let us now think about gravity in special relativity; in principle, we keep the Newtonian description, but some effects can be examined in different manner now (we have new notion/tool of inertial frame for example)

regarding gravitational redshift: reason for puzzling and seemingly inconsistent results: a frame where gravity acts is \emph{not} inertial (because we have external force in gravity, right?)!

task: finding a reference frame where the effect of gravity is not present; obviously, earths surface is not sufficient and neither is a uniformly moving one; however, in free fall we experience no gravity, that is a freely falling frame cancels out the effect of gravity

we now state this more formally

\begin{prop}[Weak Equivalence Principle]
The effects of a gravitational field are indistinguishable from an accelerated frame of reference.
\end{prop}
so basically: only in a freely falling frame we can cancel out the effect of gravity and create an inertial frame on earth


\begin{prop}[Einstein Equivalence Principle]
The laws of physics in a freely falling frame are locally described by SR without gravity. Such a frame is called \Def[local inertial frame]{local inertial frame (LIF)}.
\end{prop}
speaking strictly mathematically, \enquote{locally} means in an infinitesimally small neighbourhood of points; how much this can be extended in practice depends on the physical effects of interest

moreover, there is no global frame for Earth and no uniform direction of acceleration



from experience we know about gravity that
\begin{itemize}
\item[(a)] all bodies which start with the same initial velocity move through a gravitational field along the same curve

\item[(b)] bodies which move initially parallel to each other in a freely falling frame do not necessarily move parallel at all times if an external gravitational field is present (this effect is due to \Def{tidal forces} acting on them)
\end{itemize}




Now further examination of property (b)

for a particle with worldline $x^k(\tau)$ we have
\begin{equation*}
\dv[2]{x^k}{\tau} = - \delta^{k l} \pdv{\Phi}{x^l} = - \delta^{k l} \eval{\pdv{\Phi}{x^l}}_{\vec{x}}
\end{equation*}
because of \eqref{eq:cons_force} and Newton's second law $F^k = m \dv[2]{x^k}{\tau}$

similarly, for another particle starting close to the first one (i.e.~with worldline $x^k + \xi^k$, where $\abs{\xi^k \xi_k} \ll 1$)
\begin{align*}
\dv[2]{\qty(x^k + \xi^k)}{\tau} &= - \delta^{k l} \eval{\pdv{\Phi}{x^l}}_{\vec{x} + \vec{\xi}}
\\
&\simeq - \delta^{k l} \eval{\pdv{\Phi}{x^l}}_{\vec{x}} - \delta^{k l} \xi^m \pdv{x^m} \eval{\pdv{\Phi}{x^l}}_{\vec{x}}
\end{align*}
where we made use of a Taylor approximation to first order

since derivatives are linear, we obtain
\begin{equation}
\eqbox{
\dv[2]{\xi}{\tau} = \dv[2]{\qty(x^k + \xi^k)}{\tau} - \dv[2]{x^k}{\tau} = - \delta^{k l} \pdv[2]{\Phi}{x^m}{x^l} \, \xi^m
}
\end{equation}
This is the \Def{Newtonian deviation equation}. We see that tidal forces are governed by the tidal acceleration tensor $\pdv[2]{\Phi}{x^m}{x^l}$.

tidal forces are a way to detect gravity as opposed to constant acceleration (which would affect the worldlines $x^k$ and $x^k + \xi^k$ equally)



	\subsection{Curved Spacetime}
problem: Newtonian gravity and SR are not entirely consistent, e.g.~due to instantaneous effect of gravity (redshift also strange); however, we can come up with generalized description: for anybody familiar with differential/Riemannian geometry, the effects of gravity sound very much like the ones associated with a curved space; this motivates the (mathematical) description of gravity as a geometrical effect in Minkowskian spacetime (which will become a curved space in this process)

derive geodesic equation; motivation: path which extremizes proper distance between events/points $A, B$; because proper time is proportional to infinitesimal line element, this tells us that the worldlines of test particles are governed by the geodesic equation, which generalizes the notion of a straight line; it makes sense to demand that because in a LIF, we have to reproduce SR where test particles move along straight lines (if no force is acting on them)




		\subsection{Weakly Curved Spacetimes}
re-derive first subsection from GR, that is using the spacetime metric
\begin{equation}
ds^2 = - \qty(1 + 2 \Phi) dt^2 + \qty(1 - 2\Phi) \qty(dx^2 + dy^2 + dz^2)
\end{equation}
which is the first order approximation in case the Newtonian potential $\Phi = - \frac{m}{r}$ fulfils $\abs{\Phi} \ll 1$; we will see where it comes from later on, will be used as motivation for now



	\subsection{Einstein Equation}

\begin{equation}
\eqbox{
G^{\alpha \beta} + \Lambda g^{\alpha \beta} = 8 \pi T^{\alpha \beta}
}
\end{equation}
with
\begin{equation}
\eqbox{
G^{\alpha \beta} = R^{\alpha \beta} - \frac{1}{2} g^{\alpha \beta} R
}
\end{equation}


often, we will use
\begin{equation}
G^{\alpha \beta} = 8 \pi T^{\alpha \beta}
\end{equation}
where the contribution of the cosmological constant is thought of as stress-energy \enquote{of empty space} and absorbed in $T^{\alpha \beta}$

in vacuum, this becomes
\begin{equation}
G^{\alpha \beta} = 0
\end{equation}
but since taking the trace of the Einstein equations yields $R = - 8 \pi T$, we obtain the vacuum Einstein equation
\begin{equation}
\eqbox{
R^{\alpha \beta} = 0
}
\end{equation}


now few examples of solutions that come out of the Einstein equations (there are not too many)

\begin{ex}
weak Newtonian potential is Schwarzschild far away from source (we define a certain potential and make a gauge transformation)
\end{ex}



		\subsection{Existence of Gravitational Waves}



		\subsection{Effect of Gravitational Waves}







\end{document}