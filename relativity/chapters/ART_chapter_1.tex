%\documentclass[ART_main.tex]{subfiles}
\documentclass[ngerman, DIV=11, BCOR=0mm, paper=a4, fontsize=11pt, parskip=half, twoside=false, titlepage=true]{scrreprt}
%\graphicspath{ {Bilder/} {../Bilder/} }


\usepackage[singlespacing]{setspace}
\usepackage{lastpage}
\usepackage[automark, headsepline]{scrlayer-scrpage}
\clearscrheadings
\setlength{\headheight}{\baselineskip}
%\automark[part]{section}
\automark[chapter]{chapter}
\automark*[chapter]{section} %mithilfe des * wird nur ergänzt; bei vorhandener section soll also das in der Kopfzeile stehen
\automark*[chapter]{subsection}
\ihead[]{\headmark}
%\ohead[]{Seite~\thepage}
\cfoot{\hypersetup{linkcolor=black}Seite~\thepage~von~\pageref{LastPage}}

\usepackage[utf8]{inputenc}
\usepackage[ngerman, english]{babel}
\usepackage[expansion=true, protrusion=true]{microtype}
\usepackage{amsmath}
\usepackage{amsfonts}
\usepackage{amsthm}
\usepackage{amssymb}
\usepackage{mathtools}
\usepackage{mathdots}
\usepackage{aligned-overset} % otherwise, overset/underset shift alignment
\usepackage{upgreek}
\usepackage[free-standing-units]{siunitx}
\usepackage{esvect}
\usepackage{graphicx}
\usepackage{epstopdf}
\usepackage[hypcap]{caption}
\usepackage{booktabs}
\usepackage{flafter}
\usepackage[section]{placeins}
\usepackage{pdfpages}
\usepackage{textcomp}
\usepackage{subfig}
\usepackage[italicdiff]{physics}
\usepackage{xparse}
\usepackage{wrapfig}
\usepackage{color}
\usepackage{multirow}
\usepackage{dsfont}
\numberwithin{equation}{chapter}%{section}
\numberwithin{figure}{chapter}%{section}
\numberwithin{table}{chapter}%{section}
\usepackage{empheq}
\usepackage{tikz-cd}%für Kommutationsdiagramme
\usepackage{tikz}
\usepackage{pgfplots}
\usepackage{mdframed}
\usepackage{floatpag} % to have clear pages where figures are
%\usepackage{sidecap} % for caption on side -> not needed in the end
\usepackage{subfiles} % To put chapters into main file

\usepackage{hyperref}
\hypersetup{colorlinks=true, breaklinks=true, citecolor=linkblue, linkcolor=linkblue, menucolor=linkblue, urlcolor=linkblue} %sonst z.B. orange bei linkcolor

\usepackage{imakeidx}%für Erstellen des Index
\usepackage{xifthen}%damit bei \Def{} das Index-Arugment optional gemacht werden kann

\usepackage[printonlyused]{acronym}%withpage -> seems useless here

\usepackage{enumerate} % for custom enumerators

\usepackage{listings} % to input code

\usepackage{csquotes} % to change quotation marks all at once


%\usepackage{tgtermes} % nimmt sogar etwas weniger Platz ein als default font, aber wenn dann nur auf Text anwenden oder?
\usepackage{tgpagella} % traue mich noch nicht ^^ Bzw macht ganze Formatierung kaputt und so sehen Definitionen nicer aus
%\usepackage{euler}%sieht nichtmal soo gut aus und macht Fehler
%\usepackage{mathpazo}%macht iwie überall pagella an...
\usepackage{newtxmath}%etwas zu dick halt im Vergleich dann; wenn dann mit pagella oder überall Times gut

\setkomafont{chapter}{\fontfamily{qpl}\selectfont\Huge}%{\rmfamily\Huge\bfseries}
\setkomafont{chapterentry}{\fontfamily{qpl}\selectfont\large\bfseries}%{\rmfamily\large\bfseries}
\setkomafont{section}{\fontfamily{qpl}\selectfont\Large}%{\rmfamily\Large\bfseries}
%\setkomafont{sectionentry}{\rmfamily\large\bfseries} % man kann anscheinend nur das oberste Element aus toc setzen, hier also chapter
\setkomafont{subsection}{\fontfamily{qpl}\selectfont\large}%{\rmfamily\large}
\setkomafont{paragraph}{\rmfamily}%\bfseries\itshape}%\underline
\setkomafont{title}{\fontfamily{qpl}\selectfont\Huge\bfseries}%{\Huge\bfseries}
\setkomafont{subtitle}{\fontfamily{qpl}\selectfont\LARGE\scshape}%{\LARGE\scshape}
\setkomafont{author}{\Large\slshape}
\setkomafont{date}{\large\slshape}
\setkomafont{pagehead}{\scshape}%\slshape
\setkomafont{pagefoot}{\slshape}
\setkomafont{captionlabel}{\bfseries}



\definecolor{mygreen}{rgb}{0.8,1.00,0.8}
\definecolor{mycyan}{rgb}{0.76,1.00,1.00}
\definecolor{myyellow}{rgb}{1.00,1.00,0.76}
\definecolor{defcolor}{rgb}{0.10,0.00,0.60} %{1.00,0.49,0.00}%orange %{0.10,0.00,0.60}%aquamarin %{0.16,0.00,0.50}%persian indigo %{0.33,0.20,1.00}%midnight blue
\definecolor{linkblue}{rgb}{0.00,0.00,1.00}%{0.10,0.00,0.60}


% auch gut: green!42, cyan!42, yellow!24


\setlength{\fboxrule}{0.76pt}
\setlength{\fboxsep}{1.76pt}

%Syntax Farbboxen: in normalem Text \colorbox{mygreen}{Text} oder bei Anmerkungen in Boxen \fcolorbox{black}{myyellow}{Rest der Box}, in Mathe-Umgebung für farbige Box \begin{empheq}[box = \colorbox{mycyan}]{align}\label{eq:} Formel \end{empheq} oder farbigen Rand \begin{empheq}[box = \fcolorbox{mycyan}{white}]{align}\label{eq:} Formel \end{empheq}

% Idea for simpler syntax: renew \boxed command from amsmath; seems to work like fbox, so maybe background color can be changed there

\usepackage[most]{tcolorbox}
%\colorlet{eqcolor}{}
\tcbset{on line, 
        boxsep=4pt, left=0pt,right=0pt,top=0pt,bottom=0pt,
        colframe=cyan,colback=cyan!42,
        highlight math style={enhanced}
        }

\newcommand{\eqbox}[1]{\tcbhighmath{#1}}


\newcommand{\manyqquad}{\qquad \qquad \qquad \qquad}  % Four seems to be sweet spot



\newcommand{\rem}[1]{\fcolorbox{yellow!24}{yellow!24}{\parbox[c]{0.985\textwidth}{\textbf{Remark}: #1}}}%vorher: black als erste Farbe, das macht Rahmen schwarz%vorher: black als erste Farbe, das macht Rahmen schwarz

%\newcommand{\anm}[1]{\footnote{#1}}

\newcommand{\anmind}[1]{\fcolorbox{yellow!24}{yellow!24}{\parbox[c]{0.92 \textwidth}{\textbf{Anmerkung}: #1}}}
% wegen Einrückung in itemize-Umgebungen o.Ä.

\newcommand{\eqboxold}[1]{\fcolorbox{white}{cyan!24}{#1}}

\newcommand{\textbox}[1]{\fcolorbox{white}{cyan!24}{#1}}


\newcommand{\Def}[2][]{\textcolor{defcolor}{\fontfamily{qpl}\selectfont \textit{#2}}\ifthenelse{\isempty{#1}}{\index{#2}}{\index{#1}}}%{\colorbox{green!0}{\textit{#1}}}
% zwischendurch Test mit \textbf{#1} noch (wurde aber viel größer)

% habe jetzt Schrift/ font pagella reingehauen (mit qpl), ist mega; wobei Times auch toll (ptm statt qpl)

% wenn Farbe doch doof, einfach beide auf white :D




\mdfdefinestyle{defistyle}{topline=false, rightline=false, linewidth=1pt, frametitlebackgroundcolor=gray!12}

\mdfdefinestyle{satzstyle}{topline=true, rightline=true, leftline=true, bottomline=true, linewidth=1pt}

\mdfdefinestyle{bspstyle}{%
rightline=false,leftline=false,topline=false,%bottomline=false,%
backgroundcolor=gray!8}


\mdtheorem[style=defistyle]{defi}{Definition}[chapter]%[section]
\mdtheorem[style=satzstyle]{thm}[defi]{Theorem}
\mdtheorem[style=satzstyle]{prop}[defi]{Property}
\mdtheorem[style=satzstyle]{post}[defi]{Postulate}
\mdtheorem[style=satzstyle]{lemma}[defi]{Lemma}
\mdtheorem[style=satzstyle]{cor}[defi]{Corollary}
\mdtheorem[style=bspstyle]{ex}[defi]{Example}




% if float is too long use \thisfloatpagestyle{onlyheader}
\newpairofpagestyles{onlyheader}{%
\setlength{\headheight}{\baselineskip}
\automark[section]{section}
%\automark*[section]{subsection}
\ihead[]{\headmark}
%
% only change to previous settings is here
\cfoot{}
}




% Spacetime diagrams
%\usepackage{tikz}
%\usetikzlibrary{arrows.meta}
% -> setting styles sufficient
%\tikzset{>={Latex[scale=1.2]}}
\tikzset{>={Stealth[inset=0,angle'=27]}}

%\usepackage{tsemlines}  % To draw Dragon stuff; Bard says this works with emline, not pstricks
%\def\emline#1#2#3#4#5#6{%
%       \put(#1,#2){\special{em:moveto}}%
%       \put(#4,#5){\special{em:lineto}}}


% Inspiration: https://de.overleaf.com/latex/templates/minkowski-spacetime-diagram-generator/kqskfzgkjrvq, https://www.overleaf.com/latex/examples/spacetime-diagrams-for-uniformly-accelerating-observers/kmdvfrhhntzw

\usepackage{fp}
\usepackage{pgfkeys}


\pgfkeys{
	/spacetimediagram/.is family, /spacetimediagram,
	default/.style = {stepsize = 1, xlabel = $x$, ylabel = $c t$},
	stepsize/.estore in = \diagramStepsize,
	xlabel/.estore in = \diagramxlabel,
	ylabel/.estore in = \diagramylabel
}
	%lightcone/.estore in = \diagramlightcone  % Maybe also make optional?
	% Maybe add argument if grid is drawn or markers along axis? -> nope, they are really important

% Mandatory argument: grid size
% Optional arguments: stepsize (sets grid scale), xlabel, ylabel
\newcommand{\spacetimediagram}[2][]{%
	\pgfkeys{/spacetimediagram, default, #1}

    % Draw the x ct grid
    \draw[step=\diagramStepsize, gray!30, very thin] (-#2 * \diagramStepsize, -#2 * \diagramStepsize) grid (#2 * \diagramStepsize, #2 * \diagramStepsize);

    % Draw the x and ct axes
    \draw[->, thick] (-#2 * \diagramStepsize - \diagramStepsize, 0) -- (#2 * \diagramStepsize + \diagramStepsize, 0);
    \draw[->, thick] (0, -#2 * \diagramStepsize - \diagramStepsize) -- (0, #2 * \diagramStepsize + \diagramStepsize);

	% Draw the x and ct axes labels
    \draw (#2 * \diagramStepsize + \diagramStepsize + 0.2, 0) node {\diagramxlabel};
    \draw (0, #2 * \diagramStepsize + \diagramStepsize + 0.2) node {\diagramylabel};

	% Draw light cone
	\draw[black!10!yellow, thick] (-#2 * \diagramStepsize, -#2 * \diagramStepsize) -- (#2 * \diagramStepsize, #2 * \diagramStepsize);
	\draw[black!10!yellow, thick] (-#2 * \diagramStepsize, #2 * \diagramStepsize) -- (#2 * \diagramStepsize, -#2 * \diagramStepsize);
}



\pgfkeys{
	/addobserver/.is family, /addobserver,
	default/.style = {grid = true, stepsize = 1, xpos = 0, ypos = 0, xlabel = $x'$, ylabel = $c t'$},
	grid/.estore in = \observerGrid,
	stepsize/.estore in = \observerStepsize,
	xpos/.estore in = \observerxpos,
	ypos/.estore in = \observerypos,
	xlabel/.estore in = \observerxlabel,
	ylabel/.estore in = \observerylabel
}

% Mandatory argument: grid size, relative velocity (important: if negative, must be given as (-1) * v where v is the absolute value, otherwise error)
% Optional arguments: stepsize (sets grid scale), xlabel, ylabel
\newcommand{\addobserver}[3][]{%
	\pgfkeys{/addobserver, default, #1}

    % Evaluate the Lorentz transformation
    %\FPeval{\calcgamma}{1/((1-(#3)^2)^.5)}
    \FPeval{\calcgamma}{1/((1-((#3)*(#3)))^.5)} % More robust, allows negative v
    \FPeval{\calcbetagamma}{\calcgamma*#3}

	% Draw the x' and ct' axes
	\draw[->, thick, cm={\calcgamma,\calcbetagamma,\calcbetagamma,\calcgamma,(\observerxpos,\observerypos)}, blue] (-#2 * \observerStepsize - \observerStepsize, 0) -- (#2 * \observerStepsize + \observerStepsize, 0);
    \draw[->, thick, cm={\calcgamma,\calcbetagamma,\calcbetagamma,\calcgamma,(\observerxpos,\observerypos)}, blue] (0, -#2 * \observerStepsize - \observerStepsize) -- (0, #2 * \observerStepsize + \observerStepsize);

	% Check if grid shall be drawn
	\ifthenelse{\equal{\observerGrid}{true}}{%#
		% Draw transformed grid
		\draw[step=\diagramStepsize, blue, very thin, cm={\calcgamma,\calcbetagamma,\calcbetagamma,\calcgamma,(\observerxpos,\observerypos)}] (-#2 * \diagramStepsize, -#2 * \diagramStepsize) grid (#2 * \diagramStepsize, #2 * \diagramStepsize);
	}{} % Do nothing in else case

	% Draw the x' and ct' axes labels
    \draw[cm={\calcgamma,\calcbetagamma,\calcbetagamma,\calcgamma,(\observerxpos,\observerypos)}, blue] (#2 * \observerStepsize + \observerStepsize + 0.2, 0) node {\observerxlabel};
    \draw[cm={\calcgamma,\calcbetagamma,\calcbetagamma,\calcgamma,(\observerxpos,\observerypos)}, blue] (0, #2 * \observerStepsize + \observerStepsize + 0.2) node {\observerylabel};
}



\pgfkeys{
	/addevent/.is family, /addevent,
	default/.style = {v = 0, label =, color = red, label placement = below, radius = 5pt},
	v/.estore in = \eventVelocity,
	label/.estore in = \eventLabel,
	color/.estore in = \eventColor,
	label placement/.estore in = \eventLabelPlacement,
	radius/.estore in = \circleEventRadius
}

% Mandatory argument: x position, y position
% Optional arguments: relative velocity (important: if negative, must be given as (-1) * v where v is the absolute value, otherwise error), label, color, label placement
\newcommand{\addevent}[3][]{%
	\pgfkeys{/addevent, default, #1}

    % Evaluate the Lorentz transformation
    %\FPeval{\calcgamma}{1/((1-(#3)^2)^.5)}
    \FPeval{\calcgamma}{1/((1-((\eventVelocity)*(\eventVelocity)))^.5)} % More robust, allows negative v
    \FPeval{\calcbetagamma}{\calcgamma*\eventVelocity}

	% Draw event
	\draw[cm={\calcgamma,\calcbetagamma,\calcbetagamma,\calcgamma,(0,0)}, red] (#2,#3) node[circle, fill, \eventColor, minimum size=\circleEventRadius, label=\eventLabelPlacement:\eventLabel] {};
}



\pgfkeys{
	/lightcone/.is family, /lightcone,
	default/.style = {stepsize = 1, xpos = 0, ypos = 0, color = yellow, fill opacity = 0.42},
	stepsize/.estore in = \lightconeStepsize,
	xpos/.estore in = \lightconexpos,
	ypos/.estore in = \lightconeypos,
	color/.estore in = \lightconeColor,
	fill opacity/.estore in = \lightconeFillOpacity
}

% Mandatory arguments: cone size
% Optional arguments: stepsize (scale of grid), xpos, ypos, color, fill opacity
\newcommand{\lightcone}[2][]{
	\pgfkeys{/lightcone, default, #1}
	% Draw light cone -> idea: go from event location into the directions (1, 1), (-1, 1) for upper part of cone and then in directions (-1, -1), (1, -1) for lower part of cone
	\draw[\lightconeColor, fill, fill opacity=\lightconeFillOpacity] (\lightconexpos * \lightconeStepsize - #2 * \lightconeStepsize, \lightconeypos * \lightconeStepsize + #2 * \lightconeStepsize) -- (\lightconexpos, \lightconeypos) -- (\lightconexpos * \lightconeStepsize + #2 * \lightconeStepsize, \lightconeypos * \lightconeStepsize + #2 * \lightconeStepsize);
	\draw[\lightconeColor, fill, fill opacity=\lightconeFillOpacity] (\lightconexpos * \lightconeStepsize - #2 * \lightconeStepsize, \lightconeypos * \lightconeStepsize - #2 * \lightconeStepsize) -- (\lightconexpos, \lightconeypos) -- (\lightconexpos * \lightconeStepsize + #2 * \lightconeStepsize, \lightconeypos * \lightconeStepsize - #2 * \lightconeStepsize);
}





\begin{document}

\chapter{Mathematics}


	\section{Gravitational Physics Summary -- Math Part}
		\subsection{Basics of Manifolds}
to describe curved spacetime, we need a coordinate-independent notion of spaces; this is given by manifolds, which are described using charts=coordinates but have an independent, invariant meaning; similarly, they can often be pictured to be embedded in some higher-dimensional Euclidian space, but that need not be the case

therefore, physics happens on manifolds, so events are points on it and more


defining vectors on manifolds is a non-trivial topic, they are now completely distinct notion from points and cannot be visualized as pointing from some origin to this point (problem: thinking of an embedded manifold for now, the vectors would point out of the manifold); instead, we can define vectors locally (infinitesimally) via derivatives of curves (i.e.~as \Def{tangent vectors}; the corresponding set of all tangent vectors is called \Def{tangent space} $V$); for a function $f(x^\alpha)$ on the manifold, we can calculate the derivative along a curve $\gamma = \gamma(\sigma)$ ($\sigma \in \mathbb{R}$ parametrizes $\gamma$):
\begin{equation}
\dv{f}{\sigma} = \dv{f(x^\alpha(\sigma))}{\sigma} = \pdv{f}{x^\alpha} \dv{x^\alpha}{\sigma}
\end{equation}
this is a simply application of the chain rule and it yields the tangent vector components
\begin{equation}
t^\alpha = \dv{x^\alpha}{\sigma}
\end{equation}
because the vector is supposed to act as
\begin{equation}
\underline{t} \cdot f = \qty(t^\alpha \pdv{x^\alpha}) \cdot f = \pdv{f}{x^\alpha} t^\alpha
\end{equation}
(remember: we identify it with a derivative, which can in turn be expressed using partial derivatives with respect to the coordinates)

formally, we can express this as
\begin{equation}
\underline{t} = t^\alpha \underline{e_\alpha} = t^\alpha \pdv{x^\alpha} = \dv{\sigma}
\end{equation}
(note: $\pdv{x^\alpha}$ is often abbreviated as $\partial_\alpha$)


tangent vectors are invariant quantities, they do not (and should not!) depend on the coordinates we use to express them; their components, on the other hand, are \emph{not} invariant; they obey the following transformation rule:
\begin{equation}
\underline{t} = t^\alpha \pdv{x^\alpha} = t^\alpha \pdv{x'^\beta}{x^\alpha} \pdv{x'^\beta} = t'^\beta \pdv{x'^\beta} \qquad \Leftrightarrow \qquad \eqbox{t'^\beta = t^\alpha \pdv{x'^\beta}{x^\alpha}}
\end{equation}
once again, this is basically just an application of the chain rule


next natural step: linear maps on tangent space $V$ (= set/space of tangent vectors); these are called \Def{covectors} or \Def{one forms} (elements of the dual space or \Def{cotangent space} $V^*$) and it turns out that we can identify them with differentials/gradients of functions
\begin{equation}
df = \pdv{f}{x^\alpha} dx^\alpha
\end{equation}
where we chose a convenient basis $\qty{\underline{e^\alpha}}_\alpha$ of the dual vector space; the $\qty{dx^\alpha}_\alpha$ satisfy
\begin{equation}
dx^\alpha\qty(\pdv{x^\beta}) = \pdv{x^\alpha}{x^\beta} = \delta^\alpha_\beta
\end{equation}

more generally, such covectors $w: V \rightarrow \mathbb{R}$ obey
\begin{equation}
w\qty(\alpha \underline{a} + \beta \underline{b}) = \alpha w\qty(\underline{a}) + \beta w\qty(\underline{b}), \qquad \qquad \forall a, b \in \mathbb{R}, \; \underline{a}, \underline{b} \in V
\end{equation}
we can also characterize covectors via tuples of components
\begin{equation}
w_\alpha = w\qty(\underline{e_\alpha}) = w\qty(\partial_\alpha)
\end{equation}

in general, we can also write
\begin{equation}
w\qty(\underline{a}) = w_\alpha \underline{e^\alpha}\qty(a^\beta \underline{e_\beta}) = w_\alpha a^\alpha
\end{equation}



		\subsection{Tensors}
generalization of vectors and covectors



		\subsection{Covariant Derivative}



\newpage



	\section{Notes \& Thoughts}
to be able to develop an appropriate/meaningful notion of parallelism, we need a \enquote{better} derivative. This will be provided by a connection



\subsection{Giulini GR lectures May 19 and 26}
in Riemann normal coordinates, all Christoffel symbols vanish; but they only exist in neighbourhoods around points $p$ (are Riemann normal coordinates \emph{at $p$}); they are \emph{very} helpful in calculations because tensor equations only have to be proven in a single coordinate system, which we can choose to be Riemann normal coordinates because we have just shown that they do exist

uhh, Christoffel symbols satisfy an affine transformation law, not linear (indicator of not tensorial) because there is term without $\Gamma_{\mu \nu}^\sigma$


formel of Koszul only holds like this for torsion-free and metric connections; and by substituting $X = e_\alpha, Y = e_\beta, Z = e_\gamma$ into it and then contracting with certain component of inverse metric gives rise to formula for connection coefficients (commonly called Christoffel symbols for Levi-Civita connection) and these uniquely determine the connection (which is proof for uniqueness); note that connection coefficients (= covariant derivative with only basis vectors) already determines the connection because connection is $\mathbb{R}$-linear, tensorial ($C^\infty$-linear) and obeys the Leibniz rule

$\nabla_X Y = X^\alpha \qty(\nabla_\alpha Y^\beta) \pdv{x^\beta}$, so components of $\nabla_X Y$ are partial derivatives of components plus extra term, i.e.~$\nabla_{\pdv{x^\alpha}} Y =: \qty(\nabla_\alpha Y^\beta) \pdv{x^\alpha}$; however, they to read this is \emph{not} covariant derivative of $Y^\beta$ since this would be the covariant derivative of a function, but instead as the $\beta$-component of the covariant derivative $\nabla_\alpha Y$


here it is, reason why covariant derivative of covector (with respect to some vector field $X$) looks the way it looks:
\begin{align*}
\nabla_X \omega &= \nabla_{X^\alpha \pdv{x^\alpha}} \qty(\omega_\beta dx^\beta)
\\
&= X^\alpha \qty(dx^\beta \nabla_{\pdv{x^\alpha}} \omega_\beta + \omega_\beta \nabla_{\pdv{x^\alpha}} dx^\beta)
\\
&= X^\alpha \pdv{\omega_\beta}{x^\alpha} dx^\beta
\end{align*}
but $dx^\beta\qty(\pdv{x^\alpha}) = \delta_\alpha^\beta$, so by taking the derivative of this equation we see that ... $\qty(\nabla_{\pdv{x^\gamma}} dx^\beta)\qty(\pdv{x^\alpha}) = \qty(\nabla_{\pdv{x^\gamma}} dx^\beta)^\alpha = - \Gamma_{\gamma \alpha}^\beta$

from that we get general formula for tensors of arbitrary rank because of Leibniz rule; therefore, we get the \enquote{master formula}
\begin{align}
\nabla_X T &= T^{\alpha_1 \dots \alpha_k}_{\beta_1 \dots \beta_m} \, \pdv{x^{\alpha_1}} \otimes \dots \otimes \pdv{x^{\alpha_k}} \otimes dx^{\beta_1} \otimes \dots \otimes dx^{\beta_m}
\notag\\
&= X^\gamma \qty(\nabla_\gamma T^{\alpha_1 \dots \alpha_k}_{\beta_1 \dots \beta_m}) \, \pdv{x^{\alpha_1}} \otimes \dots \otimes \pdv{x^{\alpha_k}} \otimes dx^{\beta_1} \otimes \dots \otimes dx^{\beta_m}
\notag\\
\nabla_\gamma T^{\alpha_1 \dots \alpha_k}_{\beta_1 \dots \beta_m} &= \qty(\pdv{T^{\alpha_1 \dots \alpha_k}_{\beta_1 \dots \beta_m}}{x^\gamma} + \sum_{i = 1}^k \Gamma_{\gamma \lambda}^{\alpha_i} T^{\alpha_1 \dots \alpha_{i - 1} \lambda \alpha_{i + 1} \alpha_k}_{\beta_1 \dots \beta_m} - \sum_{j = 1}^m \Gamma_{\gamma \beta_j}^\lambda T^{\alpha_1 \dots \alpha_k}_{\beta_1 \dots \beta_{j - 1} \lambda \beta_{j + 1} \dots \beta_m})
\end{align}


there is also another notion of derivative on manifolds, the Lie derivative; to define it, we do not need any additional structure (unlike for connection, connection coefficients need metric); the Lie derivative can often be used to express symmetries; components of it are
\begin{equation}
\qty(\mathcal{L}_X T)^{\alpha_1 \dots \alpha_k}_{\beta_1 \dots \beta_m} = X^\alpha \pdv{x^\alpha} T^{\alpha_1 \dots \alpha_k}_{\beta_1 \dots \beta_m} - \sum_{i = 1}^k \pdv{X^{\alpha_i}}{x^\lambda} T^{\alpha_1 \dots \alpha_{i - 1} \lambda \alpha_{i + 1} \dots \alpha_k}_{\beta_1 \dots \beta_m} + \sum_{j = 1}^m \pdv{X^\lambda}{x^{\beta_i}} T^{\alpha_1 \dots \alpha_k}_{\beta_1 \dots \beta_{i - 1} \lambda \beta_{i + 1} \dots \beta_m}
\end{equation}
we notice: for upper index we now have minus, lower index has plus (reversed compared to connection); interesting property: all partial derivatives could be replaced by covariant derivatives without changing the formula (despite them being defined independently of each other!); if a certain vector field defines a symmetry, i.e.~the metric does not change under the flow of that vector field (stays constant along it/integral curves defined by it), then we can express that as the vanishing of the Lie-derivative of this symmetry-generating vector field; these vector fields are called Killing fields; note that $\qty(\mathcal{L}_X g)_{\alpha \beta} = \qty(\nabla_\alpha X^\gamma) g_{\gamma \beta} + \qty(\nabla_\beta X^\gamma) g_{\alpha \gamma} = \nabla_\alpha X_\beta + \nabla_\beta X_\alpha$ (we just use writing in terms of covariant derivative for first equality) -> shouldn't that be equal to $\nabla_{[\alpha} X_{\beta]}$; would also explain $\mathcal{L} = \text{Alt}(\nabla)$ statement I heard; ah no, this is the \emph{symmetrized} part... But maybe that supports view, symmetric part is zero for Killing field (but this has vanishing Lie derivative, so antisymmetric part also zero, right?)



very interesting: Einstein tensor is divergence-free, i.e.~$\nabla_\alpha G^{\alpha \beta} = 0$

we know that, in general, $\qty[\nabla_\mu, \nabla_\nu] T^{\alpha_1 \dots \alpha_k}_{\beta_1 \dots \beta_m} \neq 0$; expanding this quantity for an arbitrary vector field $X^\alpha$, $\qty[\nabla_\mu, \nabla_\nu] X^\alpha = \dots = \qty(\pdv{x^\mu} \Gamma_{\nu \beta}^\alpha - \pdv{x^\nu} \Gamma_{\mu \beta}^\alpha) X^\beta + \text{terms proportional to } \Gamma = R^\alpha_{\beta \mu \nu} X^\beta + \text{terms proportional to } \Gamma$; in Riemann-normal coordinates, $\Gamma = 0$ and $\qty[\nabla_\mu, \nabla_\nu] X^\alpha = R^\alpha_{\beta \mu \nu} X^\beta$ and since both sides are tensors, this equation holds in general; curvature is related to (Giulini said \enquote{obstruction}) commutivity of second derivatives; furthermore, pulling down the index $\alpha$ yields $\qty[\nabla_\mu, \nabla_\nu] X_\alpha = - R^\beta_{\alpha \mu \nu} X_\beta$, which tells us how this quantity acts on a covector (again, has on other sign and acts on other index, like it was for covariant derivative itself); thus, we get the general formula $\qty[\nabla_\mu, \nabla_\nu] T^{\alpha_1 \dots \alpha_k}_{\beta_1 \dots \beta_m} = \sum_{i = 1}^k R^{\alpha_i}_{\lambda \mu \nu} T^{\alpha_1 \dots \alpha_{i - 1} \lambda \alpha_{i + 1} \dots \alpha_k}_{\beta_1 \dots \beta_m} - \sum_{j = 1}^m R^\lambda_{\beta_j \mu \nu} T^{\alpha_1 \dots \alpha_k}_{\beta_1 \dots \beta_{i - 1} \lambda \beta_{i + 1} \dots \beta_m}$

application of that formula: $\nabla_\mu \nabla_\nu X_\beta = R^\alpha_{\mu \nu \beta} X_\alpha$ holds for any Killing vector field $X$; second derivatives are determined by vector field itself; similarly, $\pdv{x^\alpha} X_\beta + \pdv{x^\beta} X_\alpha = 2 \Gamma_{\alpha \beta}^\gamma X_\gamma$, the symmetric part of first derivative of Killing vector field is determined by field itself as well; the only free parameters are value of the field itself and anti-symmetric part $\pdv{x^\alpha} X_\beta - \pdv{x^\beta} X_\alpha$ of first derivative (all derivatives of higher order are determined by relation to curvature tensor); solutions of linear differential equations (no matter of partial or not) constitute a vector space because we can add them together and multiply with numbers and maximum number of dimensions is given by number of freely specifiable initial conditions; here, these are values $\eval{X^\alpha}_p$ of Killing field at a specific point, i.e.~$n = \dim(M)$, and $\eval{\pdv{x^\alpha} X_\beta - \pdv{x^\beta} X_\alpha}_p$, i.e.~$\frac{1}{2} n (n + 1)$; in total, that means there are at most $n + \frac{1}{2} n (n + 1)$ independent solutions to the Killing equation; in Minkowski space there are indeed $10$, in that sense it is maximally symmetric (these generate symmetries of the space, which are given by Poincare group)


Riemann tensor has $20$ components and there are $10$ different traces (because of antisymmetry in first two indices, which means trace is always zero); our goal is now decomposing it into trace and traceless parts, both contain information about Riemann tensor; the trace part is nothing but the Ricci tensor $R_{\alpha \beta} = R^\lambda_{\alpha \lambda \beta}$, the \enquote{rest} (trace-free part) is the Weyl tensor (which he writes down in terms of some weird product); this product has the same symmetries as the Riemann tensor, so Weyl tensor also has them and it is trace free in addition, i.e.~$W^\lambda_{\alpha \lambda \beta} = 0$ (taking these conditions into account, the Weyl tensor has $10$ independent components in $4$ dimensions; very interesting property is that \emph{only} in $4$ dimensions, the amount of information in Weyl, Ricci Tensor is the same; in $3$ dimensions, Weyl tensor has no information and in higher ones much more than Ricci); in index-form it is given by $W^\alpha_{\beta \mu \nu}$


oof:
\begin{align*}
R_{\alpha \beta \gamma \delta} &= \frac{1}{2} \qty(g \cdot \text{Ric})_{\alpha \beta \gamma \delta} - \frac{1}{12} R \qty(g \cdot g)_{\alpha \beta \gamma \delta} + W_{\alpha \beta \gamma \delta}
\\
&= \frac{1}{2} \qty(g_{\alpha \gamma} R_{\beta \delta} + g_{\beta \delta} R_{\alpha \gamma} - g_{\alpha \delta} R_{\beta \gamma} - g_{\beta \gamma} R_{\alpha \delta})
\\
&\quad - \frac{1}{12} R \qty(g_{\alpha \gamma} g_{\beta \delta} + g_{\beta \delta} g_{\alpha \gamma} - g_{\alpha \delta} g_{\beta \gamma} - g_{\beta \gamma} g_{\alpha \delta}) + W_{\alpha \beta \gamma \delta}
\\
&= \frac{1}{2} \qty(g_{\alpha \gamma} R_{\beta \delta} + g_{\beta \delta} R_{\alpha \gamma} - g_{\alpha \delta} R_{\beta \gamma} - g_{\beta \gamma} R_{\alpha \delta}) - \frac{1}{6} R \qty(g_{\alpha \gamma} g_{\beta \delta} - g_{\alpha \delta} g_{\beta \gamma}) + W_{\alpha \beta \gamma \delta}
\end{align*}


long talk about constant curvature; interesting statement (Schur's theorem): constant curvature $\Leftrightarrow$ Gaussian/sectional curvature of each point does not depend on the choice of the $2$-tangent-plane through the point

the Weyl curvature $W^\alpha_{\beta \mu \nu}$ (which is a function of the metric $g$) has the important property of being conformally invariant, i.e.~$W^\alpha_{\beta \mu \nu}(\Omega^2 g) = W^\alpha_{\beta \mu \nu}(g)$ for some function $\Omega$ and even the reverse statement is true: if $W^\alpha_{\beta \mu \nu}(g_1) = W^\alpha_{\beta \mu \nu}(g_2)$, then $\exists$ locally a function $\Omega \in C^\infty(M; \mathbb{R})$ without zeros such that $g_1 = \Omega^2 g_2$; for example, a vanishing Weyl tensor means that the space is locally, conformally flat -> all of these statements are valid only for $n \geq 3$







		\subsection{Order}
basically take order from Penrose?; other way to put it: from summary H\_Analysis, but with less math; also Carroll?



first: do manifolds; then go to tangent space (we want vectors); then go to bundles (first tangent bundle, then more general vector bundles); then define tensors and tensor bundle; then go to differential geometry



	\subsection{General Thoughts}
Schwarzschild metric contains information on many effects of BHs in its components! coefficient $1 - \frac{2 M}{r}$ in front of $dt^2$ tells us about time dilation close to BH (more $t$ goes by the closer you get) and $\frac{1}{1 - \frac{2 M}{r}}$ in front of $dr^2$ tells us about curvature of space (increases as $r$ decreases)



		\subsection{Math Stuff}
regarding tensor product: throughout the discussions, linearity of objects was very important (we have used it for the differential, many mappings, etc.); however, a very important notion that is not linear is the underlying spaces we have looked at; take for example $\mathbb{R}^2 = \mathbb{R} \cross \mathbb{R}$: here, we do not have linearity, which would require $(2, 1) = 2 \cdot (1, 1)$ and clearly, this is not true; however, we might be interested in such a space and this is what is called a tensor product space; there are more tensor products, we also have to make sense of the one of objects in this space, but this is a good motivation


goal of derivatives is approximation to first order, which is expressed in demanding linearity of operators


differentiation is linear and has Leibniz rule, so it already fulfils requirements for tensor derivative (thus it makes sense to demand $\nabla, d$ acting like $D$ on functions); multiple generalizations of $Df$ to something like \enquote{$Ds$} for sections $s$ exist, which is fine because from $\nabla$ we can easily get many others e.g.~by $d = \text{Alt}(\nabla)$ (not sure if equality is true, but from Carroll eq 1.82 it looks like this), that is by suitable mappings



	\subsection{From Wald}
the notion of curvature, intuitively, corresponds to the one of a $2$-sphere in 3D space; however, this is extrinsic curvature which is only visible in embeddings, but what we are interested in is something like intrinsic curvature; how can we detect that?



		\subsection{From Penrose}
tangent space in point $p \in M$ is immediate/infinitesimal vicinity of $M$ \enquote{stretched out}; more formally, a linearisation of the manifold

to do physics, we cannot just work with vector spaces or affine spaces like the Euclidian space (basically $\mathbb{R}^n$, but no need to fix origin), but we need manifolds; however, manifolds do not have enough natural structure to build up the theory that is needed to describe physics, so we need some additional (local) structure (e.g.~enabling us to measure infinitesimal distances in case of a metric structure); this structure is often encoded to/using the tangent spaces (which are present naturally for manifolds), which are vector spaces again

problem of abstract notion of \enquote{no structure} is for example: no general, meaningful (well-defined) notion of differentiation (does exist for functions, but not for vector fields, 1-forms or other tensors); exterior derivative is something like that, but it does not really give information about varying of the forms (nice is that it maps $p$-forms to $p + 1$-forms)


some structure can also be provided by connection; although not every structure can reproduced, metrics uniquely determine a connection (Levi-Civita connection)


goal of derivative operators: measure constancy and deviations from it; in case of vectors, this is equivalent to a notion of parallelism; note: we will go reverse route, define derivative and get parallelism from that; this notion will have the unusual feature of path-dependence, where unusual is meant with respect to what we know from Euclidian space; while it is possible to do this (see Wald), but this is mainly by making the \enquote{right} guess and thus not really helpful (idea is to say we want something where change of $v$ is proportional to difference $\Delta x$ and then we say: this works; welp)


which requirements make sense? since tangent space is linearisation of manifold, there should also be linear dependence on direction that we differentiate along; more generally, pointwise linearity means that functions can be dragged across the operator; when acting on tensors however, a product rule has to be specified: $\nabla_X (f s) = (\nabla_X f) s + f \nabla_X s$ makes sense (without argument $X$, this becomes $\nabla (f s) = (\nabla f) \otimes s + f \nabla s$) (?)


ideas come from the fact that our goal is to generalize action of derivative $D$; therefore, demanding $\nabla f = df$ also makes a lot of sense

extension to more than one tensor field is possible by demanding additivity $\nabla (s + t) = \nabla s + \nabla t$ and by specifying product/Leibniz rule $\nabla (s \otimes t) = (\nabla s) \otimes t + s \otimes (\nabla t)$; to uniquely determine this generalization, it is also necessary to demand compatibility with trace/contraction (which also helps with defining these things in the first place)


interesting: local connection can be defined uniquely from Gaussian basis vectors



\end{document}