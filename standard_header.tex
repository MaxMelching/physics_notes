\documentclass[
  fontsize=11pt,
  % twoside=false,
  twoside,
  titlepage=true
  % -- NOTE: the following are nice, but overriden by geometry package
  % paper=a4,
  % DIV=11,  % Division factor, controls ratio of text block width and margin width
  % BCOR=0mm,  % Binding correction
]{scrreprt}



% -- Fundamental Packages -----------------------------------------------------
\usepackage{subfiles}  % For larger projects, allows to split up files

\usepackage[
  utf8
]{inputenc}  % Make sure encoding is recognized correctly

\usepackage[
  % -- NOTE: order matters here! The SECOND entry is primary language, i.e. the that table of contents label etc is in
  ngerman,  % I am young enough to use ngerman = new german
  english,
]{babel}  % Get consistent behaviour for different languages

\usepackage[
  expansion=true,  % Slight modification of character widths ok to make things look better
  protrusion=true,  % Small characters like punctuation can go into margins for straighter text edges
]{microtype}  % Makes things look nice in documents, better text flow etc.#

\usepackage{textcomp}  % Extend symbols available for T1 encoding (provides, \textcelsius, \texteuro)

\usepackage{hyperref}  % Allows clicking on links
\hypersetup{
  colorlinks=true,
  breaklinks=true,
  citecolor=linkblue,
  linkcolor=linkblue,
  menucolor=linkblue,
  urlcolor=linkblue,
}

\usepackage{imakeidx}  % Allows creation of an index
\usepackage{xifthen}  % More advanced if-then-else syntax



% -- Colors -------------------------------------------------------------------
\usepackage{xcolor}  % More versatile than color package

\definecolor{mygreen}{rgb}{0.8,1.00,0.8}
\definecolor{mycyan}{rgb}{0.76,1.00,1.00}
\definecolor{myyellow}{rgb}{1.00,1.00,0.76}
\definecolor{defcolor}{rgb}{0.10,0.00,0.60} %{1.00,0.49,0.00}%orange %{0.10,0.00,0.60}%aquamarin %{0.16,0.00,0.50}%persian indigo %{0.33,0.20,1.00}%midnight blue
\definecolor{linkblue}{rgb}{0.00,0.00,1.00}%{0.10,0.00,0.60}
\definecolor{codecolor}{RGB}{235, 66, 0}  % Inspired by color on Google Slides. Supposed to be dark orange



% -- Graphics & Floats --------------------------------------------------------

% -- General Float Functionality
\usepackage{graphicx}  % Provides figure environment
% \usepackage{epstopdf}  % Enable working with eps files
\usepackage[section]{placeins}  % Make sure floats are not placed in next section
\usepackage{flafter}  % Make sure floats only appear AFTER they are first mentioned in text
\usepackage[hypcap]{caption}  % Flexible captions, supports subfigues and multi-line captions
\usepackage{wrapfig}  % Wrap text around figures
\usepackage{standalone}  % Convenient input of figures that are stored as separate standalones

% -- Figures
\usepackage{subfig}  % Provides \subfloat

% -- Tables
\usepackage{booktabs}  % Nicer tables, provides e.g. \toprule, \midrule, \bottomrule
\usepackage{array}  % Define fancy table cells
\usepackage{colortbl}  % Provides \cellcolor, \rowcolor
\usepackage{multirow}  % Provides \multirow, complementary to \multicolumn


% -- Code
\usepackage{listings}



% -- Math Packages ------------------------------------------------------------
\usepackage{amsmath}  % The basic package for everything
\usepackage{amsfonts}  % Add fonts like \mathfrac
\usepackage{amsthm}  % Used below
\usepackage{amssymb}  % Load variety of symbols
% \usepackage{dsfont}  % Provides \mathds for better fonts than amsfonts
\usepackage{mathtools}  % Better spacing + alignment, extends amsmath
\usepackage{mathdots}  % More spacing + alignment, especially for various dots commands
\usepackage{upgreek}  % Get non-italicized greek letters by using \upalpha etc.
\usepackage{aligned-overset}  % For overset to work in align environment
\usepackage{esvect}  % Get longer arrows using \vv{} or \vvv{} (and optionally different vector style)
% \usepackage{empheq}  % Put boxes around equations to emphasize them

\usepackage[
  italicdiff,
]{physics}  % Provides many commands, e.g. \qty for brackets with adjustable height

\usepackage[
  free-standing-units,
]{siunitx}  % Very convenient to display numbers (with and without units)

\usepackage{wasysym}  % More math symbols. Provides \ascnode



% -- Some Unordered Add-ons ---------------------------------------------------
\usepackage{pdfpages}  % Tools to include pdf files (\includepdf). Allows selecting certain pages.

% \usepackage[
%   printonlyused,
%   % withpage,
% ]{acronym}  % Provides \ac for consistent handling of acronyms (abbreviations)

% \usepackage{xparse}  % More convenient command definition using \NewDocumentCommand

\usepackage{enumerate} % For custom enumerators
\renewcommand{\labelitemi}{$\blacktriangleright$}%{$\vartriangleright$}
\renewcommand{\labelitemii}{\textbf{--}}
\renewcommand{\labelitemiii}{$\bullet$}

\usepackage{csquotes} % To easily get consistent quotation marks, with adjustable style

\usepackage{xspace}  % Allows calling \command instead of \command{} and still have proper spacing

% -- TikZ
\usepackage{pgfplots}  % Loads pgf and pgfkeys
\usepackage{tikz}
\usepackage{tikz-3dplot}
\usepackage{fp}


\usetikzlibrary{
	arrows.meta,
	math,
	calc,
	% 3d,
}



% -- Page Layout --------------------------------------------------------------
\usepackage[
  a4paper,
  % margin=1in,  % Default
  % bindingoffset=10mm,
  % inner=1in,
  % outer=1in,
  % -- Standard binding offset is 6mm. Instead of making document smaller, we shift.
  inner=1in+6mm,
  outer=1in-6mm,
  top=1.5in,
  bottom=1.5in,
  % -- 1in is default
]{geometry}


\usepackage[
  automark,
  headsepline=1pt,
]{scrlayer-scrpage}

\clearmainofpairofpagestyles  % Reset everything

% -- Control what is shown when calling \headmark. [] controls what is shown on
% -- left side, {} what is shown on right side (only seems to be relevant for
% -- double page documents).
\automark{chapter}  % What is used by default
\automark*{section}  % Use section instead of chapter (if there is one)
%\automark*{subsection}


% -- Control numbers displayed for labelled stuff
\numberwithin{equation}{chapter}
\numberwithin{figure}{chapter}
\numberwithin{table}{chapter}


% -- Header And Footer Handling
\usepackage{lastpage}  % To get access to what last page is
% \setlength{\headheight}{\baselineskip}


\definecolor{midnightblue}{rgb}{0.1, 0.1, 0.44}
\definecolor{darkmidnightblue}{rgb}{0.0, 0.2, 0.4}

\definecolor{headercolor}{named}{midnightblue}


% \renewcommand{\headrulewidth}{1pt}

\ifthenelse{\boolean{@twoside}}{
  \lehead{{%
    \hypersetup{hidelinks}\thepage/\pageref{LastPage}}%
  }
  \rehead{\leftmark}

  \lohead{%
    \leftmark%
  }
  \rohead{{\hypersetup{hidelinks}\thepage/\pageref{LastPage}}}
}{
  \ihead{\headmark}
  %\ohead[]{Seite~\thepage}  % Like it better in footer
  \cfoot{{\hypersetup{hidelinks}Page~\thepage~of~\pageref{LastPage}}}
  % -- NOTE: if lastpage is removed as package, use \pagemark instead of \thepage
}


% -- Some custom pagestyles to handle things like big floats. Use them via
% -- \thisfloatpagestyle{onlyheader}. Non-trivial task because floats may be
% -- inserted at different page compared to text surrounding them (package fings
% -- page they appear on and sets pagestyle there).
\usepackage{floatpag}  % Provides \floatpagestyle


% TODO: check if we can't simply do "use current style" and edit just selected stuff -> nope


% -- Control spacing in document. First between lines, then between paragraphs.
% -- Setting parskip manually is advantageous, then we can have indent+skip.
\usepackage[
  % singlespacing,
  % -- Also possible: onehalfspace, doublespace
]{setspace}
% -- NOTE: for some reason, using this package causes errors when TikZ is loaded afterwards.
% -- Thus we replace it here by setting the stretch manually. Equivalent to package use: passing "onehalfspacing" to the documentclass.
\setstretch{1.15}  % Custom value
% \renewcommand{\baselinestretch}{1.15}  % Note that 1.25 corresponds to onehalfspacing!
% -- Not recommended, changes spacing everywhere: https://tex.stackexchange.com/questions/79046/baselinestretch-vs-setspace -> indeed, also causes indent in footnotes for some reason
\setlength{\parskip}{0.5em}  % What parskip=half would do in document class
\setlength{\parindent}{1em}  % What would be indent if no parskip given in documentclass



% -- Fonts --------------------------------------------------------------------
\usepackage{newtxmath}  % Use Times in math mode
\usepackage{tgpagella}  % Use Pagella in text

\setkomafont{chapter}{\rmfamily\Huge}
\setkomafont{chapterentry}{\large\bfseries}
\setkomafont{section}{\rmfamily\Large\bfseries}
\setkomafont{subsection}{\rmfamily\large\scshape}  % slshape also not bad
\setkomafont{paragraph}{\rmfamily}  % \bfseries\itshape} %\underline

% -- Titlepage fonts
\setkomafont{title}{\bfseries}
\setkomafont{subtitle}{\Large\scshape}
\setkomafont{author}{\Large\slshape}
%\setkomafont{date}{\Large\slshape}

% -- Header & Footer
\setkomafont{pagehead}{\scshape\bfseries}
\setkomafont{pagefoot}{\slshape}
\addtokomafont{pagehead}{\color{headercolor}}
% \addtokomafont{headsepline}{\color{gray}}  % From https://esc-now.de/2018/latex-individuelle-kopf-und-fusszeilen-update/



% -- Bibliography Management --------------------------------------------------
% \usepackage[
%   nottoc,
%   notlot,
%   notlof,
%   chapter,
% ]{tocbibind} %macht automatisch ins TOC, auch index und andere Sachen; so ungenummert, es geht aber auch mit Option numbib -> nicht nötig jetzt

% \usepackage[
%   maxcitenames=3,
%   backend=biber,
% ]{biblatex}%vlt hätte maxnames=2 gepasst

\usepackage[
    maxcitenames=3,
    % style=authoryear
    style=alphabetic
    % style=alpha
    % style=numeric
    % style=phys
]{biblatex}


%\renewcommand{\bibname}{References}
\addto\captionsenglish{\renewcommand{\bibname}{References}}



% -- Custom Environments & Boxes ----------------------------------------------
\setlength{\fboxrule}{0.76pt}
\setlength{\fboxsep}{1.76pt}


% -- Syntax for/notes on colored boxes:
% -- * in normal text use \colorbox{mygreen}{some text} and for remark in boxes use
% --   \fcolorbox{black}{myyellow}{rest of box}
% -- * in math environmentas: for colored box use \begin{empheq}[box = \colorbox{mycyan}]{align}\label{eq:} ... \end{empheq}
% --   or for colored borders: \begin{empheq}[box = \fcolorbox{mycyan}{white}]{align}\label{eq:} ... \end{empheq}



% -- Some Colored Boxes For Various Purposes
%\newcommand{\anm}[1]{\footnote{#1}}  % V1
\newcommand{\anm}[1]{%
  \fcolorbox{black}{yellow!24}{\parbox[c]{0.985\textwidth}{\textbf{Anmerkung}: #1}}%
}

\newcommand{\anmind}[1]{%
  \fcolorbox{black}{yellow!24}{\parbox[c]{0.92\textwidth}{\textbf{Anmerkung}: #1}}%
}  % Indented remark, for use in itemize etc.

\newcommand{\eqbox}{%
  \fcolorbox{white}{cyan!24}%
}

\newcommand{\textbox}[1]{%
  \fcolorbox{white}{cyan!24}{#1}%
}

% Other nice colors: green!42, cyan!42, yellow!24


% -- General Commands That Draw Something Or Make Test Colored
\newcommand{\Def}[2][]{%
  \textcolor{defcolor}{\fontfamily{ptm}\selectfont \textit{#2}}%
  %{\colorbox{green!0}{\textit{#1}}}%
  \ifthenelse{\isempty{#1}}{\index{#2}}{\index{#1}}%
}  % TODO: maybe choose other font here than pagella?

\newcommand{\code}[1]{%
  \textcolor{codecolor}{\texttt{#1}}%
}

\newcommand{\todo}[1]{%
  \fcolorbox{white}{red!42}{\textit{\textbf{TODO:}} #1}%
}  % Very rudimentary, but works

\def\RArrow{
  \tikz{
    \draw[
      ->,
      fill,
      line width=3pt,  % Better than thick
      -{Triangle[length=4.4pt, width=5.8pt]}
    ] (0,0) -- (11pt, 0) {};
  }
}


% -- Theorem Boxes
% \usepackage{amsthm}  % Must be loaded beforehand to avoid error, but this is what we use here
\usepackage{mdframed}

% -- First we define some styles, then some environments based on them
\mdfdefinestyle{defistyle}{
  topline=false,
  rightline=false,
  linewidth=1pt,
  frametitlebackgroundcolor=gray!12,
}

\mdfdefinestyle{satzstyle}{
  topline=true,
  rightline=true,
  leftline=true,
  bottomline=true,
  linewidth=1pt,
}

\mdfdefinestyle{bspstyle}{
  rightline=false,
  leftline=false,
  topline=false,
  % bottomline=false,
  backgroundcolor=gray!8,
}



% -- Not really a theorem, but with same style: remark
\mdfdefinestyle{remstyle}{
  rightline=false,
  leftline=false,
  topline=false,
  bottomline=false,
  backgroundcolor=myyellow,
  innerleftmargin=.4\baselineskip,
  innerrightmargin=.4\baselineskip,
  leftmargin=-.4\baselineskip,
  rightmargin=-.4\baselineskip,
  % -- Setting the indentations is important because otherwise, everything will
  % -- be indented (.4\baselineskip is default and looks natural, so this is
  % -- chosen here; the effects of margin and innermargin have to be balanced).
  % frametitle={\textbf{remark}: }}%frametitle also makes linebreak
}

\newmdenv[style=remstyle]{remark}

% -- Goal: call this using a command \rem, not \begin{remark}. Defining such a
% -- thing in preambles, however, is tricky. Solution from
% -- https://www.mrunix.de/forums/showthread.php?59532-begin-und-end-in-newcommand
\def\brem#1\erem{\begin{remark}#1\end{remark}}
\newcommand{\rem}[1]{\brem \textbf{remark:} #1\erem}

% -- Slightly modified version (new line not always wanted for remarks in text)
\usepackage{soul}
\sethlcolor{myyellow}
\newcommand{\question}[1]{\hl{#1}}
