% Author: Izaak Neutelings (March 2020)
% Copied from https://tikz.net/magnet_fieldlines_dipoles/
\documentclass[border=3pt,tikz]{standalone}
\usepackage{amsmath} % for \dfrac
\usepackage{bm}
\usepackage{tikz,pgfplots}
\tikzset{>=latex} % for LaTeX arrow head
\pgfplotsset{compat=1.13}
\usetikzlibrary{decorations.markings,intersections,calc}
\usepackage{ifthen}
\usepackage{xcolor}
\usepackage{auto-pst-pdf}
\usepackage{pst-magneticfield}
\usepackage[outline]{contour} % glow around text
\colorlet{Bcol}{violet!90}
\colorlet{BcolFL}{violet!90}
\tikzstyle{north}=[thick,top color=red!60,bottom color=red!90,shading angle=20]
\tikzstyle{south}=[thick,top color=blue!60,bottom color=blue!90,shading angle=20]
\contourlength{1.5pt}
\tikzset{
   EFielLineArrow/.style args={#1,#2}{BcolFL,thick,decoration={markings,
          mark=at position #1 with {\arrow[rotate=#2]{latex}}},
          postaction={decorate}}
}

\makeatletter
  \newcommand{\xy}[3]{% % FIND X, Y
    \tikz@scan@one@point\pgfutil@firstofone#1\relax
    \edef#2{\the\pgf@x}%
    \edef#3{\the\pgf@y}%
  }
\makeatother
\newcommand{\EFielLineArrow}[2]{ % ELECTRIC FIELD LINE ARROW
  \pgfkeys{/pgf/fpu,/pgf/fpu/output format=fixed} % for calculation between -1*10^324 and +1*10^324
  \pgfmathsetmacro{\x}{#1/28.45pt}
  \pgfmathsetmacro{\y}{#2/28.45pt}
  \pgfmathsetmacro{\U}{\Q*((\y+\a)^2+(\x)^2)^(3/2)}
  \pgfmathsetmacro{\V}{-(\Q)*((\y-\a)^2+(\x)^2)^(3/2)}
  \pgfkeys{/pgf/fpu=false}
  \pgfmathparse{
    atan2(((\y+\a)*\V + (\y-\a)*\U),((\x)*\V + (\x)*\U))
  }
  \edef\angle{\pgfmathresult}
  \pgfmathsetmacro{\D}{int(1000*(\Q)*(\y+\a)/sqrt((\y+\a)^2+\x*\x) - 1000*(\Q)*(\y-\a)/sqrt((\y-\a)^2+\x*\x))/1000}
  \draw[EFielLineArrow={0.5,\angle}] ({\xpt+(3pt)*cos(\angle)},{\ypt+(3pt)*sin(\angle)});
}
\newcommand{\EFieldLines}{ % ELECTRIC FIELD LINES
  \message{^^JField lines {\Q} with contours range ^^J\range^^J}
  
  % FIELD LINES
  \draw[BcolFL,thick,name path=Elines] plot[id=plot, raw gnuplot, smooth]
    function{
       f(x,y) = \Q*(y+\a)/sqrt((y+\a)**2+x**2) - (\Q)*(y-\a)/sqrt((y-\a)**2+x**2);
       set xrange [\xmin:\xmax];
       set yrange [-\ymax:\ymax];
       set view 0,0;
       set isosample 800,800;
       set cont base;
       set cntrparam levels discrete \range;
       unset surface;
       splot f(x,y)
    };
  
  % ELLIPSE INTERSECTIONS
  \path[name path=ellipse] (0,0) ellipse ({1.5*\R} and {\R+\a});
  \path[name path=yline] (\xmin,0) -- (\xmax,0);
  \message{Intersections...}
  \path[name intersections={of=Elines and ellipse,total=\t}]
      \pgfextra{\xdef\Nb{\t}}; 
  \message{ found \Nb ^^J}
  \foreach \i in {1,...,\Nb}{
    \message{  \i}
    \xy{(intersection-\i)}{\xpt}{\ypt}
    \EFielLineArrow{\xpt}{\ypt}
    \message{ (\D,\x,\y)^^J} %,\xpt,\ypt
  }
  \path[name intersections={of=Elines and yline,total=\t}]
      \pgfextra{\xdef\Nb{\t}}; 
  \message{ found \Nb ^^J}
  \foreach \i in {1,...,\Nb}{
    \message{  \i}
    \xy{(intersection-\i)}{\xpt}{\ypt}
    \EFielLineArrow{\xpt}{\ypt}
    \message{ (\D,\x,\y)^^J} %,\xpt,\ypt
  }
}
\tikzset{
  pics/magnet/.style={ %args={#1}
    code={
      \def\h{0.8}
      \coordinate (-N) at (0,\h);
      \coordinate (-S) at (0,-\h);
      \draw[pic actions,thick,top color=red!60,bottom color=red!90,shading angle=20]
        (-0.8*\h/2,0) rectangle ++(0.8*\h,\h);
      \draw[pic actions,thick,top color=blue!60,bottom color=blue!90,shading angle=20]
        (-0.8*\h/2,0) rectangle ++(0.8*\h,-\h);
      \node[pic actions] at (0, \h/2) {\textbf{N}};
      \node[pic actions] at (0,-\h/2) {\textbf{S}};
  }},
  pics/minimagnet/.style={
    code={
      \def\h{0.2}
      \coordinate (-N) at (0,\h);
      \coordinate (-S) at (0,-\h);
      \draw[pic actions,very thin,fill=red!60]
        (-0.7*\h/2,0) -- (0,\h) -- (0.7*\h/2,0) -- cycle;
      \draw[pic actions,very thin,fill=blue!60]
        (-0.7*\h/2,0) -- (0,-\h) -- (0.7*\h/2,0) -- cycle;
      %\node[pic actions] at (0, \h/2) {\textbf{N}};
      %\node[pic actions] at (0,-\h/2) {\textbf{S}};
  }}
}


\begin{document}

% NOTE: this file must be compiled in terminal. Use "pdflatex -shell-escape magnetic_dipole.tex". Otherwise no permission to write .table file


% MAGNET
\begin{tikzpicture}
  \message{Magnet start. ^^J}
  \def\xmin{-3}
  \def\xmax{3}
  \def\ymax{3}
  \def\a{0.4}
  \def\Q{1.0}
  \def\R{1.9}
  \def\range{0.05,0.1,0.2,0.3,0.4}
  %\def\range{0.08,0.16,0.24,0.30,0.36}
  \EFieldLines
  \draw[thick,EFielLineArrow={0.65,180}] (0,-0.8) -- (0,-\ymax);
  \draw[thick,EFielLineArrow={0.65,0}] (0,0.8) -- (0,\ymax);
  \pic[rotate=0] (L) at (0,0) {magnet};
  %\pic[rotate=-55] at (125:\E) {minimagnet};
  \message{Magnet done. ^^J}
\end{tikzpicture}


% MAGNET with PSTricks
% \def\xmax{3.5}
% \def\ymax{3.5}
% \begin{tikzpicture}[shift={(\xmax+0.024,\ymax+0.024)}]
%   \message{Magnet (PSTricks) start. ^^J}
%   \def\H{1}
%   \def\W{0.45}
%   \draw[north] (-\W,0) rectangle (\W, \H);
%   \draw[south] (-\W,0) rectangle (\W,-\H);
%   \begin{scope} %[shift={(3,3)}]
%     \clip (-\xmax,-\ymax) rectangle (\xmax,\ymax); %[shift={(0,-10)}] 
%     %\psset{unit=0.5}
%     \psset{arrowinset=0} % does not work?
%     %\newpsstyle{sensCourant}{arrowinset=0}
%     \begin{pspicture*}(-\xmax,-\ymax)(\xmax,\ymax) %[shift=-10]
%       \psframe[linecolor=white](-\xmax,-\ymax)(\xmax,\ymax)
%       %\psframe[linecolor=black, fillstyle=solid,fillcolor=Ncol](-1,0)(1,3)
%       %\psframe[linecolor=black, fillstyle=solid,fillcolor=Scol](-1,0)(1,-3)
%       \psmagneticfield[
%           N=120,R=\W,L=2, %-0.005,
%           nL=4,pointsB=800,
%           nS=1,numSpires=10,pointsS=1500,
%           linewidth=1.0pt,linecolor=Bcol,drawSelf=false
%         ](-\xmax,-\ymax)(\xmax,\ymax)
%       %\rput(0,-5.2){\textcolor{white}{S}}
%       %\rput(0,5.2){\textcolor{white}{N}}
%     \end{pspicture*}
%   \end{scope}
%   \node[scale=1.3] at (0, \H/2) {\contour{red!80}{N}};
%   \node[scale=1.3] at (0,-\H/2) {\contour{blue!70}{S}};
%   \message{Magnet (PSTricks) done. ^^J}
% \end{tikzpicture}


% % EARTH
% \begin{tikzpicture}
%   \message{Earth start. ^^J}
%   \def\xmin{-3}
%   \def\xmax{3}
%   \def\ymax{3}
%   \def\a{0.60}
%   \def\Q{-1.0}
%   \def\R{1.76}
%   \def\E{1.2}
%   %\def\range{0.005,0.02,0.05,0.10}
%   \def\range{-0.05,-0.15,-0.25,-0.35,-0.45}
%   \EFieldLines
%   \draw[thick,EFielLineArrow={0.74,0}] (0,-0.8) -- (0,-\ymax);
%   \draw[thick,EFielLineArrow={0.64,180}] (0,0.8) -- (0,\ymax);
%   \draw[dashed,rotate=-11] (0,-1.5*\E) -- (0,1.5*\E);
%   \fill[blue!70!black!80] (0,0) circle (1.2);
%   \draw[ball color=blue!70!black!90,fill opacity=0.3] (0,0) circle (\E);
%   \begin{scope}[rotate=-11]
%     \clip (0,0) circle (\E);
%     \fill[white] (0,\E) ellipse ({0.6*\E} and {0.10*\E});
%     \fill[white] (0,-\E) ellipse ({0.8*\E} and {0.15*\E});
%     \fill[green!70!black,rotate=-30] (160:1.1*\E) ellipse ({0.2*\E} and {0.8*\E});
%     \fill[green!70!black,rotate=40] (-10:1.14*\E) ellipse ({0.2*\E} and {0.9*\E});
%     %\fill[brown!70!black!60,draw=green!60!black,very thick,rotate=-20]
%     %  (230:0.86*\E) ellipse ({0.25*\E} and {0.2*\E});
%     \fill[green!60!black,very thick,rotate=-20] % Australia
%       (230:0.86*\E) ellipse ({0.25*\E} and {0.18*\E});
%     \draw[dashed] (-\E,0) -- (\E,0);
%   \end{scope}
%   \pic[rotate=180,scale=1.0] at (0,0) {magnet};
%   %\pic[rotate=0] at (0,0) {minimagnet};
%   %\pic[rotate=0] at (-\E,0) {minimagnet};
%   %\pic[rotate=0] at (\E,0) {minimagnet};
%   \pic[rotate=-55] at (125:\E) {minimagnet};
%   \pic[rotate=-55] at (-55:\E) {minimagnet};
%   \pic[rotate=-11] at (180-11:\E) {minimagnet};
%   \pic[rotate=-11] at (-11:\E) {minimagnet};
%   \message{Earth done. ^^J}
% \end{tikzpicture}


\end{document}