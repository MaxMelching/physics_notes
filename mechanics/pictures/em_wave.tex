% Author: Max Melching (2025)
% Adapted from the TiKz manual, https://tikz.dev/library-3d#sec-40.3
% Heavily inspired by Fig. 9.10 in Griffiths Electrodynamics (4th edition)
\documentclass[border=3pt,tikz]{standalone}

\usepackage{physics}
\usepackage{tikz}
\usepackage{tikz-3dplot}
\usepackage[outline]{contour} % glow around text
\usepackage{xcolor}

\usepackage{newtxmath}
\usepackage{tgpagella}

\colorlet{mygreen}{green!60!black}
\colorlet{myred}{red!90!black}
\colorlet{myblue}{blue!80!black}


\usetikzlibrary{calc, 3d, arrows.meta, decorations.markings, decorations.pathreplacing}


\tikzset{
    >={Stealth[inset=0,angle'=27]},
    surface/.style={
        left color=gray!10,
        right color=gray!80,
    },
    labelarrow/.style={
        ->,
        thick,
        myblue,
    },
}


% \tdplotsetmaincoords{60}{105}
\tdplotsetmaincoords{70}{115}


\begin{document}

\begin{tikzpicture}[
  % % z={(10:10mm)},x={(-45:5mm)},  % Default
  % z={(-20:5mm)},x={(-45:5mm)},
  tdplot_main_coords,
  rotate around y=-90,
  rotate around x=-90,
]
  \def\wave{
    \draw[fill,thick,fill opacity=.2]
     (0,0) sin (1,1) cos (2,0) sin (3,-1) cos (4,0)
           sin (5,1) cos (6,0) sin (7,-1) cos (8,0)
          %  sin (9,1) cos (10,0)sin (11,-1)cos (12,0)
           ;
    % \foreach \shift in {0,4,8}
    \foreach \shift in {0,4}
    {
      \begin{scope}[
        xshift=\shift cm,
        thin,
        ->,
        % help lines,
        opacity=0.6,
      ]
        \draw (.5,0)  -- (0.5,0 |- 45:1cm);
        \draw (1,0)   -- (1,1);
        \draw (1.5,0) -- (1.5,0 |- 45:1cm);
        \draw (2.5,0) -- (2.5,0 |- -45:1cm);
        \draw (3,0)   -- (3,-1);
        \draw (3.5,0) -- (3.5,0 |- -45:1cm);
      \end{scope}
    }
  }

  \begin{scope}[
    ->,
    black,
    semithick,
  ]
    \draw (0,0,0) -- (2,0,0) node[pos=1.1] {$x$};
    \draw (0,0,0) -- (0,2,0) node[pos=1.1] {$y$};
    \draw (0,0,0) -- (0,0,9) node[pos=1.01] {$z$};
  \end{scope}

  \begin{scope}[canvas is zy plane at x=0,fill=blue]
    \wave
    % \node at (6,-1.5) [transform shape] {magnetic field};
  \end{scope}

  \begin{scope}[canvas is zx plane at y=0,fill=red]
    % % \draw[help lines] (0,-2) grid (12,2);
    % \draw[help lines] (0,-2) grid (8,2);
    \wave
    % \node at (6,1.5) [rotate=180,xscale=-1,transform shape] {electric field};
  \end{scope}


  \draw[dashed, thin] (0,0,0) -- ++(1,0,0) node[left] {$E_0$} -- ++(0,0,9) -- ++(-1,0,0);
  \draw[dashed, thin] (0,0,0) -- ++(0,1,0) node[above left=-3] {$B_0 = E_0 / c$} -- ++(0,0,9) -- ++ (0,-1,0);

  \node[above] at (1,0,1) {$\vec{E}$};
  \node[below left=-3] at (0,1,1) {$\vec{B}$};

  \draw[->, thick, shift={(1.25,0,4.5)}] (0,0,0) -- ++(0,0,1) node[midway, above] {$c$};
\end{tikzpicture}


\end{document}