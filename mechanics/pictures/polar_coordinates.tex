% Author: Max Melching (2025)
% Inspiration: https://tikz.net/axis3d/
\documentclass[border=3pt,tikz]{standalone}
\input{../colors.tex}

\usepackage{physics}
\usepackage{tikz}
\usepackage{tikz-3dplot}
\usepackage[outline]{contour} % glow around text
\usepackage{xcolor}

\usepackage{newtxmath}
\usepackage{tgpagella}

\colorlet{mygreen}{green!60!black}
\colorlet{myred}{red!75!black}
\colorlet{myblue}{blue!70!black}


\usetikzlibrary{calc, 3d, arrows.meta}


\tikzset{
    >={Stealth[inset=0,angle'=27]},
    vector/.style={
        ->,
        myred,
        thick,
    },
    unit vector cartesian/.style={
        ->,
        thick,
        black,
    },
    unit vector non cartesian/.style={
        ->,
        semithick,
        myblue,
    },
    vector projection/.style={
        myred!90!black,
        dashed,
        thin,
    },
    angles/.style={
        ->,
        mygreen,
    },
}


% -- Quick command for variable vector labelling
\def\unitvec#1{\hat{#1}}
% \def\unitvec#1{\vec{e}_{#1}}



\begin{document}

\def\rvec{0.8}
\def\phivec{60}

\def\axislen{1}
\def\axislabelpad{0.1}


\begin{tikzpicture}[scale=2.8]
    \coordinate (O) at (0,0);
    
    \begin{scope}[
        unit vector cartesian,
        black,
        shift={(O)},
    ]
        \draw (0,0) --++ (\axislen,0);
        \node at ({\axislen+\axislabelpad},0) {\contour{white}{$\unitvec{x}$}};
        
        \draw (0,0) --++ (0,\axislen);
        \node at (0,{\axislen+\axislabelpad}) {\contour{white}{$\unitvec{y}$}};
    \end{scope}

    
    \coordinate (P) at ({\rvec*cos(\phivec)},{\rvec*sin(\phivec)});
    \coordinate (Px) at ({\rvec*cos(\phivec)},0);
    \coordinate (Py) at (0,{\rvec*sin(\phivec)});

    \draw[vector] (O)  -- (P);% node[right=-5] {$P = (r, \phi)$};
    \node[vector] at (0.8*\axislen,0.8*\axislen) {$P = (r, \phi)$};

    
    \def\fontscale{0.9}

    \begin{scope}[
        unit vector non cartesian,
        rotate=\phivec,
        shift={(P)},
        scale=0.33,
    ]
        \draw (0,0) --++ (0,\axislen);
        \node[scale=\fontscale] at (0,{\axislen+\axislabelpad}) {\contour{white}{$\unitvec{\phi}$}};
        
        \draw (0,0) --++ (\axislen,0);
        \node[scale=\fontscale] at ({\axislen+\axislabelpad},0) {\contour{white}{$\unitvec{r}$}};
    \end{scope}


    \begin{scope}[
        vector projection,
    ]
        \def\fontscale{0.75}
        \draw (P) -- (Px) node[below,scale=\fontscale] {\contour{white}{$r \cos(\phi)$}};
        \draw (P) -- (Py) node[left,scale=\fontscale] {\contour{white}{$r \sin(\phi)$}};
    \end{scope}


    % \draw[angles] (\rvec,0) arc(0:\phivec:\rvec) node[midway,above right=-2] {$\phi$};
    \draw[angles] (Px) arc(0:\phivec:{\rvec*cos(\phivec)}) node[midway,below left=-2] {$\phi$};
    
    \coordinate (Pangle) at ({\rvec*cos(\phivec)*cos(\phivec)},{\rvec*cos(\phivec)*sin(\phivec)});  % Use that radius = projection on x-component

    \begin{scope}[
        unit vector non cartesian,
        rotate=\phivec,
        % xshift={\rvec*cos(\phivec)},
        % xshift=11.4,  % Manual, don't know why this is needed...
        shift={(Pangle)},
        scale=0.33,
        densely dotted,
    ]
        \draw (0,0) --++ (0,\axislen);
        \node[scale=\fontscale] at (0,{\axislen+\axislabelpad}) {\contour{white}{$\unitvec{\phi}$}};
    \end{scope}
\end{tikzpicture}


\end{document}