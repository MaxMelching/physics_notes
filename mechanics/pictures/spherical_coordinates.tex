% Author: Max Melching (2025)
% Inspiration: https://tikz.net/axis3d/
\documentclass[border=3pt,tikz]{standalone}
\input{../colors.tex}

\usepackage{physics}
\usepackage{tikz}
\usepackage{tikz-3dplot}
\usepackage[outline]{contour} % glow around text
\usepackage{xcolor}

\usepackage{newtxmath}
\usepackage{tgpagella}

\colorlet{mygreen}{green!60!black}
\colorlet{myred}{red!75!black}
\colorlet{myblue}{blue!70!black}


\usetikzlibrary{calc, 3d, arrows.meta}


\tikzset{
    >={Stealth[inset=0,angle'=27]},
    vector/.style={
        ->,
        myred,
        thick,
    },
    unit vector cartesian/.style={
        ->,
        thick,
        black,
    },
    unit vector non cartesian/.style={
        ->,
        semithick,
        myblue,
    },
    vector projection/.style={
        myred!90!black,
        dashed,
        thin,
    },
    angles/.style={
        ->,
        mygreen,
    },
}


% -- Quick command for variable vector labelling
\def\unitvec#1{\hat{#1}}
% \def\unitvec#1{\vec{e}_{#1}}



\begin{document}

\def\rvec{0.8}
\def\thetavec{30}
\def\phivec{60}

\def\axislen{1}
\def\axislabelpad{0.1}


\tdplotsetmaincoords{60}{110}
\begin{tikzpicture}[scale=2.8,tdplot_main_coords]
    \coordinate (O) at (0,0,0);
    
    \begin{scope}[
        unit vector cartesian,
        black,
        shift={(O)},
    ]
        \draw (0,0,0) --++ (\axislen,0,0);
        \node at ({\axislen+\axislabelpad},0,0) {\contour{white}{$\unitvec{x}$}};
        
        \draw (0,0,0) --++ (0,\axislen,0);
        \node at (0,{\axislen+\axislabelpad},0) {\contour{white}{$\unitvec{y}$}};
        
        \draw (0,0,0) --++ (0,0,\axislen);
        \node at (0,0,{\axislen+\axislabelpad}) {\contour{white}{$\unitvec{z}$}};
    \end{scope}

    
    \tdplotsetcoord{P}{\rvec}{\thetavec}{\phivec}

    \draw[vector] (O)  -- (P);% node[right=-5] {$P = (r, \theta, \phi)$};
    \node[vector] at (0.8*\axislen,0.8*\axislen,0) {$P = (r, \theta, \phi)$};

    \def\fontscale{0.9}

    \begin{scope}[
        vector projection,
    ]
        % -- V1
        % \draw (O) -- (Pxy);
        % \draw (P) -- (Pxy);
        % \draw (Px) -- (Pxy);
        % \draw (Py) -- (Pxy);
        % \draw (P) -- (Pz);
        
        % -- V2
        \def\fontscale{0.75}
        \draw (P) -- (Pxy);
        \draw (Pxy) -- (Px) node[left,scale=\fontscale] {\contour{white}{$r \cos(\phi) \sin(\theta)$}};
        \draw (Pxy) -- (Py) node[above right=-1,scale=\fontscale] {\contour{white}{$r \sin(\phi) \sin(\theta)$}};
        \draw (P) -- (Pz) node[left,scale=\fontscale] {\contour{white}{$r \cos(\theta)$}};
    \end{scope}


    \begin{scope}[
        angles,
    ]
        \tdplotdrawarc[->,scale=sin(\thetavec)]{(0,0,0)}{\rvec}{0}{\phivec}{anchor=north,scale=\fontscale}{\contour{white}{$\phi$}}
        
        \tdplotsetthetaplanecoords{\phivec}
        \tdplotdrawarc[->,tdplot_rotated_coords]{(0,0,0)}{\rvec}{0}{\thetavec}{anchor=south west,scale=\fontscale}{\hspace{-1mm}\contour{white}{$\theta$}}
    \end{scope}

    
    \begin{scope}[
        unit vector non cartesian,
        rotate around z=\phivec,
        rotate around y=\thetavec,
        shift={(P)},
        scale=0.33,
    ]
        \draw (0,0,0) --++ (\axislen,0,0);
        \node[scale=\fontscale] at ({\axislen+\axislabelpad},0,0) {\contour{white}{$\unitvec{\theta}$}};
        
        \draw (0,0,0) --++ (0,\axislen,0);
        \node[scale=\fontscale] at (0,{\axislen+\axislabelpad},0) {\contour{white}{$\unitvec{\phi}$}};
        
        \begin{scope}[
            shift={(Pxy)},
            rotate around y=-\thetavec,
            densely dotted,
        ]
            % \draw (0,0,0) --++ (0,\axislen,0);
            % \node[midway,below,scale=\fontscale] at (0,{\axislen+\axislabelpad},0) {\contour{white}{$\unitvec{\phi}$}};

            \draw (0,0,0) --++ (0,\axislen,0) node[midway,below,scale=\fontscale] at (0,{\axislen+\axislabelpad},0) {\contour{white}{$\unitvec{\phi}$}};
        \end{scope}
        
        \draw (0,0,0) --++ (0,0,\axislen);
        \node[scale=\fontscale] at (0,0,{\axislen+\axislabelpad}) {\contour{white}{$\unitvec{r}$}};
    \end{scope}
\end{tikzpicture}


\end{document}