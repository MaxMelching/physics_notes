% Author: Max Melching (2025)
% Adapted from the TiKz manual, https://tikz.dev/library-3d#sec-40.3
% Heavily inspired by Fig. 9.10 in Griffiths Electrodynamics (4th edition)
\documentclass[border=3pt,tikz]{standalone}

\usepackage{physics}
\usepackage{tikz}
\usepackage{tikz-3dplot}
\usepackage[outline]{contour} % glow around text
\usepackage{xcolor}

\usepackage{newtxmath}
\usepackage{tgpagella}

\colorlet{mygreen}{green!60!black}
\colorlet{myred}{red!90!black}
\colorlet{myblue}{blue!80!black}


\usetikzlibrary{calc, 3d, arrows.meta, decorations.markings, decorations.pathreplacing}


\tikzset{
    >={Stealth[inset=0,angle'=27]},
    surface/.style={
        left color=gray!10,
        right color=gray!80,
    },
    labelarrow/.style={
        ->,
        thick,
        myblue,
    },
}


\def\N{99}
\def\zmax{8}
\def\zdampingtime{\zmax*0.8}


% \tdplotsetmaincoords{60}{105}
\tdplotsetmaincoords{70}{115}


\begin{document}

\begin{tikzpicture}[
  % % z={(10:10mm)},x={(-45:5mm)},  % Default
  % z={(-20:5mm)},x={(-45:5mm)},
  tdplot_main_coords,
  rotate around y=-90,
  rotate around x=-90,
]
  \def\wave#1{
    % \draw[fill,thick,fill opacity=.2]
    %  (0,0) sin (1,1) cos (2,0) sin (3,-1) cos (4,0)
    %        sin (5,1) cos (6,0) sin (7,-1) cos (8,0)
    %       %  sin (9,1) cos (10,0)sin (11,-1)cos (12,0)
    %        ;
    
    % \draw[fill,thick,fill opacity=.2,samples=\N,smooth,variable=\z,domain=0:\zmax] plot ({\z},{sin(2*pi*\z*2/\zmax r+#1 r)});

    % \draw[fill,thick,fill opacity=.2,samples=\N,smooth,variable=\z,domain=0:\zmax] (0,0) -- plot%[name=\plottedwave]
    % % ({\z},{sin(2*pi*14.34*\z+#1)}) -- (\zmax,0);  % Try & error...
    % ({\z},{sin(2*pi*\z*2/\zmax r+#1 r)}) -- (\zmax,0);  % Achieves period of \zmax/2

    % -- Trick: draw full period, clip to correct range later on
    \draw[fill,thick,fill opacity=.2,samples=\N,smooth,variable=\z,domain=-#1:\zmax+#1] plot ({\z},{exp(-\z/\zdampingtime)*sin(2*pi*(\z+#1)*2/\zmax r)});  % Achieves period of \zmax/2

    % % \foreach \shift in {0,4,8}
    % \foreach \shift in {0,4}
    % {
    %   \begin{scope}[
    %     xshift=\shift cm,
    %     thin,
    %     ->,
    %     % help lines,
    %     opacity=0.6,
    %   ]
    %     \draw (.5,0)  -- (0.5,0 |- 45:1cm);
    %     \draw (1,0)   -- (1,1);
    %     \draw (1.5,0) -- (1.5,0 |- 45:1cm);
    %     \draw (2.5,0) -- (2.5,0 |- -45:1cm);
    %     \draw (3,0)   -- (3,-1);
    %     \draw (3.5,0) -- (3.5,0 |- -45:1cm);
    %   \end{scope}
    % }

    \foreach \z [evaluate=\z as \z using \z*\zmax/2-#1] in {0.125, 0.25, 0.375, 0.625, 0.75, 0.875}
    {
      \draw[->, thin, opacity=0.6] (\z, 0) -- (\z, {exp(-\z/\zdampingtime)*sin(2*pi*(\z+#1)*2/\zmax r)});

      \pgfmathsetmacro{\z}{\z+\zmax/2}  % Second period
      \draw[->, thin, opacity=0.6] (\z, 0) -- (\z, {exp(-\z/\zdampingtime)*sin(2*pi*(\z+#1)*2/\zmax r)});

      \pgfmathsetmacro{\z}{\z+\zmax/2}  % Third period (may be started due to shift)
      % \draw[->, thin, opacity=0.6] (\z, 0) -- (\z, {exp(-\z/\zdampingtime)*sin(2*pi*(\z+#1)*2/\zmax r)});
      % TODO: only draw third period if #1!=0?
    }

    \draw[dashed,thin,samples=\N,smooth,variable=\z,domain=-#1:\zmax+#1] plot ({\z},{exp(-\z/\zdampingtime)});

    \draw[dashed, thin] (0,0) -- (0,1);
    \draw[dashed, thin] (\zmax,0) -- (\zmax,{exp(-\zmax/\zdampingtime)});
  }

  \begin{scope}[
    ->,
    black,
    semithick,
  ]
    \draw (0,0,0) -- (2,0,0) node[pos=1.1] {$x$};
    \draw (0,0,0) -- (0,2,0) node[pos=1.1] {$y$};
    \draw (0,0,0) -- (0,0,9) node[pos=1.01] {$z$};
  \end{scope}

  \begin{scope}[canvas is zy plane at x=0,fill=blue]
    % \wave{pi/2}

    \clip (0,-2) rectangle (\zmax, 2);  % Generous clipping in y-direction; x-direction much more important
    \wave{1}  % For shift in time units
  \end{scope}

  \begin{scope}[canvas is zx plane at y=0,fill=red]
    % % \draw[help lines] (0,-2) grid (12,2);
    % \draw[help lines] (0,-2) grid (8,2);

    % \clip (0,-2) rectangle (\zmax, 2);  % Generous clipping in y-direction; x-direction much more important -> do not show arrows in third period
    \wave{0}
  \end{scope}


  \draw[dashed, thin] (1,0,0) node[left] {$E_0$};
  \draw[dashed, thin] (0,1,0) node[above left=-3] {$B_0 = E_0 / c$};

  \node[above] at (1,0,1) {$\vec{E}$};
  \node[below=-2] at (0,1,0) {$\vec{B}$};

  \draw[->, thick, shift={(1.25,0,4.5)}] (0,0,0) -- ++(0,0,1) node[midway, above] {$c/n$};
\end{tikzpicture}


\end{document}