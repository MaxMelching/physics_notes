% Author: Max Melching (2025)
% Inspiration: https://tikz.net/axis3d/
\documentclass[border=3pt,tikz]{standalone}
\usepackage{physics}
\usepackage{tikz}
\usepackage{tikz-3dplot}
\usepackage[outline]{contour} % glow around text
\usepackage{xcolor}

\colorlet{mygreen}{green!50!black}
\colorlet{myblue}{blue!80!black}
\colorlet{myred}{red!90!black}
\colorlet{mydarkblue}{blue!50!black}


\usetikzlibrary{calc, 3d, arrows.meta, topaths}


\tikzset{
    >={Stealth[inset=0,angle'=27]},
    vector/.style={
        ->,
        % myblue,
        myred,
        thick,
    },
    unit vector cartesian/.style={
        ->,
        thick,
        black,
    },
    unit vector non cartesian/.style={
        ->,
        % thick,
        % mygreen,
        myblue,
    },
    vector projection/.style={
        % myblue!80!black,
        myred!80!black,
        dashed,
        % line width=0.08,
        thin,
    },
    angles/.style={
        ->,
        % myblue,
        mygreen,
    },
    element/.style={
        % draw=none,
        purple,
        fill,
        fill opacity=0.5,
    },
    element lines/.style={
        purple,
        % opacity=0.5,
    },
    element helplines/.style={
        thin,
        % black,
        purple,
        dashed,
        % densely dotted,
        opacity=0.5,
    },
}


% -- Quick command for variable vector labelling
\def\unitvec#1{\hat{#1}}
% \def\unitvec#1{\vec{e}_{#1}}



\begin{document}

\def\rvec{0.8}
% \def\thetavec{30}
% \def\phivec{36}
\def\thetavec{42}
\def\phivec{50}

\def\axislen{1}
\def\axislabelpad{0.1}


\tdplotsetmaincoords{60}{110}
\begin{tikzpicture}[scale=2.8,tdplot_main_coords]
    \coordinate (O) at (0,0,0);
    
    \begin{scope}[
        unit vector cartesian,
        black,
        shift={(O)},
    ]
        \draw (0,0,0) --++ (\axislen,0,0);
        \node at ({\axislen+\axislabelpad},0,0) {$\unitvec{x}$};
        
        \draw (0,0,0) --++ (0,\axislen,0);
        \node at (0,{\axislen+\axislabelpad},0) {$\unitvec{y}$};
        
        \draw (0,0,0) --++ (0,0,\axislen);
        \node at (0,0,{\axislen+\axislabelpad}) {$\unitvec{z}$};
    \end{scope}

    
    \tdplotsetcoord{P}{\rvec}{\thetavec}{\phivec}

    \draw[vector] (O)  -- (P);


    \def\dr{0.16}
    \def\dtheta{8}
    \def\dphi{8}

    % \tdplotsetcoord{Pppp}{\rvec+\dr}{\thetavec+\dtheta}{\phivec+\dphi}
    % \tdplotsetcoord{Ppmp}{\rvec+\dr}{\thetavec-\dtheta}{\phivec+\dphi}
    % \tdplotsetcoord{Pppm}{\rvec+\dr}{\thetavec+\dtheta}{\phivec-\dphi}
    % \tdplotsetcoord{Ppmm}{\rvec+\dr}{\thetavec-\dtheta}{\phivec-\dphi}
    % -- No +dr, otherwise we have to use $(r+dr)d\theta$ etc. Could neglect terms quadratic in $dr$ in the end, so same result, but unnecessarily complicated
    \tdplotsetcoord{Pppp}{\rvec}{\thetavec+\dtheta}{\phivec+\dphi}
    \tdplotsetcoord{Ppmp}{\rvec}{\thetavec-\dtheta}{\phivec+\dphi}
    \tdplotsetcoord{Pppm}{\rvec}{\thetavec+\dtheta}{\phivec-\dphi}
    \tdplotsetcoord{Ppmm}{\rvec}{\thetavec-\dtheta}{\phivec-\dphi}
    \tdplotsetcoord{Pmpp}{\rvec-\dr}{\thetavec+\dtheta}{\phivec+\dphi}
    \tdplotsetcoord{Pmmp}{\rvec-\dr}{\thetavec-\dtheta}{\phivec+\dphi}
    \tdplotsetcoord{Pmpm}{\rvec-\dr}{\thetavec+\dtheta}{\phivec-\dphi}
    \tdplotsetcoord{Pmmm}{\rvec-\dr}{\thetavec-\dtheta}{\phivec-\dphi}


    \begin{scope}[
        element helplines,
    ]
        \draw (O) -- (Pmpp);
        \draw (O) -- (Pmmp);
        \draw (O) -- (Pmpm);
        \draw (O) -- (Pmmm);
    \end{scope}
    
    \draw[element] (Pppp) -- (Pppm) -- (Ppmm) -- (Ppmp) -- cycle;
    % \draw[element] (Pmpp) -- (Pmpm) -- (Pmmm) -- (Pmmp) -- cycle;

    % \draw[element] (Pppp) -- (Pmpp) -- (Pmmp) -- (Ppmp) -- cycle;
    \draw[element] (Pppm) -- (Pmpm) -- (Pmmm) -- (Ppmm) -- cycle;

    \draw[element] (Pppp) -- (Pmpp) -- (Pmpm) -- (Pppm) -- cycle;
    % \draw[element] (Ppmp) -- (Pmmp) -- (Pmmm) -- (Ppmm) -- cycle;

    % -- Omitting some fillings here to get better fill color


    \def\fontscale{0.8}

    \begin{scope}[
        element lines,
    ]
        % +dr
        % \tdplotsetthetaplanecoords{\phivec-\dphi}
        % \tdplotdrawarc[tdplot_rotated_coords]{(0,0,0)}{{\rvec+\dr}}{\thetavec-\dtheta}{\thetavec+\dtheta}{}{}

        % \tdplotdrawarc[element helplines,tdplot_rotated_coords]{(0,0,0)}{{\rvec+\dr}}{0}{\thetavec-\dtheta}{}{}
        % \tdplotdrawarc[element helplines,tdplot_rotated_coords]{(0,0,0)}{{\rvec+\dr}}{\thetavec+\dtheta}{90}{}{}

        % \tdplotsetthetaplanecoords{\phivec+\dphi}
        % \tdplotdrawarc[tdplot_rotated_coords]{(0,0,0)}{{\rvec+\dr}}{\thetavec-\dtheta}{\thetavec+\dtheta}{above right=-2,scale=\fontscale}{$r d\theta$}

        
        % \tdplotdrawarc[element helplines,tdplot_rotated_coords]{(0,0,0)}{{\rvec+\dr}}{0}{\thetavec-\dtheta}{}{}
        % \tdplotdrawarc[element helplines,tdplot_rotated_coords]{(0,0,0)}{{\rvec+\dr}}{\thetavec+\dtheta}{90}{}{}
        
        % \tdplotdrawarc[element helplines]{(0,0,0)}{{\rvec+\dr}}{\phivec-\dphi}{\phivec+\dphi}{}{}

        
        \tdplotsetthetaplanecoords{\phivec-\dphi}
        \tdplotdrawarc[tdplot_rotated_coords]{(0,0,0)}{\rvec}{\thetavec-\dtheta}{\thetavec+\dtheta}{}{}

        \tdplotdrawarc[element helplines,tdplot_rotated_coords]{(0,0,0)}{\rvec}{0}{\thetavec-\dtheta}{}{}
        \tdplotdrawarc[element helplines,tdplot_rotated_coords]{(0,0,0)}{\rvec}{\thetavec+\dtheta}{90}{}{}

        \tdplotsetthetaplanecoords{\phivec+\dphi}
        \tdplotdrawarc[tdplot_rotated_coords]{(0,0,0)}{\rvec}{\thetavec-\dtheta}{\thetavec+\dtheta}{above right=-2,scale=\fontscale}{\contour{white}{$r d\theta$}}

        
        \tdplotdrawarc[element helplines,tdplot_rotated_coords]{(0,0,0)}{\rvec}{0}{\thetavec-\dtheta}{}{}
        \tdplotdrawarc[element helplines,tdplot_rotated_coords]{(0,0,0)}{\rvec}{\thetavec+\dtheta}{90}{}{}
        
        \tdplotdrawarc[element helplines]{(0,0,0)}{\rvec}{\phivec-\dphi}{\phivec+\dphi}{}{}


        

        % -dr
        \tdplotsetthetaplanecoords{\phivec-\dphi}
        \tdplotdrawarc[tdplot_rotated_coords]{(0,0,0)}{{\rvec-\dr}}{\thetavec-\dtheta}{\thetavec+\dtheta}{}{}

        \tdplotdrawarc[element helplines,tdplot_rotated_coords]{(0,0,0)}{{\rvec-\dr}}{0}{\thetavec-\dtheta}{}{}
        \tdplotdrawarc[element helplines,tdplot_rotated_coords]{(0,0,0)}{{\rvec-\dr}}{\thetavec+\dtheta}{90}{}{}

        \tdplotsetthetaplanecoords{\phivec+\dphi}
        \tdplotdrawarc[tdplot_rotated_coords]{(0,0,0)}{{\rvec-\dr}}{\thetavec-\dtheta}{\thetavec+\dtheta}{}{}

        
        \tdplotdrawarc[element helplines,tdplot_rotated_coords]{(0,0,0)}{{\rvec-\dr}}{0}{\thetavec-\dtheta}{}{}
        \tdplotdrawarc[element helplines,tdplot_rotated_coords]{(0,0,0)}{{\rvec-\dr}}{\thetavec+\dtheta}{90}{}{}
        
        \tdplotdrawarc[element helplines]{(0,0,0)}{{\rvec-\dr}}{\phivec-\dphi}{\phivec+\dphi}{}{}
    \end{scope}

    % \draw[element lines] (Pppm) -- (Pppp) node[midway,below right=-2,scale=\fontscale] {$r \sin(\theta) d\phi$};

    % \tdplotsetcoord{Plabel}{\rvec}{\thetavec+\dtheta}{\phivec-0.5\dphi}
    \tdplotsetcoord{Plabel}{\rvec-\dr}{\thetavec+\dtheta}{\phivec-0.5\dphi}  % Technically not correct position, but looks much better
    \node[element lines,below right=-4,scale=\fontscale] at (Plabel) {\contour{white}{$r \sin(\theta) d\phi$}};
    \tdplotsetcoord{Plabel}{\rvec-\dr}{\thetavec-\dtheta}{\phivec-0.5*\dphi}
    \node[element lines,above left=-1,scale=\fontscale] at (Plabel) {\contour{white}{$dr$}};


    % -- Some more helper lines
    \begin{scope}[
        element helplines,
    ]
        \draw[rotate around z=\phivec-\dphi] (\rvec-\dr,0,0) -- (\rvec,0,0);
        \draw[rotate around z=\phivec+\dphi] (\rvec-\dr,0,0) -- (\rvec,0,0);

        \draw (0,0,\rvec-\dr) -- (0,0,\rvec);
    \end{scope}
\end{tikzpicture}


\end{document}