% Author: Max Melching (2025)
% Inspiration: https://tikz.net/axis3d/
\documentclass[border=3pt,tikz]{standalone}
\usepackage{physics}
\usepackage{tikz}
\usepackage{tikz-3dplot}
\usepackage[outline]{contour} % glow around text
\usepackage{xcolor}

\colorlet{mygreen}{green!50!black}
\colorlet{myblue}{blue!80!black}
\colorlet{myred}{red!90!black}
\colorlet{mydarkblue}{blue!50!black}


\usetikzlibrary{calc, 3d, arrows.meta, topaths}


\tikzset{
    >={Stealth[inset=0,angle'=27]},
    vector/.style={
        ->,
        % myblue,
        myred,
        thick,
    },
    unit vector cartesian/.style={
        ->,
        thick,
        black,
    },
    unit vector non cartesian/.style={
        ->,
        % thick,
        % mygreen,
        myblue,
    },
    vector projection/.style={
        % myblue!80!black,
        myred!80!black,
        dashed,
        % line width=0.08,
        thin,
    },
    angles/.style={
        ->,
        % myblue,
        mygreen,
    },
    element/.style={
        draw=none,
        purple,
        fill,
        fill opacity=0.5,
    },
    element lines/.style={
        purple,
        % opacity=0.5,
    },
    element helplines/.style={
        thin,
        % black,
        purple,
        dashed,
        % densely dotted,
        opacity=0.5,
    },
}


% -- Quick command for variable vector labelling
\def\unitvec#1{\hat{#1}}
% \def\unitvec#1{\vec{e}_{#1}}



\begin{document}

\def\rvec{0.8}
% \def\thetavec{30}
% \def\phivec{36}
\def\thetavec{42}
\def\phivec{42}

\def\axislen{1}
\def\axislabelpad{0.1}


\tdplotsetmaincoords{60}{110}
\begin{tikzpicture}[scale=2.8,tdplot_main_coords]
    \coordinate (O) at (0,0,0);
    
    \begin{scope}[
        unit vector cartesian,
        black,
        shift={(O)},
    ]
        \draw (0,0,0) --++ (\axislen,0,0);
        \node at ({\axislen+\axislabelpad},0,0) {$\unitvec{x}$};
        
        \draw (0,0,0) --++ (0,\axislen,0);
        \node at (0,{\axislen+\axislabelpad},0) {$\unitvec{y}$};
        
        \draw (0,0,0) --++ (0,0,\axislen);
        \node at (0,0,{\axislen+\axislabelpad}) {$\unitvec{z}$};
    \end{scope}

    
    \tdplotsetcoord{P}{\rvec}{\thetavec}{\phivec}

    \draw[vector] (O)  -- (P);


    \def\dtheta{8}
    \def\dphi{8}

    \tdplotsetcoord{Ppp}{\rvec}{\thetavec+\dtheta}{\phivec+\dphi}
    \tdplotsetcoord{Pmp}{\rvec}{\thetavec-\dtheta}{\phivec+\dphi}
    \tdplotsetcoord{Ppm}{\rvec}{\thetavec+\dtheta}{\phivec-\dphi}
    \tdplotsetcoord{Pmm}{\rvec}{\thetavec-\dtheta}{\phivec-\dphi}

    \begin{scope}[
        element helplines,
    ]
        \draw (O) -- (Ppp);
        \draw (O) -- (Pmp);
        \draw (O) -- (Ppm);
        \draw (O) -- (Pmm);
    \end{scope}
    
    \draw[element] (Ppp) -- (Ppm) -- (Pmm) -- (Pmp) -- cycle;
    % \draw[element] (Ppp) to (Ppm) to (Pmm) to (Pmp) to cycle;


    \begin{scope}[
        element lines,
    ]
        \def\fontscale{0.8}

        \tdplotsetthetaplanecoords{\phivec-\dphi}
        \tdplotdrawarc[tdplot_rotated_coords]{(0,0,0)}{\rvec}{\thetavec-\dtheta}{\thetavec+\dtheta}{}{}

        % \tdplotdrawarc[element lines,tdplot_rotated_coords]{(0,0,0)}{\rvec}{0}{90}{}{}
        \tdplotdrawarc[element helplines,tdplot_rotated_coords]{(0,0,0)}{\rvec}{0}{\thetavec-\dtheta}{}{}
        \tdplotdrawarc[element helplines,tdplot_rotated_coords]{(0,0,0)}{\rvec}{\thetavec+\dtheta}{90}{}{}

        \tdplotsetthetaplanecoords{\phivec+\dphi}
        \tdplotdrawarc[tdplot_rotated_coords]{(0,0,0)}{\rvec}{\thetavec-\dtheta}{\thetavec+\dtheta}{above right=-2,scale=\fontscale}{\contour{white}{$r d\theta$}}

        % \tdplotdrawarc[element helplines,tdplot_rotated_coords]{(0,0,0)}{\rvec}{0}{90}{}{}
        \tdplotdrawarc[element helplines,tdplot_rotated_coords]{(0,0,0)}{\rvec}{0}{\thetavec-\dtheta}{}{}
        \tdplotdrawarc[element helplines,tdplot_rotated_coords]{(0,0,0)}{\rvec}{\thetavec+\dtheta}{90}{}{}

        % \tdplotsetphiplanecoords{\thetavec+\theta}
        % \tdplotsetrotatedcoords{60}{40}{30}
        % \tdplotdrawarc[tdplot_rotated_coords]{(0,0,0)}{\rvec}{\phivec-\dphi}{\phivec+\dphi}{}{}

        \tdplotdrawarc[]{(0,0,{\rvec*cos(\thetavec-\dtheta)})}{{\rvec*sin(\thetavec-\dtheta)}}{\phivec-\dphi}{\phivec+\dphi}{}{}

        \tdplotdrawarc[]{(0,0,{\rvec*cos(\thetavec+\dtheta)})}{{\rvec*sin(\thetavec+\dtheta)}}{\phivec-\dphi}{\phivec+\dphi}{below right=-3, scale=\fontscale}{\contour{white}{$r \sin(\theta) d\phi$}}
        % TODO: this is not correct!!!
        % -> but best we can do probably...
        
        \tdplotdrawarc[element helplines]{(0,0,0)}{\rvec}{\phivec-\dphi}{\phivec+\dphi}{}{}  % This ok because at phi=0

        
        % -- Testing with rotating into correct position -> difficult with scaling...
        % \tdplotsetrotatedcoords{\phivec}{\thetavec+\dtheta-90}{0}
        % \tdplotdrawarc[tdplot_rotated_coords]{(0,0,0)}{\rvec}{-\dphi}{\dphi}{}{}
    \end{scope}
\end{tikzpicture}


\end{document}