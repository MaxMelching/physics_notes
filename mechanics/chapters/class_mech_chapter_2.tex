\documentclass[../class_mech_main.tex]{subfiles}
%\documentclass[DIV=11, BCOR=0mm, paper=a4, fontsize=11pt, parskip=half, twoside=false, titlepage=true]{scrartcl}

\usepackage{subfiles}


\usepackage[singlespacing]{setspace} 
\usepackage{lastpage}
\usepackage[automark, headsepline]{scrlayer-scrpage}
\clearscrheadings
\setlength{\headheight}{\baselineskip}
\automark{section} % mit [] wird Argument in [] für links, {} rechts genommen
\automark*{subsection} % write section in footline instead of chapter (if there is one)
%\automark*{subsection}
\ihead{\headmark}
%\ohead[]{Seite~\thepage}
\cfoot{{\hypersetup{linkcolor=black}Page~\thepage~of~\pageref{LastPage}}}

\usepackage[utf8]{inputenc}
\usepackage[ngerman, english]{babel}
\usepackage[expansion=true, protrusion=true]{microtype}
\usepackage{amsmath}
\usepackage{amsfonts}
\usepackage{amsthm}
\usepackage{amssymb}
\usepackage{mathtools}
\usepackage{mathdots}
\usepackage{upgreek}
\usepackage[free-standing-units]{siunitx}
\usepackage{esvect}
\usepackage{graphicx}
\usepackage{epstopdf}
\usepackage[hypcap]{caption}
\usepackage{booktabs}
\usepackage{flafter}
\usepackage[section]{placeins}
\usepackage{pdfpages}
\usepackage{textcomp}
\usepackage{subfig}
\usepackage{floatpag} % to have clear pages where figures are
\usepackage[italicdiff]{physics}
\usepackage{xparse}
\usepackage{wrapfig}
\usepackage{color}
\usepackage{xcolor}
\usepackage{colortbl}
\usepackage{multirow}
\usepackage{array} % needed to define fancy table cells
\usepackage{diagbox} % needed for double colored table cells
\usepackage{dsfont}
\numberwithin{equation}{section}
\numberwithin{figure}{section}
\numberwithin{table}{section}
\usepackage{empheq}
\usepackage{tikz}
\usepackage{tikz-cd}%für Kommutationsdiagramme
\usepackage{forest}%Baumdiagramme
\usepackage{mdframed}

\usepackage{hyperref}
\hypersetup{colorlinks=true, breaklinks=true, citecolor=linkblue, linkcolor=linkblue, menucolor=linkblue, urlcolor=linkblue} %sonst z.B. orange bei linkcolor

\usepackage{imakeidx}%für Erstellen des Index
\usepackage{xifthen}%damit bei \Def{} das Index-Arugment optional gemacht werden kann

\usepackage[printonlyused]{acronym}%withpage -> seems useless here

\usepackage{enumerate} % for custom enumerators

\usepackage{listings} % to input code

\usepackage{csquotes} % to change quotation marks all at once

%\usepackage[nottoc, notlot, notlof, chapter]{tocbibind} %macht automatisch ins TOC, auch index und andere Sachen; so ungenummert, es geht aber auch mit Option numbib -> nicht nötig jetzt

%\usepackage[maxcitenames=3, backend=biber]{biblatex}%vlt hätte maxnames=2 gepasst


%man muss wohl Pakete mit Matheschrift zuerst laden
%\usepackage{mathpazo}%hä lol, das stellt überall pagella ein, erlaubt aber noch Modifikation?! Besser als pagella einzeln laden sogar -> ah, man kann aber z.B. noch Times auch einstellen hinterher; sieht jetzt aber nicht unbedingt überragend aus, Times da mein Favorit
%\usepackage{euler} %macht Fehler und sieht nichtmal so nice aus

%Versuch nur in Mathe Modus anzumachen, geht wohl in pdflatex nicht
%\usepackage{xfrac,unicode-math}
%\defaultfontfeatures{Scale=MatchLowercase}
%\setmathfont{TeX Gyre Termes Math}{version=termes}
%\setmathfont{TeX Gyre Pagella Math}{version=pagella}

% Versuch zwei -> nope, man braucht wohl XeLatex
%\usepackage{fontenc,xunicode}
%\setmathrm{Optima}

% Version 3
\usepackage{newtxmath} %geil, macht Times an in Mathe (ist stark, wenn auch zu dick bei Nutzen von Standard Computer Modern); muss auf jeden Fall rein bei Schrift Times, sonst sieht das im Vergleich viel zu dünn aus (auch bei pagella eigentlich)
%newtxtext funktioniert nicht, aber dafür ist ja auch tgtermes da

%\usepackage{tgtermes}
%\usepackage{cmbright}%ihhhhhhhh
\usepackage{tgpagella}
\setkomafont{section}{\rmfamily\Large\bfseries}
\setkomafont{sectionentry}{\large\bfseries}
\setkomafont{subsection}{\rmfamily\large\scshape}%textsc%\textsl auch not bad
\setkomafont{title}{\bfseries}%von pagella ein
\setkomafont{subtitle}{\Large\scshape}
\setkomafont{author}{\Large\slshape}
%\setkomafont{date}{\Large\slshape}
\setkomafont{pagehead}{\scshape}
\setkomafont{pagefoot}{\slshape}
\setkomafont{captionlabel}{\bfseries}
%\mathversion{qpl}



\definecolor{mygreen}{rgb}{0.8,1.00,0.8}
\definecolor{mycyan}{rgb}{0.76,1.00,1.00}
\definecolor{myyellow}{rgb}{1.00,1.00,0.76}
\definecolor{defcolor}{rgb}{0.10,0.00,0.60} %{1.00,0.49,0.00}%orange %{0.10,0.00,0.60}%aquamarin %{0.16,0.00,0.50}%persian indigo %{0.33,0.20,1.00}%midnight blue
\definecolor{linkblue}{rgb}{0.00,0.00,1.00}%{0.10,0.00,0.60}


% auch gut: green!42, cyan!42, yellow!24

%Syntax Farbboxen: in normalem Text \colorbox{mygreen}{Text} oder bei Anmerkungen in Boxen \fcolorbox{black}{myyellow}{Rest der Box}, in Mathe-Umgebung für farbige Box \begin{empheq}[box = \colorbox{mycyan}]{align}\label{eq:} Formel \end{empheq} oder farbigen Rand \begin{empheq}[box = \fcolorbox{mycyan}{white}]{align}\label{eq:} Formel \end{empheq}

\setlength{\fboxrule}{0.76pt}
\setlength{\fboxsep}{1.76pt}

\newcommand{\anm}[1]{\fcolorbox{black}{yellow!24}{\parbox[c]{0.985\textwidth}{\textbf{Anmerkung}: #1}}}

%\newcommand{\anm}[1]{\footnote{#1}}

\newcommand{\anmind}[1]{\fcolorbox{black}{yellow!24}{\parbox[c]{0.92 \textwidth}{\textbf{Anmerkung}: #1}}}
% wegen Einrückung in itemize-Umgebungen o.Ä.

\newcommand{\eqbox}{\fcolorbox{white}{cyan!24}}

\newcommand{\textbox}[1]{\fcolorbox{white}{cyan!24}{#1}}


\newcommand{\Def}[2][]{\textcolor{defcolor}{\fontfamily{ptm}\selectfont \textit{#2}}\ifthenelse{\isempty{#1}}{\index{#2}}{\index{#1}}}%{\colorbox{green!0}{\textit{#1}}}
% zwischendurch Test mit \textbf{#1} noch (wurde aber viel größer)

% habe jetzt Schrift (font) pagella reingehauen, ist mega

% wenn Farbe doch doof, einfach beide auf white :D




\mdfdefinestyle{defistyle}{topline=false, rightline=false, linewidth=1pt, frametitlebackgroundcolor=gray!12}

\mdfdefinestyle{satzstyle}{topline=true, rightline=true, leftline=true, bottomline=true, linewidth=1pt}

\mdfdefinestyle{bspstyle}{%
rightline=false,leftline=false,topline=false,%bottomline=false,%
backgroundcolor=gray!8}% tried imitation of spruce from beamer with black!20!white


\mdtheorem[style=defistyle]{defi}{Definition}[section]
\mdtheorem[style=satzstyle]{thm}[defi]{Theorem}
\mdtheorem[style=satzstyle]{lem}[defi]{Lemma}
\mdtheorem[style=satzstyle]{cor}[defi]{Corollary}
\mdtheorem[style=satzstyle]{prop}[defi]{Property}
\mdtheorem[style=bspstyle]{ex}[defi]{Example}
% just have one, Property, instead of Theorem, Lemma, Corollary?


\newtheoremstyle{rem}
  {\topsep}{\topsep}
  {}{}%{\centering}{0.1\textwidth}
  {\bfseries}{\textbf{remark}:}
  { }{}
\theoremstyle{rem}
% might be unnecessary now

\mdfdefinestyle{remstyle}{%
rightline=false,leftline=false,topline=false,bottomline=false,%
backgroundcolor=myyellow,innerleftmargin=.4\baselineskip,innerrightmargin=.4\baselineskip,leftmargin=-.4\baselineskip,rightmargin=-.4\baselineskip}%setting the indentations is important because otherwise, everything will be indented (.4\baselineskip is default and looks natural, so this is chosen; the effects of margin and innermargin have to be balanced)
%,frametitle={\textbf{remark}: }}%frametitle also makes linebreak

\newmdenv[style=remstyle]{remark}%{remark}
%\newmdtheoremenv[style=remstyle]{rem}{remark}
%\mdtheorem[style=remstyle]{rem}{remark:}%allows use of \begin{rem*} for no numbering

%\newcommand{remark}[1]{\begin{rem*}: #1\end{rem*}}
%use of begin, end is not allowed before \begin{document}


%Lösung (also Umgehen von Verbot \begin{} in Präambel) kommt von: https://www.mrunix.de/forums/showthread.php?59532-begin-und-end-in-newcommand
\def\brem#1\erem{\begin{remark}#1\end{remark}}
\newcommand{\rem}[1]{\brem \textbf{remark:} #1\erem}
% finally, now \rem{} is a shortcut for \begin{remark} etc.

% new line not always wanted for remarks, thus change to this here
\usepackage{soul}
\sethlcolor{myyellow}
\newcommand{\question}[1]{\hl{#1}}


% Anpassung von itemize-Symbolen
\renewcommand{\labelitemi}{$\blacktriangleright$}%{$\vartriangleright$}
\renewcommand{\labelitemii}{\textbf{--}} % is also default there
\renewcommand{\labelitemiii}{$\bullet$}


% Shortcuts -> falls man Abkürzung mal ändern will, muss man dann nicht den ganzen Text durchgehen
\usepackage{xspace} %weil man sonst \gw{} callen muss, damit Leerzeichen danach erkannt werden.
\newcommand{\gw}{{\hypersetup{linkcolor=black}\ac{gw}}\xspace}
\newcommand{\gws}{{\hypersetup{linkcolor=black}\acp{gw}}\xspace}

\newcommand{\mi}{{\hypersetup{linkcolor=black}\ac{mi}}\xspace}

\newcommand{\art}{{\hypersetup{linkcolor=black}\ac{art}}\xspace}

% wenn was nicht klappt, dann \gw{} callen
% mit diesem Ding leider kene Nutzung in Überschriften möglich

%\newcommand{\Var}{{\fontfamily{ptm}\selectfont\text{var}}}
%\newcommand{\Cov}{{\fontfamily{ptm}\selectfont\text{cov}}}
%\newcommand{\Corr}{{\fontfamily{ptm}\selectfont\text{corr}}}

% this is better, auto-select fonts etc
\DeclareMathOperator{\Var}{var}
\DeclareMathOperator{\Cov}{cov}
\DeclareMathOperator{\Corr}{corr}


%\renewcommand{\bibname}{References}
\addto\captionsenglish{\renewcommand{\bibname}{References}}



% if float is too long use \thisfloatpagestyle{onlyheader}
\newpairofpagestyles{onlyheader}{%
\setlength{\headheight}{\baselineskip}
\automark[section]{section}
%\automark*[section]{subsection}
\ihead[]{\headmark}
%
% only change to previous settings is here
\cfoot{}
}


\newpairofpagestyles{onlyfooter}{%
\setlength{\headheight}{\baselineskip}
\automark[section]{section}
%\automark*[section]{subsection}
\ihead[]{}
%
% only change to previous settings is here
\cfoot{{\hypersetup{linkcolor=black}Page~\thepage~of~\pageref{LastPage}}}
}



% for dartboard (from https://de.overleaf.com/latex/templates/dartboard/bhpfmdvjsjmk)
\tikzstyle{wired}=[draw=gray!30, line width=0.15mm]
\tikzstyle{number}=[anchor=center, color=white]
%%%<
\usepackage{verbatim}
%%%>
\begin{comment}
:Title: Dartboard
:Tags: Foreach; Node positioning
:Author: Roberto Bonvallet
:Slug: dartboard
\end{comment}

% Sectors are numbered 0-19 counterclockwise from the top.

% \strip{color}{sector}{outer_radius}{inner_radius}
\newcommand{\strip}[4]{
    \filldraw[#1, wired]
      ({18 *  #2}      :                   #3) arc
      ({18 *  #2}      : {18 * (#2 + 1)} : #3) --
      ({18 * (#2 + 1)} :                   #4) arc
      ({18 * (#2 + 1)} : {18 *  #2}      : #4) -- cycle;
}

% \sector{color}{sector}{radius}
\newcommand{\sector}[3]{
    \filldraw[#1, wired]
      (0, 0) --
      ({18 * #2} :                   #3) arc
      ({18 * #2} : {18 * (#2 + 1)} : #3) -- cycle;
} \graphicspath{../}


\begin{document}


% \chapter{Erweiterungen von Newton}
\chapter{Analytical Mechanics}

Newtonian mechanics provided a description of the world in terms of forces and momenta. This is convenient for many applications, but not for all. For this reason, it is sometimes worthwhile to adopt different viewpoints on the physical world, and we will present several alternative frameworks in this chapter. These are often summarized under the term \enquote{analytical mechanics}.

-> maybe say that we switch to fields (?) and generalized coordinates + momenta



	\section{Lagrange-Formalism}

		\subsection{Zwangsbedingungen}

		\subsection{D'Alembert Principle}

		\subsection{Hamilton Principle}

		% \subsection{The Mathematical Viewpoint: Transition from Lagrange to Hamilton}
		% \subsection{The Mathematical Perspective: Moving on from Lagrange}
		\subsection{Gaining Momentum(a): Moving on from Lagrange}

\todo{make section? As nice transition between Lagrange and Hamilton, that still provides something new in the mathematical formulation}

introduce the very formal way of looking at formalism

what we have done is describe physical system in terms of time $t$, some generalized positions $q$, and their derivatives $\dot{q} = \dv{q}{t}$

\footnote{Although it has components, we do not use $\vec{q}$, since it refers to a point on a manifold now. On this abstract object, points are now distinct notions from vectors, unlike in Euclidean space.}

-> turns out, the generalized coordinates $q_i$ span a manifold, known as \Def{configuration space}; the \Def{configuration} of a system at each time $t$ is an element of this space, \todo{so $q(t)$ is a curve in this space}

-> one further step is to include $t$ in the space as well, which yields the \Def{event space}; the Lagrange formalism can be visualized in this space, evolution of physical system is a curve in this space \todo{really, is it? or in configuration space -> ah, I guess in both; in one, we have curve as something like graph, in other we vary "external" parameter, in the sense that it is not on one of the axes, and get curve in this way}



on a manifold, derivatives of coordinates are called tangent vectors; in our case, this role is taken by the $\dot{q}_i = \dv{q_i}{t}$, which form a basis of the tangent space; the pair $(q, \dot{q})$ is then referred to as tangent bundle, and Lagrangian is function on that (that has additional parameter in the time $t$)

-> for next section: now, we want description in terms of $p_i = \dv{\mathcal{L}}{q_i}$, which are also called generalized momenta and form what is called a conjugate variable to the $q_i$ (\todo{provide justification for that} -> maybe that from derivative in its definition, we can already guess it will help with reducing order? I mean just hand-wavy here)\footnote{This justification is a little hand-wavy. To me, this looks like a classical case of: two approaches exist (Lagrangian, Hamiltonian), and people found this cool mathematical relation between them. So this serves as our transition between them. I might be wrong, and a better justification other than \enquote{something good comes out of switching to the conjugate variables} might exist.}; next question: how can we do same physics using $p_i$ that we did before? Answer: by finding the analogue of $\mathcal{L}$ formulated in terms of the $p_i$; turns out, this is a frequently asked problem, so there is a good and general solution for it: the Legendre transform (which works here because Lagrangian is a convex function for each $q$)\footnote{Fun fact: the Legendre transform is also a prominent part of thermodynamics.}; yields Hamiltonian $H$, which essentially lets us do same theory as before, but now on the cotangent bundle of the manifold for which $q$ are coordinates 

-> ok, what this means is that using Hamiltonian is just a reformulation of what we have been able to do anyway; so, what's cool about it, why should we use it? well, one practical issue with Lagrangian is that the general form of equations of motion are second-order differential equations, and (as we will see) by including derivatives directly into the generalized momenta, the Hamiltonian is able to reduce this to first-order differential equations (at the cost of doubling the equations; but this is price we are often absolutely willing to pay, as it still results in overall simplification of the problem); also, Hamiltonian itself corresponds to total energy under certain circumstances, so unlike Lagrangian it does have a physical meaning

-> note that tangent space and cotangent space are manifolds themselves\todo{verify again}; in particular, one can do things like integrals on them, something that we will see later on; they even got a name, phase space (or state space \todo{this is the german name, certainly not english}, if time is included as well)

-> talk about Hamiltonian flow (?) -> nah, rather in math section below


purpose of Euler-Lagrange equations (ELE) is to give us a way to find $q$ as function of $t$ for a given Lagrangian; when switching to Hamiltonian, this transition would not be complete without a similar way of obtaining $(q, p)$ from a given Hamiltonian (be it that we know it straight up, or get via Legendre from Lagrangian); basically, what we want is turning ELE, i.e.~conditions on derivatives of Lagrangian, into conditions on derivatives of Hamiltonian; fortunately, we can relate Hamiltonian to Lagrangian; it is thus straightforward idea to write out the differential of Hamiltonian and then compare general coefficients to coefficients in terms of the Lagrangian (for the latter, we hope to be able to apply ELE)



	\section{Hamilton-Formalism}
basically description in terms of energy instead of momentum, right? since Hamiltonian = energy (?)




	\section{The Mathematical Deep Dive: Symplectic Structure Of Classical Mechanics}

summary of all Wikipedia articles I could find on this:
% - https://en.wikipedia.org/wiki/Hamiltonian_mechanics#From_symplectic_geometry_to_Hamilton's_equations
% - https://en.wikipedia.org/wiki/Hamiltonian_vector_field
% - https://en.wikipedia.org/wiki/Tautological_one-form
% - https://en.wikipedia.org/wiki/Legendre_transformation#Analytical_mechanics

Darboux's theorem forms basis for this; states that every symplectic manifold is the phase space of some physical system (is equivalence; every phase space is also symplectic)! So this type of manifolds is really the arena of Hamiltonian mechanics!


idea: we have phase space; at every point in this, I can look at tangent space, and this point-wise mapping induces a tangent bundle

-> there is a natural mapping, given by symplectic form (also: tautological one-form), between tangent and cotangent bundle; and in a small neighborhood of certain point, Darboux guarantees this can be expressed in terms of the canonical coordinates

-> we can then write down a vector field in terms of this (the Hamiltonian vector field), $X_H = J dH$ (sometimes expressed as $J(dH)$), and solving the differential equation for this vector field on the phase space (there is a natural way to formulate this on vector fields) turns out to be nothing but solving the canonical equations (from Hamiltonian) that we have derived (if we also use knowledge from ELE)



\hrule


hier halt etwas mathematischere Beschreibung; die kanonischen Vertauschungsrelationen definieren nämlich mathematisch gesehen eine Basis sowie die dazu duale Basis eines Tangentialraums bzw. sogar -bündels

-> Beispiele für symplektische VR sind insbesondere die Tangentialräume von symplektischen MF

-> Satz von Darboux zeigt: jede symplektische MF ist Phasenraum eines physikalischen Systems (habe ich zumindest mal so gelesen)


gute Quelle: Nakahara DiffGeo + Topologie Abschnitt 5.4.3

	\subsection{Phasenraum}
%Die Idee der Klassischen Statistischen Mechanik ist es, ein makroskopisches Vielteilchen-System durch seine klassischen, mikroskopischen Eigenschaften zu beschreiben und die Quantenmechanik erst einmal zu vernachlässigen. Manchmal ist das hilfreich bei der Veranschaulichung, aber man muss immer beachten, dass diese Beschreibung aus verschiedensten Gründen nicht exakt ist.\\
Mathematisch ist ein Teilchen in einem Gebiet $\Omega$ beschreibbar als Phase $\gamma = (\vec{p}, \vec{q})$ bestehend aus den \Def[Koordinaten! generalisierte]{generalisierten Koordinaten} $\vec{p}, \vec{q}$ im zugehörigen \Def{Phasenraum}
\begin{equation}
\Gamma = \mathbb{R}^3 \cross \Omega = \qty{\gamma: \; \gamma = (\vec{p}, \vec{q}) \equiv \text{ Zustand}} \equiv \text{ Zustandsraum} \, ,
\end{equation}
wobei unter Umständen noch Nebenbedingungen berücksichtigt werden müssen.

? eher Konfiguration ? (p,q) ist Zustand statt Konfiguration, wenn stationäres Problem vorliegt. Dann ist ja Zeit egal und kann weggelassen werden (wie bei $\Psi$, das ja meist zeitunabhängig und dann ist eben bereits $\Psi(\vec{r})$ Zustand)

	\anm{meist ist $\vec{q} \in \Omega \subset \mathbb{R}^3$ ein Element des Ortsraumes ($\Omega$ ist natürlich $\subset \mathbb{R}^3$, da dies ja der von uns Menschen beobachtbare Raum ist) und $\vec{p} \in \mathbb{R}^3$ ein Element des Impulsraumes, der den Dualraum zum Ortsraum bildet (anschaulich, da man mit dem Impuls, der die Bewegungsrichtung und -geschwindigkeit enthält, quasi Orte aufeinander abbilden kann, sich also von einem zum anderen bewegen). Der Wechsel zwischen Orts- und Impulsraum ist übrigens mithilfe der Fourier-Trafo möglich.}


Betrachtet man nun $N$ Teilchen in Zuständen $(\vec{p}_i, \vec{q}_i)$, so ist der Phasenraum des Gesamtsystems gerade das kartesische Produkt der einzelnen Phasenräume, also
\begin{equation}
\begin{split}
\Gamma &:= \Gamma_{Ges} = \Gamma_1 \cross \dots \cross \Gamma_N = \qty(\mathbb{R}^3 \cross \Omega) \cross \dots \cross \qty(\mathbb{R}^3 \cross \Omega) = \qty(\mathbb{R}^3 \cross \Omega)^N
\\
&= \qty{\gamma: \; \gamma = \qty((\vec{p_1}, \vec{q_1}), \dots, (\vec{p_n}, \vec{q_n}))} = \qty{\gamma: \, \gamma = \qty(\vec{p_1}, \dots, \vec{p_n}, \vec{q_1}, \dots \vec{q_n})} \, .
\end{split}
\end{equation}
Durch Angabe von $\gamma \in \Gamma$ ist das $N$-Teilchen-System also vollständig beschreibbar.

	\anm{man sieht aber, dass $\dim\qty(\Gamma) = 6N$, was in realen Anwendungen sehr groß werden kann (Avogadro-Zahl ist ja $\approx 6 \cdot 10^{23}$ !), daher ist das nicht immer die beste Möglichkeit (siehe StaPhy).}\\

Oft ändert man nun noch die Indizes, da nicht immer alle drei Teilchenkoordinaten zusammen benötigt werden und schreibt sie in einen Gesamtorts-/ impulsvektor
\begin{align}
p &= (p_{1,1}, p_{1,2}, p_{1,3}, \dots, p_{N,1}, p_{N,2}, p_{N,3}) = (p_1, \dots, p_{3N})
\\
q &= (q_{1,1}, q_{1,2}, q_{1,3}, \dots, q_{N,1}, q_{N,2}, q_{N,3}) = (q_1, \dots, q_{3N})
\\
\Rightarrow \quad \gamma &= (p, q) \in \qty(\mathbb{R}^{3N} \cross \Omega^{N}) = \Gamma \, .
\end{align}

! Beispiel Plot reinmachen von Phasenraum Harmonischer Ossi, da sieht man mögliche (p,q)-Paare zu fester Energie, Masse = Nebenbedingung !



	\subsection{Funktionen auf $\Gamma$}
Ein Beispiel für eine Funktion auf dem Phasenraum ist die Hamilton-Funktion $H$ (wie Hamilton-Operator), die gleichzeitig die Energie beschreibt und definiert ist als
\begin{align}
H: \Gamma \rightarrow \mathbb{R}, \; \gamma \mapsto H(\gamma) = \sum\limits_{i = 1}^N \qty(\frac{\vec{p}^{\;2}_i}{2m} + V_1(\vec{q}_i)) + \sum\limits_{i,j = 1; i \neq j}^N V_2(\vec{q}_i - \vec{q}_j) \, .
%\\
%&= \sum\limits_{j = 1}^{3N} \qty(\frac{p_j^2}{2m} + V_1(q_j)) + \sum\limits_{i,j = 1; i \neq j}^{3N} V_2(\vec{q}_i - \vec{q}_j) .
\end{align}

	\anm{damit das Phasenraumvolumen eines Teilchens endlich ist, braucht man ein Potential, das im Unendlichen divergiert, da es sonst frei beweglich ist und somit ein unendliches Volumen beim Ortsteil eingenommen wird.}



Wie in Vektorräumen üblich, existiert auch in $\Gamma$ noch mehr Struktur und zwar eine Art Skalarprodukt, es hilft hier der Satz von Darboux: er besagt, dass jeder Phasenraum in der Hamilton'schen Mechanik eine \Def[symplektisch! -e Mannigfaltigkeit]{symplektische Mannigfaltigkeit} bildet (Begriffe sind äquivalent). Das ist ein Paar $(M, \omega)$ bestehend aus einer glatten Mannigfaltigkeit $M$ mit einer \Def[symplektisch! -e Differentialform]{symplektischen Differentialform} $\omega = \sum\limits_{i,j} \omega_{ij} \, dq_i \wedge dp_j$ (glatt und geschlossen, also $d\omega = 0$), die analoge Rollen zu Vektorräumen mit alternierenden Bilinearformen/ Skalarprodukten einnehmen.

! siehe OneNote $\rightarrow$ StaPhy Zusammenfassung für sehr hilfreiche Aufgabe ! dazu in Übung 10 MfP angucken, die man $\pdv{x}$ in einer Differentialform auswertet und paar andere coole Sachen ! w anscheinend mit Koeffizienten 1 !

Auf dieser Grundlage kann man die sogenannte \Def{Poisson-Klammer} definieren als
\begin{equation}
\qty{f,g}_{p,q} = \sum\limits_{i = 1}^{3N} \pdv{f}{q_i} \pdv{g}{p_i} - \pdv{f}{p_i} \pdv{g}{q_i} = \sum\limits_{i,j} \omega^{ij} \, \partial_i f \, \partial_j g \, ,
\end{equation}
das gilt nur in gewählten Koordinaten !!! ? hatte vorher 6N, aber das kann nicht sein oder ? -> dort wahrscheinlich mit symplektischer Matrix gearbeitet

wobei die Indizes $p,q$ oft weggelassen werden, da sie die Basis kennzeichnen und die Auswertung basisunabhängig ist (folgt aus den Eigenschaften von Differentialformen).

Es lassen sich direkt einige grundlegende Relationen nachrechnen, die \Def[Poisson-Klammer! fundamentale]{fundamentalen Poisson-Klammern}:
\begin{equation}
\{q_k, q_l\} = 0 \hspace{1cm} \{p_k, p_l\} = 0 \hspace{1cm} \{q_k, p_l\} = \delta_{kl} \, .
\end{equation}

Dabei muss man lediglich die folgenden Zusammenhänge ausnutzen:
\begin{equation}
\pdv{q_k}{q_l} = \delta_{kl} = \pdv{p_k}{p_l} \hspace{2cm} \pdv{q_k}{p_l} = 0 = \pdv{p_k}{q_l} .
\end{equation}

Man kann nun zeigen, dass für eine beliebige Phasenraumfunktion $F(p,q)$ gilt:
\begin{equation}
\dv{F}{t} = \qty{F,H} + \pdv{F}{t} .
\end{equation}

Um nun die Dynamik dieses Systems (Annahme: $N$ Teilchen gleicher Masse $m$) zu beschreiben, kann man einfach die generalisierten Orts- und Impulsfunktionen $p_i, q_i$ (ordnen ja letztendlich jedem Zustand $\gamma \in \Gamma$ gewisse Größen zu, sind gerade Phasenraumfunktionen; sind nicht explizit zeitabhängig) in die Poisson-Klammer einsetzen. Das Ergebnis sind die \Def[kanonisch! -e Bewegungsgleichungen]{Hamilton'schen/ kanonischen Bewegungsgleichungen}:
\begin{equation}\label{eq:kanGl}
\dot{q}_i = \qty{q_i, H} = \pdv{H}{p_i} \qquad \dot{p}_i = \qty{p_i, H} = -\pdv{H}{q_i}, \qquad i = 1, \dots, 3N \, .
\end{equation}



	\subsection{Zeitentwicklung}
Die allgemeine Lösung der kanonischen Gleichungen \eqref{eq:kanGl} zum beliebigen Anfangswert $\gamma$ beschreibt ja die Zeitentwicklung eines Systems, ordnet also jedem Zeitpunkt $t$ einen Punkt $\gamma(t) = \tilde{\gamma} = (p,q) \in \Gamma$ zu und bildet somit eine Kurve im Phasenraum (was einer Abfolge von Systemzuständen zu verschiedenen Zeiten entspricht).\\
Diese allgemeine Lösung lässt sich zu beliebigen Startwerten bestimmen (zumindest theoretisch ist das möglich, hier aber gar nicht explizit nötig) und ist deshalb allgemein mithilfe einer Abbildungs-Schar darstellbar, dem \Def[Hamilton-Fluss]{Hamilton'schen Fluss}
\begin{equation}
\mathcal{F}_t : \Gamma \rightarrow \Gamma, \; \gamma \mapsto \mathcal{F}_t \gamma \equiv \gamma(t) \, .
\end{equation}
Es folgen quasi per Definition die Eigenschaften $\mathcal{F}_0 \gamma = \gamma$, $\mathcal{F}_s \circ \mathcal{F}_t = \mathcal{F}_{s+t}, \, s,t \in \mathbb{R}$.

Benutzung Fluss ergibt hier voll Sinn, weil man so zu jedem Anfangszustand die komplette Zeitentwicklung beschreiben kann und das in einer Abbildung (fasse also den Parameter der Schar als Parameter einer Abbildung auf)
\begin{equation}
\mathcal{F}: (\mathbb{R}, \Gamma) \rightarrow \Gamma, \; (t, \gamma) \mapsto \gamma(t)
\end{equation}

	\anm{wir nehmen an, dass am Rand des Gebietes $\Omega$ die Lösung weiter gültig bleibt, das Teilchen aber $"$reflektiert$"$ bzw. $"$gespiegelt$"$ wird: $(\vec{p}_i, \vec{q}_i) \mapsto (-\vec{p}_i, \vec{q}_i)$.}\\

Vorgriff zur Statistischen Mechanik: hier beschreibt man den Zustand bzw. die Präparation von Systemen als Dichtefunktion $\rho$ (kommt z.B. aus mikrokanonischer/ kanonischer Gesamtheit), so ist dort die Zeitentwicklung des Systems gegeben durch:
\begin{equation}
\dot{\rho}_t = \qty{\rho, H} \, .
\end{equation}
Dies ist die \Def{Liouville-Gleichung}, die dem Schrödinger-Bild $\dot{\rho}_t = i [\rho, H]$ in der QM entspricht. Der Zusammenhang zum Hamilton'schen Fluss $\mathcal{F}_t$ ist $\rho_t(\gamma) = \rho\qty(\mathcal{F}_{-t} \gamma)$.\\

Will man nun ein Volumen oder eine Fläche messen, so ist dazu immer ein Maß nötig. Das auf $\Gamma$ verwendete Liouville-Maß ist dabei wie das Lebesgue-Maß definiert:
\begin{equation}
d\gamma = d^{\,3N}p \, d^{\,3N}q = dp_1 \, dq_1 \dots dp_{3N} \, dq_{3N} \equiv d^{\,3N}p \wedge d^{\,3N}q \, .
\end{equation}

Mit diesem Hilfsmittel kann man nun die beiden wesentlichen Erhaltungsgrößen im Phasenraum mit den zugehörigen Konsequenzen untersuchen, Volumen und Energie.

Die Erhaltung des Volumens (damit ist nicht die Erhaltung von $d\gamma$ gemeint, sondern des Integrals davon über eine Menge $M$ !) bedeutet einfach, dass für eine beliebige integrierbare Funktion $f$ zu jedem Zeitpunkt $t \in \mathbb{R}$ gilt:
\begin{equation}\label{eq:zeitentw}
\int_M f\qty(\mathcal{F}_t \gamma) \, d\gamma = \int_{\mathcal{F}_t^{-1} M} f\qty(\gamma) \, d\gamma = \int_M f(\gamma) \, d\gamma \, .
\end{equation}

Idee: Anwendung der Transformationsformel auf die linke Seite, das ergibt dann $f\qty(\mathcal{F}_t^{-1}(\mathcal{F}_t \gamma)) = f(\gamma)$, aber auch auf Gebiet nötig, es kommt dann $\mathcal{F}_t^{-1}M$ raus

%-> müsste links nicht auch $d\mathcal{F}_t \gamma$ stehen ??? evtl nicht, da wir ja über gleiche kurve integrieren (?)

Daraus erhält man dann mit $f = \chi_M: \Gamma \rightarrow [0,1]$ als charakteristische Funktion einer beliebigen Menge $M \subset \Gamma$ für die Volumina
%\begin{equation}
%V_{\mathcal{F}_t^{-1}M} = \int f\qty(\mathcal{F}_t \gamma) \, d\gamma %= \int f(\gamma) \, d\gamma = V_M
%\end{equation}
\begin{equation}
\int_{\mathcal{F}_t^{-1} M} d\gamma = \int_M d\gamma  \equiv \int_M \omega^{3N} \Leftrightarrow V_M = V_{\mathcal{F}_t^{-1}M} \, .
\end{equation}
Anmerkung: $\omega^{3N}$ ist gerade das $\omega$ von oben mit mehr Dimensionen (evtl. nur $\omega^N$)

(sicher -1 etc. ??? $\rightarrow$ ja, jetzt eigentlich schon; das kann man zudem safe umschreiben zu $\mathcal{F}_{-t}$ nach den Eigenschaften von Flüssen)

$\Rightarrow$ evtl. ist Bezeichnung nur missverständlich; er schreibt $\mathcal{F}_t^{-1}M = \{\gamma: \mathcal{F}_t \gamma \in M\}$, vlt. ist also gemeint, dass man sich das Volumen der zeitentwickelten $\gamma$ anschaut und wenn man das dann quasi zurückentwickelt, erhält man das gleiche Volumen wie von $M$ ? Das heißt nicht, dass gleich viele Zustände oder so (heißt es das doch evtl. ? Haben hier noch gar keine Dichte; eigentlich Aussage ist doch, dass Zustände aus $M$ auch bei Zeitentwicklung in $M$ bleiben oder ?), sondern eben nur, dass von allen Zuständen multipliziert mit der Dichte das gleiche Volumen eingenommen wird.
-> hier geht es tatsächlich einfach nur mit Transformationsformel

Wie so oft in der Physik wird (hier auch auf Grundlage des 1. Hauptsatzes -> we are at interface of mechanics and thermodynamics here!) zudem die Energieerhaltung angenommen, es soll während der Zeitentwicklung nichts davon verloren gehen. Das bedeutet einfach
\begin{equation}
H(\mathcal{F}_t \gamma) = H(\gamma), \, \forall t \, .
\end{equation}
Betrachtet man nun alle Punkte mit gleicher Energie $E$, so bilden diese eine Energiefläche $\qty{\gamma | \, H(\gamma) = E}$, die als Fläche im $6N$-dimensionalen Phasenraum genau $(6N-1)$-dimensional ist. Das heißt aber, dass sie ein Liouville-Maß von 0 haben !

Um trotzdem eine Aussage über Energieflächen treffen zu können, fügt man ihnen in der fehlenden Dimension die Ausdehnung $\epsilon$ hinzu und konstruiert so eine Energieschale $\qty{\gamma: \; E-\epsilon \leq H(\gamma) \leq E}$, die nun Liouville-messbar ist mit Maß $\neq 0$.\\
Um diese kleine Schummelei auszugleichen, teilen wir bei der Definition des Flächenintegrals über eine Energiefläche (oder einer Teilmenge davon) wieder durch $\epsilon$ und erhalten so als Integral einer beliebigen Funktion $f: \Gamma \rightarrow \mathbb{R}$ (z.B. eine Observable) 
\begin{align}\label{eq:energy}
\expval{f}_{\qty[E = H(\gamma), \, E + \epsilon]} &= \int \chi_{\qty[E = H(\gamma), E - \epsilon]} f(\gamma) \, d\gamma
\notag\\
&= \int_{\qty{\gamma | \, H(\gamma) = E}} f(\gamma) \, d\gamma = \int \delta(H(\gamma) - E) \, f(\gamma) \, d\gamma
\notag\\
&:= \lim_{\epsilon \rightarrow 0} \frac{1}{\epsilon} \int_{E-\epsilon \leq H(\gamma) \leq E} f(\gamma) \, d\gamma = \dv{E} \int_{H(\gamma) \leq E} f(\gamma) \, d\gamma \, .
\end{align}

-> müsste es nicht nur über $H(\gamma) = E$ sein, da $\epsilon = 0$, also $E - \epsilon = E$; zweite Schreibweise überhaupt nötig ???


Nach \eqref{eq:zeitentw} ist auch dieses Integral zeitunabhängig, jedoch hängen die Schalenbreite und somit der Wert des Integrals vom Inversen des Gradienten der Energie $dE$ ab.

Anmerkung: wir können die $p_i, q_i$ auf einer Energiefläche nicht beliebig groß machen, da sonst zu viel Energie dazu benötigt wird (haben aber nur konstant viel zur Verfügung)\\

Die hier Beschreibung eines Systems über eine Phase $\gamma \in \Gamma$ ist meistens jedoch überhaupt nicht praktisch, da $\Gamma$ ja $6N$-dimensional ist und $N$ oft Werte im Bereich von $10^{23}$ hat - man könnte die nötigen Datenmengen nicht speichern, geschweige denn damit rechnen. Zudem wäre es aufgrund der Quantenmechanik gar nicht möglich, ein System genau zu vermessen oder in einer Phase $\gamma$ zu präparieren.

Wie bereits ganz am Anfang angedeutet, ist es sinnvoll, auf eine nicht exakte, statistische Beschreibung zurückzugreifen, da eine genaue nicht funktioniert. Die klassische Vielteilchen-Mechanik ist also eine Statistische Mechanik.



		\subsection{Random Ergänzungen}

kanonische Trafos sind Basiswechsel (bzw. eigentlich sogar Kartenwechsel oder ?) -> jo, Karten- und Basiswechsel sind ja quasi das Gleiche; deshalb auch ruhig Interpretation als symplektische MF bringen in Kapitel nach Hamilton, da sieht man dann mathematisch genau die Physik von vorher wieder

Liouville-Gleichung folgt, weil $\rho$ Erhaltungsgröße quasi (haben $\dv{\rho}{t} = 0$ und dann Umstellen)

Nolting Band 6, Abschnitt 1.2.1 - 1.2.3 ist super zu Phasenraum !


haben auch Poisson-Klammern bei Drehimpuls (werden auf jeden Fall auf Seite 5, Band 5.2 erwähnt, sehen aber genau so aus wie die Kommutatorrelationen) !



Formulierung mit Mannigfaltigkeiten: $(q,\dot{q}) \equiv$ Punkt, Tangentialvektor ergibt Sinn als Basis des Tangentialbündels (das den Konfigurationsraum beschreibt) -> ist ja bis auf Faktor $m$ gleich mit $(q, p)$ !! Und dann ist auch $(q,p)$ als Basis des Kotangentialbündels richtig/ sinnvoll (weil wegen Trivialisierung right; Eselsbrücke: Impuls erzeugt Translationen/ Bewegung, also die Abbildung eines Punktes auf andere Punkte und ist daher was Duales)



die ganz allgemeine Form von Lagrange- und Hamilton-Formalismus ist das Wirkungsprinzip -> da dann auch guter Verweis möglich, dass GR den gleichen Lagrangian hat und nur andere Metrik, da die dort durch Gravitation und Massen beeinflußt wird (daher dann auch bisschen andere Physik)

Legendre-Trafo wechselt zwischen Tangential- und Kotangentialbündel oder?


\end{document}