\documentclass[../class_mech_main.tex]{subfiles}
%\documentclass[DIV=11, BCOR=0mm, paper=a4, fontsize=11pt, parskip=half, twoside=false, titlepage=true]{scrartcl}

\usepackage{subfiles}


\usepackage[singlespacing]{setspace} 
\usepackage{lastpage}
\usepackage[automark, headsepline]{scrlayer-scrpage}
\clearscrheadings
\setlength{\headheight}{\baselineskip}
\automark{section} % mit [] wird Argument in [] für links, {} rechts genommen
\automark*{subsection} % write section in footline instead of chapter (if there is one)
%\automark*{subsection}
\ihead{\headmark}
%\ohead[]{Seite~\thepage}
\cfoot{{\hypersetup{linkcolor=black}Page~\thepage~of~\pageref{LastPage}}}

\usepackage[utf8]{inputenc}
\usepackage[ngerman, english]{babel}
\usepackage[expansion=true, protrusion=true]{microtype}
\usepackage{amsmath}
\usepackage{amsfonts}
\usepackage{amsthm}
\usepackage{amssymb}
\usepackage{mathtools}
\usepackage{mathdots}
\usepackage{upgreek}
\usepackage[free-standing-units]{siunitx}
\usepackage{esvect}
\usepackage{graphicx}
\usepackage{epstopdf}
\usepackage[hypcap]{caption}
\usepackage{booktabs}
\usepackage{flafter}
\usepackage[section]{placeins}
\usepackage{pdfpages}
\usepackage{textcomp}
\usepackage{subfig}
\usepackage{floatpag} % to have clear pages where figures are
\usepackage[italicdiff]{physics}
\usepackage{xparse}
\usepackage{wrapfig}
\usepackage{color}
\usepackage{xcolor}
\usepackage{colortbl}
\usepackage{multirow}
\usepackage{array} % needed to define fancy table cells
\usepackage{diagbox} % needed for double colored table cells
\usepackage{dsfont}
\numberwithin{equation}{section}
\numberwithin{figure}{section}
\numberwithin{table}{section}
\usepackage{empheq}
\usepackage{tikz}
\usepackage{tikz-cd}%für Kommutationsdiagramme
\usepackage{forest}%Baumdiagramme
\usepackage{mdframed}

\usepackage{hyperref}
\hypersetup{colorlinks=true, breaklinks=true, citecolor=linkblue, linkcolor=linkblue, menucolor=linkblue, urlcolor=linkblue} %sonst z.B. orange bei linkcolor

\usepackage{imakeidx}%für Erstellen des Index
\usepackage{xifthen}%damit bei \Def{} das Index-Arugment optional gemacht werden kann

\usepackage[printonlyused]{acronym}%withpage -> seems useless here

\usepackage{enumerate} % for custom enumerators

\usepackage{listings} % to input code

\usepackage{csquotes} % to change quotation marks all at once

%\usepackage[nottoc, notlot, notlof, chapter]{tocbibind} %macht automatisch ins TOC, auch index und andere Sachen; so ungenummert, es geht aber auch mit Option numbib -> nicht nötig jetzt

%\usepackage[maxcitenames=3, backend=biber]{biblatex}%vlt hätte maxnames=2 gepasst


%man muss wohl Pakete mit Matheschrift zuerst laden
%\usepackage{mathpazo}%hä lol, das stellt überall pagella ein, erlaubt aber noch Modifikation?! Besser als pagella einzeln laden sogar -> ah, man kann aber z.B. noch Times auch einstellen hinterher; sieht jetzt aber nicht unbedingt überragend aus, Times da mein Favorit
%\usepackage{euler} %macht Fehler und sieht nichtmal so nice aus

%Versuch nur in Mathe Modus anzumachen, geht wohl in pdflatex nicht
%\usepackage{xfrac,unicode-math}
%\defaultfontfeatures{Scale=MatchLowercase}
%\setmathfont{TeX Gyre Termes Math}{version=termes}
%\setmathfont{TeX Gyre Pagella Math}{version=pagella}

% Versuch zwei -> nope, man braucht wohl XeLatex
%\usepackage{fontenc,xunicode}
%\setmathrm{Optima}

% Version 3
\usepackage{newtxmath} %geil, macht Times an in Mathe (ist stark, wenn auch zu dick bei Nutzen von Standard Computer Modern); muss auf jeden Fall rein bei Schrift Times, sonst sieht das im Vergleich viel zu dünn aus (auch bei pagella eigentlich)
%newtxtext funktioniert nicht, aber dafür ist ja auch tgtermes da

%\usepackage{tgtermes}
%\usepackage{cmbright}%ihhhhhhhh
\usepackage{tgpagella}
\setkomafont{section}{\rmfamily\Large\bfseries}
\setkomafont{sectionentry}{\large\bfseries}
\setkomafont{subsection}{\rmfamily\large\scshape}%textsc%\textsl auch not bad
\setkomafont{title}{\bfseries}%von pagella ein
\setkomafont{subtitle}{\Large\scshape}
\setkomafont{author}{\Large\slshape}
%\setkomafont{date}{\Large\slshape}
\setkomafont{pagehead}{\scshape}
\setkomafont{pagefoot}{\slshape}
\setkomafont{captionlabel}{\bfseries}
%\mathversion{qpl}



\definecolor{mygreen}{rgb}{0.8,1.00,0.8}
\definecolor{mycyan}{rgb}{0.76,1.00,1.00}
\definecolor{myyellow}{rgb}{1.00,1.00,0.76}
\definecolor{defcolor}{rgb}{0.10,0.00,0.60} %{1.00,0.49,0.00}%orange %{0.10,0.00,0.60}%aquamarin %{0.16,0.00,0.50}%persian indigo %{0.33,0.20,1.00}%midnight blue
\definecolor{linkblue}{rgb}{0.00,0.00,1.00}%{0.10,0.00,0.60}


% auch gut: green!42, cyan!42, yellow!24

%Syntax Farbboxen: in normalem Text \colorbox{mygreen}{Text} oder bei Anmerkungen in Boxen \fcolorbox{black}{myyellow}{Rest der Box}, in Mathe-Umgebung für farbige Box \begin{empheq}[box = \colorbox{mycyan}]{align}\label{eq:} Formel \end{empheq} oder farbigen Rand \begin{empheq}[box = \fcolorbox{mycyan}{white}]{align}\label{eq:} Formel \end{empheq}

\setlength{\fboxrule}{0.76pt}
\setlength{\fboxsep}{1.76pt}

\newcommand{\anm}[1]{\fcolorbox{black}{yellow!24}{\parbox[c]{0.985\textwidth}{\textbf{Anmerkung}: #1}}}

%\newcommand{\anm}[1]{\footnote{#1}}

\newcommand{\anmind}[1]{\fcolorbox{black}{yellow!24}{\parbox[c]{0.92 \textwidth}{\textbf{Anmerkung}: #1}}}
% wegen Einrückung in itemize-Umgebungen o.Ä.

\newcommand{\eqbox}{\fcolorbox{white}{cyan!24}}

\newcommand{\textbox}[1]{\fcolorbox{white}{cyan!24}{#1}}


\newcommand{\Def}[2][]{\textcolor{defcolor}{\fontfamily{ptm}\selectfont \textit{#2}}\ifthenelse{\isempty{#1}}{\index{#2}}{\index{#1}}}%{\colorbox{green!0}{\textit{#1}}}
% zwischendurch Test mit \textbf{#1} noch (wurde aber viel größer)

% habe jetzt Schrift (font) pagella reingehauen, ist mega

% wenn Farbe doch doof, einfach beide auf white :D




\mdfdefinestyle{defistyle}{topline=false, rightline=false, linewidth=1pt, frametitlebackgroundcolor=gray!12}

\mdfdefinestyle{satzstyle}{topline=true, rightline=true, leftline=true, bottomline=true, linewidth=1pt}

\mdfdefinestyle{bspstyle}{%
rightline=false,leftline=false,topline=false,%bottomline=false,%
backgroundcolor=gray!8}% tried imitation of spruce from beamer with black!20!white


\mdtheorem[style=defistyle]{defi}{Definition}[section]
\mdtheorem[style=satzstyle]{thm}[defi]{Theorem}
\mdtheorem[style=satzstyle]{lem}[defi]{Lemma}
\mdtheorem[style=satzstyle]{cor}[defi]{Corollary}
\mdtheorem[style=satzstyle]{prop}[defi]{Property}
\mdtheorem[style=bspstyle]{ex}[defi]{Example}
% just have one, Property, instead of Theorem, Lemma, Corollary?


\newtheoremstyle{rem}
  {\topsep}{\topsep}
  {}{}%{\centering}{0.1\textwidth}
  {\bfseries}{\textbf{remark}:}
  { }{}
\theoremstyle{rem}
% might be unnecessary now

\mdfdefinestyle{remstyle}{%
rightline=false,leftline=false,topline=false,bottomline=false,%
backgroundcolor=myyellow,innerleftmargin=.4\baselineskip,innerrightmargin=.4\baselineskip,leftmargin=-.4\baselineskip,rightmargin=-.4\baselineskip}%setting the indentations is important because otherwise, everything will be indented (.4\baselineskip is default and looks natural, so this is chosen; the effects of margin and innermargin have to be balanced)
%,frametitle={\textbf{remark}: }}%frametitle also makes linebreak

\newmdenv[style=remstyle]{remark}%{remark}
%\newmdtheoremenv[style=remstyle]{rem}{remark}
%\mdtheorem[style=remstyle]{rem}{remark:}%allows use of \begin{rem*} for no numbering

%\newcommand{remark}[1]{\begin{rem*}: #1\end{rem*}}
%use of begin, end is not allowed before \begin{document}


%Lösung (also Umgehen von Verbot \begin{} in Präambel) kommt von: https://www.mrunix.de/forums/showthread.php?59532-begin-und-end-in-newcommand
\def\brem#1\erem{\begin{remark}#1\end{remark}}
\newcommand{\rem}[1]{\brem \textbf{remark:} #1\erem}
% finally, now \rem{} is a shortcut for \begin{remark} etc.

% new line not always wanted for remarks, thus change to this here
\usepackage{soul}
\sethlcolor{myyellow}
\newcommand{\question}[1]{\hl{#1}}


% Anpassung von itemize-Symbolen
\renewcommand{\labelitemi}{$\blacktriangleright$}%{$\vartriangleright$}
\renewcommand{\labelitemii}{\textbf{--}} % is also default there
\renewcommand{\labelitemiii}{$\bullet$}


% Shortcuts -> falls man Abkürzung mal ändern will, muss man dann nicht den ganzen Text durchgehen
\usepackage{xspace} %weil man sonst \gw{} callen muss, damit Leerzeichen danach erkannt werden.
\newcommand{\gw}{{\hypersetup{linkcolor=black}\ac{gw}}\xspace}
\newcommand{\gws}{{\hypersetup{linkcolor=black}\acp{gw}}\xspace}

\newcommand{\mi}{{\hypersetup{linkcolor=black}\ac{mi}}\xspace}

\newcommand{\art}{{\hypersetup{linkcolor=black}\ac{art}}\xspace}

% wenn was nicht klappt, dann \gw{} callen
% mit diesem Ding leider kene Nutzung in Überschriften möglich

%\newcommand{\Var}{{\fontfamily{ptm}\selectfont\text{var}}}
%\newcommand{\Cov}{{\fontfamily{ptm}\selectfont\text{cov}}}
%\newcommand{\Corr}{{\fontfamily{ptm}\selectfont\text{corr}}}

% this is better, auto-select fonts etc
\DeclareMathOperator{\Var}{var}
\DeclareMathOperator{\Cov}{cov}
\DeclareMathOperator{\Corr}{corr}


%\renewcommand{\bibname}{References}
\addto\captionsenglish{\renewcommand{\bibname}{References}}



% if float is too long use \thisfloatpagestyle{onlyheader}
\newpairofpagestyles{onlyheader}{%
\setlength{\headheight}{\baselineskip}
\automark[section]{section}
%\automark*[section]{subsection}
\ihead[]{\headmark}
%
% only change to previous settings is here
\cfoot{}
}


\newpairofpagestyles{onlyfooter}{%
\setlength{\headheight}{\baselineskip}
\automark[section]{section}
%\automark*[section]{subsection}
\ihead[]{}
%
% only change to previous settings is here
\cfoot{{\hypersetup{linkcolor=black}Page~\thepage~of~\pageref{LastPage}}}
}



% for dartboard (from https://de.overleaf.com/latex/templates/dartboard/bhpfmdvjsjmk)
\tikzstyle{wired}=[draw=gray!30, line width=0.15mm]
\tikzstyle{number}=[anchor=center, color=white]
%%%<
\usepackage{verbatim}
%%%>
\begin{comment}
:Title: Dartboard
:Tags: Foreach; Node positioning
:Author: Roberto Bonvallet
:Slug: dartboard
\end{comment}

% Sectors are numbered 0-19 counterclockwise from the top.

% \strip{color}{sector}{outer_radius}{inner_radius}
\newcommand{\strip}[4]{
    \filldraw[#1, wired]
      ({18 *  #2}      :                   #3) arc
      ({18 *  #2}      : {18 * (#2 + 1)} : #3) --
      ({18 * (#2 + 1)} :                   #4) arc
      ({18 * (#2 + 1)} : {18 *  #2}      : #4) -- cycle;
}

% \sector{color}{sector}{radius}
\newcommand{\sector}[3]{
    \filldraw[#1, wired]
      (0, 0) --
      ({18 * #2} :                   #3) arc
      ({18 * #2} : {18 * (#2 + 1)} : #3) -- cycle;
} \graphicspath{../}


\begin{document}



\chapter{Newtonian Mechanics}

The ultimate goal of physics is to study the interaction between different objects in the physical world that we live in. For decades, such interactions have typically been described in terms of forces, which are used to describe the motion of objects that is caused by a force or rather the interaction it corresponds to. This backbone of physics goes back to laws (or rather axioms) that Sir Isaac Newton has formulated and published in his famous \enquote{Philosophiae Naturalis Principia Mathematica} in 1687. Stunningly, the Newtonian mechanics derived from this accurately describes the dynamics of many physical scenarios, with the exception of very extreme situations (such as very small objects, where quantum effects must be accounted for, very fast objects, where relativistic effects come into play, or objects in very strong gravitational fields). 



    \section{Newton's Laws of Motion}
% -- Sources:
% - https://en.wikipedia.org/wiki/Newton%27s_laws_of_motion
% - https://en.wikipedia.org/wiki/Inertial_frame_of_reference
% - https://physics.stackexchange.com/questions/70186/are-newtons-laws-of-motion-laws-or-definitions-of-force-and-mass/339561#339561 -> answer by joshphysics is just incredible; also discussion in the comments, on existence of inertial frames when taking into account that gravity = acceleration (they don't, only local inertial frames exist! In small patches, we can have freely falling property, e.g., because all gravitational fields cancel, I think that's what they say)
% - https://physics.stackexchange.com/questions/66057/can-one-of-newtons-laws-of-motion-be-derived-from-other-newtons-laws-of-motion?rq=1
% 
% 
These axioms define how objects (which are sometimes given the abstract name \Def{observer} $\mathcal{O}$; these give us viewpoints for our descriptions) experience physics. 

-> introduce reference frame (?); feels natural now that we have defined observer; Wikipedia defines them as \enquote{an abstract coordinate system, whose origin, orientation, and scale have been specified in physical space}

-> In order to describe this action mathematically, we also need explicit ways to assign the position $\vec{r}$ of observers to points $\in \mathbb{R}^3$ in the Euclidean space we live in. Such an assignment is what physicists call \Def{coordinates} or \Def{frames}. -> make clear that $\vec{r}$ is symbol reserved for positions of bodies/particles/observers, while $\vec{x}$ is general form to denote coordinates

-> this is good place to talk about what Thorne+Blandford mention; to write down things, we do \emph{not} need coordinates, Newtonian physics can be approached as a purely geometric theory; however, coordinates are useful and often needed when we are working on real-world problems


following Thorne + Blandford in Sec.~1.4, emphasize how we can formulate Newtons laws entirely geometrically, without any coordinates! Of course, we \emph{can} use coordinates (as we do frequently, by writing laws in terms of components), but there is no \emph{need} to do so. This is very special



		\subsection{First Law}
\begin{axiom}[First Law]
	\centering
	Every body remains at rest or in a uniform motion, unless a force acts upon it.
\end{axiom}
One interpretation of this statement is that it asserts the existence of a special class of reference frames, which are best suited to describe physics.
% -> \url{https://physics.stackexchange.com/questions/70186/are-newtons-laws-of-motion-laws-or-definitions-of-force-and-mass/339561#339561}
\begin{defi}[Inertial Frame]
	An \Def[inertial frame]{inertial frame (of reference)} is a frame where the first law holds.
\end{defi}

The reason that we like inertial frames is that physical laws take their simplest form in these frames (explicit expressions can be determined using the second law). We discuss complications that come with non-inertial in a dedicated Section (cf.~Sec.~\ref{sec:non_inertial_frames}).\footnote{These \enquote{complications} manifest as extra terms, so-called \enquote{fictitious forces}, that have to be added manually.} Therefore, they are simply a convenient choice when working out physical problems.


The way that Newton thought about inertial frames was an absolute one. He envisioned that there is one \enquote{master inertial frame}, defined in terms of fixed stars on the night sky. In other words, Newton believed in an \Def{absolute space}. In a modern interpretation of physics,\footnote{It is worth pointing out that even back in Newton's time, there were people that did not believe an absolute space existed, e.g., Leibniz.} this is not believed to be true anymore, the concept of absolute space has been superseded by Einstein's relativity principle. Nonetheless, there is no need to modify the first law because it just postulates the existence of \emph{some} inertial frame, not necessarily of absolute space. 

-> \todo{rephrase previous paragraph by incorporating some of following} It was brilliant Newton to formulate these postulates from what people knew about everyday life, but he was a little absolutistic in their interpretation: in particular, he believed in an absolute space where a preferred inertial frame existed, from which all other inertial frame can be derived/defined (i.e.~that there is an an absolute frame of reference that we can fall back to when defining inertial frames). Albert Einstein challenged this thought more than two decades later, by proclaiming that there is no absolute space. This resolved many issues that were present at that time, making it the preferred interpretation until today. The relativity principle states that there is no absolute notion of rest when it comes to uniform motion. It is, however, possible to detect if ones frame is accelerated, i.e.~it is possible to know whether one is in an inertial frame of reference or not (we will expand on this later on).


% In addition to that, it gives us a systematic way to obtain new inertial frames
In addition to serving as a definition of inertial frames, the first law implicitly specifies even more, namely how to define a second inertial frame $\Sigma'$ given a first such frame $\Sigma$. We can simply choose $\Sigma'$ to move uniformly with respect to $\Sigma$; this still obeys the requirements on an inertial frame because an observer in $\Sigma'$ sees (i) resting bodies in $\Sigma$ move uniformly, and (ii) objects moving with uniform velocity in $\Sigma$ still move with uniform, albeit different, velocity in $\Sigma'$. This relation is reflected mathematically in the set of transformation between different two inertial frames (a subset of all possible coordinate transformations), the elements of which are called \Def{Galilei transform}. In case $\Sigma'$ moves with velocity $v$ relative to $\Sigma$, one can map coordinates between according to
\begin{equation}\label{eq:galilei_transform}
	\eqbox{
		x \rightarrow x' = x + v t
	}
	\qquad \qquad
	\eqbox{
		y \rightarrow y' = y
	}
	\qquad \qquad
	\eqbox{
		z \rightarrow z' = z
	} \, .
\end{equation}




		\subsection{Second Law}
\begin{axiom}
	\centering
	If a force $\vec{F}$ acts upon an object of mass $m$, it causes a change in momentum
	\begin{equation}\label{eq:newton_second_law}
		\eqbox{
			\vec{F} = \dv{\vec{p}}{t}
		}
	\end{equation}
	where momentum is defined as
	\begin{equation}
		\eqbox{
			\vec{p} = m \vec{v}
		} \, .
	\end{equation}

	If more than one force is present, $\vec{F}$ must be replaced by the superposition of all forces,
	\begin{equation}
		\eqbox{
			\vec{F}_\mathrm{tot} = \sum_k \vec{F}_k
		} \, .
	\end{equation}
\end{axiom}
The change in momentum in Eq.~\eqref{eq:newton_second_law} is also called \Def{impulse}. Both of the terms that the second law relates in force and momentum are frequently used in essentially all of mechanics, but their meaning is not so clear at this point. In fact, it can be hard to grasp what they even represent. We elaborate on this, and the role of the second law in their definition (spoiler: none of them are defined by the second law), in a dedicated section (cf.~Sec.~\ref{sec:forces_and_momenta}).


If $m = \text{const.}$, as is commonly the case, the second law simplifies to
\begin{equation}
	\vec{F} = m \vec{a} % = m \dv[2]{\vec{r} \todo{$\vec{x}$?}}{t}
\end{equation}
which is perhaps the second most famous formula in physics (right after $E = m c^2$). In either form, the second law determines dynamics in the Newtonian physics.


We have chosen to include the superposition principle in the formulation of the second law, which may not be really common. Note that this property is not trivial, although it feels very natural because forces are vectors and can thus be added together in a straightforward manner; still, a priori there is no obvious reasons why it is their linear combination that yields $\vec{F}_\mathrm{tot}$, instead of something like $\vec{F}_\mathrm{tot}^2 = \sum_k \vec{F}_k^2$. In the literature, there are many ways in which this is handled, some even choose to propose this as a fourth law of motion. No matter what is your preferred option, the most important takeaway is that the superposition principle is \emph{not} something that holds by definition, it is an empirical insight and should thus be considered (part of) an axiom.\\


Coming back to Eq.~\eqref{eq:newton_second_law}, this is the \enquote{simplest form} of physical laws that has already been alluded to. In particular, this is what physical laws look like in inertial frames. The fact that the relation between forces and momenta takes this particular form in an inertial frame is one of the big reasons why this type of reference frame is so important and widely used. One quick sanity check that this is actually true is the case where no external forces are present, i.e.~$\vec{F} = 0$. The second law then implies $\vec{a} = 0$, provided we look at a test particle with fixed mass; in other words, the test particle remains at rest or moves uniformly, depending on the initial state of motion. This is the definition of an inertial frame.\footnote{I have seen definitions of the term inertial frame as \enquote{a frame where $\vec{F} = m \vec{a}$ holds}. In my opinion, this is confusing because it suggests that the first and second law are somewhat dependent on each other. However, this is not really the case. The term inertial frame makes perfect sense without the second law and the second law is a statement that cannot be deduced from the first law.} That the second law generally holds in an inertial frame is, again, an empirical insight (for this reason, it is sometimes explicitly included in the second law that Eq.~\eqref{eq:newton_second_law} holds in in inertial frames). Another indicator for the special role of inertial frames in this regard is that extra terms are needed in non-inertial frames to correct predictions of the second law in those frames (which we show in Sec.~\ref{sec:non_inertial_frames}).

% Not only do we get valid physical laws in every inertial frame -- the laws obtained for the same physical situation in each frame are also equivalent.
The previous discussion established that one does get valid physical laws in every inertial frame -- but what is the relation between those laws?
We can see this as follows: by definition, velocities $\vec{v}, \vec{v}'$ measured in different inertial frames $\Sigma, \Sigma'$ measure at maximum by a uniform component, $\vec{v} - \vec{v}' = \text{const}$. Therefore,
\begin{equation}
	\vec{a} = \dv{\vec{v}}{t} = \dv{\vec{v}'}{t} = \vec{a}'
	\quad \Rightarrow \quad
	\vec{F} = m \vec{a} = m \vec{a}' = \vec{F}'
	\, .
\end{equation}
This means Newton's second law looks exactly the same in both (and thus all) inertial frames -- the laws obtained for the same physical situation in each inertial frame are the same, and thus also the inferred physics. A very convenient consequence of this is that we as applicants of Newton's theory are free to choose the frame from which we treat problems, there is no preferred frame that we must use.\footnote{This is a manifestation of the relativity principle (the origins of which date back to Galileo Galilei; but it was Einstein who first recognized and explored the full consequences of it). It is also this point where cracks in the interpretation of absolute space begin to show. While it does not hinder us in any way, it is also not required to do physics -- then, what is it even good/required for?}

% -> while the values of $\vec{v}$ and $\vec{v}'_2'$ might differ and thus depend on the inertial frame we choose, the physics inferred from them does not (forces are what we observe \todo{but this is not decisive factor, is it? Is simply that interplay of chosen coordinates and forces acting in them is always equivalent}). This explains the preferred role of inertial frames when describing physics. Laws stated in inertial frames hold in all of them, while laws in non-inertial frames do not, whence it is often non-trivial to find out whether effects can be attributed to some physical process or to the frame/coordinates used to describe the situation (for example, the Coriolis force is needed to explain effects on Earth's surface, which is a rotating and thus non-inertial frame of reference).


-> another consequence: it is impossible for scientists to set up an experiment that determines the velocity that the frame moves with (tells us already that absolute space is wrong as a concept)



		\subsection{Third Law}
\begin{axiom}
	\centering
	If two bodies exert forces on each other, these forces have the same magnitude but opposite directions,
	\begin{equation}\label{eq:newton_third_law}
		\eqbox{
			\vec{F}_{12} = - \vec{F}_{21}
		} \, .
	\end{equation}
\end{axiom}
This motto is sometimes stated as \enquote{actio = reactio}, though this is much too short to properly encapsulate what the third law expresses.


-> in the end, this tells us something about equilibrium of objects (how we can quantify); not directly from expression, but third law allows us to calculate net forces (e.g., between system of multiple particles) that we need in second law; in particular, tells us how a system of particles can have no net force if no external forces are present, despite intrinsic interactions taking place


-> manifestation of a more fundamental principle, conservation of momentum (Wikipedia says that, think about; though I guess discussion in previous paragraph is just that)



The third law is probably the hardest to grasp, but it has profound implications on how one should think about forces etc.~intuitively.

-> when I push table, I feel the force it exerts on me (not the one I exert)


mathematically speaking, it tells us something about symmetries of Newtonian dynamics (don't know how to elaborate further on that...)


\hrule

\begin{ex}
	astronaut example who pushes stones away in space
\end{ex}


\begin{ex}[Momentum Conservation]
	A very interesting consequence of the third law arises when we remember how forces are related to momenta, i.e.~the second law. If we consider a physical system with no external forces acting on it, all forces acting are between the respective particles. The third law now implies that the total force must vanish since for each pair of particles
	\begin{equation}
		\vec{F}_{ij} + \vec{F}_{ji} = \vec{F}_{ij} - \vec{F}_{ij} = 0
	\end{equation}
	where we use indices $i, j$ to refer to the $i$-th and $j$-th particle. Therefore, the total force when considering all particles also vanishes,
	\begin{equation}\label{eq:force_equilibrium}
		\eqbox{
			F_\mathrm{total}^k = \sum_{i, j} F_i^k = 0
		}
	\end{equation}
	where we denote by $\vec{F}_i$ the total force acting on particle $i$.

	In terms of the corresponding momenta, this implies
	\begin{equation}\label{eq:momentum_conservation}
		\eqbox{
			\sum_{i} \dv{p_i^k}{t} = 0
		} \, ,
	\end{equation}
	the total momentum of the system is constant. A common way to phrase this is saying the momentum is \Def[conservation]{conserved}.

	Exploiting this property alone is already enough to treat many physical problems. One example is the collision of two (or more) particles, where momentum conservation implies
	\begin{equation}\label{eq:implication_momentum_conservation}
		\eqbox{
			\sum_j p_j^k = \sum_j \tilde{p}_j^k
		} \, .
	\end{equation}
	Here, $p_j^k$ denotes the momentum of the $j$-th object before the collision and $\tilde{p}_j^k$ after.

	-> this property is big reason why momenta are useful, despite their kind of abstract meaning

	-> mention that this condition is also called dynamic equilibrium; because of course $\vec{F} = 0$ does not mean nothing happens here with all particles at rest, just means there are no forces acting on particles from the outside
\end{ex}

As we have seen in this example, if there are no net external forces on a system, the total force vanishes (cf.~Eq.~\eqref{eq:force_equilibrium}). In particular, this characterizes the equilibrium of the system and is thus also a common way to determine how particles behave in such a system, given the forces between them. \todo{check wording here. because no external force does not always imply equilibrium, this is not what we are trying to say} -> point is: to determine how system in equilibrium looks like, we can analyze all forces acting on the system and demand that they cancel each other in the end (because then Newton 2 tells us momentum is conserved); this is frequently done to get relation between different forces in different physical situations

important: momentum is conserved when no \emph{net} external force is present! There can be forces, as long as total force cancels (which corresponds to equilibrium)

-> if net force on a single object is zero, then it is in equilibrium; this is something we can say (avoids what we had above, potential issue with interaction in closed system, e.g. a collision, where total system may not have force, but calling this equilibrium would also be tough...)


\hrule


-> first law tells us about good frame we may use (inertial), where laws look same; second one tells us how laws look like; third one helps us in practice with writing down second one explicitly

one of the big achievements (advancements?) of these axioms is that they give us a systematic way to write down an equation of motion, which determines how the current state of a system will evolve with time (mainly comes from Newton 2, but insights from other axioms help formulating these equations as well; e.g., trick with no net force can often only be applied due to Newton 3, etc.); granted, we need state of the system at some point for that, i.e.~an initial condition, but this is typically what we have: we ask questions like \enquote{what happens if I drop a ball}, which naturally includes knowledge of the initial setup of the experiment. Solving the equations of motion then gives us the final state of such an experiment, as well as state at every time between beginning and end (we are guaranteed that solution can be found, at least numerically, from the existence of procedures such as the Picard-Lindelöf iteration)



		\subsection{Gravitation}
% https://en.wikipedia.org/wiki/Newton%27s_law_of_universal_gravitation
Along with the axioms that bear his name, Newton published his \Def{universal law of gravitation} in the Principia, another revolutionary accomplishment. It states that the gravitational force exerted by a body with mass $M$ onto another body of mass $m$ is
\begin{equation}\label{eq:newton_gravitation_law}
	\eqbox{
		\vec{F} = - \frac{G M m}{\norm{\vec{x}_m - \vec{x}_M}^3} (\vec{x}_m - \vec{x}_M) \eqqcolon - \frac{G M m}{r^3} \vec{e}_r
	}
\end{equation}
where $\vec{e}_r \hat{=} \vec{e}_{Mm}$ is the unit vector pointing from $M$ to $m$, and $r$ is the norm of this connecting vector (the reason that it is usually denotes by $\vec{e}_r$ or $\hat{r}$ is that $r$ is the distance of bodies from Earth's origin, a common application \todo{improve wording here}). This law naturally obeys the third law, since $\vec{e}_{mM} = - \vec{e}_{Mm}$.

\todo{start using $\vec{r}$ to denote positions? Does have advantage that $\vec{x}$ could then be used to describe vectors of arbitrary kind}

-> $G$ is \Def{Newtonian constant of gravitation}, experimentally determined to be
\begin{equation}
	\eqbox{
		G = 6.674 \cdot 10^{-11} \, \frac{\mathrm{m}^3}{\mathrm{kg} \, \mathrm{s}^2}
	}
\end{equation}


\Def{gravitational field} is $\vec{g} = \frac{\vec{F}}{m}$ (\enquote{force per unit mass exerted on some particle that is subject to gravitational force from body of mass $M$})

there is a gravitational potential $\Phi$ with $\vec{g} = - \grad \Phi$; it is easy to show that $\Phi = - \frac{M}{r}$; implies force is conservative, $\vec{F} = - \grad U$ with $U = m \Phi$



        % \subsection{What Is A Force?}
        \subsection{Forces \& Momenta}
		\label{sec:forces_and_momenta}
% -- Sources:
% - https://en.wikipedia.org/wiki/Force
% - https://en.wikipedia.org/wiki/Momentum
% - https://physics.stackexchange.com/questions/832588/what-is-the-definition-of-force/832621#832621
% - https://physics.stackexchange.com/questions/818140/philosophical-discussion-of-newtons-second-law/818299#818299
% 
% 

% -> force = something that can change state of motion (gives more precise meaning to/is mathematical formalization of things like pushing or pulling); state of motion is described by momentum; Newton 2 relates these two (though I guess, to some degree, it also defines them)

% in the same spirit, we may ask: what is momentum? Well, the answer is: the quantity that is changed in time by an external force, i.e.~the quantity showing up in Newton's second law. If this law had been different (because of different underlying physics of the world), we would perhaps not be bothering to deal with the expression $m \vec{x}$ because there would be no immediate physical meaning to it.


Now, let us come to a very philosophical question: Does Newton's second law define forces or does it define momenta? What even is a force and what is momentum?

Intuitively speaking, a force is something that changes the state of motion of an object. This state of motion is quantified by its mass and velocity, i.e.~through the notion of momentum. The second law then relates the two. But in some sense, it also serves as the definition of a force: it formalizes the intuitive concepts of pushing and pulling that we associate a force with into a proper (though the concept itself exists without the law), mathematical definition, by fixing its relation to the change of momentum (fixing in the sense that we choose to include no additional constant factors in the definition, which could alter the units and values of forces, without changing how we think about them).
% The second law cannot, however, be seen as a true definition of a specific force because it quantifies the net effect that all present forces have on a certain object; in other words, while this can be used to determine a specific force (having made sure no other forces are present), it is not sufficient as a definition of such a force. \todo{improve wording here, do not like it}

% Forces are our way to quantify interactions between objects and it formalizes the intuitive concepts of pushing and pulling that we associate a force with into a proper, mathematical definition.
-> forces are our way to quantify interactions between objects

Momentum, on the other hand is an independent concept and we can see why a description of state of motion should involve both mass and velocity. The latter should be kind of obvious, objects move with certain velocities. But it is also intuitive that a fly moving at $42 km/h$ is very different from a car moving at the same speed, these two situations represent very different motions.
Now, part of this also comes back to the intuition that is encapsulated in the second law, namely that more force is required to get the car to move at such a speed, but that just shows how the second law is an incarnation of some relation that is present in our physical reality. \todo{improve this part, starting at "intuition that is encapsulated"}


Admittedly, I myself am not perfectly suited to point out all the little details and pitfalls that come up when decoupling the different definitions and laws; in essence, this is a deeply philosophical topic and I have simply not thought about it long enough (yet). In the end, the boring answer as to why forces and momenta are defined in the way they are is that these definitions happen to have really convenient properties -- they define things in the \enquote{right} way, as experience has shown us. For example, momentum happens to be conserved in the absence of external forces -- thus it makes sense to study it.


inertia = how much something resists to being moved; straightforward way to measure this abstract concept is mass


-> second law is truly a law, neither force nor momentum are defined by it


this has some nice insights: \url{https://physics.stackexchange.com/a/589005}


excellent takes and clarifications: \url{https://physics.stackexchange.com/questions/832588/what-is-the-definition-of-force/832621#832621}



Because of their relation to forces, some fundamental principles involve momenta. \todo{is that true? Or are momenta they also relevant independently of second law?}



mention that forces are assumed to have instantaneous effect; poses problem in relativity (could that even be reason for one of the remarks, that we have to shift to fields?)


\hrule


Now that we have defined forces, here are some of its properties that we should be aware of. Usually when talking about forces, we actually refer to force fields that vary in space and time, i.e.~$\vec{F} = \vec{F}(\vec{x}, t)$ (unlike for momentum, which is defined purely as function of time).


emphasize properties of a real force here, to be able to contrast it with fictitious forces later on


% -> note conservative property, maybe even connect with following property already: Dass man konservative Kräfte (für die $\nabla \cross F = 0$, die also ein Potential haben) in die Form $F = - \nabla V$ bringen kann, sagt uns schon unglaublich viel ! Wenn man dann nämlich den GGW-Zustand eines Systems finden will, so entspricht das offenbar gerade $F = 0$ und daher sind dann die Extrema von $V$ interessant ! Weil uns Kräfte zudem immer in die Richtung des Minimums von $V$ drücken (so ist es quasi nach Definition), sucht man spezieller Minima (Maxima sind instabil). Daher kommt es, dass Systeme immer in den Zustand minimaler Energie streben, das Potential entspricht nämlich einer Energie und soll ja minimiert werden (daher wird auch Gesamtenergie minimiert, weil ja kinetische im Kräfte-GGW auch konstant ist oder ? -> ok, so doch nicht). Allgemeiner sucht man bei nicht konservativen Kräften dann Kräfte-Gleichgewichte (weil ja immer noch $F = 0$ sein muss) und setzt die daher gleich dann.



		\subsection{Energy}
Another concept of fundamental importance in physics is \Def{energy}. What is a good definition of it? And why are we interested in it? This is another tough question and my answer to it is similar to what we noted about forces and momenta: It is a choice that we study this quantity, and this choice is (i) made for historical reasons, but (ii) also because they turn out to be convenient because we can ultimately derive things like conservation laws for them -- it simply turns out to be a good quantity to study the behavior of physical systems.

Now let's come to some definitions. The \Def[energy!kinetic]{kinetic energy} that a particle with momentum $\vec{p}$ and mass $m$ carries is
\begin{equation}
	\eqbox{
		E_\mathrm{kin} = \frac{p^2}{2 m} = \frac{m v^2}{2}
	} \, .
\end{equation}


But it is also possible to analyze and classify forces in terms of energy. Some forces have a \Def{potential} $V = V(\vec{x})$ that can be used to express them as
\begin{equation}\label{eq:conservative_force}
	\eqbox{
		\vec{F} = - \grad V = \pdv{\vec{x}} V
	} \, .
\end{equation}
This type of forces is called \Def[force!conservative]{conservative}, while non-conservative forces are also called \Def[force!dissipative]{dissipative}. $V$ has dimensions of energy and is also called \Def[energy!potential]{potential energy}. Conservative forces play a special role in how they interact with the energy of a system. This is quantifies by the \Def{work} done by a force on a particle moving on a path between two positions $\vec{x}_1, \vec{x}_2$,
\begin{equation}\label{eq:work}
	\eqbox{
		W = \int_{\vec{x}_1}^{\vec{x}_2} \vec{F} \cdot d\vec{x} = \int_{t_1}^{t_2} \vec{F} \cdot \dot{\vec{x}} \, dt
	}
	\quad \Leftrightarrow \quad
	\eqbox{
		\dv{W}{t} = \vec{F} \cdot \vec{v}
	}
\end{equation}
-> subtracted when it takes energy away from particle

-> implicit reference point for $V$ is $\infty$, in the sense that $V = \int_r^\infty F(r')\, dr'$

By some (quite beautiful) mathematical theorem, the property we noted as a definition of conservative forces implies
\begin{equation}\label{eq:work_conservative_force}
	\eqbox{
		W = \int \vec{F} \cdot d\vec{s} = \int_{s_0}^{s_1} - \nabla V \cdot d\vec{s} = - [V(s_1) - V(s_0)] = V(s_0) - V(s_1)
	} \, ,
\end{equation}
work done by them is path-independent; as a bonus, it may be expressed as a potential difference. In particular, work done on a closed path vanishes,
\begin{equation}\label{eq:work_conservative_force_closed_path}
	\eqbox{
		W = \oint \vec{F} \cdot d\vec{x} = 0
	}
\end{equation}
Both Eq.~\eqref{eq:work_conservative_force} and Eq.~\eqref{eq:work_conservative_force_closed_path} are equivalent definitions of conservative forces. A sufficient condition to check whether a force is conservative is that its curl vanishes,
\begin{equation}\label{eq:conservative_force_condition}
	\eqbox{
		\curl \vec{F} = 0
	}
\end{equation}
(which is due to the same beautiful theorem as before; you should really go read up on Stokes' theorem if you don't know it yet).

For dissipative forces like friction, all of these characteristics are not true. Energy lost on closed path gets turned into heat (so, of course, it is still somewhat conserved, but does not go into the motion of the system that we are primarily studying in this chapter), \enquote{mechanical energy = kinetic + potential energy} is not true anymore.\\


% -> work is path-dependent in general; if not, we have special type of force, called \Def{conservative force} and one of the characteristics of them is that they have a potential $V$ -> I also like this motivation, but right now things are kind of nicer in order


% -> nice motivation: say we have particle moving at velocity $v$; what is change in kinetic energy when an external force now acts on particle? Well, should involve $\vec{F} \cdot \vec{v}$ somehow -- turns out, that's already it! This is work -> not sure this is proper motivation for correct units and so on


Now some notes on this definition of work: we see that a force can do no work when acting on a particle, which is precisely the case when $\vec{F} \perp \vec{v}$; however, that does not mean the force has no effect, its interaction simply does not lead to a change in the speed as the magnitude of $\vec{v}$, only in its direction. Moreover, the \enquote{beautiful theorem} used to relate work and potential could very well be applied if we had $\vec{F} = \grad V$ as well; the minus sign in this definition is purely conventional, and in the chosen convention, work is gained from a loss of potential energy. As far as I can see, this is made for consistency with the behavior inferred from kinetic energy, where no motion means no energy. In a similar fashion, the minus sign means that no force is acting when $V$ is minimal (instead of maximal). This is actually a profound insight: Newton's laws dictate that $\vec{F} = 0$ corresponds to particles maintaining their state of motion, so in particular they remain at rest if they are initially. From what we just noticed, an equivalent way to say this is that the particle is in equilibrium when its energy is minimized. Moreover, we see that $\vec{F} = 0 \Rightarrow \dv{E}{t}$, so the total energy is also conserved if a system is in equilibrium, no matter if the equilibrium is static or dynamic.


This insight is a first indication that many laws and ways to think about system can be formulated both in terms of forces and energies. While both formulations are equivalent in terms of the physics they imply, depending on the situation it might be much easier to formulate one of them. Hence it is always good to think about which approach seems more promising. We will demonstrate this in an example.

\begin{ex}[Throwing A Ball Up]
	let's say we launch a ball in straight vertical line (to avoid angles \todo{perhaps not so hard to include them} -> $\theta$ is angle from ground to $\vec{v}$, so $\theta = \pi/2$ means $\parallel \vec{e}_z$) so that it has initial velocity $v_0$

	some interesting questions: how high does ball fly (above point it was launched), as function of $v_0$? how long does ball take until this point?

	\paragraph{Analysis With Forces}
	only force acting is gravity, $F_g = - m g$

	determine trajectory from second law: $m \vec{a} = - m g \vec{e}_z$, which implies $\vec{v}(t) = - g \vec{e}_z t + v_0$ and $\vec{x}(t) = - 1/2 g \vec{e}_z t^2 + \vec{v}_0 t + \vec{x}_0$

	from initial conditions we have initial velocity and without loss of generality, $\vec{x}_0 = 0$ (we measure difference to initial position)

	maximum height? look for time $t_\mathrm{max}$ where vertical velocity is zero, i.e.~$\dv{z}{t} = - g t_\mathrm{max} + v_0 \sin \theta = 0$, which yields $t_\mathrm{max} = \frac{v_0 \sin \theta}{g}$

	-> substitute into trajectory to get maximum height: $z(t_\mathrm{max}) = - 1/2 g (\frac{v_0 \sin \theta}{g})^2 + v_0 \sin \theta \frac{v_0 \sin \theta}{g} = 1/2 \frac{v_0^2 \sin^2 \theta}{g}$



	\paragraph{Analysis With Energies}

	initial kinetic energy: $E_0 = \frac{1}{2} m v_0^2$ -> ah, we are only interested in $z$-component of this, right? Because any horizontal component of motion does not change question we want to address

	gravitational force constantly does work on ball, which effectively reduces energy $E_0$; total amount of work done at height $h$ (as long as it is still going ups) above starting point is $W = - m g h$

	in point of turnaround, ball briefly has no energy (after turnaround, it gains energy again due to gravity pulling on it; velocity in different motion, but energy of course still positive -> ah no, work actually becomes positive after turnaround, since then $\vec{F} \parallel \vec{v}$ instead of anti-parallel); from $E_0 + W = 0$ we get $h = \frac{v_0^2}{2 g}$, same we got from force analysis (but much quicker, actual calculation is very short)
\end{ex}


therefore, we can circumvent solving equations of motion if energy is conserved! similar for other quantities, this is why \Def{integrals of motion} are so useful



	\section{(Harmonic) Oscillations}
back to one of the absolutely fundamental systems in physics, the harmonic oscillator

first, let us motivate why we should even deal with it -- turns out, plays pivotal role in physics!



		\subsection{Oscillations Around Equilibrium}
		\label{subsec:osc_around_equ}
physicists like Taylor expansions and one of the places where it is really useful is the analysis of equilibrium states

reason: we know that equilibrium means no net force; in terms of the potential, equilibrium means $\nabla V = 0$, potential is at a minimum

-> for a particle that moves around equilibrium position $x_0$, we can Taylor-expand potential:
\begin{equation*}
	V(x) = V(x_0) + \dv{V}{x}(x_0) \, (x - x_0) + \frac{1}{2} \dv[2]{V}{x}(x_0) \, (x - x_0)^2 + \mathcal{O}((x - x_0)^3) = V(x_0) + \frac{1}{2} \dv[2]{V}{x}(x_0) \, (x - x_0)^2 + \mathcal{O}((x - x_0)^3)
\end{equation*}
since the equilibrium condition translates to $\dv{V}{x}(x_0) = 0$. Therefore, choosing $V(x_0)$ as our reference for the potential (in other words: setting it to zero), the potential around an equilibrium is approximately quadratic,
\begin{equation}
	\eqbox{
		V(x) \simeq \frac{1}{2} \dv[2]{V}{x} \, (x - x_0)^2 \eqqcolon \frac{k}{2} (x - x_0)^2
	}
	\, .
\end{equation}

-> equilibrium is \Def{stable} if $\dv[2]{V}{x}(x_0) > 0$ (minimum, upon a small perturbation the system will tend back into $x_0$) and unstable if $\dv[2]{V}{x}(x_0) < 0$ (maximum, any small perturbation will lead to evolution into very different position); in case of $\dv[2]{V}{x}(x_0) = 0$, we must go to higher terms



but what about associated forces? For that, we assume equilibrium $x_0$ exists and choose coordinates so that $x_0 = 0$ (1D for now)

-> restoring force: we also do expansion to leading order; where do we stop? Well, let's use $V(x) = \frac{k}{2} (x - x_0)^2 = \frac{k}{2} x^2$ (where we know that $k = \dv[2]{V}{x}(x_0 = 0)$; but this explicit form does not matter, most important part is that it is constant with respect to $x$) and
\begin{equation}\label{eq:restoring_force}
	\eqbox{
		\vec{F} = - \grad V = - k x
	} \, .
\end{equation}
Thus linear order seems sufficient, and this force we call \Def{restoring force}. Basically, it pushes any displacement from equilibrium back to the equilibrium at $x_0 = 0$. (Note that this is equivalent to Taylor-expanding the force, i.e.~to $F(x) \simeq F(0) + \dv{F}{x}(x_0) \, x \eqqcolon - k x$.) This implies $k > 0$ (makes sense, we assume stable equilibrium), and we also see that restoring forces are linear forces because only $x$ is appearing.

\todo{but how does that work here, just because one of the forces is conservative, this does not mean its potential is at minimum, right? I mean if total force is vanishing, then this just means total potential is at minimum, right? Or do we refer to dynamical equilibrium here, i.e.~we have restoring force equal to $m a$? -> yeah, think so; don't worry about it I guess...}


systems with forces of type $F = - k x$ are said to obey \Def{Hooke's law} (for many systems, this is approximate, but for some it holds more exactly)



		\subsection{The Harmonic Oscillator}
If a restoring force is acting on a physical system, then Newton's second law tells us
\begin{equation}
	- k x = m \ddot{x}
	\quad \Leftrightarrow \quad
	\eqbox{
		\ddot{x} + \omega_0^2 x = 0
	}
	, \quad
	\eqbox{
		% \omega_0 = \sqrt{\frac{k}{m}}
		\omega_0^2 = \frac{k}{m}
	}
\end{equation}
where we have used the common reformulation in terms of the quantity $\omega_0$, the meaning of which we do not know yet.

-> note: just as something general to remember, here we have second order differential equation; means two integrations are required to solve it, so we anticipate two constants will appear and are to be determined


-> general solution is complex exponential, but since physical trajectories are real functions, we can always rewrite them as either one of the following two:
\begin{subequations}
	\begin{align}
		x(t) &= A \sin(\omega_0 t - \delta_1)
		\\
		x(t) &= A \cos(\omega_0 t - \delta_2)
	\end{align}
\end{subequations}
where $\abs{\delta_1 - \delta_2} = \pi/2$ (this is just relation between sine and cosine). $A$ and $\delta_i$ are determined by the initial conditions of the problem.

\todo{think about using general solution with complex exponentials... Like that more in ExPhy 1 summary} -> at least take expressions for $\delta$ from there



Period:
\begin{equation}
	\eqbox{
		T_0 = \frac{2\pi}{\omega_0} = 2\pi \sqrt{\frac{m}{k}}
	}
\end{equation}
(since this is time where argument of sine/cosine increases by $2\pi$). Therefore, $\omega_0$ is the angular (!) frequency of the motion (which just means it has an additional factor of $2\pi$ compared to frequencies).

-> more facts from Thornton: period is independent of energy, thus system is called \Def{isochronous}



\begin{ex}[Pendulum]
	Thornton example 3.3

	visualization: Fig.~\ref{fig:pendulum}


	idea: gravity = restoring force; $- m \vec{g}$ always acts downwards, i.e.~$\parallel \vec{e}_y$, but we want component that acts along angular direction (pointing towards origin), and this is $- m g \sin(\phi)$ -> see \url{https://en.wikipedia.org/wiki/Pendulum_(mechanics)#/media/File:Pendulum_gravity.svg}

	therefore, end up with
	\begin{equation}
		m l \ddot{\theta} + m g \sin(\theta) = 0
	\end{equation}
	this is pretty nasty (elliptical integral \todo{right?}, so no analytical/closed-form solution)

	-> why $l \ddot{\theta}$? Well, pendulum moves along circle (pendulum arm is assumed to be stiff, i.e.~length is always $l$), so we must calculate arc length; when moving by some angle $\theta$, the corresponding arc length is circumference of circle element cut out by this, which is simply $l \theta$ (we know this, for whole circle $\theta = 2\pi$ and yields formula for circumference of circle

	assuming oscillation around equilibrium are small, which corresponds to small angles $\theta$, we can use $\sin(\theta) \simeq \theta$ and find harmonic oscillator equation
	\begin{equation}
		\ddot{\theta} + \omega_0^2 \theta = 0, \quad \omega_0 = \sqrt{\frac{g}{l}}
	\end{equation}
\end{ex}



\begin{figure}
	\centering

	\includegraphics[width=0.3\textwidth]{pictures/pendulum.pdf}

	\caption{Pendulum setup. Taken from \url{https://tikz.net/dynamics_pendulum/}}
	\label{fig:pendulum}
\end{figure}



		\subsection{Phase Space Portraits}
general idea: eliminate dependence on time from solutions, in most cases by relating position $x$ and velocity $\dot{x}$ somehow (since these are variables that constitute the phase space) -> we can also model time evolution in there, i.e.~position for each time step; but for pictures, we are usually interested in the entirety of phase plots

-> for free (undamped, not driven, etc.) harmonic oscillator for example we get ellipse



\begin{figure}
	\centering

	% \includepdf[pages={1-4},nup=2x2]{pictures/dynamics_phaseportrait/dynamics_phaseportrait.pdf}

	\subfloat[Free harmonic oscillator (initial condition indicated as dark red point)]{\includegraphics[width=0.42\textwidth,page=1]{pictures/dynamics_phaseportrait/dynamics_phaseportrait.pdf}}%
	% 
	\subfloat[Two free harmonic oscillators with different periods]{\includegraphics[width=0.42\textwidth,page=2]{pictures/dynamics_phaseportrait/dynamics_phaseportrait.pdf}}

	\subfloat[Damped harmonic oscillator]{\includegraphics[width=0.42\textwidth,page=3]{pictures/dynamics_phaseportrait/dynamics_phaseportrait.pdf}}%
	% 
	\subfloat[Pendulum]{\includegraphics[width=0.42\textwidth,page=4]{pictures/dynamics_phaseportrait/dynamics_phaseportrait.pdf}}

	\caption{Phase space portraits. Taken from \url{https://tikz.net/dynamics_phaseportrait/}.}
	\label{fig:harm_osc_phase_space}
\end{figure}



		\subsection{The Damped Harmonic Oscillator}
Until now we have considered oscillators where the energy is conserved, so in particular we have eternal conversion from potential to kinetic energy and vice versa, the oscillation would never stop. Of course, this is not a good model of reality, where dissipative processes lead to a loss of energy. We now start to take this into account, by adding another force. This force is \enquote{restrictive} in nature, it acts against the restoring force and leads to dissipation a of energy.

-> commonly, $\vec{F} = \alpha \vec{v}$ is used as model for \Def{damping force}; for 1D, we like to express as $F = -b \dot{x}$ for $b > 0$ ($b < 0$ would not represent damping, but quite the opposite)


new equation of motion:
\begin{equation}
	\eqbox{
		m \ddot{x} + b \dot{x} + k x = 0
	}
	\quad \Leftrightarrow \quad
	\eqbox{
		\ddot{x} + 2 \gamma \dot{x} + \omega_0^2 x = 0
	}
\end{equation}
with the \Def{damping parameter}
\begin{equation}
	\eqbox{
		\gamma = \frac{b}{2m}
	}
\end{equation}


Making same ansatz as before yields solution
\begin{equation}
	\eqbox{
		x(t) = e^{-\gamma t} \qty[A_1 e^{\sqrt{\gamma^2 - \omega_0^2} t} + A_1 e^{-\sqrt{\gamma^2 - \omega_0^2} t}]
	} \, .
\end{equation}

-> type of motion now depends on the values of $\gamma$ relative to $\omega_0$



			\paragraph{$\gamma^2 = \omega_0^2$}
			% Yes, we need squares. There is nothing that tells us \gamma, \omega_0 must be greater than 1
Mathematically speaking the simplest case since exponentials in the brackets disappear.

\begin{equation}
	\eqbox{
		x(t) = C(t) e^{- \gamma t}
	}
\end{equation}
where $C(t)$ may be determined from the initial conditions (possibly consists of two constants, thus the time dependence; i.e.~$C(t) = a + b t$)


-> this is called \Def{critical damping} because amplitude falls off quickest, quicker than in both over-, underdamped cases. Such a behavior is desirable for many applications in practice, e.g., damping in cars where we do not want continuous oscillation after hitting a speed bump, or a door that closes with one of those fancy pneumatic tubes



			\paragraph{$\gamma^2 > \omega_0^2$}
relatively boring, there is no oscillatory behavior since the exponent is real, not imaginary; the motion just slowly creeps to zero


\Def{overdamped motion}



			\paragraph{$\gamma^2 < \omega_0^2$}
Now things finally become interesting. We get solution that looks very similar to the one from before:
\begin{equation}
	\eqbox{
		x(t) = A e^{- \gamma t} \cos(\omega_1 t - \delta)
	}, \quad
	\eqbox{
		\omega_1 = \pm \sqrt{\omega_0^2 - \gamma^2}
	} \, .
\end{equation}
We do have oscillations in this case of \Def{underdamped motion}.

Some subtle differences to free harmonic oscillator exist, though. First the overall damping factor $e^{- \gamma t}$, which we have encountered before already. This slowly \enquote{damps away} the oscillation until it eventually, and inevitably, reaches an amplitude of zero. One can quantify this by comparing the amplitude of successive maxima $T_{1, 2}$, which is called the \Def{decrement}
\begin{equation}
	\eqbox{
		\frac{A e^{- \gamma T_1}}{A e^{- \gamma (T_1 + T_2)}} = e^{\gamma T_2}
	} \, .
\end{equation}
This factor determines the decrease in amplitude over the period $T_2$. Note that it is also common to look at the \Def[decrement!logarithmic]{logarithmic decrement} $\gamma T_2$.

But besides impacting the amplitude, the frequency is also slightly different compared to the free motion. Generally speaking $\omega_1 < \omega_0$. But we must take a little more care to even speak about $\omega_1$ as a frequency because the motion is not actually periodic. We could try to talk ourselves out of this by noting that the underlying periodic motion, which is damped away by $\gamma$, warrants this interpretation; but on the other hand, the frequency of this underlying oscillation is $\omega_0$. This is an issue because the (temporal) distance of successive zero crossings for instance is determined by $\omega_1$. Therefore, we still use $\omega_1$ as a (kind of) frequency and also rely on a definition via the distance of zero crossings: two of them mean one period and from this we get the (angular) frequency.


% regarding period: there is an underlying periodic motion that is damped down in amplitude; we have to think about period of this underlying motion when talking about period of the damped motion -> nope, wrong; is about zero crossings


energy:



\begin{ex}[Damped Pendulum]
	Thornton example 3.3
\end{ex}



		\subsection{The Driven Harmonic Oscillator}
Driven means we apply an external force to the harmonic oscillator as well. We here consider the simplest case of a sinusoidal driving force (still in 1D)
\begin{equation}
	F_d = F_0 \cos(\omega_d t) \, .
\end{equation}
This modifies the equation of motion to
\begin{equation}
	\eqbox{
		m \ddot{x} + b \dot{x} + k x = F_0 \cos(\omega_d t)
	}
	\quad \Leftrightarrow \quad
	\eqbox{
		\ddot{x} + 2 \gamma \dot{x} + \omega_0^2 x = \kappa \cos(\omega_d t) \eqqcolon x_d(t)
	}
\end{equation}
with $\kappa = \frac{F_0}{m} = x_d(0)$. We are now dealing with an inhomogenous differential equation; the solution of this consists of two parts, one solution (\Def{complementary solution}) $x_\mathrm{hom}$ of the homogenous equation (with $x_d = 0$) and one solution (the \Def{particular solution}) $x_\mathrm{inhom}$ of the full differential equation. The latter can be shown to read
\begin{equation}\label{eq:driven_harm_osc_params}
	\eqbox{
		x_\mathrm{inhom}(t) = \frac{x_d(0)}{\sqrt{(\omega_0^2 - \omega_d^2)^2 + 4 \gamma^2 \omega_d^2}} \cos(\omega_d t + \delta) = x_{i0} \cos(\omega_d t + \delta)
	}, \quad
	\eqbox{
		\tan(\delta) = \frac{2 \omega_d \gamma}{\omega_0^2 - \omega_d^2}
	} \, .
\end{equation}
$\delta$ is a phase difference between driving force and motion of the harmonic oscillator that occurs as a consequence. It goes from $\delta(0) = 0$ through $\delta(\omega_0) = \frac{\pi}{2}$ up to $\delta(\infty) = \pi$, with the exact shape (especially slope) of the curve depending on the ratio between $\omega_0, \gamma, \omega_d$.

While in general, the motion is modelled by $x(t) = x_\mathrm{hom}(t) + x_\mathrm{inhom}(t)$, we have seen before that in the realistic case where damping losses are present, $x_\mathrm{hom}$ is a transient will go to zero at some point in time (at $t \ll 1/\gamma$). After this period of initial mixing of the two solutions, the driven motion will thus eventually dominate (i.e.~it represents the steady-state solution). During this initial transient period, the motion is heavily dependent on the initial conditions, e.g., the state of the oscillator (in both phase and amplitude) when the force is first applied.


One phenomenon of great importance in harmonic oscillators is \Def{resonance}. It basically deals with the question under which conditions $x_{i0}$ attains a maximum. From its expression, we see that this depends on the parameters $x_d(0), \omega_d$. The linear decrease with $x_d(0)$ poses a clear condition on the overall amplitude of the driving force (the more, the better), but the dependence on $\omega_d$ is more complicated. Differentiate $x_{i0}$ with respect to it yields that it is maximal at
\begin{equation}
	\eqbox{
		\omega_r = \sqrt{\omega_0^2 - 2 \gamma^2}
	} \, .
\end{equation}
(Note how compared to the frequencies of the free and damped oscillator, $\omega_r < \omega_1 < \omega_0$.)

An interesting follow-up on resonance is the following: what is shape of $x_{i0}$ as function of frequency around $\omega_r$? Eq.~\eqref{eq:driven_harm_osc_params} reveals a characteristic profile that is known as \Def{Lorentz curve}. It peaks at $\omega_r$, and its width is heavily dependent on the damping parameter $\gamma$. The width is thus customarily quantified by the \Def{Q-factor} (\Def{quality factor})
\begin{equation}
	\eqbox{
		Q = \frac{\omega_r}{2\gamma}
	} \, .
\end{equation}
For a fixed $\omega_0$, larger $\gamma$ lead to a broader line shape (along with the shift in peak location $\omega_r$). In the limiting case $\gamma = 0$, we encounter what is called \Def{catastrophic resonance} in German: $x_{i0}$ becomes infinite, with vanishing width (i.e.~in essence a $\delta$-function).

-> many definitions of this exist; also (for damped, so perhaps not for driven) $\frac{\omega_0}{2\gamma}$ and $2\pi \frac{\text{energy stored}}{\text{energy lost per cycle}}$


Besides its impact on $x_{i0}$, $Q$ also has an effect on how $\delta$ evolves as a function of frequency, the phase lag between driven and oscillator motion. High $Q$ correspond to a quick transition between $0$ and $\pi$, with very steep slope, i.e.~there is basically a phase jump $\omega_0$; low $Q$, on the other hand, produce curves of almost uniform slope.

(Note: if the damping is small, $Q \approx \frac{\omega_0}{\Delta \omega}$ with the full-width-at-half-maximum [FWHM] $\Delta \omega$, which is defined by $x_{i0}(\omega_r \pm \Delta \omega) = x_{i0}(\omega_r)/\sqrt{2}$.)

-> some examples for $Q$-values: loudspeakers have roughly $100$; precisely tuned electrical circuits can have $10^4$; atoms have $10^7 - 10^8$; lasers can have $10^{14}$



		\subsection{Energy}
Let us now look at the energy of the different harmonic oscillator setups.


For the free case, we can simply calculate the kinetic energy
\begin{equation}\label{eq:free_harm_osc_total_energy}
	\eqbox{E} = T + V = \frac{m}{2} \dot{x}^2 + \frac{k}{2} x^2 \eqbox{= \frac{1}{2} k A^2} \, .
\end{equation}
$E \propto A^2$ is a general result for linear systems. As expected, $E$ is conserved (we are looking at a closed systems without losses at the moment).

Once we take into account damping, energy will not be conserved anymore. Losses to friction lead to
\begin{equation}
	\eqbox{
		\dv{E}{t} = - \vec{F} \cdot \vec{v} = - b \dot{x} = - 2 \gamma m \dot{x}^2
	} \, .
\end{equation}
Hence, the initial energy Eq.~\eqref{eq:free_harm_osc_total_energy} is reduced by
\begin{equation}
	W_T = \int_0^{T_1} \dv{E}{t} \, dt = \frac{m}{2} A^2 (\omega_0^2 + \gamma^2) (e^{- 2 \gamma T_1} - 1)
\end{equation}
per period ($W_{T_1}$ is turned into heat). If $\gamma$ is small, $e^{- 2 \gamma t} \approx 1 \Rightarrow W_{T_1} = 0$ and Eq.~\eqref{eq:free_harm_osc_total_energy} holds to good approximation.


For the driven oscillator, on the other hand,
\begin{equation}
	\dv{E}{t} = F_d(t) \dot{x} - b \dot{x}^2
	\, .
\end{equation}
Therefore, the system can reach a stationary state (after the initial transient period), where the velocity is such that the energy entering the system is precisely balanced by heat losses. This is why the energy does not diverge, despite the constant pouring of energy into the system (only divergence if $b = 0 = \gamma$, i.e.~in case of catastrophic resonance).


-> energy average of kinetic and potential for free case are equal, $\frac{1}{4} k A^2$ each


-> for driven oscillator, kinetic $T$ attains maximum at $\omega = \omega_0$, \emph{not} $\omega_r$; potential energy, on the other, is maximal at $\omega_r$ (since it is directly proportional to $x_{i0}$); the two energies resonating at different frequencies is a consequence of the system being non-conservative, there is this steady exchange of energy that was already mentioned



    \section{Non-Inertial Frames}\label{sec:non_inertial_frames}
% -- Sources:
% - https://en.wikipedia.org/wiki/Non-inertial_reference_frame
% - https://en.wikipedia.org/wiki/Rotating_reference_frame
% - https://en.wikipedia.org/wiki/Centrifugal_force
% - https://en.wikipedia.org/wiki/Centripetal_force
% - https://en.wikipedia.org/wiki/Coriolis_force
% - https://en.wikipedia.org/wiki/Euler_force
% - https://taleinav.github.io/Lectures/Ph%201a/Lecture%205%20-%202017-10-12.pdf
% 
% 

% calculations are following Nolting closely -> only if we keep delta x0


Until now, different observers could use Newton's second law to determine from given forces the acceleration that an object experiences. More explicitly,
\begin{equation}\label{eq:second_law_invariance}
	\vec{F} = m \vec{a}_\Sigma = m \vec{a}_{\Sigma'}
\end{equation}
where $\Sigma, \Sigma'$ denote different reference frames. However, our discussions were restricted to inertial frames. We will now see why this restriction was made. For that, we apply a frame transformation from an inertial frame $\Sigma$, where the second law holds, to a frame $\Sigma'$ that is accelerated with respect to $\Sigma$, and see what happens in this non-inertial frame. The idea how this transformation can be achieved is to look at the position $\vec{x}$ of a test particle from both frames by expanding it in terms of the coordinate axes $\unitvec{i}, \unitvec{i'}$ of the respective frame. In general, $\Sigma'$ being non-inertial can be caused by both an accelerated motion relative to $\Sigma$ or a rotation, which means the relation between vectors in the two coordinate systems can be expressed as
\begin{equation}\label{eq:vector_different_frames}
	\eqbox{
		\vec{x}
		= \sum_k x^k \unitvec{k}
		= \delta \vec{x}(t) + \sum_k x'^k \unitvec{k'}
		= \delta \vec{x}(t) + \sum_k x'^k R(t) \unitvec{k}
		= \delta \vec{x}(t) + R(t) \sum_k x'^k \unitvec{k}
	} \, .
\end{equation}
$\delta \vec{x}(t)$ denotes a displacement between the origins of the two frames and $R(t)$ the rotation matrix of a rotation about this shifted origin. The derivative of this expression is
\begin{align}\label{eq:vel_non_inert_frame_v1}
	\dv{\vec{x}}{t} &= \dv{(\delta \vec{x}_0 + \vec{x'})}{t}
	= \dv{\delta \vec{x}_0}{t} + \dv{\vec{x'}}{t}
	= \dv{\delta \vec{x}_0}{t} + \sum_k \dv{x'^k}{t} \unitvec{k'} + x'^k \dv{\unitvec{k'}}{t}
	\, .
\end{align}
It reflects that change in the position $\vec{x}$, when expressed in the unprimed coordinate system, can come from three sources: (i) motion of the coordinate origin, $\dv{\delta \vec{x}_0}{t}$ (ii) motion of the object in the unprimed coordinate system, i.e.~change of components relative to the axes, $\sum_k \dv{x'^k}{t} \unitvec{k'}$ (iii) changes of the coordinate axes themselves, i.e.~$\sum_k x'^k \dv{\unitvec{k'}}{t}$.

-> we could choose to re-express


Taking another derivative yields
\begin{align}\label{eq:accel_non_inert_frame_v1}
	\dv[2]{\vec{x}}{t} &= \dv[2]{\delta \vec{x}_0}{t} + \dv{t} \sum_k \dv{x'^k}{t} \vec{k'} + x'^k \dv{\unitvec{k'}}{t}
	\\
	&= \dv[2]{\delta \vec{x}_0}{t} + \sum_k \dv[2]{x'^k}{t} \vec{e'}_k + 2 \dv{x'^k}{t} \dv{\vec{e'}_k}{t} + x'^k \dv[2]{\vec{e'}_k}{t}
\end{align}
Before continuing with physical implications, here is a quick note on notation. To avoid overly cluttered and confusing expressions, it makes sense to proceed with the following definitions (to avoid overly cluttered expressions):
\begin{equation}
	\eqbox{
		\vec{v}_\Sigma = \sum_k v^k \unitvec{k} \coloneqq \sum_k \dv{x^k}{t} \unitvec{k}
	},
	\eqbox{
		\vec{v}_{\Sigma'} = \sum_k v'^k \unitvec{k'} \coloneqq \sum_k \dv{x'^k}{t} \unitvec{k'}
	} \, .
\end{equation}
These are the velocities in the unprimed and primed frames, respectively (they represent motion with respect to the respective coordinate axes $\unitvec{k}, \unitvec{k'}$). In the same spirit, we define the accelerations
\begin{equation}
	\eqbox{
		\vec{a}_\Sigma = \sum_k a^k \unitvec{k} \coloneqq \sum_k \dv[2]{x^k}{t} \unitvec{k}
	},
	\eqbox{
		\vec{a}_{\Sigma'} = \sum_k a'^k \unitvec{k'} \coloneqq \sum_k \dv[2]{x'^k}{t} \unitvec{k'}
	}
	\, .
\end{equation}
Additionally, we introduce abbreviations for quantities related to the motion of the origins:
\begin{equation}
	\eqbox{
		\vec{v}_0 \coloneqq \dv{t} \delta \vec{x}_0
	}, \quad
	\eqbox{
		\vec{a}_0 \coloneqq \dv[2]{t} \delta \vec{x}_0
	} \, .
\end{equation}

Eqs.~\eqref{eq:vel_non_inert_frame_v1}, \eqref{eq:accel_non_inert_frame_v1} then become
\begin{align}
	&\eqbox{
		\vec{v}_\Sigma = \vec{v}_0 + \vec{v}_{\Sigma'} + \sum_k x'^k \dv{\unitvec{k'}}{t}
	}
	\label{eq:vel_non_inert_frame_final}
	\\
	&\eqbox{
		\vec{a}_\Sigma = \vec{a}_0 + \vec{a}_{\Sigma'} + \sum_k x'^k 2 a'^k \dv{\unitvec{k}}{t} + \dv[2]{\unitvec{k'}}{t}
	}
	\label{eq:accel_non_inert_frame_final}
\end{align}
where we used that velocity, acceleration in the inertial frame fulfill\footnote{Technically, this assumes $\dv{t} \vec{e}_k = 0$ in the frame where we write things out, i.e.~that we are in rest frame of the observer. We do this without loss of generality since even if we had a term there, it would be constant and thus vanish in second derivative, which is the more relevant quantity here. That we can make the assumption to evaluate from an inertial frame at all simply comes from the fact that Newtonian physics is formulated in these frames, and we do not intend to leave this realm here.}
\begin{equation}
	\dv{\vec{x}}{t} = \dv{t} \sum_k x^k \unitvec{k} = \vec{v}_\Sigma, \quad \dv[2]{\vec{x}}{t} = \dv[2]{t} \sum_k x^k \unitvec{k} = \vec{a}_\Sigma
	\, .
\end{equation}
Eqs.~\eqref{eq:vel_non_inert_frame_final}, \eqref{eq:accel_non_inert_frame_final} then follows directly from Eq.~\eqref{eq:vector_different_frames}.


Therefore, Newton's second law in the non-inertial frame reads
\begin{equation}\label{eq:second_law_non_inert_frame}
	\frac{\vec{F}}{m} = \vec{a}_0 + \vec{a}_{\Sigma'} + \sum_k x'^k 2 v'^k \dv{\unitvec{k}}{t} \dv[2]{\unitvec{k'}}{t} \neq \vec{a}_{\Sigma'}
	\, .
\end{equation}
In other words: the invariance of Newton's second law, which has been noted in Eq.~\eqref{eq:second_law_invariance}, does not translate to a non-inertial frame. This should not be too surprising because, observers in an accelerated frame experience an acceleration without any force by definition of such a frame. This finding is the reason that we demanded an inertial frame when stating Newton's second law. And while this result may not be surprising, it poses a practical challenge: how do we determine equations of motion from a non-inertial frame? (This is relevant not just out of academic interesting, but also because we live in a non-inertial frame -- Earth is rotating!)


-> in some sense, Newton's laws are only invariant in frames that differ to first order of Taylor expansion; as soon as we get second-order differences, things go bad

-> different inertial frames disagree on notion of rest; but frames that are accelerated relative to each other can't even agree on a notion of force!



Well, luckily for us, Eq.~\eqref{eq:second_law_non_inert_frame} shows how: we have to add extra terms manually to what Newton's second law gives us. Sometimes, these terms are pulled over to the left hand side and labelled \Def[force!fictitious]{fictitious forces} (also: \Def[force!pseudo]{pseudo-forces})
\begin{equation}\label{eq:pseudo_force_general}
	\eqbox{
		\vec{F}_\mathrm{fict}
		% = - m \dv[2]{\delta \vec{x}_0}{t} - m \sum_k 2 \dv{x'^k}{t} \dv{\unitvec{k'}}{t} + x'^k \dv[2]{\unitvec{k'}}{t}
		= - m \vec{a}_0 - m \sum_k 2 v'^k \dv{\unitvec{k'}}{t} + x'^k \dv[2]{\unitvec{k'}}{t}
	}	\, .
\end{equation}
The assignment of these terms to the total \enquote{saves} Newton's second law because the only remaining acceleration on the right hand side of Eq.~\eqref{eq:second_law_non_inert_frame} is $\vec{a}_{\Sigma'}$, but we must be very careful with the interpretation here: while fictitious forces are very much a real effect, accelerated observers do experience them. This is because we have defined forces as something that all (inertial) observers can agree on, and fictitious forces do not qualify for that. Another way to see that fictitious forces do not represent a real force is that they are not arising from interactions between particles, but are merely a product of doing physics from a specific frame of reference.

% With these new terms in place, a recipe of how to work with non-inertial frames goes as follows: \enquote{Treat the fictitious forces like real forces, and pretend you are in an inertial frame}\footnote{I have taken this quote from Wikipedia, which in turn cites Louis N. Hand, Janet D. Finch Analytical Mechanics, p. 267.}, i.e.~apply Newton's second law.



\todo{merge following in there:}
Make no mistake, although terms like \enquote{pseudo} and \enquote{fictitious} suggest that they are something imaginary, pseudo-forces are a real phenomenon that observers in non-inertial frames experience; it is just that we should try to avoid the term \enquote{force} for this phenomenon. (We will see plenty examples in the rest of the section.) In fact, these pseudo-forces are how an observer can detect whether his rest frame is accelerating or not. A profound insight from this is that acceleration is an absolute quantity in Newtonian physics.\footnote{Once again, this is not true anymore in a more modern understanding of the universe, the theory of relativity. You may, however, be surprised to hear that this interpretation does not change in special relativity. It is only in general relativity that gravity becomes equivalent to acceleration, so that acceleration loses its role as an absolute quantity.}



% Here is an important note, in the vein of how \cite{Thorne_2018} approaches things: $\vec{x}$ is a purely geometric quantity that describes a vector in Euclidean space. In other words, while we can choose to express in the primed or unprimed coordinate system, it has a clear meaning \emph{without any} coordinate system. The same is true for the velocity $\vec{v} = \dv{\vec{x}}{t}$ and acceleration $\vec{a} = \dv[2]{\vec{x}}{t}$.

% -> then we would also have $\vec{x}$ as a purely abstract, geometric quantity, which does not require any coordinate system yet -- it CAN be expanded in the different coordinates, upon choice of origin and axes orientation, but is just generally used to denote the position (this does not need coordinates; also emphasized by Thorne+Blandford I think)




		\subsection{A Detailed Examination}

Before we examine two types of accelerations in more detail for the remainder of this section, there is a final remark to be made that will lead us down a rabbit hole (or perhaps two rabbit holes). At the heart of this section lies the problem that an accelerated observer ignores acceleration that comes from his own coordinates/frame of reference, which do appear upon a frame transformation. These have to be added manually to get valid equations of motions. We can correct for this in two ways: (i) by using four-vectors and different derivative operator (this is done in relativity) or (ii) we fully lean into the geometric interpretation of quantities that, e.g., \cite{Thorne_2018} advocates for (which I agree with).


Let's do (ii) first, requires less new mathematical notions. 

-> we have this done implicitly already in Eq.~\eqref{eq:vector_different_frames} (we have not really made an explicit frame transformation here to be honest, just written things out in different frame); this uses that $\vec{x}$ is an entity in Euclidean space that has meaning \emph{without any coordinate system}. One can write it out in different frames, as we have done, but this is not needed for it to be meaningful. We have not fully explored the consequences of this. It actually provides a solution to the problem we had, namely formulating the second law in a geometric manner as $\vec{F} = m \vec{a}$ where the derivatives appearing in $\vec{a}$ are applied to both components and basis vectors.

-> always yields correct results, pseudo-forces come naturally out of this geometric approach; in fact, we can see this immediately because in fact, this is exactly what we did before and which led to Eq.~\eqref{eq:vel_non_inert_frame_final}; we just applied \enquote{wrong} version of second law

-> if we had required substituting $\vec{a}$, then things would have been equal already; after all, for the situation we observed where $\vec{a}$ represented an acceleration in $\Sigma$, we could have used $\vec{a} = \vec{a}_\Sigma$; by definition, this coincides with what formulation of second law from $\Sigma$ yields, but since the right hand side of Eq.~\eqref{eq:accel_non_inert_frame_final} just represents $\dv[2]{t} \vec{x}$, it too is equal to $\vec{a}$!!!

Thus, using $\vec{F} = m \vec{a}$ always yields the correct equation of motion, without the artificial introduction of pseudo-forces. These come naturally out of the derivatives in $\vec{a}$.



-> this equation works both ways: when analyzing inertial observer in $\Sigma$ with, e.g., $a^k = 0$ from the non-inertial frame $\Sigma'$, this equation yields correct result $\vec{a}$ because the acceleration $\sum_k a'^k \unitvec{k'}$ that the non-inertial observer sees this observer to have will be canceled by the other terms (\enquote{pseudo-forces}) in $\vec{a}$; and secondly, when analyzing an observer at rest in $\Sigma'$, i.e.~with $a'^k = 0$, the non-inertial observer will infer same total acceleration as the inertial (who sees it due to frame being accelerated) because the pseudo-force terms are still non-zero; non-inertial observer knows that he has to take into account pseudo-force terms because one can measure absolute acceleration! (at least in Newton, in GR we can locally only measure differential effect of acceleration and gravitation, though this is side-note here)


% -> we could even use this approach to analyze physics from a non-inertial frame; in that case, inertial frame would be seen to experience pseudo-forces, and things would work the same -> nah, rather not do; is not like analysis from two different inertial frames here, where we have to show equivalence (because, in fact, the frames are not equivalent; we \emph{can} detect whether we are in non-inertial or not) -> actually, I think this is a helpful remark (explains why we are free to assume inertial frame axes do not change) -> reason why we analyze from inertial frame: in Newtonian physics, we can only expect things to work there, which is consequence of second axiom (and we know from experience and experiments that this works reasonably well); in other words, if we choose to stay in Newtonian realm, then inertial frames are the tool we should fall back too


regarding (i):


-> In differential geometry, the concept of a covariant derivative automatically incorporates information from the underlying frame (such as $\dv{t} \unitvec{k} \neq 0$), and happens to produce exactly the same terms that come out of the derivative when it is carried out in $\Sigma$. These additional terms are called Christoffel symbols. If we were to formulate Newton's second law using covariant derivatives, we would automatically get consistent results in all reference frames. As far as I know, this is done in a field called \enquote{Newon-Cartan gravity}; it is certainly done in general relativity.


-> pseudo-force = non-zero Christoffel symbol. In some sense, that reflects pseudo-force = any force that is proportional to $m$, except for gravity. The \enquote{except for} part is resolved by general relativity, where gravity shows up as curvature of space and thus also non-zero Christoffel symbols, in some sense making gravity the pseudo-force of general relativity.



-> I still stand by following (I think this is finally correct): in the geometric way of expressing things, applying $\vec{F} = m \vec{a}$ still yields correct equations of motion, because change of coordinate axes is directly incorporated if we use $\vec{a} = \dv[2]{t} \vec{x}$ and not the expression in a specific frame, $\vec{a} = \sum_k \unitvec{k} \dv[2]{t} x^k$; yes, then we get different forces experienced in different coordinate systems, but still correct equation of motion; in that sense, geometric approach is universal, though with subtleties in physical implication (e.g., what is force and what not) -> even an accelerated observer can test if he is in non-inertial frame (and thus account for the terms that describe axis changing), this is why interpretation works: we can check if the component version holds, $F^k = m a^k = m \dv[2]{t} x^k$, which corresponds to checking $\vec{F} = m \vec{a}_\Sigma$ (with $\Sigma$ denoting rest frame of observer); if not, observer knows he is in accelerating frame and needs to account for extra terms when determining $\vec{a}$, the geometrical acceleration -> that is why $\nabla_t \vec{p} = \vec{F}$ works with covariant derivative works too; in fact, by taking into potential account action of $\dv{t}$ onto the axis vectors, we basically use covariant derivative, right?
% -> can we maybe even say that this treatment reveals that forces are not a fully in-/covariant concept? Because their form depends on frame? (Or maybe we must say that they are only invariant in inertial frames; but they loose their important in GR anyway, don't they? Since gravity = acceleration, so any frame with gravity = non-inertial frame. So perhaps it is useful to not even start thinking about them, at least we should always bear in mind they are not most general thing) -> yep, think so; this is why we are being so specific in formulation of second law and must exclude non-inertial frames, because Newtonian physics is not a fully covariant theory!!! Is due to fact that it is formulated in terms of forces (?)
-> regarding previous paragraph (now commented): no, is quite the opposite; in using $\nabla_t \vec{a} = \vec{F}$, we basically make things covariant! Because then, $\vec{F} = 0$ yields the correct equation of motion from every frame, since pseudo-forces appear naturally when analyzing from non-inertial frame, instead of having to introduce them manually
-> uhh, but interesting: this requires regarding things in 4D-spacetime already! Because only then does derivative with respect to time also take into account changes of underlying space! These are viewed as separate things in the initial, Newtonian interpretation



\hrule

general stuff about acceleration:

cool thought experiment: what does not accelerate \url{https://physics.stackexchange.com/questions/568969/what-does-not-accelerate}


-> it \emph{is} possible to detect acceleration! Denk an bremsen im auto, zeug auf dem armaturenbrett fliegt dir entgegen! Der Grund warum wir (in dem beschleunigten frame) am sitz in ruhe bleiben ist die kraft, die der sitz auf uns ausübt (richtig?). bei rotation auf erde geht das zB mit Foucalt-Pendel (das sollte für Coriolis sein), allgemeiner in nem karussell über zentripetalkraft (die zentripetalkraft ist halt das, was man im Endeffekt spürt; also das gegen den Sitz gedrückt werden, man spürt halt wieder nur die Gegenkraft zur Zentrifugalkraft, die einen eigentlich nach außen drückt) -> guess that's more of an empirical insight, right?

-> you may ask: why don't we feel centrifugal force on Earth then? Well, we do! Is just subdominant to a force that acts as centripetal force pulling us back to Earth, gravitation (at equator, where centrifugal is maximum, it is still just $0.3\%$ of gravity since angular velocity is really small at $\approx 7 \cdot 10^{-5} \unit{ \radian\per\second}$)

-> in fact, there are many sources of acceleration, also motion of Earth around the sun etc. But they all have incredibly small angular velocities (because of large radii of the motion; absolute velocities are usually quite high, also for motion of the whole solar system through galaxy; but we cannot feel constant velocity via force, so this is not relevant)



		\subsection{Acceleration Without Rotation}
For any accelerated system with acceleration that is constant in direction (not necessarily magnitude), the two frames $\Sigma, \Sigma'$ are related by a shift in coordinate origin
\begin{equation}
	\delta \vec{x}_0 = \frac{\vec{a}_u t^2}{2} + \vec{x}_0
\end{equation}
where $\vec{x}_0$ is the initial separation of coordinate origins and $\vec{a}_u$ is a (potentially time-dependent) acceleration. Setting $\vec{x}_0 = 0$, the expression \eqref{eq:pseudo_force_general} for pseudo-forces evaluates to
\begin{align}
	\vec{F}_\mathrm{fict}&= - m \dv[2]{t} \qty(\unitvec{k} + \vec{a}_u \frac{t^2}{2})
	\notag\\
	&= - m \dv{t} \qty(\dv{\vec{a}_u}{t} \frac{t^2}{2} + \vec{a}_u t)
	\notag\\
	&= - m \qty(\dv[2]{\vec{a}_u}{t} \frac{t^2}{2} + \dv{\vec{a}_u}{t} t + \dv{\vec{a}_u}{t} t + \vec{a}_u)
	\notag\\
	&= \eqbox{
		- m \dv[2]{\vec{a}_u}{t} \frac{t^2}{2} - 2 m \dv{\vec{a}_u}{t} t - m \vec{a}_u
	}
	\, .
\end{align}
For a uniformly accelerated system, this simplifies to
\begin{equation}
	\eqbox{
		\vec{F}_\mathrm{fict} = - m \vec{a}_u
	} \, .
\end{equation}
As we know very well from the second law, this does indeed represent a force.



\begin{ex}[Car]
	braking in a car

	we are sitting still in seat upon braking, but feel force mediated by seat behind us; that we don't move can be explained by aid of pseudo-force of acceleration, which causes observer in the accelerating frame to experience no net acceleration

	-> what is pushing us back is pseudo-force


	-> same thing: taking a turn; if we turn left, then we are basically in rotating coordinate system; we get pushed left due to centrifugal force (what acts as centrifugal here is seat and seatbelt, this keeps us in place \todo{right?})


	also love example from Tal Einav, who is talking about person standing on car and trying to balance; but as soon as car accelerates, he falls off; in the accelerated frame, this is caused by pseudo-force pushing him back so that he does not stay on car; from inertial frame, we see that this is due to his inertia, which keeps him in same place, because no force is acting upon him (the car just goes, does not affect him)
\end{ex}




        \subsection{Rotating Frames}
Next, we turn to the case where $\Sigma'$ rotates with respect to $\Sigma$ with (orbital) angular velocity $\vec{\omega}$. The time evolution of a vector in a rotating coordinate system is well-known. \todo{perhaps we should derive this} In the rotating frame $x'$, this implies
\begin{equation}
    \dv{\vec{x'}}{t} = \vec{\omega} \cross \vec{x'} \, .
\end{equation}
Using this relation for the coordinate axes $\vec{e}'_i$, Eq.~\eqref{eq:pseudo_force_general} turns into
\begin{align}
	\vec{F}_\mathrm{fict} &= - m \sum_k 2 \dv{x'^k}{t} \vec{\omega} \cross \unitvec{k'} + x'^k \dv{t} \qty(\vec{\omega} \cross \unitvec{k'})
	\notag\\
	&= - m \sum_k 2 \dv{x'^k}{t} \vec{\omega} \cross \unitvec{k'} + x'^k \qty(\dv{\vec{\omega}}{t} \cross \unitvec{k'} + \vec{\omega} \cross \dv{\unitvec{k'}}{t})
	\notag\\
	&= - m \sum_k 2 \dv{x'^k}{t} \vec{\omega} \cross \unitvec{k'} + x'^k \qty(\dv{\vec{\omega}}{t} \cross \unitvec{k'} + \vec{\omega} \cross (\vec{\omega} \cross \unitvec{k'}))
	\notag\\
	&= \eqbox{
		- m \qty(2 \vec{\omega} \cross \sum_k \dv{x'^k}{t} \unitvec{k'} + \dv{\vec{\omega}}{t} \cross \vec{x} + \vec{\omega} \cross (\vec{\omega} \cross \vec{x}))
	}
    \label{eq:pseudo_force_rot_frame}
\end{align}


Now, let us come back to expression \eqref{eq:pseudo_force_rot_frame} and analyze what is going on. first term is analogous to what acceleration of any observer in any inertial frame (including $x$) looks like, change of components. but there are more terms to acceleration of an observer placed in $x'$:
\begin{itemize}
    \item Centrifugal force:
    \begin{equation}\label{eq:f_centrifugal}
		\eqbox{
			\vec{F}_\mathrm{centri} = - m \vec{\omega} \cross (\vec{\omega} \vec{x})
		}
	\end{equation}
	for $\vec{\omega} \parallel \vec{x}$:
	\begin{equation}
		\eqbox{
			\vec{F}_\mathrm{centri} = m \omega r^2 \, \unitvec{r}
		}
	\end{equation}
	
	centrifugal can be interpreted as follows: if you do not move actively, but are suddenly subject to acceleration, you will experience a force; say you are standing on a disk that suddenly starts rotating: then you get pushed outwards radially, and this is effect of centrifugal force (which is \emph{not} the same as centripetal force, which acts towards center of acceleration = origin and not away from it)
    
	-> IMPORTANT: I am hesitant to write $\dv{x'^k}{t} \vec{e}'_k$ as $\vec{v}$ because this assumes we work in a frame where the $\vec{e}'_k$ are constant, i.e.~in the non-inertial frame; but this expression $\dv{x'^k}{t} \vec{e}'_k$ does represent a velocity, namely the one measured from the non-inertial frame
    
    good example is also car that takes a turn
    
    -> not to be confused with centripetal (similar idea/concept, but used in different contexts; used to describe radial movement in non-rotating frames, i.e.~quantifies force needed to keep something moving on a circle (force radial movement lol); important in particle accelerators, where it determines current we need for attraction towards some position; or for ISS, where it determines how much velocity is needed to force things on circle around Earth, whence gravitational pull of Earth is supposed to act as the centripetal force of the motion, so that ISS remains on this circular path)

    -> Earth is not a perfect sphere, but more of an oblate; is due to centrifugal force (apart from imperfections on surface, coming from mountains or so), which acts on east and west side of Earth, but not north/south. I think this is explained more quantitatively here: \url{https://taleinav.github.io/Lectures/Ph%201a/Lecture%205%20-%202017-10-12.pdf}


    -> nice summary from Wikipedia: Motion relative to a rotating frame results in another fictitious force: the Coriolis force. If the rate of rotation of the frame changes, a third fictitious force (the Euler force) is required. These fictitious forces are necessary for the formulation of correct equations of motion in a rotating reference frame and allow Newton's laws to be used in their normal form in such a frame (with one exception: the fictitious forces do not obey Newton's third law: they have no equal and opposite counterparts). Newton's third law requires the counterparts to exist within the same frame of reference, hence centrifugal and centripetal force, which do not, are not action and reaction (as is sometimes erroneously contended).


    \item Coriolis force:
    \begin{equation}\label{eq:f_coriolis}
		\eqbox{
			\vec{F}_\mathrm{cor} = - 2 m \vec{\omega} \cross \sum_k \dv{x'^k}{t} \vec{e}'_k
		}
	\end{equation}

	effect: any observer that moves on straight line in $\Sigma$ does not move on such a line in the accelerated frame (Wikipedia has super nice visualization: \url{https://en.wikipedia.org/wiki/File:Corioliskraftanimation.gif}) -> see also Neutelings plot

	
	Coriolis can be interpreted as follows: say we move inward radially in the rotating frame so that we cancel effect of Coriolis. Will inertial observer see us as resting? Nope, still not, there would still be rotation, we are pushed in angular direction around the disk (to stay in this example/situation) and this is Coriolis force
    
    -> we have assumed constant angular velocity here, but this is kind of intuitive
    
    
    -> this is how hurricanes are created: air flows into areas of low pressure; but instead flowing radially inward as it would in an inertial frame, there is an additional component to the movement that comes from the rotation of the Earth, the resulting air kind of spins up, due to the component $\vec{\omega} \cross \vec{x}$ (note that this term appears in both acceleration and velocity!); also explains why direction of rotation of these storms is determined by hemisphere they are created on

    -> haha, this is crazy, Wikipedia takes viewpoint that Coriolis is causing movement of the Sun around Earth (huge force due to huge mass of the Sun (?))


    \item Euler force: 
    
    -> from Wikipedia: The Euler force will be felt by a person riding a merry-go-round. As the ride starts, the Euler force will be the apparent force pushing the person to the back of the horse; and as the ride comes to a stop, it will be the apparent force pushing the person towards the front of the horse. A person on a horse close to the perimeter of the merry-go-round will perceive a greater apparent force than a person on a horse closer to the axis of rotation. 
    
    -> usually neglected, on Earth it is only the first two that really appear
\end{itemize}



\begin{ex}[Merry Go-Round]
	% Thornton 10.2

	let's look at puck let go with velocity $\vec{v}$ on a merry-go-round moving at constant speed
\end{ex}



\begin{ex}[Foucault Pendulum]
	Thornton example 10.5
\end{ex}



	\section{Rotation}
% -- Sources:
% - https://en.wikipedia.org/wiki/Angular_momentum
% - https://en.wikipedia.org/wiki/Torque
% 
% 
The preceding discussion of rotating frames of reference showed that this type/state of motion is not properly described by just the momentum $\vec{p}$. One indicator of this is the occurrence of pseudo-forces, which cause an acceleration even in the absence of forces and accordingly, a change in momentum.

One way to look at this is to argue that $\vec{p}$ represents linear momentum, which is best suited to describe motion into constant direction. If a rotational component is present, we need a new notion. In analogy to $\vec{p}$, this new notion is called \Def{angular momentum} and denoted by $\vec{L}$. For the sake of definition, it makes sense to focus on the situation in which the motion is of purely rotational nature, and the previous section showed that the pivotal factor controlling a rotation is the angular velocity $\vec{\omega}$. It also makes sense to have an \emph{angular momentum} being proportional to \emph{(orbital) angular velocity},
\begin{equation}\label{eq:angular_momentum_1}
	\eqbox{
		\vec{L} = I \vec{\omega}
	} \, .
\end{equation}
What we are yet to determine is the \Def{moment of inertia} $I$, which is the analogue of inertia for the linear momentum. Continuing with the thought process of this being \enquote{how much an object resists against rotation} \todo{improve this}, it should not be surprising that $I \propto m$.
But also depends on distance from origin; and on the circumference

-> makes sense for angular momentum to have angular velocity; then define with proper notion of inertia and we're done; is not just mass now because also depends on distance from origin (quadratically; I guess we can argue via formula for circumference of circle, something like rotating with same angular velocity means sweeping out same part of this, i.e. we rescale by radius squared?)

-> my argument: inertia upon rotation must increase as we increase distance between object and origin (this is factor of $r$); then also we have to account for increase in circumference covered if we keep angular velocity fixed -> not sure if this is repetitive, I mean it does get harder to rotate stuff further away at same velocity due to increase in same circumference swept out, right? Meaning we potentially account for this twice here -> I guess we could argue via area swept out during this, but I do not see how this can be related to inertia

-> note that $I = m r^2$ for case of test mass, i.e.~point particle, that we still look at!

\todo{finish intuitive interpretation}

If you find this intuitive motivation for the expression of $I$ (and $\vec{L}$ in general) unsatisfactory, then the following remark might be helpful for you. This particular expression is of interest because it turns out to be what is conserved in physical setups that have rotational invariance. Akin to linear momentum, which is conserved when translational symmetry is present (meaning I can move test mass whereever I want in space and get same results; not fulfilled if there are objects mediating forces, such as a planet or charges, since these have particular position, so force would change if we were to move test mass)


motivation why this particular expression is interesting: we had seen that momentum is very useful quantity when studying motion of objects, partly because it is conserved in the absence of external forces (and more generally because Newton 2 involves it). well, it turns out that another such quantity exists, $\vec{x} \cross \vec{p}$. is especially useful when studying rotations, which motivates name angular momentum.\footnote{For the advanced reader: this is related to what is arguably one of the most important (and beautiful) theorems in physics, \Def{Noether's theorem}.}

important note: angular momentum depends on coordinates because it involves position vector, which is measured relative to origin! momentum (also, to distinguish from angular momentum: linear momentum) can also depend on the frame of reference, but this is different from depending on the chosen coordinate frame (for frame of reference, the origin is not so important)


It made sense to define angular momentum in the way we did, but in many textbooks you might encounter another expression for it. We can recover this by noticing that the orbital angular velocity of a test particle about the origin is
\begin{equation}
	\vec{\omega} = \frac{\vec{r} \cross \vec{v}}{r^2} \, .
\end{equation}
This yields
\begin{equation}\label{eq:angular_momentum_2}
    \eqbox{
		\vec{L}} = I \vec{\omega} = m r^2 \frac{\vec{r} \cross \vec{v}}{r^2} \eqbox{= \vec{r} \cross \vec{p}
	} \, .
\end{equation}



we can see how it fulfills intended usage by rewriting in polar (spherical?) coordinates:
\begin{equation}
    \vec{L} = m r^2 \dots \vec{e}_L \todo{can this be done?}
\end{equation}
and in case the motion occurs in a 2D-plane, choosing this orbital plane to be the $xy$-plane yields
\begin{equation}
    % \vec{L} = m r^2 \dv{\phi}{t} \vec{e}_z \todo{can this be done?}
    \vec{L} = - m r^2 \dv{\phi}{t} \vec{e}_\theta
	\, .
\end{equation}
more generally (from Wikipedia):
\begin{equation}
	\vec{L} = m \vec{r} \cross \vec{v} = m r^2 (\dv{\theta}{t} \vec{e}_\phi - \dv{\phi}{t} \vec{e}_\theta)
	\, .
\end{equation}

in 2D, i.e.~in polar coordinates,
\begin{equation}
	L = m r^2 \dv{\phi}{t}
\end{equation}


\hrule


We have seen a lot of similarities between momentum and angular momentum, i.e.~motion with and without rotation. However, we have not yet seen the corresponding equation of motion. To obtain it, we must calculate the rate of change in angular momentum is called \Def{torque} (also: \Def[impulse!angular]{angular impulse}):
\begin{equation}
    \eqbox{\vec{M}} \coloneqq \dv{t} \vec{L} = \dv{\vec{r}}{t} \cross m \vec{v} + \vec{r} \cross \dv{\vec{p}}{t} = m \underbrace{\vec{v} \cross \vec{v}}_{= 0} + \vec{r} \cross \dv{\vec{p}}{t} \eqbox{= \vec{r} \cross \vec{F}} \, .
\end{equation}
This is the rotational analogue of Newton's second law.

% \todo{do we have to note that this in inertial frame? Because we use $\dv{\vec{x}}{t} = \vec{v}$, no derivative from axes -> hmm no, right? Because change in basis vector would also correspond to particle having some velocity, so this is all fine; $\vec{v} \neq \dv{x^k}{t} \vec{e}_k$} -> in other words, wouldn't this change in basis vectors also show up in the other $\vec{v}$? I think we're good here

Equivalently, we can express this angular impulse as a change in angular rather than linear momentum, which is most easily done by differentiating the initial definition of angular momentum:
\begin{align}
	\vec{M} = \dv{I \vec{\omega}}{t} = \dv{I}{t} \vec{\omega} + I \dv{\vec{\omega}}{t}
	= \qty(\dv{m}{t} r^2 + 2 m r \dv{r}{t}) \vec{\omega} + I \dv{\vec{\omega}}{t}
\end{align}
Similarly to Newton's second law, we obtain a simpler form if the moment of inertia is constant, namely
\begin{equation}
	\vec{M} = I \dv{\vec{\omega}}{t} \eqqcolon I \vec{\alpha} \, .
\end{equation}
In particular, this is valid for the case of a purely rotational motion, where no change in radius is present), assuming there is no change in mass too. If, on the other hand, linear momentum parallel to $\vec{r}$ is present, i.e.~some push or pull in radial direction, this is not true (any perpendicular component of $\vec{p}$ just corresponds to an increase in $\omega$).


\todo{note expression for rotational energy}

-> work is $dW = F dx$ \todo{I think no minus here, right?}, so for rotational quantities we get $dW = M d\theta = I \dv{\omega}{t} d\theta$ \todo{vectors?} -> nah, we want $\dv{W}{t} = \vec{F} \cdot \vec{v}$, which in terms of rotational quantities becomes $\dv{W}{t} = \vec{M} \vec{\omega} = I \dv{\vec{\omega}}{t} \vec{\omega}$ and integration yields
\begin{equation}
	\eqbox{
		E = W = \frac{1}{2} I \omega^2 = \frac{1}{2} \vec{\omega} \cdot \vec{L}
	}
\end{equation}



	\section{$N$-Body Mechanics}
	\label{sec:n_body_mechanics}
talk about center of mass system; motivation: if we have two particles, we have coupled equations of motion, much harder to solve than two separate differential equations; but by going into CMS, we can decouple the equations


usually sums, but sometimes also integrals; of course, integrals with densities etc effectively correspond to treating an infinite number of particles (the \enquote{continuum limit});
% (this is actually a theorem from measure theory; every set with finite number of points in it has zero measure, which just means integral over it vanishes; since we usually deal with particles where integral does not vanish, there are infinitely many points in each volume -> on other hand, density does not necessarily assign non-zero value to each point, after all outcome is finite -> unnecessary discussion, just leave out)
this is not realistic, but sometimes required because it becomes infeasible to treat huge number of particles using sums; but this is realm of statistical physics, in this section we focus on systems where sums can still be used



very frequently used tool: center of mass (COM) and the related center of mass coordinate system (CMS); we then have COM position and each particle position relative to this, but not all of these vectors are independent, so degrees of freedom do of course not change upon this coordinate transformation


-> notable properties: Newton 2 holds for COM, since all internal interactions between particles vanish (due to Newton 3); most quantities behave additively, such as momentum or angular momentum, so we can divide into contribution from COM and from relative coordinates



	    \subsection{Two Body Problem}
We will now treat the special case $N = 2$ since it turns out that it can be formulated in a particularly simple manner in the CMS. Besides the COM
\begin{equation}
	\vec{R} = \frac{1}{m_1 + m_2} \qty(m_1 \vec{r}_1 + m_2 \vec{r}_2)
	\, ,
\end{equation}
we usually use coordinates relative to the COM, i.e.~something like $\vec{r}_i - \vec{R}$. However, introducing the \Def{relative coordinate}
\begin{equation}\label{eq:com_relative_coord}
	\eqbox{
		\vec{r} \coloneqq \vec{r}_1 - \vec{r}_2
	}
\end{equation}
as the vector from second to first body, we notice that
\begin{equation}\label{eq:trafo_com_components}
	\eqbox{
		\vec{r}_1(t) = \vec{R}(t) + \frac{\mu}{m_1} \vec{r}(t) = \vec{R}(t) + \frac{m_2}{M} \vec{r}(t)
	}
	\, , \quad
	\eqbox{
		\vec{r}_2(t) = \vec{R}(t) - \frac{\mu}{m_2} \vec{r}(t) = \vec{R}(t) - \frac{m_1}{M} \vec{r}(t)
	}
\end{equation}
In other words, COM and relative coordinate are already sufficient to describe the two-body problem of $\vec{r}_1, \vec{r}_2$, they are merely a reparametrization. You may very well argue that this should not be surprising, as this is the number of quantities we expect to be needed from an analysis of the available degrees of freedom.


But until now, we have not seen the catch. For that, let us look at the equations of motion for the system in the CMS. Newton's second law for the new quantities can be derived from the equations for the individual particles (we operate under the common assumption that all masses $m_i$ are constant here):
\begin{align}
	\eqbox{\ddot{\vec{R}}} &= \frac{1}{\sum_i m_i} \sum_i m_i \ddot{\vec{r}}_i = \frac{1}{\sum_i m_i} \vec{F}_i = \eqbox{\frac{1}{M} \sum_i \vec{F}^\mathrm{ext}}_i
	\\
	\eqbox{\ddot{\vec{r}}} &= \ddot{\vec{r}}_1 - \ddot{\vec{r}}_2 = \frac{\vec{F}^\mathrm{ext}_1 + \vec{F}_{2 \rightarrow 1}}{m_1} - \frac{\vec{F}^\mathrm{ext}_2 + \vec{F}_{1 \rightarrow 2}}{m_2} = \eqbox{\frac{\vec{F}^\mathrm{ext}_1}{m_1} - \frac{\vec{F}^\mathrm{ext}_2}{m_2} + \qty(\frac{1}{m_1} + \frac{1}{m_2}) \vec{F}_{2 \rightarrow 1}}
	\, .
	\footnotemark
\end{align}%
\footnotetext{In many textbooks, you will find $\vec{F}_{1, 2}$ instead of $\vec{F}_{21}$. However, they are referring to the body is exerted on in the first index, with the second index referring to the body that is exerting the force. When we write $\vec{F}_{21}$, we mean the same thing, namely the force exerted from body $2$ onto body $1$. To avoid the ambiguity, we use an arrow that clarifies which body is exerting the force and which one it is exerted on.}%
At this point, it is common to introduce the \Def{reduced mass}
\begin{equation}
	\eqbox{
		\frac{1}{\mu} = \frac{1}{m_1} + \frac{1}{m_2}
	} \quad \Leftrightarrow \quad
	\eqbox{
		\mu = \frac{m_1 m_2}{m_1 + m_2} = \frac{m_1 m_2}{M}
	} \, .
\end{equation}
Therefore, we have now transformed the equations of motion for the particles at $\vec{r}_1, \vec{r}_2$ that have masses $m_1, m_2$ to equations for two bodies at positions $\vec{R}, \vec{r}$ and with masses $M, \mu$.

In an isolated system, where no external forces are present, these equations turn into
\begin{subequations}
	\begin{align}
		&\eqbox{\ddot{\vec{R}} = 0}
		\label{eq:eq_of_motion_two_body_com}
		\\
		&\eqbox{\mu \ddot{\vec{r}} = \vec{F}_{2 \rightarrow 1}(\vec{r})}
		\label{eq:eq_of_motion_two_body_relcoord}
		\, .
	\end{align}
\end{subequations}
This is the catch: we only have to solve a single equation now -- the two-body system has been mapped to an effective one-body system (of a single particle with mass $\mu$). Part of what makes this set of two equations special is that they are decoupled now, i.e.~the equation for $\ddot{\vec{r}}$ does not depend on $\vec{R}$ and vice versa (a consequence of the fact that the force $\vec{F}_{2 \rightarrow 1}$ will only depend on $\vec{r}_1, \vec{r}_2$ through the vector connecting body $2$ and body $1$, which is nothing but the relative coordinate $\vec{r}$). The Newtonian equations, on the other hand, were always coupled, i.e.~the equation for $\ddot{\vec{r}_1}$ contained $\vec{r}_2$ and vice versa.



-> many other quantities behave additively here/can be decoupled, similar to how things were in general for $N$-body systems:
\begin{subequations}
	\begin{align}
		T &= T_{\vec{R}} + T_{\vec{r}} = \frac{1}{2} M \dot{\vec{R}}^2 + \frac{1}{2} \mu \dot{\vec{r}}^2
		\\
		E &= T + V = T_{\vec{R}} + T_{\vec{r}} + V_{12}(\vec{r}) = E_{\vec{R}} + E_{\vec{r}}
		\\
		\vec{L} &= \vec{L}_{\vec{R}} + \vec{L}_{\vec{r}} = M \vec{R} \cross \dot{\vec{R}} + \mu \vec{r} \cross \dot{\vec{r}}
	\end{align}
\end{subequations}
where we have assumed $\vec{F}_{2 \rightarrow 1}$ to be conservative, so that it has a potential $V_{12} = V_{21}$ (note: $\vec{F}_{2 \rightarrow 1} = - \grad_{\vec{r}} V_{21} = - \grad_{-\vec{r}} V_{12} = - \vec{F}_{1 \rightarrow 2}$)



% -> advantage: one can decouple equations of motion by using COM and relative coordinate

% -> mention that two-body problem has closed-form solution, but once more bodies get involved, this is not the case anymore -> careful; what we can say though is that three-body systems are chaotic



% -> seems that the relative coordinate just means we transform into system where one (typically the more massive) body is at rest; then we only model motion of other body (so ellipse is around other body, not around COM; and in expressions, we have to replace second mass by reduced mass, although we do model motion of second body) -> but sun is then approximately also in center of mass

% -> hmm I think ChatGPT was wrong here; what we do is model two-body system via center of mass and relative coordinate; then we can write down two equations of motion for the two associated \enquote{bodies}; 



	    \subsection{Central Forces}
% https://en.wikipedia.org/wiki/Classical_central-force_problem
% https://en.wikipedia.org/wiki/Two-body_problem
% https://en.wikipedia.org/wiki/Kepler%27s_laws_of_planetary_motion
% https://physics.stackexchange.com/questions/599329/how-does-the-orbital-motion-of-reduced-mass-tell-us-how-the-individual-planets-s/602997#602997
% https://physics.stackexchange.com/questions/275822/solving-a-two-body-problem-using-relative-motion-and-reduced-mass?rq=1

Physically speaking, formulating the two-body problem in the CMS corresponds to switching to a (potentially) non-inertial frame of reference, centered in the second body at $\vec{r}_2$. We then solve for the relative separation of body $1$ in this frame, which is nothing but the relative coordinate $\vec{r}$. Usually, we try to avoid non-inertial frames because they can have strange effects such as pseudo-forces. However, this particular frame allows the formulation of much simpler equations of motion in Eqs.\eqref{eq:eq_of_motion_two_body_com}, \eqref{eq:eq_of_motion_two_body_relcoord}. Also, after solving for the evolution of $\vec{r}$, Eq.~\eqref{eq:trafo_com_components} allows us to easily transform back to what we really want in $\vec{r}_1, \vec{r}_2$ (the time evolution of $\vec{R}$ is determined by the initial conditions already).


Another advantage of reformulating ths two-body problem in the CMS is that, with no external forces, the only force left is $\vec{F}_{2 \rightarrow 1}(\vec{r})$. Many physical forces do not depend on the orientation between the interacting objects, but only on their distance (the forces are isotropic in a certain sense). If $\vec{F}_{2 \rightarrow 1}(\vec{r})$ is of such nature, the relative coordinate $\vec{r}$ being a position vector (i.e.~it is centered at the origin at all times) implies that the only force present in $\vec{F}_{2 \rightarrow 1}(\vec{r})$ is a \Def{central force}. Central forces pose a very common type of problem in physics. They occur when is a body is subject to a force\footnote{In principle, central forces are characterized by $F(r) = - k r^n$. However, $n = -2$ is the most common case, so we will focus on that.}
\begin{equation}\label{eq:central_force}
	\eqbox{
		% \vec{F}(\vec{r}) = - k \frac{\vec{r}}{r^3} = - k \frac{1}{r^2} \vec{e}_r = F(r) \vec{e}_r
		\vec{F}(\vec{r}) = - k \, \vec{r} \, r^{n - 1} = - k \, r^n \, \vec{e}_r = F(r) \vec{e}_r
	}
	\; \text{ where } \; k > 0
\end{equation}
that is pointing radially outwards from the origin (note that $k > 0$ ensures the force is attractive, not repulsive). Due to general laws of the gradient, it can be shown that such a force always has a potential, namely
\begin{equation}\label{eq:central_force_potential}
	\eqbox{
		% V(\vec{r}) = V(r) = - \frac{k}{r}
		V(\vec{r}) = V(r) = - k \, r^{n + 1}
	}
	\, .
\end{equation}
The most common example is $n = -2$ with $F(r) = - \frac{k}{r^2}, \; V(r) = - \frac{k}{r}$, which includes Newton's law of gravitation, the Coulomb interaction of charged particles, and many more.


Time to back up for a second. Our initial goal of understanding the two-body problem of Newtonian physics has now lead us the the classical central-force problem. Therefore, our current objective is to find the trajectory $\vec{r} = \vec{r}(t)$ of a body subject to a central force. (Note that the mass of this \enquote{body} we are looking at is $\mu$. Body is in quotation marks here because we do not actually have a physical object with mass $\mu$ in our problem, but rather have this as an intermediate step and reformulation.) There are many ways in which this problem can be approached. In almost all of them, polar coordinates $(r, \phi)$ are used to describe $\vec{r}$. This should make you curious because polar coordinates describe positions in 2D, while we are dealing with a 3D problem -- or do we?

Well, actually, no. Objects in a central force perform a planar motion, i.e.~we can indeed describe this motion using polar coordinates that are assigned to this plane in which the body moves. We can see this (and actually find the corresponding plane) by considering angular momentum. Defining angular momentum with respect to the center of mass (reduces to $\vec{L}$ if $\vec{R}$ is in origin), i.e.~looking only at the component $\vec{L}_{\vec{r}}$, we trivially have
\begin{equation}
	\eqbox{
		\vec{r} \cdot \vec{L}_{\vec{r}} = \vec{r} \cdot \qty(\vec{r} \cross \mu \dot{\vec{r}}) = 0
	} \, ,
\end{equation}
the relative coordinate is always perpendicular to the angular momentum. This defines a plane because
\begin{equation}
	\eqbox{
		\dot{\vec{L}}_{\vec{r}} = \dv{t} \vec{r} \cross \mu \dot{\vec{r}} = \vec{r} \cross \vec{F}_{2 \rightarrow 1}(\vec{r}) = \vec{F}_{2 \rightarrow 1}(r) \, \vec{r} \cross \vec{e}_r = 0
	} \, ,
\end{equation}
angular momentum is conserved in both direction and magnitude. This holds true for all central forces.% \footnote{More generally, conservative forces with potential $V(\vec{r}) = V(r) = - \frac{k}{r}$ or $V(\vec{r}) = V(r) = - k r^2$ are required since both will yield forces $\propto \vec{e}_r$. The first case includes central forces.}
Thus, the statement $\vec{r} \perp \vec{L}_{\vec{r}}$ is sufficient to prove the claim that $\vec{r}$ can be described in a plane.


Having justified the use of polar coordinates, let us proceed with actually solving for $\vec{r}$. Furthermore, it is common to give the solution as a trajectory $r(\phi)$ rather than $r(t), \phi(t)$ (for the, slightly unsatisfactory, reason that the solutions are more intuitive). One way to proceed further is to look at the energy of the system. For that, we note that in polar coordinates
\begin{align}
	\vec{r} &= x \vec{e}_x + y \vec{e}_y = r \cos(\phi) \vec{e}_x + r \sin(\phi) \vec{e}_y = r \vec{e}_r \equiv \mqty(r \cos(\phi) \\ r \sin(\phi)) \equiv \mqty(x \\ y)
	\, ,
	\\
	\Rightarrow \quad \dot{\vec{r}} &= \dot{r} \mqty(\cos(\phi) \\ \sin(\phi)) + r \dot{\phi} \mqty(- \sin(\phi) \\ \cos(\phi)) = \dot{r} \vec{e}_r + r \dot{\phi} \vec{e}_\phi
	\, .
\end{align}
Bearing in mind that $\vec{e}_r, \vec{e}_\phi$ are orthogonal, the total energy then becomes
\begin{equation}\label{eq:energy_central_force}
	E = T + V = \frac{1}{2} \mu \dot{\vec{r}}^2 + V = \frac{1}{2} \mu \dot{r}^2 + \frac{L}{2 \mu r^2} + V \eqqcolon \frac{1}{2} \mu \dot{r}^2 + U
\end{equation}
where we have substituted the angular momentum, which reads\footnote{Here we drop the index ${}_{\vec{r}}$ since we are just writing out the angular momentum in polar coordinates. It is to be understood that when interpreted as the angular momentum in the CMS, this is only the component $\vec{L}_{\vec{r}}$.}
\begin{equation}
	\vec{L} = \mu \vec{r} \cross \dot{\vec{r}} = m r^2 \dot{\phi} \vec{e}_z
	\, .
\end{equation}
$U$ is an effective potential that contains the energy coming from the interaction in $V$ and the centrifugal term $\frac{L}{2 \mu r^2}$ that comes about from $\vec{r}$ orbiting (i.e.~rotating) around the origin.


-> we get effective potential due to contribution from centrifugal potential; is centrifugal term consequence of analysis non-inertial frame? Nahh, don't think so, just comes from particle rotating, i.e.~accelerating.


Equation \eqref{eq:energy_central_force} is equivalent to
\begin{equation}
	\dot{r} = \pm \sqrt{\frac{2}{\mu} (E - U)}
	\, .
\end{equation}
Forming the quotient $\flatfrac{\dot{{r}}}{\dot{\phi}}$ then gives an inhomogenous differential equation for $\dv{r}{\phi}$, which can be solved by making a harmonic oscillator ansatz. The solution reads
\begin{equation}\label{eq:central_force_r_of_phi}
	\eqbox{
		r(\phi) = \frac{p}{1 + \epsilon \cos(\phi)}
	} \, .
\end{equation}
This is the general formula for a \Def{conic section}. (Note that it takes this particular form because we have chosen to $\phi$ to be the \Def{true anomaly} here, which is measured with respect to the point of minimal distance $r_\mathrm{min}$.) The exact nature of the graph of $r(\phi)$ depends on the values of the new parameters, and some other quantities related to ellipses.



\begin{figure}
	\centering

	\includegraphics[width=0.5\textwidth]{pictures/central_force_ellipse.pdf}

	\caption{Quantities in an ellipse.}
	\label{fig:central_force_ellipse}
\end{figure}



First, we have $\epsilon$, which is called \Def{numerical eccentricity}:
\begin{center}
	$\epsilon < 1$: ellipse
	\qquad
	$\epsilon = 1$: parabola
	\qquad
	$\epsilon > 1$: hyperbola
\end{center}
Note that $\epsilon = 0$ represents a special form of ellipse, a circle. In terms of physical quantities,
\begin{equation}
	\eqbox{
		\epsilon = \sqrt{1 + 2 \frac{E L^2}{m k^2}}
	} \, .
\end{equation}
Since all quantities in the square root are greater than zero, except for the energy, statements about the form of the orbit can be translated to statements in terms of energy: $E < 0$ means $\epsilon < 1$ and thus an elliptical orbit, which matches our understanding of a bound orbit; $E >= 0$ then corresponds to unbound orbits like parabola or hyperbola.


$p$ is another prominent one, the \Def{semi-latus rectum} of the conic section, which basically determines the size of the orbit. It involves the same physical quantities as $E$ (but combined in a different way, it plays a different role than $\epsilon$) and reads
\begin{equation}
	\eqbox{
		p = \frac{L^2}{m k}
	}
	\, .
\end{equation}


These quantities play a sort of hybrid role because they can be interpreted physically or mathematically, via their role in the general description of conic sections (cf.~Fig.~\ref{fig:central_force_ellipse}). 


The lengths along these axes are called \Def[major axis]{major} and \Def{minor axis}, describing the longer and shorter axis. In Fig.~\ref{fig:central_force_ellipse}, we elect to draw the \Def{semi-major axis}
\begin{equation}
	\eqbox{
		a = \frac{p}{1 - \epsilon^2} = \frac{r_\mathrm{min} + r_\mathrm{max}}{2}
	}
\end{equation}
and \Def{semi-minor axis}
\begin{equation}
	\eqbox{
		b = \frac{p}{\sqrt{1 - \epsilon^2}} = \sqrt{r_\mathrm{min} r_\mathrm{max}}
	} \, ,
\end{equation}
as is commonly done (\enquote{semi} = half). Here we have chosen to also express them in terms of the minimal and maximal distance. In celestial mechanics, one calls those \Def{perihelion}\footnote{\enquote{Peri} $\hat{=}$ close, to memorize which point represents minimum/maximum.}
\begin{equation}
	\eqbox{
		r_\mathrm{min} = \frac{p}{1 + \epsilon}
	}
\end{equation}
and \Def{aphelion}
\begin{equation}
	\eqbox{
		r_\mathrm{max} = \frac{p}{1 - \epsilon}
	}
	\, .
\end{equation}
For the convention on the angle $\phi$ that we have chosen, these occur at $\phi = 0\degree, 180\degree$, respectively (where $\cos(\phi) = \pm 1$).

Besides the arithmetic ($a$) and geometric ($b$) mean of perihelion and aphelion, their harmonic mean also plays an important role. In fact, we have seen it already in the semi-latus rectum
\begin{equation}
	\eqbox{
		p = \qty(\frac{r_\mathrm{min}^{-1} + r_\mathrm{max}^{-1}}{2})^{-1} = \frac{b^2}{a}
	}
\end{equation}
which is the radius at $\phi = 90\degree, 270\degree$. Additionally, the numerical eccentricity receives a nicer interpretation: it is the coefficient of variation between perihelion, aphelion, i.e.
\begin{equation}
	\eqbox{
		\epsilon = \frac{r_\mathrm{max} - r_\mathrm{min}}{r_\mathrm{max} + r_\mathrm{min}}
	}
	\, .
\end{equation}
For completeness, we note that there is also another notion of eccentricity in the \Def{linear eccentricity}
\begin{equation}
	\eqbox{
		e = \epsilon a
	}
	\, .
\end{equation}\\

% -> nice expressions here: \url{https://en.wikipedia.org/wiki/Kepler%27s_laws_of_planetary_motion#First_law}


-> possibly some nice things on trajectory: \url{https://en.wikipedia.org/wiki/Kepler_orbit#Properties_of_trajectory_equation}



Let us get back to physics now. The first result of note is that, when fixing the position of body $2$, body $1$ performs an elliptical motion around it with the first body sitting at a focal point of this ellipse. (This is a direct consequence of the result we have calculated in Eq.~\eqref{eq:central_force_r_of_phi}, which shows that the reduced mass $\mu$ performs such a motion around the origin; body $1$ need not be in the origin for this interpretation to be true.) However, this is not how the actual orbit of body $1$ looks like, when viewed from a proper, inertial frame; similarly, in such a frame, body $2$ is not necessarily at rest for all times.

To obtain the trajectories of both bodies, we must use the transformations from $\vec{R}, \vec{r}$ back to $\vec{r}_{1, 2}$ in Eq.~\eqref{eq:trafo_com_components}. Those tranformation equations immediately tell us that, generally speaking, the physical trajectories of both bodies are ellipses (of different extent, determined by $\frac{\mu}{m_i} = \frac{m_j}{M} (1 - \delta_{ij})$ for the $i$-th body), and that they both have $\vec{R}$ in one of their focal points (as this is the constant offset from the focal point $O = (0, 0)$ of $\vec{r}$). We show some examples for different mass ratios $\flatfrac{m_1}{m_2}$ in Fig.~\ref{fig:central_force_orbits}.



% -> in the more physical/intuitive terms, this means m2 orbits around m1 in elliptical orbit, with m1 sitting at focal point (but note that these are all \emph{not} statements from an inertial frame, unless in Keplerian limit; normally, we have to transform to an inertial frame, which can be done by calculating $\vec{r}_1 = \vec{R} + \frac{m_2}{M} \vec{r}$; then we see that, in general, both bodies perform elliptical motion, and that they both have center of mass in focal point)

% -> fact that solution of problem using COM \enquote{looks like} problem for particle of mass $\mu$ only helps us make easy transition in Keplerian case, whence it allows direct transition from $\mu$ to $m_2 \approx \mu$; in most general case, it is probably not suited well because it can easily mess up intuition (about where center of mass is, whether in focal point or not, or how orbits of the two bodies look like); in this general setting, going to COM is really just a coordinate transformation applied to get easier solutions to the equations of motion, but for the really interesting, physical results we still have to map back to the initial (inertial) frame (though some limited statements can be inferred from $\vec{r}$ already, namely about motion of m2 around m1; ultimately, this is not relevant though, since actual motion of m2 might be substantially different)



\begin{figure}
	\centering

	\includegraphics[height=0.1\textheight]{pictures/central_force_orbits.pdf}

	\caption{Some orbits. Keplerian limit is uppermost one \todo{fix plot}}
	\label{fig:central_force_orbits}
\end{figure}




		\subsection{Kepler's Laws}
From a historical point of view, the solution of the two-body problem in the way presented here was found by Newton, a few decades after Johannes Kepler famously formulated his three laws about the orbits of planets in our solar system. For Kepler, these laws were empirical insights gained from data collected by Tycho Brahe; for us, they will come quite naturally as corollaries from the general two-body and central-force problem over the last few subsections. To see that, let us consider the case $m_2 \gg m_1$, which means $M \approx m_2, \mu \approx m_1$.


Kepler considered two objects that interact gravitationally, which means the potential is given by
\begin{equation}
	\eqbox{
		V(r) = - \frac{G M m}{r}
	}
	\, .
\end{equation}
Here we have started to use $M$ for the mass $m_2$ of the larger body (the sun) and $m$ for the mass $m_1$ of the smaller body (the planet in consideration).




% mention fatal misconception by Kepler, trying to describe everything using ellipses? Which made things unbelievably complicated -> would have to read up on this, but sounds like a kind of funny anecdote



			\paragraph{First Law}

\begin{center}
	% The orbit of the relative coordinate $\vec{r}$ around the COM $\vec{R}$ is an ellipse with $\vec{R}$ at one of its foci. -> this is WRONG! Relative coordinate is not starting in COM, and COM is not always in focal point
	The orbit of every planet is an ellipse with the sun at one of the two foci.
\end{center}

% -> we are not satisfied with this, let's solve for the actual orbits (this part was done by Newton, a few decades after Kepler noted nature of orbits)


Other than the implication that all planets have inclination $< 1$, this is really just a corollary of planets orbiting the sun being an example of the two-body problem in a central force. Schematically, Fig.~\ref{fig:kepler_orbits} shows how the orbits of several planets look like; notice in particular how the sun is always in a focal point. In the Keplerian limit, Fig.~\ref{fig:central_force_ellipse} depicts accurately the orbit of a planet if we place $\vec{R}$ in $F_1$ (which is the origin in this plot, though this is not required), since $\vec{r}_1 \approx \vec{R}$ and $\vec{r}_2 \approx \vec{r}$.



\begin{figure}
	\centering

	\includegraphics[width=0.5\textwidth]{pictures/central_force_planets.pdf}

	\caption{Some planet orbits. 3D, unlike in previous plots}
	\label{fig:kepler_orbits}
\end{figure}




			\paragraph{Second Law}
The first Keplerian law tells us the motion of a body in a central potential follows an elliptical orbit. In other words, the body moves in a plane. The second law can be stated as:

\begin{center}
	% A line segement connecting the two bodies sweeps out equal ares during equal time intervals.
	A line segement along the relative coordinate $\vec{r}$ sweeps out equal areas during equal time intervals.
\end{center}
\todo{really with center?}


In a more explicit (and perhaps modern) interpretation, the second law is usually stated in terms of angular momentum. We had seen that $\vec{L}$ is conserved, i.e.~constant in time (both direction and magnitude). The area swept out for a given time can be calculated as
\begin{equation}
	\eqbox{
		A = \int_{t_1}^{t_2} \frac{L}{2m} \, dt = \frac{L}{2m} (t_2 - t_1)
	} \, .
\end{equation}
This means we are already done proving Kepler's second law: the area swept out is a constant multiplied with the time interval. Due to its close relation to angular momentum, the second Keplerian law is often seen as an early formulation of conservation of angular momentum.

% -> notice how potential only depends on distance, not direction, crucial assumption here; moreover, this particular form of gradient means we have radial or quadratic force -> hmm nah, is general law, right? We only use chain rule here

% further characterization of the motion: for any force $\propto \vec{r}$, $\dot{\vec{L}} = \vec{r} \cross \vec{F} \propto \vec{r} \cross \vec{r} = 0$; in terms of potential, that means $\propto \frac{1}{r}$ or $\propto r^2$; so for central force problem as special case of this, we have angular momentum conservation


-> constancy of $L$ tells us same about velocity: for smaller $r$, we need larger $v$ to keep $L$ constant (which neglects dependence on angle between $\vec{r}, \vec{v}$, but point stands, think about periastron and apostron for example)


-> in Keplerian case, $\vec{R}$ coincides with $\vec{r}_1$; thus $\vec{L} = \vec{L}_{\vec{r}}$ and all claims involving angular momentum do not require so much case \todo{mention here or after starting Kepler section?}




			\paragraph{Third Law}
There exists a general formula for the area of an ellipse,
\begin{equation}
	A = \pi a b
	\, .
\end{equation}
However, we can also calculate the area in terms of other physical quantities by integrating Kepler's second law over an entire orbit. This results in
\begin{equation}
	A = T \frac{L}{2m}
\end{equation}
which in turn yields
\begin{equation}
	\eqbox{
		\frac{T^2}{a^3} = \frac{\pi^2 \mu k^2}{- 2 E^3} \frac{- 8 E^3}{k^3} = \frac{4\pi^2 \mu}{k}
	}
	\, .
\end{equation}
In other words, if we take two different bodies in a central force,
\begin{equation}
	\eqbox{
		\qty(\frac{T_1}{T_2})^2 = \qty(\frac{a_1}{a_2})^3
	} \, .
\end{equation}
For our solar system, this means the orbits of planets farther out in the solar system (larger $a$) have a longer orbital period $T$.



		\subsection{Collisions}
\todo{wouldn't that fit really well here?}



		\subsection{Scattering}
We have looked at $\epsilon < 1$ case extensively now, but what about $\epsilon \geq 1$. Formally, this is the case of hyperbolic trajectories and we can think of it as \enquote{collisions at a distance}: $m_1$ flies in, interacts with the potential generated by $m_2$, and flies off again. This is what physicists call \Def{scattering}. It occurs in a variety of places throughout virtually all physical theories (celestial bodies, nuclear physics, particles moving through the atmosphere, ...). % This is also the scope that we discuss it in, i.e.~we drop the assumption of gravitational interaction for now and return to a general central force. \todo{really? Even at this stage? Right now we just need conservative, don't we? -> hmm not 100\% sure, I think we assume central force actually}

% -> this is hyperbolic case, and there are some interesting dynamics going on as well; make transition to scattering, talk about deflection angle

% -> this central force problem not only occurs for celestial objects that pass by a planet or something, but all over the place in physics; another example is Rayleigh scattering of particles in the atmosphere (then talk about cross section)

We follow the discussion provided in \cite{Thornton_2003}, so we have to pay a little attention to the scope of certain formulas. 


We will analyze the situation from a frame (\enquote{lab frame}) where $m_2$ is initially at rest, while $m_1$ enters the frame with velocity $v_1$. Given this information (or equivalent one, like the charges if a Coulomb force is considered), the strength of the interaction taking place is entirely determined by $b$, the \Def{impact parameter}, which is the minimal distance of the particles if no force was acting (Fig.~\ref{fig:scattering}). $b$ determines what trajectories after scattering process look like, which can be encapsulated in the \Def{scattering angle} $\theta$.


The previous paragraph needs some clarification. It follows an argument from \cite{Thornton_2003}, which goes as follows: $b$ is sufficient to determine trajectories etc.~because all expression we saw for central force involved just energy $E$, potential $V$, angular momentum $L$. However, assuming that $m_2$ is at rest before the particles interact, the total angular momentum is given entirely by the angular momentum of $m_1$. Since we are free to choose a reference point for $\vec{L}$, we can choose to look at angular momentum of $m_1$ about $m_2$. For $m_1$ far away from $m_2$, this is given by
\begin{equation}
	L = m_1 v_1 b = b \sqrt{2 m_1 T'_1}
\end{equation}
where $v_1$ also refers to the velocity far away from $m_2$ (which just means the force exerted by $m_2$ has a negligible effect). Since the magnitude $L$ is conserved in a central force, $b$ is the quantity that determines the whole interaction (because mass, initial velocity are considered to be known).\footnote{It might be possible to get away with the less restrictive requirement of conserved angular momentum, not necessarily central force. But we chose to focus on central forces here.} The fact that we have to assume $\dot{L} = 0$ for this argument was not entirely obvious to me from reading the discussion in \cite{Thornton_2003}, which is why we emphasize it here.

-> also, $E = T'_1$ where $T'_1$ is kinetic energy far away from origin, since we have energy conservation (prime indicates we compute it in CMS)

% better justification of why $b$ is sufficient: ; we assume $E$ to be given (via mass, velocity) and $V$ as well (since nature of force is known); angular momentum of $m_1$ about $m_2$, however, can be expressed in terms of $b$ as $L = m_1 v_1 b$ (which is value of $L$ at particular point, but the magnitude of this value is conserved, so we can just use it); therefore, $b$ already determines quantities such as $r(\phi)$



\begin{figure}
	\centering

	\includegraphics[width=0.5\textwidth]{pictures/scattering.pdf}

	\caption{Scattering. $\theta$ is same angle as $\psi$, but when viewed from CMS (whence motion of $m_1$ in this plot becomes motion of $\mu$; i.e.~$\theta$ is angle of trajectory of $m_1$ relative to $m_2$).}
	\label{fig:scattering}
\end{figure}



Treating the problem of scattering in a manner suitable for general application requires a paradigm shift because most of the time when we try to model scattering, we do not know what value $b$ exactly has. For planets as macroscopic objects, we may be able to find it; but when scattering electrons off of atoms, the story is very different because we have no way to measure $b$ in this case and thus no way to determine $\theta(b)$ exactly. Instead, we have to resort to assigning probabilities to ranges of values for $\theta$.

% -> i.e.~we would like to have $d\theta(db)$ for some range of $b$-values; factor of proportionality involves cross section

But first, we need some meaningful notions to describe the whole situation. For this, we can drop the assumption of a central force or conserved angular momentum, at least temporarily. Say we have particle beam with intensity $I$ that is incident on $m_2$. Then one can write the number of particles $dN$ scattered into a solid angle element $d\Omega'$ at scattering angle $\theta$ (and per unit time) as\footnote{The fact that we use $\theta$ means this is a definition in the CMS. We do this to be consistent with the treatment in \cite{Thornton_2003}, but note that a definition in the lab frame takes the same form, upon replacing $\theta \mapsto \psi, d\Omega' \mapsto d\Omega$ (since $\sigma(\theta) d\Omega' = \sigma(\psi) d\Omega$; must be the case since the amount of scattering into a given path of the sphere is something that does not depend on coordinates, and this is what both expressions represent in CMS, lab frame, respectively; this is visualized nicely in Fig.9-20 of \cite{Thornton_2003}).}
\begin{equation}\label{eq:cross_section_cms}
	\eqbox{
		dN = I \sigma(\theta) \, d\Omega'
	}
	\quad \Leftrightarrow \quad
	\eqbox{
		\sigma(\theta) = \dv{\sigma}{\Omega'} = \frac{1}{I} \dv{N}{\Omega'}
	}
\end{equation}
for some factor of proportionality $\sigma$. This factor quantifies how effectively particles from the beam are scattered into a specific solid angle element $d\Omega'$ by computing the ratio of particles scattered into $d\Omega'$ ($\dv{N}{\Omega'}$) and the total number of incident particles (per unit area and time, which is quantified by $I$). This ratio is called \Def{differential scattering cross section}.

% where the \Def{differential scattering cross section} $\sigma(\theta)$ quantifies \enquote{the number of interactions per target particle that lead to scattering into $d\Omega'$ at the angle $\theta$ (this is $dN$), divided by number of incident particles per unit area (this is $I$)}

% -> idea: intensity is also something like number per area (and time), but for the incoming particle stream; cross section thus compares incoming particles per area to particles per are (solid angle element) at a given angle; in other words, it tells us how effectively beam is scattered into the specific part of the sky that this solid angle element represents -> is number of particles scattered into specific solid angle element, normalized by total number of particles that are interacting (right?) -> intensity is power per area = energy per time and per area (= flux density)


The definition of $\sigma$ we have just provided can be used no matter what the nature of the force exerted by the second body is. For more detailed discussions, however, we now return to the assumption of a central force. This is a force that has axial symmetry, so
\begin{equation}
	d\Omega' = \sin(\theta) d\theta d\phi = 2\pi \sin(\theta) d\theta
\end{equation}
($d\Omega'$ is intended for integration, and we have carried one out now). We had stated that $b$ is sufficient to describe the scattering process, which means our goal is to find an expression for some range of scattering values in terms of a range of impact parameter values, i.e.~$d\theta = d\theta(db)$. For ease of analysis, we continue in the CMS and deal with the transformation back to the lab frame later on. In our axially symmetric setup, if we consider a disk of width $db$ and radius $b$ (centered on $b = 0$), then the area of this disk is just $2\pi b \, db$. Axial symmetry also implies all particles incident on this area are scattered to the same solid angle element $d\Omega'(\theta)$ (remember that we have \enquote{integrated out} the azimuthal dependence, the integration over the whole circumference of the circle here accounts for that), and since $\mathrm{number} = \mathrm{intensity} \cdot \mathrm{area}$,
% In our current setup, we have an intensity $I$ that is incident on the range $db$. Continuing continued axially symmetric must, upon multiplication with circumference $2\pi b$ (due to symmetry), correspond to particles scattered into $d\Omega'(\theta)$, i.e.
\begin{equation}
	\eqbox{
		I 2\pi b \, db = dN = - I \sigma(\theta) d\Omega' %= - I \sigma(\theta) 2\pi \sin(\theta) d\theta
	} \, .
\end{equation}
The minus comes from the fact that increasing the impact parameter $b$ decreases the amount of angular deflection $\theta$, i.e.~$\dv{b}{\theta} < 0$. Therefore, remembering the axial symmetry in expressing $d\Omega'$, we get
\begin{equation}\label{eq:cross_section_cms_2}
	\eqbox{
		\sigma(\theta) = \frac{b}{\sin(\theta)} \abs{\dv{b}{\theta}}
	} \, .
\end{equation}

This expression is helpful because for a central force interaction, one can show that the change in angle between the points of closest and largest distance is
\begin{equation}
	\Theta = \frac{\pi - \theta}{2} = \int_{r_\mathrm{min}}^{r_\mathrm{max}} \frac{L}{r^2 \sqrt{2 \mu \qty(E - V - \frac{L^2}{2 \mu r^2})}} \, dr
	\, .
\end{equation}
Using that $L = \mu v_1 b = b \sqrt{2 m_1 T'_1}$ (where, again, $T'_1$ denotes the kinetic energy \emph{before} any interaction with the central potential exerted from the origin takes place, and measured in the CMS), we obtain
\begin{equation}\label{eq:angle_as_func_of_b}
	\Theta = \int_{r_\mathrm{min}}^{\infty} \frac{b}{\sqrt{1 - \frac{b^2}{r^2} - \frac{U}{T'_1}}} \, dr
	\, .
\end{equation}
This determines how $\theta$ depends on $b$, which in turn allows determining $\sigma$ via Eq.~\eqref{eq:cross_section_cms_2}.\\


Having found an expression for the cross section in the CMS, our next task is to transform this expression to the lab frame. This involves a substantial amount of math (the calculation is based on $\sigma(\theta) d\Omega' = \sigma(\psi) d\Omega$), and the result also reflects this:
\begin{equation}\label{eq:cross_section_general}
	\eqbox{
		\sigma(\psi) = \eval{\sigma(\theta)}_{\theta = \arcsin(q \sin(\psi)) + \psi} \frac{\qty(1 \cos(\psi) + \sqrt{1 - q^2\sin[2](\psi)})^2}{\sqrt{1 - q^2\sin[2](\psi)}}
		, \quad q = \frac{m_1}{m_2}
	}
	\, .
\end{equation}
% -> note that this assumes $q \leq 1$; is fine, because we can always transform so that it is larger of the masses that is at rest, and reassign larger mass to $m_2$ -> uuum nope, we do not have something greater 1 in arcsin
Evaluating this expression is complicated for general $q$, but at least we have simple special cases: for $q = 1 \Leftrightarrow m_1 = m_2$, both particles get scattered at same angle in lab frame, so the relative angle between particles (which is what we have from CMS) being $\theta$, implies $\psi = \frac{\theta}{2}$, so that
\begin{equation}\label{eq:cross_section_q1}
	\eqbox{
		\sigma(\psi) = \eval{\sigma(\theta)}_{\theta = 2\psi} 4 \cos(\psi)
	}
\end{equation}
Similarly, for $m_1 \ll m_2 \Leftrightarrow q \approx 0$, the second particle remains at rest so that $\psi \approx \theta$ and
\begin{equation}\label{eq:cross_section_q0}
	\eqbox{
		\sigma(\psi) = \eval{\sigma(\theta)}_{\theta = \psi}
	}
\end{equation}



Although we have mostly used the symbol $\sigma$ for now, the cross sections used up until this point have been differential ones, i.e.~they only referred to an infinitesimal solid angle element. To obtain a \Def{total scattering cross section}, we thus have to integrate over the angular dependence
\begin{equation}
	\eqbox{
		\sigma = \int \sigma(\theta) \, d\Omega' = \int \sigma(\psi) \, d\Omega
	}
	\, .
\end{equation}

-> comment on different integration ranges

-> interpretation: this is what we should look at if we are interested in probability that any interaction with second particle $m_2$ takes place; resulting $\sigma$ is effective area of target particle (strongly interacting = large effective area)


\todo{show example? For solid sphere for example}




			\paragraph{Rutherford Scattering}
Let us discuss a particular example of a central force, namely once again
\begin{equation*}
	V(r) = - \frac{k}{r} \, ,
\end{equation*}
which describes, e.g., a massive particle in a gravitational field or a charged particle in a Coulomb field. For this force, one can evaluate Eq.~\eqref{eq:angle_as_func_of_b} to obtain
\begin{equation}
	\cos(\Theta) = \frac{k}{2 T'_1 \sqrt{1 + \qty(\frac{k}{2 T'_1})^2}}
\end{equation}
and this can be rearranged to
\begin{equation}
	\eqbox{
		b = \frac{k \cot(\theta/2)}{2 T'_1} = \frac{\mu k \cot(\theta/2)}{p^2}
	}
	\,. 
\end{equation}
Differentiating with respect to $\theta$ gives
\begin{equation}
	\dv{b}{\theta} = - \frac{k}{4 T'_1} \frac{1}{\sin[2](\theta/2)}
\end{equation}
and substitution of this into Eq.~\eqref{eq:cross_section_cms_2} then yields the final result
\begin{equation}\label{eq:rutherford_scat}
	\eqbox{
		\sigma(\theta) = \frac{k^2}{(4 T'_1)^2} \frac{1}{\sin[4](\theta/2)}
	}
	\, .
\end{equation}
This is the \Def{Rutherford scattering formula}. We see a strong dependence on $\theta$. Recall that $\sin[4](\theta/2) \simeq (\theta/2)^4$ for small angles, so the cross section falls off extremely steeply. On the other hand, there is no dependence on sign of $k$. That means we get same cross sections for attractive and repulsive forces.

Also, as a little aside, this classical result actually agrees with experimental evidence -- which means it agrees with QM-prediction of the same phenomenon (quite remarkably).\\


-> interesting: total cross section diverges!


For $m_1 = m_2$, $T'_1 = T_1/2$, so Eq.~\eqref{eq:rutherford_scat} becomes
\begin{equation}
	\sigma(\theta) = \frac{k^2}{(4 T_1)^2} \frac{1}{\sin[4](\theta/2)}
\end{equation}



    \section{Rigid Bodies}
In this section, we finally go away from the model of a point particle/test mass to bodies that have a spatial extent and shape. Granted, some idealized assumptions are still made, for example that distances between different particles in the body remain fixed at all times and no matter if it is moving or rotating (i.e.~undergoing some kind of dynamics). This is where \enquote{rigidity} comes from.\footnote{In reality, things do not work like that. Atoms and molecules are constantly vibrating due to thermal energy (Brownian motion\todo{right?}) and relativity theory also tells us that the maximum speed at which information can be transmitted is finite at $c$ (in matter, the propagation speed is the speed of sound of the respective material and thus usually much smaller than $c$). Hence, if I push one end of a rod, the other one does not move immediately, but we assume this anyway for our model of a rigid body.} Moreover, although bodies are always composed of a finite number of particles, for a macroscopic number of particles we will sometimes approximate them as continuous bodies (corresponds to what is done in statistical physics). This has mostly mathematical consequences, in that sums over particles must be replaced by integrals over appropriate densities (the former can be recovered from the latter using densities that involve $\delta$-functions).


-> in essence, rigid body = $N$-particle system for $N$ large; so we can utilize many results from Sec.~\ref{sec:n_body_mechanics}


Before proceeding with some physics, we note down how a number of definitions translate from particles to rigid bodies. Given the mass density $\rho(\vec{x})$ of a continuous body that occupies some volume $V$, we can define its (total) mass
\begin{equation}
	M = \int_V dm
	% = \int_V \rho dV
	= \int_V \rho(\vec{x}) d^3x
\end{equation}
its center of mass
\begin{equation}
	\vec{R} = \frac{1}{M} \int_V \rho(\vec{x}) \vec{x} d^3x
\end{equation}
and its momentum
\begin{equation}
	\vec{P} = \int_V \rho(\vec{x}) \dot{\vec{x}} d^3x
\end{equation}

\todo{shouldn't we also have analogous expression for inner angular momentum? Or does this not exist for a rigid body?}

-> good for us that all of these are additive quantities (sometimes this is also called extensive, a notion coming from thermodynamics)


inner forces in a rigid body? Must be zero, because of the third law. Therefore, second law for rigid body reads
\begin{equation}
	\eqbox{
		\vec{F}_\mathrm{ext} = \dv{\vec{P}}{t}
	}
\end{equation}


its moment of inertia when rotated about some axis $\vec{n} = \frac{\vec{\omega}}{\omega}$
\begin{equation}
	I \coloneqq \vec{n}^T J \vec{n}, \quad J_{ij} = \int_V \rho(\vec{x}) \qty[\vec{x} \cdot \vec{x} \delta_{ij} - x_i x_j] d^3x
\end{equation}


component of $\vec{L} \parallel \vec{\omega}$:
\begin{equation}
	% \vec{L}_\omega = \vec{L} \cdot \vec{n} \underset{\todo{understand this}}{=} \vec{n}^T J \vec{n} \omega = \vec{M}_\mathrm{ext} \cdot \vec{n}
	\vec{L}_\omega = \vec{L} \cdot \vec{n} = I_\omega \vec{\omega} \cdot \vec{n} = \vec{n}^T J \vec{n} \omega \underset{\todo{understand this}}{=} \vec{M}_\mathrm{ext} \cdot \vec{n}
\end{equation}
hmm I mean we have $\vec{L} = I \vec{\omega}$ where $I$ is moment of inertia around $\vec{n}$
-> I think this does not make sense here

-> one has to exert moment of inertia around the axis of rotation in order to change the angular momentum (total angular momentum of the body)

-> not sure we can see this from the equation here; but this holds in general, right? Not just for rigid body




nice for work on rigid body: \url{https://en.wikipedia.org/wiki/Work_(physics)#Work_of_forces_acting_on_a_rigid_body}



gravitation of rigid bodies: \url{https://en.wikipedia.org/wiki/Shell_theorem} -> Thornton treats this in detail as well, see Chapter 5



		\subsection{Rotating Rigid Bodies}
Turns out to be useful to define one inertial system (\enquote{fixed/inertial coordinate system}) and one attached to the body in a specific way (\enquote{body coordinate system}). The general transformation between the two systems requires a translation (from origin of the inertial frame to the origin of the body coordinate system with the latter usually being defined as the bodie's center of mass) and a rotation about the body system's origin.


The rotation is normally described in terms of \Def{Euler angles}. These are basically an algorithm for transformation between arbitrarily oriented coordinate systems; mathematically speaking, rotation matrix between the systems is decomposed into three rotation matrices about the $x$-/$y$-/$z$-axis, i.e.~we can parametrize the transformation using three angles -> order is $z$-axis, rotated $x$-axis, rotated $z$-axis


-> transformation from fixed to body system (and even more generally between different frames; not necessarily between all coordinate systems, though; e.g., not from Cartesian to spherical coordinates) can be described as matrix multiplication $\vec{x} = \tensor{\lambda} \vec{x}_f$ -> idea of Euler angles is that we can decompose $\tensor{\lambda} = \tensor{\lambda}_\psi \tensor{\lambda}_\theta \tensor{\lambda}_\phi$



why three though, after all we can specify a position in spherical coordinates using two angles? That is true but to rotate an entire coordinate system, we need three (because relative orientation of the axis shall be maintained)

-> basically: specifying position is something very different; it is not even clear from which axis we should start rotations here, could be $x$- or $z$-axis, from all positions we can find rotation using the two angles $\theta, \phi$ that rotates us into correct position; but now we want to rotate a body and not a point, so we want to preserve orientation in a certain way
-> not sure if this discussion is helpful...
-> I think following is best: we can specify how points and thus single vectors transform using two angles already, e.g., polar and azimuthal to transform $z$-axis into $P$; but this transformation does not fix how $x, y$ shall be oriented in the plane perpendicular to $z$ yet, so to preserve the orientation of different unit axes, we need three rotations with corresponding rotation angles

-> this is best phrasing: two points are sufficient to fix the orientation of a point, i.e.~of a single axis, say the $z$-axis; but this is not sufficient to uniquely fix the orientation of a whole coordinate system, for this we must also fix how $x, y$ shall be oriented in the plane perpendicular to $z$; this required another angle, so in total there are three rotations with corresponding rotation angles



		% \subsection{Moment Of Inertia \& Angular Momentum}
		\subsection{Kinetic Energy, Angular Momentum \& Moment Of Inertia}
The goal of this subsection is to derive an expression for the angular momentum of a rigid body (and based on that, for the kinetic energy). More specifically, we are referring to the angular momentum in the body system, i.e.~relative to the center of mass of the body. (Careful with wording: we will derive an expression in the fixed coordinate frame. But what it will represent is angular momentum in the body system.) To keep the discussion as general as possible, we consider a body rotating with angular velocity $\vec{\omega}$, which means the body system is a non-inertial frame. We must therefore use Eq.~\eqref{eq:vel_non_inert_frame_final} to determine the velocity of each point. Since we assume the body to be rigid, there can be no velocity relative to the body system. Hence the velocity of any point on (or particle in) the body, as measured from the fixed frame, is
\begin{equation}
	\eqbox{
		\vec{v}_f = \vec{V} + \vec{\omega} \cross \vec{r}
	}
\end{equation}
i.e.~it is composed of motion of the center of mass $\vec{V}$ and rotation about the center of mass. We omit the subscript ${}_f$ from now on.


Using this expression, we can do many things now, and all of them require quite a bit of algebra.


angular momentum, with respect to some point in the body coordinate system (e.g., the center of mass): relative to the body coordinate system,
\begin{equation}
	\vec{p}_\alpha = m_\alpha \vec{v}_\alpha = m_\alpha \vec{\omega} \cross \vec{r}_\alpha
\end{equation}
and angular momentum
\begin{equation*}
	\vec{L} = \sum_\alpha \vec{r}_\alpha \cross \vec{p}_\alpha
\end{equation*}
becomes
\begin{equation}
	\eqbox{
		\vec{L} = \tensor{I} \cdot \vec{\omega} = \sum_i \unitvec{i} \sum_j I_{ij} \omega_j
	}
\end{equation}
where we are introduced to the \Def{inertia tensor}
\begin{equation}
	\eqbox{
		% I_{ij} = \sum_\alpha m_\alpha \qty[\delta_{ij} \sum_k x_{\alpha, k} x_{\alpha, k} - x_{\alpha, i} x_{\alpha, j}]
		% -> denoting sum differently below
		I_{ij} = \sum_\alpha m_\alpha \qty[\delta_{ij} \norm{x_{\alpha}}^2 - x_{\alpha, i} x_{\alpha, j}]
	}
\end{equation}



% Although we have mostly used $\vec{L} = \vec{r} \cross \vec{p}$ for the angular momentum, it also reads 
-> this is actually reminiscent of alternative definition of angular momentum that we had, $\vec{L} = I \vec{\omega}$ (cf.~Eq.~\eqref{eq:angular_momentum_1}) -- but now $I$ is not a scalar anymore, as it was for point particle, but rather a two-tensor (matrix); this is now the case because due to the geometry of the body, it can be the case that $\vec{\omega} \nparallel \vec{L}$ and this cannot be achieved with a scalar moment of inertia; let's immediately look at an example for this


\begin{ex}[Rotating Dumbbell]
	Thornton around Eq (11.20)

	-> does he actually deal with that?
\end{ex}



kinetic energy (note that we here need center of mass in body system origin; unlike before, where angular momentum was given with respect to arbitrary point):
\begin{equation}
	\eqbox{
		T = T_\mathrm{transl} + T_\mathrm{rot} = \frac{1}{2} M V^2 + \frac{1}{2} \sum_\alpha m_\alpha \qty(\vec{\omega} \cross \vec{r}_\alpha)^2
	}
\end{equation}
\todo{Thornton uses leading up to 11.6 that $\vec{R} = 0$ in body system; does that make sense?} -> ah, is just an odd phrasing: he has there sum over all relative coordinates (because I guess this is something like center of mass, which is origin of center of mass system of course), which just yields $\sum_\alpha \vec{r}_\alpha = \sum_ \alpha \vec{r}_{f, \alpha} - \vec{R}$ and due to the factors of $m_\alpha$ that I cannot remember, this vanishes by definition
-> this is just splitting into center-of-mass terms and relative coordinate terms, with the particular form of relative velocity substituted

-> this can be further massaged into
\begin{equation}
	\eqbox{
		T_\mathrm{rot} = \frac{1}{2} \vec{\omega} \cdot \vec{L} = \frac{1}{2} \vec{\omega} \cdot \tensor{I} \cdot \vec{\omega} = \frac{1}{2} \sum_{i, j} I_{ij} \omega_i \omega_j
	}
\end{equation}



		\subsection{Moment Of Inertia}
We found a new object, the moment of inertia $\tensor{I}$, so it makes sense to look at some of its properties. Initially, it has been defined for an $N$-particle system, the generalization for continuous rigid body reads
\begin{equation}
	\eqbox{
		I_{ij}
		% = \sum_\alpha m_\alpha \qty[\delta_{ij} \sum_k x_{\alpha, k} x_{\alpha, k} - x_{\alpha, i} x_{\alpha, j}]
		% = \int_V \rho(\vec{x}) \, [\delta_{ij} \underbrace{\sum_k x_k x_k}_{\vec{x} \cdot \vec{x}} - x_i x_j] dV
		% -> denoting square of x differently below
		= \sum_\alpha m_\alpha \qty[\delta_{ij} \norm{\vec{x}_{\alpha}}^2 - x_{\alpha, i} x_{\alpha, j}]
		= \int_V \rho(\vec{x}) \, \qty[\delta_{ij} \norm{x}^2 - x_i x_j] dV
	} \, .
\end{equation}



In the previous section, we mainly derived two expressions: the one for angular momentum and the one for kinetic energy. Both involved the moment of inertia, but one subtle difference in the derivation was the placement of the respective reference point (and thus the origin) of the body system. The kinetic energy required the center of mass to be in the origin, while the angular momentum was less strict and could be calculated with respect to an arbitrary point in the body. Thinking about this in more detail, the latter is actually very convenient because it may be easier to calculate the moment of inertia for instance with respect to a different origin/reference point (depending on the geometry of the body). Therefore, it is worthwhile looking into how moments of inertia around different, parallel rotation axes are connected.

Such a situation corresponds to a coordinate transformation into a system whose origin is shifted by some displacment vector $\vec{a}$ from the center of mass (the orientation of the axes remains the same), i.e.~$\vec{r}_a = \vec{r} + \vec{a}$. The moment of inertia tensor in this shifted coordinate system then reads
\begin{align*}
	(\tensor{I}^a)_{ij}
	&= \sum_\alpha m_\alpha \qty[\delta_{ij} \norm{(\vec{x}_a)_\alpha}^2 - (x_a)_{\alpha, i} (x_a)_{\alpha, j}]
	\\
	&= \sum_\alpha m_\alpha \qty[\delta_{ij} \qty(\norm{\vec{x}_\alpha}^2 + 2 \vec{x} \cdot \vec{a} + \norm{\vec{a}}^2) - x_{\alpha, i} x_{\alpha, j} - x_{\alpha, i} a_j - x_{\alpha, j} a_i - a_i a_j]
	\\
	&= \sum_\alpha m_\alpha \qty(
	\qty[\delta_{ij} \norm{\vec{x}_\alpha}^2 - x_{\alpha, i} x_{\alpha, j}] + 
	\qty[\delta_{ij} 2 \vec{x} \cdot \vec{a} - x_{\alpha, i} a_j - x_{\alpha, j} a_i]
	\qty[\delta_{ij} \norm{\vec{a}}^2 -  - a_i a_j]
	)
	\, .
\end{align*}
This looks very messy. However, we must notice that the first term in the bracket is nothing but the (components of the) moment of inertia computed in a body system with origin in the center of mass, $\tensor{I} = \tensor{I}^{\vec{R}}$. Moreover, the second term contains in all constituents a term like $\sum_\alpha m_\alpha x_{\alpha, k}$. This is proportional to the center of mass of the $\vec{x}_{\alpha, k}$, but since they come from a coordinate system that is defined relative to the center of mass, all terms of this kind vanish. The resulting equation is called \Def{parallel-axis theorem},
\begin{equation}\label{eq:parallel_axis_theorem}
	\eqbox{
		(\tensor{I}^a)_{ij} = I_{ij} + M (a^2 \delta_{ij} - a_i a_j)
	} \, .
\end{equation}
Note that there is no particular restriction on $\vec{a}$, for instance it need not be located in the body.

-> structure of this formula is very similar to definition moment of inertia tensor itself (which should not be surprising); so basically moment is sum (difference) of moments along the different axes, and the $M$ appears because the displacements do not depend on the particle

You may know a less general version of this theorem, \Def{Steiner's parallel-axis theorem}. It determines how the moment of inertia $I_{\vec{n}}$\footnote{Yes, we are running out of symbols here. In an attempt to distinguish the moment of inertia tensor calculated in some frame and the moment of inertia around a certain axis, we use super- and subscript.} around a chosen axis $\vec{n} = \frac{\vec{\omega}}{\omega}$ evolves upon a displacement, which can be inferred as a corollary of Eq.~\eqref{eq:parallel_axis_theorem}:
\begin{equation}\label{eq:steiner_theorem}
	\begin{split}
		\eqbox{
			(I^a)_{\vec{n}}
		}
		&= \vec{n} \cdot \tensor{I}^a \cdot \vec{n}
		= \sum_{i, j} n_i (\tensor{I}^a)_{ij} n_j
		= \sum_{i, j} n_i \tensor{I} n_j + M (\norm{\vec{a}}^2 \delta_{ij} n_i n_j - a_i a_j \, n_i n_j)
		\\
		&= I_{\vec{n}} + \sum_{i, j} M (\norm{\vec{a}}^2 \norm{\vec{n}}^2 - a_i n_i \, a_j n_j)
		\eqbox{
			= I_{\vec{n}} + M (\norm{\vec{a}}^2 - (\vec{a} \cdot \vec{n})^2)
		}
	\end{split}
\end{equation}
-> this is Steiner Satz in a little disguise, is usually formulated for displacements perpendicular to rotation axis; but we can see equivalence by looking at said case, $\vec{a} \cdot \vec{n} = 0$, which yields the familiar result $(I^a)_{\vec{n}} = I_{\vec{n}} + M \norm{\vec{a}}^2$; for a displacement of the axis parallel to the axis, we expect that the moment of inertia does not change (body is still rotated in same way), and indeed since $\vec{a} \cdot \vec{n} = \norm{a}$ implies that the term in the parentheses vanishes, so $I_{\vec{n}} = J_{\vec{n}}$\\


A general result from the mathematical theory of matrices (equivalent: two-tensors) is that there exists a basis, which for us represents a coordinate system, in which their component form is diagonal, i.e.~$I_{ij} = I_{ii} \delta_{ij}$ (under certain conditions, which we assume to be fulfilled). In physical terms, this corresponds to axes where rotation axis and the resulting angular momentum are aligned, $\vec{L} \parallel \vec{\omega}$. We call these axes \Def[principal axis]{principal axes}.


-> classification of rigid bodies based on eigenvalues of inertia tensor: $I_1 = I_2 = I_3$ means body = spherical top; $I_1 = I_2 \neq I_3$ means body = symmetric top; none of them equal means body = asymmetric top; very special case is $I_1 = 0, I_2 = I_3$, whence body is called rotor (two pointmasses connected by weightless shaft for example)

-> only around axis for largest and smallest eigenvalue we have stable rotations, middle one is unstable


-> Thornton 11.7 is just some algebra to show properties of tensor (eigenvalues are real, so are eigenvectors, etc.)



\end{document}